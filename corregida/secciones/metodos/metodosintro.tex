Como ya mencionamos, el Teorema~\ref{thm:correspondenciajl} proporciona una
manera de reconstruir los espacios $\spitzH{k}{\frak{N}}$ de formas de Hilbert
cuspidales. El problema de describir la estructura de Hecke de
$\spitzH{k}{\frak{N}}$ se reduce a poder calcular $\spitzH[B]{k}{\frak{N}'}$
para un \'{a}lgebra de cuaterniones $B$ y un nivel dado $\frak{N}'$, en un
principio, arbitrarios. Por \emph{calcular} se entiende, dado un ideal primo
$\frak{p}$ de $\oka{F}$, coprimo con $\frak{N}'$, describir la acci\'{o}n de un
operador $T_{\frak{p}}$ en t\'{e}rminos de una base independiente de
$\frak{p}$. Seg\'{u}n lo desarrollado en la Secci\'{o}n
\S~\ref{sec:dehilbertformasviejasformasnuevas}, esto equivale a determinar los
posibles sistemas de autovalores $\{a_{\frak{p}}\}_{\frak{p}}$ asociados a la
familia de operadores $T_{\frak{p}}$.
%Ampliar sobre esto. En realidad, el resultado de los m\'{e}todos est\'{a} dado
%por los posibles sistemas de autovalores para la acci\'{o}n de Hecke. Explicar
%qu\'{e} sentido tiene esto: formas nuevas (Atkin-Lehner), operadores normales,
%diagonalizaci\'{o}n en paralelo de los operadores, multiplicidad uno. Estas
%observaciones son comunes a ambos m\'{e}todos.
El objetivo de este cap\'{\i}tulo es describir dos m\'{e}todos para llevar a
cabo este procedimiento. Estos m\'{e}todos se dividen en el \emph{m\'{e}todo %
indefinido} y el \emph{m\'{e}todo definido}, de acuerdo con el \'{a}lgebra $B$.
