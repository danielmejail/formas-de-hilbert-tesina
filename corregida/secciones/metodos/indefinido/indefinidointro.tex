Sea $B/F$ un \'{a}lgebra indefinida, distinta de matrices, y sea $\cal{O}$ un
orden de Eichler de nivel $\frak{N}$ en $B$. Empecemos describiendo el
m\'{o}dulo de Hecke $\spitzH[B]{k}{\frak{N}}$. La variedad de Shimura
$\shimura[B]{\frak{N}}$ es compacta%
\footnote{
	ver \S~\ref{sec:cuaternionicasvariedadesdeshimura}
}
y no tiene c\'{u}spides. Tiene sentido decir, entonces, que toda forma modular
para $B$ es cuspidal, es decir, definimos
\begin{math}
	\spitzH[B]{k}{\frak{N}}:=\modularH[B]{k}{\frak{N}}
\end{math}~.
Asumimos dado un sistema de representantes $\{\frak{a}\}$ del grupo de clases
estrictas de $F$; para cada representante $\frak{a}$, elegimos
$\hhat{a}\in\Idfin{F}$ tal que $\frak{a}=\hhat{a}\Adfin{\oka{F}}\cap F$ y
$\hhat{\alpha}\in\Idfin{B}$ tal que $\nrd(\hhat{\alpha})=\hhat{a}$.
Utilizaremos la siguiente notaci\'{o}n:
\begin{align*}
	\Adfin{\cal{O}}_{\frak{a}} \,=\,
		\hhat{\alpha}\Adfin{\cal{O}}\hhat{\alpha}^{-1} & \text{ ,}\quad
		\cal{O}_{\frak{a}}\,=\,\Adfin{\cal{O}}_{\frak{a}}\cap B
			\quad\text{y}\quad
		\Gamma_{\frak{a}}\,=\,\cal{O}_{\frak{a},+}^{\times}\,=\,
			\cal{O}_{\frak{a}}^{\times}\cap B_{+}^{\times}
	\text{ .}
\end{align*}
%

El espacio $\spitzH[B]{k}{\frak{N}}$ se descompone como suma directa de
$\spitzH[B]{k}{\frak{N},\frak{a}}$ v\'{\i}a el isomorfismo
\eqref{eq:descomposicionmodularesindefinida}, dado por
$f\mapsto(f_{\frak{a}})_{\frak{a}}$ donde
$f_{\frak{a}}(z)=f(z,\hhat{\alpha}\Idfin{\cal{O}})$. Los operadores de Hecke
act\'{u}an permutando estos subespacios. Sea $\frak{b}$ el representante de las
clases estrictas que verifica $[\frak{b}]=[\frak{a}\frak{p}^{-1}]$. Entonces,
para cada representante $\hhat{\pi}$ de las \'{o}rbitas en $\Theta(\frak{p})$,
el ret\'{\i}culo
\begin{math}
	\hhat{\alpha}\hhat{\pi}^{-1}\Adfin{\cal{O}}\hhat{\beta}^{-1}\cap B
\end{math}
es un ideal de $B$ cuyo orden a derecha es $\cal{O}_{\frak{b}}$ y su norma
reducida es un ideal en la clase (estricta) principal. En consecuencia,
%por aproximaci\'{o}n fuerte,
existe $\varpi\in B_{+}^{\times}$ tal que
\begin{math}
	\hhat{\alpha}\hhat{\pi}^{-1}\Adfin{\cal{O}}\hhat{\beta}^{-1}\cap B=
		\varpi^{-1}\cal{O}_{\frak{b}}
\end{math}
y una unidad $\hhat{u}\in\Idfin{\cal{O}}$ tal que
\begin{math}
	\hhat{\alpha}\hhat{\pi}^{-1}\hhat{u}\hhat{\beta}^{-1}=\varpi^{-1}
\end{math}~.
Si ahora miramos las componentes de la forma $T_{\frak{p}}f$ se deduce que,
usando la invarianza de $f$,
\begin{align*}
	(T_{\frak{p}}f)_{\frak{a}}(z) & \,=\,
		\sum_{\hhat{\pi}\in\Theta(\frak{p})}\,
			f(z,\hhat{\alpha}\hhat{\pi}^{-1}\Idfin{\cal{O}})
		\,=\, \sum_{\varpi}\,
			f(z,\varpi^{-1}\hhat{\beta}\Idfin{\cal{O}})
		\,=\, \sum_{\varpi}\,\big(
			f_{\frak{b}}\operadormatrices{\peso{k}}{\varpi}\big)(z)
	\text{ .}
\end{align*}
%
Si bien esto muestra que la descripci\'{o}n de $T_{\frak{p}}$ en t\'{e}rminos
de la descomposici\'{o}n de $\spitzH[B]{k}{\frak{N}}$ es sencilla, la utilidad
de la igualdad
\begin{math}
	\big(T_{\frak{p}}f\big)_{\frak{a}}=
		\sum_{\varpi}\,f_{\frak{b}}\operadormatrices{\peso{k}}{\varpi}
\end{math}
depende de poder hallar todos aquellos elementos $\varpi$ cuya existencia
est\'{a} garantizada por aproximaci\'{o}n fuerte. Estos elementos se pueden
caracterizar globalmente. La sumatoria se realiza sobre un sistema de
representantes de
\begin{align*}
	\Theta(\frak{p})_{\frak{a},\frak{b}} & \,=\,
		\Gamma_{\frak{b}}\backslash\Big\{\varpi\in
			I_{\frak{b}}I_{\frak{a}}^{-1}\cap B_{+}^{\times}\,:\,
			\nrd(\varpi)\frak{a}\frak{p}^{-1}=\frak{b}
		\Big\}
	\text{ ,}
\end{align*}
%
donde $I_{\frak{a}}=\hhat{\alpha}\Adfin{\cal{O}}\cap B$ e
$I_{\frak{b}}=\hhat{\beta}\Adfin{\cal{O}}\cap B$. El argumento es similar al
dado en la Observaci\'{o}n \ref{obs:idelesdenormapporglobales}. El resultado
\ref{propo:descomposicioninducedescomposicion} tambi\'{e}n es v\'{a}lido en
este caso, es decir, siendo $B$ un \'{a}lgebra de divisi\'{o}n indefinida.%
\footnote{
	\cite[Propo.~2.3]{ShimuraDirichletSeriesAbelianVarieties}
}

\begin{obsHeckeParaIndefinidosPorBloques}%
	\label{obs:heckeparaindefinidosporbloques}
	Los operadores de Hecke act\'{u}an por bloques en
	$\spitzH[B]{k}{\frak{N}}$, permutando en cierto sentido los sumandos
	$\spitzH[B]{k}{\frak{N},\frak{a}}$: dados
	$\frak{p}\nmid\frak{D}\frak{N}$ primo y $\frak{a}$ y $\frak{b}$ tales
	que $[\frak{b}]=[\frak{a}\frak{p}^{-1}]$, definimos
	\begin{math}
		\big(T_{\frak{p}}\big)_{\frak{a},\frak{b}}:\,
			\spitzH[B]{k}{\frak{N},\frak{b}}\rightarrow
			\spitzH[B]{k}{\frak{N},\frak{a}}
	\end{math}
	%
	por
	\begin{align*}
		\big(T_{\frak{p}}\big)_{\frak{a},\frak{b}}f_{\frak{b}} & \,=\,
		\sum_{\varpi\in\Theta(\frak{p})_{\frak{a},\frak{b}}}\,
			f_{\frak{b}}\operadormatrices{\peso{k}}{\varpi}
		\text{ .}
	\end{align*}
	%
	Entonces el operador $T_{\frak{p}}$ act\'{u}a como la matriz de
	operadores
	\begin{math}
		\big[\big(T_{\frak{p}}\big)_{\frak{a},\frak{b}}
			\big]_{\frak{a},\frak{b}}
	\end{math}~,
	donde $\frak{a}$ y $\frak{b}$ recorren los representantes de las
	clases estrictas de $F$ y $(T_{\frak{p}})_{\frak{a},\frak{b}}$ es
	el operador reci\'{e}n definido, si $[\frak{a}]=[\frak{b}\frak{p}]$, y
	es igual a $0$, en caso contrario.
\end{obsHeckeParaIndefinidosPorBloques}

El m\'{e}todo indefinido se basa en poder resolver el siguiente problema de
manera expl\'{\i}cita.

\begin{probFormasIndefinida}\label{prob:formasindefinida}
	Dado un cuerpo de n\'{u}meros totalmente real $F$ de grado
	$n=[F:\bb{Q}]$, un \'{a}lgebra de cuaterniones indefinida $B/F$, un
	ideal $\frak{N}$ de $F$ coprimo con el discriminante $\frak{D}$ de $B$
	y un peso $\peso{k}\in\bb{Z}^{n}$, calcular los posibles
	sistemas de autovalores de los operadores de Hecke $T_{\frak{p}}$
	actuando en $\spitzH[B]{k}{\frak{N}}$ para
	$(\frak{p},\frak{D}\frak{N})=1$.
\end{probFormasIndefinida}

En el caso en que el \'{a}lgebra $B/F$ ramifica en $n-1$ lugares arquimedianos
y el peso verifica $\peso{k}\in (2\bb{Z}_{\geq 1})^{n}$, este problema admite
una soluci\'{o}n mediante el c\'{a}lculo en la cohomolog\'{\i}a de curvas de
Shimura.%
\footnote{
	\cite{GreenbergVoight} y \cite{VoightComputingOverArbitrary}
}
En particular, si el grado de la extensi\'{o}n, $n$, es impar, eligiendo $B$ de
discriminante $\frak{D}=1$, es decir, $B$ ramifica exactamente en $n-1$ lugares
arquimedianos (y en ning\'{u}n lugar finito), por la correspondencia de
Jacquet-Langlands, se obtiene un algoritmo que permite calcular los sistemas de
autovalores en el espacio de formas de Hilbert cuspidales de nivel $\frak{N}$
de los operadores de Hecke coprimos con el nivel.

Sean $F$, $B$, $\frak{N}$ y $\peso{k}$ como en el enunciado del problema y sean
$\lista{v}{n}$ los lugares arquimedianos de $F$. Supongamos que $B$ ramifica en
$v_{i}$ para $i\geq 2$. En tal caso,
\begin{math}
	B_{\infty}\simeq\MM_{2\times 2}(\bb{R})\times\bb{H}^{n-1}
\end{math}~.
Supongamos, tambi\'{e}n, que contamos con un orden de Eichler $\cal{O}$ en $B$
de nivel $\frak{N}$, coprimo con $\frak{D}$. Existe una descomposici\'{o}n de la variedad de Shimura asociada
a $B$ y a $\cal{O}$ dada por
\begin{align*}
	\shimura[B]{\frak{N}}(\bb{C}) & \,=\,\bigsqcup_{\frak{a}}\,
		\Gamma_{\frak{a}}\backslash\hP
	\text{ ,}
\end{align*}
%
donde cada componente
\begin{math}
	\shimura[B]{\frak{N},\frak{a}}(\bb{C})=\Gamma_{\frak{a}}\backslash\hP
\end{math}
es una \emph{curva} conexa y compacta.
En lo que resta de esta secci\'{o}n, reservamos la notaci\'{o}n
$\Gamma_{\frak{a}}$ para la imagen de $\cal{O}_{\frak{a},+}^{\times}$ en
$\PGL_{2}(\bb{R})$. Fijamos una inmersi\'{o}n
\begin{math}
	\iota_{1}:\,B^{\times}\rightarrow\GL_{2}(\bb{R})
\end{math}~,
asociada al lugar arquimediano en donde $B$ es no ramificada, y denotamos por
$\iota_{\infty}$ la imagen de $\iota_{1}$ en el cociente.
%$\PGL_{2}(\bb{R})$.
Entonces
\begin{align*}
	\Gamma_{\frak{a}} & \,:=\,\iota_{\infty}\big(
		\cal{O}_{\frak{a},+}^{\times}\big) \,\subset\,
		\PGLtp_{2}(\bb{R})
		% \,\simeq\,\cal{O}_{\frak{a},+}^{\times}/\oka{F}^{\times}
	\text{ .}
\end{align*}
%

Si $f\in\spitzH[B]{k}{\frak{N}}$, las componentes
$f_{\frak{a}}:\,\hP\rightarrow W_{\peso{k}}(\bb{C})$ est\'{a}n dadas por
$f_{\frak{a}}(z)=f(z,\hhat{\alpha}\Idfin{\cal{O}})$ y verifican
\begin{equation}
	\label{eq:indefinidaoperadordepesokmetodos}
	\big(f_{\frak{a}}\operadormatrices{\peso{k}}{\gamma}\big)(z) \,:=\,
		\frac{\det(\gamma_{1})^{m_{1}+k_{1}-1}}{%
			j(\gamma_{1},z_{1})^{k_{1}}}\,
		f_{\frak{a}}(\gamma_{1} z)^{\gamma} \,=\,f_{\frak{a}}(z)
\end{equation}
%
para toda $\gamma\in\cal{O}_{\frak{a},+}^{\times}$. Si
$\peso{k}\in(2\bb{Z})^{n}$ es par, entonces la expresi\'{o}n anterior tiene
sentido tomando $\gamma\in\Gamma_{\frak{a}}$ e identificando
$\Gamma_{\frak{a}}\simeq\cal{O}_{\frak{a},+}^{\times}/\oka{F}^{\times}$.

Los espacios de formas cuaterni\'{o}nicas $\spitzH[B]{k}{\frak{N}}$ se realizan
en la cohomolog\'{\i}a de las curvas $\Gamma_{\frak{a}}\backslash\hP$ a
trav\'{e}s de los isomorfismos de Eichler-Shimura
\cite[Propo.~4.4]{MatsushimaShimura}. Esto permite estudiar la
estructura del m\'{o}dulo de Hecke de formas modulares cuaterni\'{o}nicas,
utilizando algoritmos espec\'{\i}ficos para el c\'{a}lculo en cohomolog\'{\i}a.
Es necesario, entonces, entender c\'{o}mo se traslada la acci\'{o}n de los
operadores de Hecke a una acci\'{o}n en cohomolog\'{\i}a.

