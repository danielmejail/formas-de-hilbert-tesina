En base a los resultados enunciados anteriormente, concluimos que es posible
realizar los espacios de formas cuaterni\'{o}nicas $\spitzH[B]{k}{\frak{N}}$
en la cohomolog\'{\i}a de los grupos $\Gamma_{\frak{a}}$.
% Esto permite dar una descripci\'{o}n de los operadores de Hecke
% $\big\{T(\frak{p})\,:\,\frak{p}\nmid\frak{D}\frak{N}\big\}$ actuando en
% \begin{math}
	% H^{+}=\bigoplus_{\frak{a}}\,\cohomolo[1]{%
		% \Gamma_{\frak{a}},L_{\peso{k}}(\bb{C})}
% \end{math}
% usando t\'{e}cnicas de \'{a}lgebra lineal.
Expresar este pasaje en forma algor\'{\i}tmica presenta sus propias
dificultades. Para concluir la descripci\'{o}n del m\'{e}todo indefinido,
mencionamos algunos de los pasos involucrados y los problemas que presentan. Se
puede encontrar una descripci\'{o}n m\'{a}s detallada en
\cite{GreenbergVoight}.

Asumimos dadas instancias concretas del cuerpo $F$ de grado $n$, del
\'{a}lgebra $B/F$ de discriminante $\frak{D}$ que ramifica en $n-1$ lugares
arquimedianos y del ideal $\frak{N}$ coprimo con el discriminante. La primera
reducci\'{o}n consiste en hallar un \emph{splitting} de $B$, es decir, una
extensi\'{o}n galoisiana $K/F$, de manera que
$B\otimes_{F}K\simeq\MM_{2\times 2}(K)$. Todos los c\'{a}lculos se realizan,
entonces, de manera exacta sobre el cuerpo $K$. Restringiendo a los m\'{o}dulos
$L_{\peso{k}}(K)$, la acci\'{o}n de los operadores de Hecke se determina en el
$K$-espacio vectorial
\begin{align*}
	H & \,=\,\bigoplus_{\frak{a}}\,\cohomolo[1]{%
		\Gamma_{\frak{a}},L_{\peso{k}}(K)}
	\text{ .}
\end{align*}
%
Para poder llevar esto a cabo, es necesario hallar una descripci\'{o}n de este
espacio en t\'{e}rminos de alguna base conveniente.

Supongamos dado, entonces, un orden de Eichler $\cal{O}$ de nivel $\frak{N}$ en
$B$. En primer lugar, se obtiene un sistema de representantes $\{\frak{a}\}$ de
las clases estrictas. Esto se realiza con el requerimiento adicional de que los
representantes sean \'{\i}ntegros y coprimos con el discriminante $\frak{D}$ y
con el nivel $\frak{N}$. Habiendo hecho esto, se busca un conjunto
$\{I_{\frak{a}}\}_{\frak{a}}$ de ideales del \'{a}lgebra $B$ tal que
\begin{align*}
	I_{\frak{a}}\,\subset\,\cal{O} & \quad\text{,}\quad
		\Oder(I_{\frak{a}})=\cal{O} \quad\text{y}\quad
		\nrd(I_{\frak{a}})=\frak{a}
\end{align*}
%
para cada representante $\frak{a}$.%
\footnote{
	ver \cite[Ch.~II, Thm.~2.3]{Vigneras} y
	\cite[Lemma 7.2]{KirschmerVoight}
}
Notamos que, entonces $I_{\frak{a}}=\hhat{\alpha}\Adfin{\cal{O}}\cap B$ para
alg\'{u}n $\hhat{\alpha}\in\Adfin{\cal{O}}$ y que los \'{o}rdenes
``compa\~{n}eros'' $\cal{O}_{\frak{a}}$ est\'{a}n dados por
\begin{align*}
	\cal{O}_{\frak{a}} & \,=\,\Oizq(I_{\frak{a}})
	\text{ .}
\end{align*}
%
% El paso siguiente es, pues, determinar los \'{o}rdenes a izquierda
% $\cal{O}_{\frak{a}}$.

Usando los algoritmos descriptos en \cite{VoightFuchsian} se pueden hallar
presentaciones $\Gamma_{\frak{a}}=\langle S_{\frak{a}}|R_{\frak{a}}\rangle$
finitas minimales para cada representante $\frak{a}$ junto con una soluci\'{o}n
al problema de la palabra para dichas presentaciones. \'{E}ste es el paso
m\'{a}s costoso,%
\footnote{
	\cite{GreenbergVoight}
}
pero es lo que permite realizar los c\'{a}lculos de manera expl\'{\i}cita.

%
% \begin{obsCohomologiaCeroExplicita}\label{obs:cohomolgiaceroexplicita}
	% Si $S\subset\Gamma$ es un conjunto de generadores, entonces
	% $M^{\Gamma}$ est\'{a} compuesto por los elementos de $M$ que verifican
	% $x\cdot s=x$ para todo $s\in S$, o, lo que es lo mismo,
	% \begin{align*}
		% M^{\Gamma} & \,=\,\bigcap_{s\in S}\,\ker(1-s)
		% \text{ ,}
	% \end{align*}
	% %
	% viendo $K[\Gamma]$ como subanillo de $\Endo_{K}(M)$.
% \end{obsCohomologiaCeroExplicita}
%%
%% observaciones acerca de la implementaci\'on (?)
% \begin{obsCohomologiaUnoExplicita}\label{obs:cohomologiaunoexplicita}
	% Sea $\Gamma$ un grupo, $K$ un anillo conmutativo con unidad y sea
	% $M$ un $K[\Gamma]$-m\'{o}dulo. Sea $\Gamma=\langle S|R\rangle$ una
	% presentaci\'{o}n de $\Gamma$ y sea $V$ el $K[\Gamma]$-m\'{o}dulo de
	% funciones $S\rightarrow M$.
	% % Supongamos que el conjunto de generadores $S$ es finito.
	% Entonces
	% \begin{align*}
		% V & \,=\,\prod_{s\in S}\,M_{s}
		% \text{ ,}
	% \end{align*}
	% %
	% donde $M_{s}=M$, identificando una funci\'{o}n $u:\,S\rightarrow M$ con
	% el conjunto indexado $(u(s))_{s\in S}$. La inclusi\'{o}n
	% $S\subset\Gamma$ determina un morfismo
	% \begin{math}
		% C^{1}(\Gamma,M) \rightarrow V
	% \end{math}
	% por restricci\'{o}n y, en particular, existe un morfismo
	% \begin{align*}
		% \cociclos[1]{\Gamma,M} &\,\rightarrow\,V \\
		% \qquad f & \,\mapsto\,(f(s))_{s\in S}
	% \end{align*}
	% %
	% Afirmamos que este morfismo es inyectivo. Dada una funci\'{o}n $f\in V$
	% es posible extenderla a una funci\'{o}n $F(S)\rightarrow M$, donde
	% $F(S)$ denota el grupo libre en $S$, de manera que se verifique la
	% condici\'{o}n de cociclo y, adem\'{a}s, tal extensi\'{o}n es \'{u}nica.
	% El grupo libre $F(S)$ act\'{u}a en $M$ a derecha v\'{\i}a el
	% epimorfismo $F(S)\rightarrow F(S)/N\simeq\Gamma$, donde $N$ es el
	% subgrupo normal generado en $F(S)$ por el conjunto de relaciones $R$.
	% Si $s\in S$, el valor $f(s)$ est\'{a} definido y se define
	% \begin{align*}
		% f(e) \,:=\,0 & \quad\text{y}\quad
			% f(s^{-1}) \,:=\,-f(s)\cdot s^{-1}\text{ .}
	% \end{align*}
	% %
	% Inductivamente, si $k\geq 1$ y $\lista{s}{k+1}\in S\cup S^{-1}$,
	% \begin{align*}
		% f(s_{1}\cdots s_{k+1}) & \,:=\,
			% f(s_{1}\cdots s_{k})\cdot s_{k+1}\,+\,f(s_{k+1})
		% \text{ .}
	% \end{align*}
	% %
	% (ver \cite[Ch.~IV, \S\S~1, 2]{MacLaneHomology}). De las ecuaciones
	% \eqref{eq:propiedadesdeloscociclos}, con $F(S)$ en lugar de $\Gamma$,
	% $\delta=r\in R$ y $\gamma=w\in F(S)$,
	% \begin{align*}
		% f(wrw^{-1})\cdot w & \,=\,f(r) \,-\, f(w)\cdot(1-r)\,=\, f(r)
		% \text{ .}
	% \end{align*}
	 % %
	% En particular, $f=0$ en el subgrupo $N$, si y s\'{o}lo si $f(r)=0$ para
	% todo $r\in R$. Si ahora $w,w'\in F(S)$ son dos palabras en
	% $S\cup S^{-1}$ que representan el mismo elemento de $\Gamma$, es decir,
	% $w^{-1}w'=n\in N$, entonces
	% \begin{align*}
		% f(w') & \,=\,f(w\,w^{-1}w')\,=\,f(w)\cdot n\,+\,f(n)\,=\,
			% f(w)\,+\,f(n)
		% \text{ .}
	% \end{align*}
	% %
	% De esto se deduce que $f$ define una funci\'{o}n en $\Gamma$, si y
	% s\'{o}lo si $f(r)=0$ para todo $r\in R$. En particular, el cociclo en
	% $F(S)$ que se obtiene a partir de $(f(s))_{s\in S}\in V$, con
	% $f\in\cociclos[1]{\Gamma,M}$, coincide con $f$. En definitiva, el
	% morfismo $\cociclos[1]{\Gamma,M}\rightarrow V$ es inyectivo y su imagen
	% es el subm\'{o}dulo de funciones $f:\,S\rightarrow M$ tales que, luego
	% de extenderlas a $F(S)$ como cociclos, verifican $f(r)=0$ para
	% $r\in R$. Es decir,
	% \begin{align*}
		% \cociclos[1]{\Gamma,M} & \,\simeq\,
			% \big\{f(r)=0\,\forall r\in R\big\}\,\subset\,
			% \cociclos[1]{F(S),M}\,\simeq\,V
		% \text{ .}
	% \end{align*}
	% %
% 
	% De manera similar, es posible caracterizar $\cobordes[1]{\Gamma,M}$
	% como subm\'{o}dulo de $V$. Una funci\'{o}n $f:\,\Gamma\rightarrow M$ es
	% un coborde, si y s\'{o}lo si existe $x\in M$ tal que $f=\partial x$.
	% En tal caso, $f(s)=x-x\cdot s$ para todo elemento $s\in S$ del conjunto
	% generador. Rec\'{\i}procamente, si $f\in V$ es tal que
	% $f(s)=x-x\cdot s$ para todo $s\in S$, entonces el morfismo cruzado en
	% $F(S)$ determinado por $f$ cumple que
	% \begin{align*}
		% f(s_{1}s_{2}) & \,=\,f(s_{1})\cdot s_{2}\,+\,f(s_{2}) \\
		% & \,=\,(x-x\cdot s_{1})\cdot s_{2}\,+\,(x-x\cdot s_{2})
			% \,=\,x-x\cdot (s_{1}s_{2})
		% \text{ ,}
	% \end{align*}
	% %
	% e, inductivamente, si $w=s_{1}\cdots s_{k+1}$, $f(w)=x-x\cdot w$. En
	% particular, $f(r)=0$ para toda $r\in R$ y $f\in\cociclos[1]{\Gamma,M}$.
	% M\'{a}s aun, $f(\gamma)=x-x\cdot\gamma$ implica que $f$ es un coborde.
	% As\'{\i}, se comprueba que $\cobordes[1]{\Gamma,M}$ es el subm\'{o}dulo
	% de $V$ compuesto por las funciones de la forma $f(s)=x-x\cdot s$ para
	% cierto $x\in M$.
% \end{obsCohomologiaUnoExplicita}
% 
% A partir de haber hallado presentaciones
% \begin{math}
	% \Gamma_{\frak{a}}=\generado{S_{\frak{a}}\mid R_{\frak{a}}}
% \end{math}~,
% los cociclos $\cociclos[1]{\Gamma_{\frak{a}},L_{\peso{k}}(K)}$ se identifican
% con subespacios
% \begin{align*}
	% \cociclos[1]{\Gamma_{\frak{a}},L_{\peso{k}}(K)} & \,\subset\,
		% \bigoplus_{s\in S_{\frak{a}}}\,L_{\peso{k}}(K)_{s}
% \end{align*}
% %
% v\'{\i}a $f\mapsto (f(s))_{s\in S_{\frak{a}}}$; las ecuaciones lineales que los
% definen se obtienen de las condiciones $f(r)=0$ con $r\in R_{\frak{a}}$,
% escribiendo cada elemento $r$ en t\'{e}rminos de los generadores. De manera
% similar, se encuentran los subespacios correspondientes a los cobordes
% $\cobordes[1]{\Gamma_{\frak{a}},L_{\peso{k}}(K)}$ y se calcula una base del
% cociente $\cohomolo[1]{\Gamma_{\frak{a}},L_{\peso{k}}(K)}$, para cada
% $\frak{a}$.
% 
% Las presentaciones de los grupos $\Gamma_{\frak{a}}$ resultan esenciales,
% tambi\'{e}n, para calcular las matrices de los operadores $T(\frak{p})$ y
% $W_{\infty}$. Tanto los operadores $T(\frak{p})$ como $W_{\infty}$ act\'{u}an
% por bloques, posiblemente permutando las componentes de $H$. Para cada
% representante $\frak{a}$ de las clases estrictas, se determinan, en primera
% instancia, los representantes $\frak{b}$ y $\frak{a}'$ que verifican
% \begin{math}
	% [\frak{a}\frak{p}^{-1}]=[\frak{b}]
% \end{math}
% y
% \begin{math}
	% [\frak{a}\frak{m}]=[\frak{a}']
% \end{math}~.
% La matriz de $T(\frak{p})$ es entonces la matriz por bloques compuesta por
% las matrices asociadas a los operadores parciales
% \begin{align*}
	% T(\frak{p})_{\frak{a},\frak{b}} & \,:\,\cohomolo[1]{%
		% \Gamma_{\frak{b}},L_{\peso{k}}(K)}\,\rightarrow\,
		% \cohomolo[1]{\Gamma_{\frak{a}},L_{\peso{k}}(K)}
	% \text{ .}
% \end{align*}
% %
% An\'{a}logamente, los bloques de la matriz de $W_{\infty}$ son las matrices de
% los operadores
% \begin{align*}
	% W_{\infty,\frak{a},\frak{a}'} & \,:\,\cohomolo[1]{%
		% \Gamma_{\frak{a}'},L_{\peso{k}}(K)}\,\rightarrow\,
		% \cohomolo[1]{\Gamma_{\frak{a}},L_{\peso{k}}(K)}
	% \text{ .}
% \end{align*}
% %
% Seg\'{u}n \eqref{eq:involucionencomponentesmetodos}, para obtener la matriz de
% $W_{\infty,\frak{a},\frak{a}'}$, es necesario expresar el elemento
% \begin{align*}
	% & f(\mu_{\frak{a},\frak{a}'}\,s\,\mu_{\frak{a},\frak{a}'}^{-1})^{%
		% \mu_{\frak{a},\frak{a}'}}
% \end{align*}
% %
% en la base de la componente en $\frak{a}$, para cada generador
% $s\in S_{\frak{a}}$ de $\Gamma_{\frak{a}}$ y $f$ en la base de la componente en
% $\frak{a}'$. Similarmente, por \eqref{eq:heckeencohomologia}, para cada
% $s\in S_{\frak{a}}$, se calculan la permutaci\'{o}n $\varpi\mapsto\varpi_{s}$ y
% los elementos $\delta_{\varpi}$ que verifican
% \eqref{eq:permutacionnormapmetodos} y se expresa, luego,
% \begin{align*}
	% & \sum_{\varpi}\,f(\delta_{\varpi})^{\varpi_{s}}
% \end{align*}
% %
% en la base de la componente en $\frak{a}$ para cada $f$ perteneciente a la base
% de la componente en $\frak{b}$.
