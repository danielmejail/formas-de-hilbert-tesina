Sea $\frak{p}$ un ideal primo de $\oka{F}$ que no divide a $\frak{D}\frak{N}$ y
sean $\frak{a}$ y $\frak{b}$ representantes que cumplen con
$[\frak{b}]=[\frak{a}\frak{p}^{-1}]$. Si
$f\in\spitzH[B]{k}{\frak{N},\frak{b}}$, el operador de Hecke en $f$,
\begin{align*}
	T_{\frak{p}}f & \,=\,\sum_{\varpi}\,
		f\operadormatrices{\peso{k}}{\varpi}
	\text{ ,}
\end{align*}
%
define una forma en $\spitzH[B]{k}{\frak{N},\frak{a}}$, donde la suma se
realiza sobre representantes de $\Theta(\frak{p})_{\frak{a},\frak{b}}$. El
grupo $\Gamma_{\frak{a}}$ act\'{u}a \emph{derecha} sobre este conjunto. Si
$\varpi\in\Theta(\frak{p})_{\frak{a},\frak{b}}$ y $\gamma\in\Gamma_{\frak{a}}$,
existen $\varpi_{\gamma}\in\Theta(\frak{p})_{\frak{a},\frak{b}}$ y
$\delta_{\varpi}\in\Gamma_{\frak{b}}$ tales que
\begin{equation}
	\label{eq:permutacionnormapmetodos}
	\varpi\gamma \,=\, \delta_{\varpi}\varpi_{\gamma}
	\text{ .}
\end{equation}
%
As\'{\i}, en analog\'{\i}a con la definici\'{o}n de $T_{\frak{p}}$,
se define, dada $f\in H^{1}(\Gamma_{\frak{b}},L_{\peso{k}}(\bb{C}))$,
un elemento $T(\frak{p})f$ del grupo
$H^{1}(\Gamma_{\frak{a}},L_{\peso{k}}(\bb{C}))$ como la clase del cociclo
\begin{equation}
	\label{eq:heckeencohomologia}
	T(\frak{p})f \,:\, \gamma\,\mapsto\,
		\sum_{\varpi}\,f(\delta_{\varpi})^{\varpi_{\gamma}}
	\text{ .}
\end{equation}
%

\begin{obsHeckeEnCohomologia}\label{obs:heckeencohomologia}
	La expresi\'{o}n \eqref{eq:heckeencohomologia} define un cociclo en
	$\cociclos[1]{\Gamma_{a},L_{\peso{k}}(\bb{C})}$. Dadas
	$\gamma,\gamma'\in\Gamma_{\frak{a}}$,
	\begin{align*}
		\varpi\,(\gamma\gamma') & \,=\,
			(\delta_{\varpi}\varpi_{\gamma})\,\gamma' \,=\,
			(\delta_{\varpi}\delta_{\varpi_{\gamma}}')\,
				(\varpi_{\gamma})_{\gamma'}
		\text{ ,}
	\end{align*}
	%
	para ciertas
	\begin{math}
		\delta_{\varpi},\delta_{\varpi_{\gamma}}'\in\Gamma_{\frak{b}}
	\end{math}~.
	En particular, la permutaci\'{o}n
	\begin{math}
		\varpi\mapsto \varpi_{\gamma\gamma'}
	\end{math}
	es la composici\'{o}n de las permutaciones determinadas por $\gamma$ y
	por $\gamma'$ y, adem\'{a}s, si denotamos $\delta_{\varpi}''$ al
	elemento que verifica
	\begin{math}
		\varpi\,(\gamma\gamma')=\delta_{\varpi}''\varpi_{\gamma\gamma'}
	\end{math}~,
	vale que $\delta_{\varpi}''=\delta_{\varpi}\delta_{\varpi_{\gamma}}'$.
	Entonces
	\begin{align*}
		\big(T(\frak{p})f\big)(\gamma\gamma') & \,=\,\sum_{\varpi}\,
			f(\delta_{\varpi}\delta_{\varpi_{\gamma}}'
				)\cdot(\varpi_{\gamma})_{\gamma'} \,=\,
			\sum_{\varpi}\,f(\delta_{\varpi})\cdot
				\delta_{\varpi_{\gamma}}'
				(\varpi_{\gamma})_{\gamma'} \,+\,
				f(\delta_{\varpi_{\gamma}}')\cdot
				(\varpi_{\gamma})_{\gamma'} \\
		& \,=\,\Big(\sum_{\varpi}\,f(\delta_{\varpi})\cdot
			\varpi_{\gamma}\Big)\cdot\gamma' \,+\,
			\sum_{\varpi}\,f(\delta_{\varpi_{\gamma}}')\cdot
				(\varpi_{\gamma})_{\gamma'} \\
		& \,=\,\big(T(\frak{p})f\big)(\gamma)\cdot\gamma' \,+\,
			\big(T(\frak{p})f\big)(\gamma')
		\text{ .}
	\end{align*}
	%
	Si
	\begin{math}
		f =\partial x\in
			\cobordes[1]{\Gamma_{\frak{b}},L_{\peso{k}}(\bb{C})}
	\end{math}~,
	\begin{align*}
		\big(T(\frak{p})f\big)(\gamma) & \,=\,\sum_{\varpi}\,
			x\cdot (1-\delta_{\varpi})\,\varpi_{\gamma} \,=\,
			\Big(\sum_{\varpi}\,x\cdot\varpi_{\gamma}\Big)\,-\,
			\sum_{\varpi}\,x\cdot\varpi\gamma \\
		& \,=\,\Big(\sum_{\varpi}\,x\cdot\varpi\Big)\cdot(1-\gamma)
		\text{ .}
	\end{align*}
	%
\end{obsHeckeEnCohomologia}

Las transformaciones
\begin{math}
	T(\frak{p}):\,\cohomolo[1]{\Gamma_{\frak{b}},L_{\peso{k}}(\bb{C})}
		\rightarrow
		\cohomolo[1]{\Gamma_{\frak{a}},L_{\peso{k}}(\bb{C})}
\end{math}
inducen un endomorfismo $T(\frak{p})$ en la suma directa $H$, que, como los
operadores $T_{\frak{p}}$ en formas cuspidales, permuta las componentes.

\begin{propoEichlerShimuraHeckeEquivariante}%
	\label{propo:eichlershimuraheckeequivariante}
	Con estas definiciones,
	\begin{align*}
		T(\frak{p})\big(\ES(f)\big) & \,=\,\ES\big(T_{\frak{p}}f\big)
		\text{ ,}
	\end{align*}
	%
	para toda $f\in\spitzH[B]{k}{\frak{N}}$.
\end{propoEichlerShimuraHeckeEquivariante}

\begin{proof}
	Sean $\frak{a},\frak{b}$ tales que $[\frak{a}]=[\frak{b}\frak{p}]$.
	\begin{align*}
		& \ES_{\frak{a}}\big(T_{\frak{p}}f_{\frak{b}}\big)(\gamma)
			\,=\,
			\int_{\tau_{0}}^{\gamma^{-1}\tau_{0}}\,\sum_{\varpi}\,
			\frak{v}\big(f_{\frak{b}}\operadormatrices{\peso{k}}{%
				\varpi_{\gamma}}\big) \,=\,
			\int_{\tau_{0}}^{\gamma^{-1}\tau_{0}}\,\sum_{\varpi}\,
			\varpi_{\gamma}^{*}\frak{v}(f_{\frak{b}})\cdot
				\varpi_{\gamma} \\
		& \qquad\,=\,
			\sum_{\varpi}\,\int_{\varpi_{\gamma}\tau_{0}}^{%
			\varpi_{\gamma}\gamma^{-1}\tau_{0}}\,
			\frak{v}(f_{\frak{b}})\cdot\varpi_{\gamma} \,=\,
			\sum_{\varpi}\,\Big(
			\int_{\tau_{0}}^{\delta_{\varpi}^{-1}\varpi\tau_{0}}\,
				\frak{v}(f_{\frak{b}})\,-\,
			\int_{\tau_{0}}^{\varpi_{\gamma}\tau_{0}}\,
				\frak{v}(f_{\frak{b}})\Big)\cdot\varpi_{\gamma}
				\\
		& \qquad\,=\,
			\sum_{\varpi}\,\Big(
			\int_{\tau_{1}}^{\delta_{\varpi}^{-1}\tau_{1}}\,
				\frak{v}(f_{\frak{b}})\,+\,
				p_{\varpi}\cdot\delta_{\varpi}\Big)\cdot
					\varpi_{\gamma}\,-\,
			\sum_{\varpi}\,p_{\varpi_{\gamma}}\cdot\varpi_{\gamma}
		\text{ .}
	\end{align*}
	%
	La \'{u}ltima igualdad es cierta m\'{o}dulo cobordes, donde
	$\tau_{1}=\varpi\tau_{0}$ y
	\begin{math}
		p_{\varpi}=\int_{\tau_{0}}^{\varpi\tau_{0}}\,
			\frak{v}(f_{\frak{b}})
	\end{math}~.
	Pero, entonces, dado que $\delta_{\varpi}\varpi_{\gamma}=\varpi\gamma$,
	\begin{align*}
		\ES_{\frak{a}}\big(T_{\frak{p}}f_{\frak{b}}\big)(\gamma)
			& \,=\,
		\sum_{\varpi}\,\int_{\tau_{1}}^{\delta_{\varpi}^{-1}\tau_{1}}\,
			\frak{v}(f_{\frak{b}})\cdot\varpi_{\gamma}\,-\,
			x\cdot(1-\gamma) \,=\,
			T(\frak{p})\big(\ES_{\frak{b}}(f_{\frak{b}})\big)
				(\gamma)
		\text{ ,}
	\end{align*}
	%
	con
	\begin{math}
		x=\sum_{\varpi}\,\int_{\tau_{0}}^{\varpi\tau_{0}}\,
			\frak{v}(f_{\frak{b}})\cdot\varpi
	\end{math}~.
\end{proof}

\begin{propoInvolucionConmutaConHecke}\label{propo:involucionconmutaconhecke}
	Los operadores de Hecke $T(\frak{p})$ conmutan con la involuci\'{o}n
	$W_{\infty}$. Es decir,
	\begin{align*}
		T(\frak{p})\big(f\big\vert W_{\infty}\big) & \,=\,
			\big(T(\frak{p})f\big)\big\vert W_{\infty}
		\text{ ,}
	\end{align*}
	%
	para toda $f\in H$.
\end{propoInvolucionConmutaConHecke}

\begin{proof}
	Supongamos que $[\frak{a}]=[\frak{b}\frak{p}]$,
	$[\frak{a}\frak{m}]=[\frak{a}']$ y que
	$[\frak{b}\frak{m}]=[\frak{b}']$. Entonces
	$[\frak{a}']=[\frak{b}'\frak{p}]$. Sea
	\begin{math}
		f\in\cohomolo[1]{\Gamma_{\frak{b}'},L_{\peso{k}}(\bb{C})}
	\end{math}~.
	Por un lado,
	\begin{align*}
		T(\frak{p})\big(f\big\vert W_{\infty,\frak{b},\frak{b}'}\big)
			(\gamma) & \,=\,\sum_{\varpi}\,\big(
			f\big\vert W_{\infty,\frak{b},\frak{b}'}\big)
				(\delta_{\varpi})\cdot\varpi_{\gamma} \,=\,
			\sum_{\varpi}\,f(\mu_{\frak{b},\frak{b}'}
			\delta_{\varpi}\mu_{\frak{b},\frak{b}'}^{-1})\cdot
				\mu_{\frak{b},\frak{b}'}\varpi_{\gamma}
		\text{ .}
	\end{align*}
	%
	Evaluando en el orden inverso,
	\begin{align*}
		\big(T(\frak{p})f\big)\big\vert W_{\infty,\frak{a},\frak{a}'}
			(\gamma) & \,=\,\big(T(\frak{p})f\big)
				(\mu_{\frak{a},\frak{a}'}\gamma
					\mu_{\frak{a},\frak{a}'}^{-1})\cdot
					\mu_{\frak{a},\frak{a}'} \,=\,
			\sum_{\varpi'}\,f(\delta_{\varpi'}')\cdot
				\varpi_{\gamma'}'\mu_{\frak{a},\frak{a}'}
		\text{ ,}
	\end{align*}
	%
	donde $\varpi'$ recorre un sistema de representantes de
	$\Theta(\frak{p})_{\frak{a}',\frak{b}'}$, $\gamma'$ denota el conjugado
	\begin{math}
		\gamma'=\mu_{\frak{a},\frak{a}'}\gamma
			\mu_{\frak{a},\frak{a}'}^{-1}\in\Gamma_{\frak{a}'}
	\end{math}
	y los elementos $\delta_{\varpi'}'\in\Gamma_{\frak{b}'}$ son tales que
	\begin{math}
		\varpi'\gamma'=\delta_{\varpi'}'\varpi_{\gamma'}'
	\end{math}~.
	Pero
	\begin{align*}
		\varpi'\gamma'\,=\,\delta_{\varpi'}'\varpi_{\gamma'}' &
			\quad\Leftrightarrow\quad
		(\mu_{\frak{b},\frak{b}'}^{-1}\varpi'\mu_{\frak{a},\frak{a}'})
			\,\gamma\,=\,
		\mu_{\frak{b},\frak{b}'}^{-1}\delta_{\varpi'}'
			\mu_{\frak{b},\frak{b}'}\,
			(\mu_{\frak{b},\frak{b}'}^{-1}\varpi_{\gamma'}'
				\mu_{\frak{a},\frak{a}'})
		\text{ ,}
	\end{align*}
	%
	con
	\begin{math}
		\mu_{\frak{b},\frak{b}'}^{-1}\delta_{\varpi'}'
			\mu_{\frak{b},\frak{b}'}\in\Gamma_{\frak{b}}
	\end{math}
	y la aplicaci\'{o}n
	\begin{math}
		\varpi\mapsto\mu_{\frak{b},\frak{b}'}\varpi
			\mu_{\frak{a},\frak{a}'}^{-1}
	\end{math}
	es una correspondencia
	\begin{math}
		\Theta(\frak{p})_{\frak{a},\frak{b}}\xrightarrow{\sim}
			\Theta(\frak{p})_{\frak{a}',\frak{b}'}
	\end{math}~.
	En definitiva,
	\begin{align*}
		\sum_{\varpi'}\,f(\delta_{\varpi'}')\cdot\varpi_{\gamma'}'
			\mu_{\frak{a},\frak{a}'} & \,=\,
			\sum_{\varpi}\,
				f(\mu_{\frak{b},\frak{b}'}\delta_{\varpi}
			\mu_{\frak{b},\frak{b}'}^{-1})\cdot
			\mu_{\frak{b},\frak{b}'}\varpi_{\gamma}
		\text{ ,}
	\end{align*}
	%
	lo que concluye la demostraci\'{o}n.
\end{proof}

\begin{coroEichlerShimura}\label{coro:eichlershimura}
	El isomorfismo
	\begin{math}
		\ES^{+} :\,\spitzH[B]{k}{\frak{N}}\,\xrightarrow{\sim}\,
		\bigoplus_{\frak{a}}\,\cohomolo[1]{%
			\Gamma_{\frak{a}},L_{\peso{k}}(\bb{C})}^{+}
	\end{math}
	es equivariante respecto de la acci\'{o}n de Hecke.
\end{coroEichlerShimura}
%

