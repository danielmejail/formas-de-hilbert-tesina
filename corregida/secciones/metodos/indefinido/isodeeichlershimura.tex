Definimos, ahora, una aplicaci\'{o}n
\begin{math}
	\frak{v}:\,\spitzH[B]{k}{\frak{N},\frak{a}}\rightarrow
		\Omega^{1}(\hP,L_{\peso{k}}(\bb{C}))
\end{math}~,
que, a una forma cuspidal $f$ le asigna una $1$-forma (holomorfa) en $\hP$ a
valores en el espacio vectorial complejo $L_{\peso{k}}(\bb{C})$. Dado un punto
$z\in\hP$, sea $v(z)\in\wpoli{1}(\bb{C})$ el polinomio
\begin{align*}
	\mathrm{v}(z) & \,=\, zX\,+\,Y \,=\,
		\begin{bmatrix} z & 1 \end{bmatrix}\,
		\begin{bmatrix} X \\ Y \end{bmatrix}
	\text{ .}
\end{align*}
%
Si $\gamma\in\GL_{2}(\bb{R})$, se cumple
\begin{align*}
	\mathrm{v}(\gamma z) & \,=\,
		\begin{bmatrix} \frac{az+b}{cz+d} & 1 \end{bmatrix}\,
		\begin{bmatrix} X \\ Y \end{bmatrix}
	\,=\, j(\gamma,z)^{-1} \begin{bmatrix} z & 1 \end{bmatrix}
		\begin{bmatrix} a & c \\ b & d \end{bmatrix}
		\begin{bmatrix} X \\ Y \end{bmatrix}
\end{align*}
%
Dados $w\geq 0$ y $m\in\bb{Z}$, miramos a $\mathrm{v}(z)^{w}$ como elemento de
$\wmpoli{w}{m}(\bb{C})$ y, recordando que
$\gamma^\iota=\det(\gamma)\gamma^{-1}$, deducimos que
\begin{align*}
	\mathrm{v}(z)^{w}\cdot\gamma^{-1} & \,=\,\det(\gamma)^{-m}\,
		\big(\mathrm{v}(z)^{w}\big) \Big(
		\begin{bmatrix} X & Y \end{bmatrix}\,
			(\det(\gamma)^{-1}\,\gamma^\iota)^\iota\Big) \\
	& \,=\,\det(\gamma)^{-m}\,\Big(
		\begin{bmatrix} z & 1 \end{bmatrix}\,\det(\gamma)^{-1}\,
			\gamma^t
		\begin{bmatrix} X \\ Y \end{bmatrix}\Big)^{w}
	\text{ .}
\end{align*}
%
Es decir,
\begin{align*}
	\mathrm{v}(\gamma z)^{w} & \,=\,j(\gamma,z)^{-w}\,\det(\gamma)^{w+m}\,
		\big(\mathrm{v}(z)^{w}\cdot\gamma^{-1}\big)
\end{align*}
%
Si $f\in\spitzH[B]{k}{\frak{N},\frak{a}}$ es una forma cuspidal, definimos
\begin{align*}
	\frak{v}(f) & \,=\, \mathrm{v}(z)^{w_{1}}\otimes f(z)\,dz
	\text{ ,}
\end{align*}
%
con $\mathrm{v}(z)^{w_{1}}\in\wmpoli{w_{1}}{m_{1}}(\bb{C})$.

\begin{obsPullbackDeUnoForma}\label{obs:pullbackdeunoforma}
	Calculamos el pullback de $\frak{v}(f)$ por una transformaci\'{o}n
	$\gamma$. De la expresi\'{o}n
	\eqref{eq:indefinidaoperadordepesokmetodos} y recordando que
	$w_{1}=k_{1}-2$, se deduce que
	\begin{align*}
		\gamma^{*}\frak{v}(f) & \,=\, \mathrm{v}(\gamma z)^{w_{1}}
			\otimes	f(\gamma z)\,d(\gamma z) \\
		& \,=\, \frac{\det(\gamma_{1})^{w_{1}+m_{1}}}{%
				j(\gamma_{1},z)^{w_{1}}}
			\,\big(\mathrm{v}(z)^{w_{1}}\cdot\gamma^{-1}\big)
				\otimes f(\gamma z)\,
			\frac{\det(\gamma_{1})}{j(\gamma_{1},z)^{2}}\,dz \\
		& \,=\,\big(\mathrm{v}(z)^{w_{1}}\cdot\gamma^{-1}\big)
			\otimes\bigg(
			\frac{\det(\gamma_{1})^{k_{1}+m_{1}-1}}{%
				j(\gamma_{1},z)^{k_{1}}}f(\gamma z)\bigg)\,dz
		\text{ .}
	\end{align*}
	%
	Es decir, en general,
	\begin{equation}
		\label{eq:pullbackdeunoforma}
		\gamma^{*}\frak{v}(f) \,=\,\big(
			\mathrm{v}(z)^{w_{1}}\otimes
				f\operadormatrices{\peso{k}}{%
				\gamma}(z)\big)\cdot\gamma^{-1}\,dz \,=\,
			\frak{v}\big(f\operadormatrices{\peso{k}}{\gamma}\big)
				\cdot\gamma^{-1}
		\text{ .}
	\end{equation}
	%
	Si $\gamma\in\Gamma_{\frak{a}}$, entonces
	\begin{math}
		\gamma^{*}\frak{v}(f) =\frak{v}(f)\cdot\gamma^{-1}
	\end{math}~.
\end{obsPullbackDeUnoForma}

Sea $\tau_{0}\in\hP$ y sea $p_{0}\in L_{\peso{k}}(\bb{C})$. Dada una forma
cuspidal $f$, la expresi\'{o}n
\begin{align*}
	\varphi(\tau) & \,=\,\int_{\tau_{0}}^{\tau}\,\frak{v}(f)\,+\,p_{0}
\end{align*}
%
define una funci\'{o}n $\hP\rightarrow L_{\peso{k}}(\bb{C})$. Por la
Observaci\'{o}n~\ref{obs:pullbackdeunoforma}, dada
$\gamma\in\Gamma_{\frak{a}}$,
\begin{align*}
	\varphi(\gamma\tau) & \,=\,\int_{\gamma\tau_{0}}^{\gamma\tau}\,
		\frak{v}(f)\,+\,\int_{\tau_{0}}^{\gamma\tau_{0}}\,\frak{v}(f)
		\,+\,p_{0} \,=\,
		\bigg(\int_{\tau_{0}}^{\tau}\,\frak{v}(f)\bigg)\cdot\gamma^{-1}
		\,+\,\int_{\tau_{0}}^{\gamma\tau_{0}}\,\frak{v}(f)\,+\,p_{0} \\
	& \,=\,	\varphi(\tau)\cdot\gamma^{-1}\,+\,\tilde{t}(\gamma)
	\text{ ,}
\end{align*}
%
donde
\begin{align*}
	\tilde{t}(\gamma) & \,:=\,\int_{\tau_{0}}^{\gamma\tau_{0}}\,
		\frak{v}(f)\,+\,p_{0}\cdot(1-\gamma^{-1})
	\text{ .}
\end{align*}
%
La funci\'{o}n
\begin{math}
	t:\,\Gamma_{\frak{a}}\rightarrow L_{\peso{k}}(\bb{C})
\end{math}
dada por $t(\gamma)=\tilde{t}(\gamma^{-1})$ verifica
\begin{align*}
	t(\gamma\delta) & \,=\,t(\gamma)\cdot\delta\,+\,t(\delta)
	\text{ .}
\end{align*}
%
Es decir, $t\in\cociclos[1]{\Gamma_{\frak{a}},L_{\peso{k}}(\bb{C})}$. Cambiando
$p_{0}$ por alg\'{u}n otro elemento del m\'{o}dulo $L_{\peso{k}}(\bb{C})$, la
funci\'{o}n que se obtiene en lugar de $t$ difiere de \'{e}sta en un coborde.
Definimos, dada una forma cuspidal $f$, una funci\'{o}n
$t_{f}:\,\Gamma_{\frak{a}}\rightarrow L_{\peso{k}}(\bb{C})$ por
\begin{align*}
	t_{f}(\gamma) & \,=\,\int_{\tau_{0}}^{\gamma^{-1}\tau_{0}}\,\frak{v}(f)
\end{align*}
%
y, al igual que $t$, la funci\'{o}n $t_{f}$ est\'{a} definida, m\'{o}dulo
$\cobordes[1]{\Gamma_{\frak{a}},L_{\peso{k}}(\bb{C})}$,
independientemente de $\tau_{0}$ y satisface
\begin{math}
	t_{f}(\gamma\delta)=t_{f}(\gamma)\cdot\delta+t_{f}(\delta)
\end{math}~.
La aplicaci\'{o}n $f\mapsto [t_{f}]$ determina una transformaci\'{o}n
$\bb{C}$-lineal
\begin{align*}
	& \ES_{\frak{a}} \,:\,\spitzH[B]{k}{\frak{N},\frak{a}} \,\rightarrow\,
		\cohomolo[1]{\Gamma_{\frak{a}},L_{\peso{k}}(\bb{C})}
\end{align*}
%
Como, los operadores de Hecke act\'{u}an en los espacios
$\spitzH[B]{k}{\frak{N},\frak{a}}$ de manera conjunta, consideramos
\begin{align*}
	& \ES \,:\,\spitzH[B]{k}{\frak{N}}\,\rightarrow\, H\,=\,
		\bigoplus_{\frak{a}}\,
			\cohomolo[1]{\Gamma_{\frak{a}},L_{\peso{k}}(\bb{C})}
\end{align*}
%
dada por
\begin{math}
	\ES =\bigoplus_{\frak{a}}\,\ES_{\frak{a}}
\end{math}~.

\begin{obsCohomologiaReal}\label{obs:cohomologiareal}
	Repitiendo el razonamiento anterior con la parte real
	$\Re(\frak{v}(f))$ en lugar de $\frak{v}(f)$ y $\bb{R}$ en lugar de
	$\bb{C}$, se deduce que la aplicaci\'{o}n
	\begin{math}
		f\mapsto\text{clase}\Big[\gamma\mapsto
			\int_{\tau_{0}}^{\gamma^{-1}\tau_{0}}\,
			\Re(\frak{v}(f))\Big]
	\end{math}~determina una transformaci\'{o}n lineal \emph{real}
	\begin{math}
		\spitzH[B]{k}{\frak{N},\frak{a}}\rightarrow
			\cohomolo[1]{\Gamma_{\frak{a}},L_{\peso{k}}(\bb{R})}
	\end{math}~.
	Esta transformaci\'{o}n es un isomorfismo de $\bb{R}$-espacios
	vectoriales (ver \cite[Thm.~8.4]{ShimuraIntroduction}).
	%
	% Si repetimos el razonamiento anterior con $\Re(\frak{v}(f))$ en lugar
	% de $\frak{v}(f)$, definiendo una funci\'{o}n
	% $\tau\mapsto\int_{\tau_{0}}^{\tau}\,\Re(\frak{v}(f))+r$, con
	% $r\in L_{\peso{k}}(\bb{R})$, obtenemos una funci\'{o}n
	% $u_{f}:\,\Gamma_{\frak{a}}\rightarrow L_{\peso{k}}(\bb{R})$
	% perteneciente a $Z^{1}(\Gamma_{\frak{a}},L_{\peso{k}}(\bb{R}))$. Y,
	% como antes, si cambiamos $r$ por alg\'{u}n otro polinomio $r'$,
	% obtenemos un cociclo $u'$ que cumple que $u_{f}-u'$ pertenece a
	% $B^{1}(\Gamma_{\frak{a}},L_{\peso{k}}(\bb{R}))$. Finalmente, definimos
	% \begin{align*}
		% & \varphi \,:\, \spitzH[B]{k}{\frak{N},\frak{a}}\,\rightarrow\,
			% H^{1}(\Gamma_{\frak{a}},L_{\peso{k}}(\bb{R})) \\
		% & \qquad\qquad f \,\mapsto\, [u_{f}]
		% \text{ .}
	% \end{align*}
	%
	Se puede obtener un isomorfismo de $\bb{C}$-espacios, por medio de una
	involuci\'{o}n en $H$. La definici\'{o}n que proporcionamos de dicha
	involuci\'{o}n est\'{a} adaptada de
	\cite[\S~2.4]{VoightComputingOverArbitrary}.
\end{obsCohomologiaReal}

Recordemos de \S~\ref{sec:clasesdeideales} el grupo de clases
\begin{math}
	\Class[(+)]{F}=
		F_{(+)}^{\times}\backslash\Idfin{F}/\Idfin{\oka{F}}
\end{math}~, donde $F_{(+)}^{\times}=\nrd(B^{\times})$ es la imagen de la norma
reducida en $F^{\times}$. El n\'{u}cleo del epimorfismo can\'{o}nico
\begin{math}
	\pClass{F}=
		F_{+}^{\times}\backslash\Idfin{F}/\Idfin{\oka{F}}
			\rightarrow\Class[(+)]{F}
\end{math}~est\'{a} dado por
\begin{math}
	F_{(+)}^{\times}/F_{+}^{\times}=\big(\bb{Z}/2\bb{Z}\big)^{r}
\end{math}~, donde $r$ es la cantidad de lugares arquimedianos no ramificados
de $B$.%
\footnote{
	ver la Observaci\'{o}n~\ref{obs:eichlernorma}.
}
Puesto que la cantidad de lugares arquimedianos en donde el \'{a}lgebra
$B/F$ no ramifica es $r=1$, el n\'{u}cleo es trivial --este es el caso, si y
s\'{o}lo si existe una unidad $u\in\oka{F}^{\times}$ que cumple $v_{i}(u)>0$
para $i\geq 2$ y $v_{1}(u)<0$-- o bien es c\'{\i}clico de orden $2$. En todo
caso, podemos elegir un ideal $\frak{m}$ de $F$ cuya clase estricta genere el
n\'{u}cleo del morfismo can\'{o}nico $\pClass{F}\rightarrow\Class[(+)]{F}$.
Dado un representante $\frak{a}$ del grupo de clases estrictas, existe un
\'{u}nico representante $\frak{a}'$ tal que $[\frak{a}\frak{m}]=[\frak{a}']$.
% Observaci\'{o}n tonta, $a=a'$, si y s\'{o}lo si $[m]=1$ (es decir, el
%R n\'{u}cleo es trivial). En ese caso, la involuci\'{o}n preservar\'{a} las
% componentes; en otro caso, las permutar\'{a} de manera no trivial (con
% orden $2$). En todo caso, la $\mu_{a,a'}$ se deber\'{\i}a poder hallar.
En particular, las clases de $\frak{a}$ y de $\frak{a}'$ son iguales en
$\Class[(+)]{F}$. Ahora, como $B/F$ es indefinida, seg\'{u}n el Teorema~%
\ref{thm:eichlernorma}, la norma reducida induce un isomorfismo entre este
grupo de clases y el cociente
\begin{math}
	B^{\times}\backslash\Idfin{B}/\Idfin{\cal{O}}
\end{math}~. Deducimos que existe $\mu_{\frak{a},\frak{a}'}\in B^{\times}$ tal
que $\det((\mu_{\frak{a},\frak{a}'})_{1})<0$ y
\begin{math}
	\hhat{\alpha}\Idfin{\cal{O}}=\mu_{\frak{a},\frak{a}'}^{-1}
		\hhat{\alpha}'\Idfin{\cal{O}}
\end{math}~,
o, lo que es lo mismo,
\begin{align*}
	\mu_{\frak{a},\frak{a}'}\,\Gamma_{\frak{a}}\,
		\mu_{\frak{a},\frak{a}'}^{-1} & \,=\,\Gamma_{\frak{a}'}
	\text{ .}
\end{align*}
%
Dado $f\in\cociclos[1]{\Gamma_{\frak{a}'},L_{\peso{k}}(\bb{C})}$, sea
\begin{math}
	f\big\vert W_{\infty,\frak{a},\frak{a}'}\in
		\cociclos[1]{\Gamma_{\frak{a}},L_{\peso{k}}(\bb{C})}
\end{math}
el cociclo definido por
\begin{equation}
	\label{eq:involucionencomponentesmetodos}
	f\big\vert W_{\infty,\frak{a},\frak{a}'}(\gamma) \,=\,
		f(\mu_{\frak{a},\frak{a}'}\gamma\mu_{\frak{a},\frak{a}'}^{-1}
			)\cdot\mu_{\frak{a},\frak{a}'}
	\text{ .}
\end{equation}
%
Si $f\in\cobordes[1]{\Gamma_{\frak{a}'},L_{\peso{k}}(\bb{C})}$, entonces la
igualdad $(1-\mu\gamma\mu^{-1})\mu=\mu(1-\gamma)$ implica que
$f\big\vert W_{\infty,\frak{a},\frak{a}'}$ es un coborde. Queda determinada una
transformaci\'{o}n lineal
\begin{align*}
	& W_{\infty,\frak{a},\frak{a}'} \,:\,
		\cohomolo[1]{\Gamma_{\frak{a}'},L_{\peso{k}}(\bb{C})}
			\,\rightarrow\,
		\cohomolo[1]{\Gamma_{\frak{a}},L_{\peso{k}}(\bb{C})}
	\text{ ,}
\end{align*}
%
para cada par de representantes $\frak{a}$ y $\frak{a}'$ tales que
$[\frak{a}\frak{m}]=[\frak{a}']$. Denotamos por $W_{\infty}$ el endomorfismo
inducido en $H$:
\begin{align*}
	(f_{\frak{a}'})_{\frak{a}'}\big\vert W_{\infty} & \,=\,
		(f_{\frak{a}'}\big\vert W_{\infty,\frak{a},\frak{a}'}
			)_{\frak{a}}
	\text{ .}
\end{align*}
%

\begin{obsInvolucionNoDependeDeMuCohomologia}%
	\label{obs:involucionnodependedemucohomologia}
	La definici\'{o}n de $W_{\infty,\frak{a},\frak{a}'}$ no depende de la
	elecci\'{o}n de $\mu_{\frak{a},\frak{a}'}$. Sea
	$\nu=\nu_{\frak{a},\frak{a}'}\in B^{\times}$ tal que $\det(\nu_{1})<0$
	y
	\begin{math}
		\hhat{\alpha}\Idfin{\cal{O}}=
			\nu^{-1}\hhat{\alpha}'\Idfin{\cal{O}}
	\end{math}
	y sea $\mu=\mu_{\frak{a},\frak{a}'}$. Sea
	$f\in\cociclos[1]{\Gamma_{\frak{a}'},L_{\peso{k}}(\bb{C})}$ y sea
	$g$ el cociclo en $\Gamma_{\frak{a}}$ dado por
	$g(\gamma)=f(\mu\gamma\mu^{-1})\cdot\mu$. La igualdad
	$\nu\hhat{\alpha}\Idfin{\cal{O}}=\mu\hhat{\alpha}\Idfin{\cal{O}}$
	implica que $\mu^{-1}\nu\in\Gamma_{\frak{a}}$, con lo cual, por
	la expresi\'{o}n \eqref{eq:propiedadesdeloscociclos},
	\begin{align*}
		f(\nu\gamma\nu^{-1})\cdot\nu & \,=\,
			g\big((\mu^{-1}\nu)\,\gamma\,(\mu^{-1}\nu)^{-1}\big)
				\cdot (\mu^{-1}\nu) \\
		& \,=\,g(\gamma)\,-\,g(\mu^{-1}\nu)\cdot(1-\gamma)\,\in\,
			f(\mu\gamma\mu^{-1})\cdot\mu\,+\,
			\cobordes[1]{\Gamma_{\frak{a}},L_{\peso{k}}(\bb{C})}
		\text{ .}
	\end{align*}
	%
\end{obsInvolucionNoDependeDeMuCohomologia}

Por la Observaci\'{o}n~\ref{obs:involucionnodependedemucohomologia}, la
igualdad $[\frak{m}]^{2}=[1]$ en $\pClass{F}$ implica que $W_{\infty}$ es una
involuci\'{o}n en $H$. Si $[\frak{a}\frak{m}]=[\frak{a}']$ entonces
$[\frak{a}'\frak{m}]=[\frak{a}]$. En particular, $W_{\infty}^{2}$ es
diagonal y es igual a
\begin{math}
	W_{\infty,\frak{a}',\frak{a}}\circ W_{\infty,\frak{a},\frak{a}'}
\end{math}
en la componente correspondiente a $\frak{a}$. Tomando
$\mu_{\frak{a}',\frak{a}}=\mu_{\frak{a},\frak{a}'}^{-1}$ en la definici\'{o}n
de $W_{\infty,\frak{a}',\frak{a}}$, se deduce que $W_{\infty}^{2}$ es la
identidad en cada componente.

Por medio de $W_{\infty}$, podemos descomponer el espacio $H$ como suma directa
de autoespacios para la involuci\'{o}n: $H=H^{+}\oplus H^{-}$, donde
\begin{align*}
	H^{\pm} & \,=\,\Big\{f\in H\,:\,f\big\vert W_{\infty}=\pm f\Big\}
	\text{ .}
\end{align*}
%

\begin{teoEichlerShimura}[{\cite[Propo.~4.4]{MatsushimaShimura}}]%
	\label{thm:eichlershimura}
	Si $\ES^{+}:\,\spitzH[B]{k}{\frak{N}}\rightarrow H^{+}$ denota la
	composici\'{o}n de $\ES$ con la proyecci\'{o}n en el autoespacio
	$H^{+}$, entonces $\ES$ es un isomorfismo de $\bb{C}$-espacios
	vectoriales.
\end{teoEichlerShimura}
