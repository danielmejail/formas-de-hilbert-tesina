% El problema an\'{a}logo a \ref{prob:formasindefinida} para un \'{a}lgebra
% totalmente definida $B/F$ admite una soluci\'{o}n. La descripci\'{o}n dada en
% \S~\ref{sec:cuaternionicasheckedefinida} es lo suficientemente
% expl\'{\i}cita como para ser implementada.
% 
% ?`Mencionar los m\'{e}todos necesarios para llevar esto a cabo? C\'{a}lculo de
% un orden de Eichler, representantes de las clases de ideales, \'{o}rdenes a
% izquierda de estos ideales, sus grupos de unidades; calcular los conjuntos
% $\mathscr{T}_{\frak{p}}(I)$.
% 
Sea $B/F$ un \'{a}lgebra totalmente definida y sea $\cal{O}\subset B$ un orden
de Eichler. Denotamos por $\ideales{\cal{O}}$ el conjunto de ideales
(invertibles) $I$ de $B$ cuyo orden a derecha es $\Oder(I)=\cal{O}$. Sea
$H=\#\lClass{\cal{O}}$ y fijemos un sistema de representantes $\{I_{t}\}_{t}$
de las clases a izquierda. Dado un ideal $I\in\ideales{\cal{O}}$, denotamos por
$[I]$, tanto su clase en $\lClass{\cal{O}}$, como la funci\'{o}n
caracter\'{\i}stica de la misma. El \emph{estabilizador} de un ideal $I$ es el
grupo
\begin{align*}
	\Gamma_{I} & \,:=\,\big\{b\in B^{\times}\,:\,bI=I\big\} \,=\,
		\Oizq(I)^{\times}
	\text{ .}
\end{align*}
%
Los elementos centrales contenidos en este grupo son precisamente
$\Gamma_{I}\cap F^{\times}=\oka{F}^{\times}$. Las inmersiones reales de $F$
determinan una inclusi\'{o}n discreta en un compacto:
\begin{align*}
	& \Gamma_{I}/\oka{F}^{\times}\,\hookrightarrow\,
		\Idinf{B}/\centre(\bb{R})
		% \,\simeq\,(\bb{H}^{\times}/\bb{R}^{\times})^{n}
		\,\simeq\,(S^{3})^{n}
	\text{ ,}
\end{align*}
%
de donde se deduce que
\begin{align*}
	w_{I} & \,:=\, \big|\Gamma_{I}/\oka{F}^{\times}\big|\,<\,\infty
	\text{ .}
\end{align*}
%
Todo ideal invertible de $B$ es localmente principal y la correspondecia
$\Idfin{B}/\Idfin{\cal{O}}\xrightarrow{\sim}\ideales{\cal{O}}$ est\'{a}
determinada por
% dada por $\hhat{\alpha}\Idfin{\cal{O}}\mapsto I$, si
$I=\hhat{\alpha}\Adfin{\cal{O}}\cap B$.%
\footnote{
	ver \S~\ref{sec:clasesdeideales}
}
Sean $\hhat{\alpha}_{t}\in\Idfin{B}$ tales que
$I_{t}=\hhat{\alpha}_{t}\Adfin{\cal{O}}\cap B$. Si
$I=\hhat{\alpha}\Adfin{\cal{O}}\cap B$, escribimos
$[\hhat{\alpha}\Idfin{\cal{O}}]$ para indicar la clase, o bien la funci\'{o}n
caracter\'{\i}stica en $\Idfin{B}/\Idfin{\cal{O}}$, correspondiente a $[I]$.

Una forma modular cuaterni\'{o}nica para $B$ de nivel $\cal{O}$ y peso
$\peso{k}$ es una funci\'{o}n
\begin{math}
	f:\,\Idfin{B}/\Idfin{\cal{O}}\rightarrow W_{\peso{k}}(\bb{C})
\end{math}
tal que
\begin{align*}
	f(\gamma\hhat{\alpha}\Idfin{\cal{O}}) & \,=\,
		f(\hhat{\alpha}\Idfin{\cal{O}})^{\gamma^{-1}}
	\text{ .}
\end{align*}
%
Alternativamente, una forma modular es una funci\'{o}n
$f:\,\ideales{\cal{O}}\rightarrow W_{\peso{k}}(\bb{C})$ equivariante respecto
de la acci\'{o}n a izquierda de $B^{\times}$. Esta equivalencia permite dar una
descripci\'{o}n alternativa de los operadores de Hecke.

Seg\'{u}n \eqref{eq:cuaternionicasoperadorcoclase},
\begin{align*}
	\big(T_{\frak{p}}f\big)(\hhat{\alpha}\Idfin{\cal{O}}) & \,=\,
		\sum_{i}\,f(\hhat{\alpha}\hhat{\pi}_{i}^{-1}\Idfin{\cal{O}})
	\text{ .}
\end{align*}
%
Fijado $\hhat{\alpha}\Idfin{\cal{O}}$, para cada $\hhat{\pi}_{i}$ existe un
\'{u}nico $t\in [\![1,H]\!]$ tal que
\begin{math}
	[\hhat\alpha\hhat\pi_i^{-1}\Idfin{\cal O}]=
		[\hhat\alpha_t\Idfin{\cal O}]
\end{math}. Existe, entonces, un elemento $\rho_{i}\in B^{\times}$ que verifica
\begin{equation}
	\label{eq:cuaternionicadefinidaidealesequivalentes}
	\rho_{i}\hhat{\alpha}\hhat{\pi}_{i}^{-1}\Idfin{\cal{O}} \,=\,
		\hhat{\alpha}_{t}\Idfin{\cal{O}}
	\text{ .}
\end{equation}
%
Entonces podemos comprobar que
\begin{align*}
	\big(T_{\frak p}f\big)(\hhat\alpha\Idfin{\cal O}) & \,=\,
		\sum_{t=1}^H\,\sum_i\,f(\hhat\alpha_t\Idfin{\cal O})^{\rho_i}\,
			[\hhat\alpha_t\Idfin{\cal O}]
				(\hhat\alpha\hhat\pi_i^{-1}\Idfin{\cal O})
	\text{ .}
\end{align*}
%
Si $\rho_{i}'\in B^\times$ cumple con
\eqref{eq:cuaternionicadefinidaidealesequivalentes}, entonces
\begin{math}
	\rho_i^{-1}\hhat\alpha_t\Idfin{\cal O}=
		{\rho_i'}^{-1}\hhat\alpha_t\Idfin{\cal O}
\end{math}, lo que significa que $\rho_i$ y $\rho_i'$ difieren en una unidad
del orden $\cal{O}_{t}=\Oizq(I_{t})$:
\begin{align*}
	\rho_i' & \,\in\,\Gamma_t\,\rho_i
	\text{ .}
\end{align*}
%
% ($\rho_{i}'\rho_{i}^{-1}\in\Gamma_{t}$).

Sea $\hhat\pi\in\frak I(\frak p)$, es decir, $\hhat\pi\in\Adfin{\cal O}$ que
cumple $\nrd(\hhat\pi)\in\hhat p\Idfin{\oka F}$. Como antes, existe un
\'{u}nico $t\in [\![1,H]\!]$ y un $\rho\in B^\times$ tales que
$\rho\hhat\alpha\hhat\pi^{-1}\Idfin{\cal O}=\hhat\alpha_t\Idfin{\cal O}$. El
elemento $\rho$ es \'{u}nico m\'{o}dulo multiplicar a izquierda por
$\Gamma_t$. Notamos que
\begin{align*}
	\rho & \,\in\,\hhat\alpha_t\Idfin{\cal O}\hhat\pi\hhat\alpha^{-1}
		\,\cap\,B^\times
	\text{ .}
\end{align*}
%
Llamamos $I=\hhat\alpha\Adfin{\cal O}\cap B$,
$I_t=\hhat\alpha_t\Adfin{\cal O}\cap B$ y definimos
\begin{align*}
	L & \,=\,II_t^{-1}\rho \,=\,
		\hhat\alpha\Adfin{\cal O}\hhat\alpha_t^{-1}\rho\,\cap\,B \,=\,
		\big(\hhat\alpha\Adfin{\cal O}\hhat\alpha^{-1}\big)\,
			(\hhat\alpha\hhat\pi\hhat\alpha^{-1})\,\cap\,B
	\text{ .}
\end{align*}
%
Cambiando $\rho$ por $\rho'\in\Gamma_t\rho$, se obtiene el mismo ret\'{\i}culo
$L$ en $B$ y cambiando $\hhat\pi$ por $\hhat\pi'\in\Idfin{\cal O}\hhat\pi$,
tambi\'{e}n. Este ret\'{\i}culo de $B$ (no necesariamente es un $\cal O$-ideal
a derecha) cumple:
\begin{align*}
	\Oizq(L)\,=\,\Oizq(I) & \quad\text{,}\quad
		L\,\subset\,\Oizq(L) \quad\text{y}\quad
		\nrd(L)\,=\,\frak p
	\text{ .}
\end{align*}
%
Rec\'{\i}procamente, si $L\subset B$ posee estas propiedades, entonces
\begin{math}
	L=\big(\hhat\alpha\Adfin{\cal O}\hhat\alpha^{-1}\big)\,
		(\hhat\alpha\hhat\pi\hhat\alpha^{-1})\cap B
\end{math} para cierto $\hhat\pi\in\frak I(\frak p)$. Desandando los pasos
previos, vemos que existe un \'{u}nico $t\in[\![1,H]\!]$ tal que
\begin{math}
	[\hhat\alpha\hhat\pi^{-1}\Idfin{\cal O}]=
		[\hhat\alpha_t\Idfin{\cal O}]
\end{math} y existe un $\rho\in B^\times$ tal que
\begin{math}
	\rho\hhat\alpha\hhat\pi^{-1}\Idfin{\cal O}=
		\hhat\alpha_t\Idfin{\cal O}
\end{math}; es decir,
\begin{math}
	\rho\in\hhat\alpha_t\Idfin{\cal O}\hhat\pi\hhat\alpha^{-1}
\end{math} y
\begin{align*}
	L & \,=\,\big(\hhat\alpha\Adfin{\cal O}\hhat\alpha_t^{-1}\big)\,
		\big(\hhat\alpha_t\Idfin{\cal O}\hhat\pi\hhat\alpha^{-1}\big)
		\,\cap\,B \,=\,II_t^{-1}\rho
	\text{ .}
\end{align*}
%

Para expresar la conclusi\'{o}n del argumento anterior de manera simple,
introducimos la siguiente notaci\'{o}n:
\begin{align*}
	\Theta(\frak p) & \,=\,\Idfin{\cal O}\backslash\frak I(\frak p) \,=\,
		\Idfin{\cal O}\backslash
			\Big\{\hhat\pi\in\Adfin{\cal O}\,:\,
				\nrd(\hhat\pi)\in\hhat p\Idfin{\oka F}\Big\}
		\text{ ,} \\
	\mathscr R_{\frak p}(I)_t & \,=\,\Gamma_t\backslash
		\Big\{\rho\in I_tI^{-1}\,:\,\nrd(II_t^{-1}\rho)=\frak p\Big\}
		\quad\text{y} \\
	\mathscr L_{\frak p}(I) & \,=\,\Big\{L\subset B \text{ ret\'{\i}culo completo}
		\,:\,L\subset\Oizq(L)=\Oizq(I),\,\nrd(L)=\frak p
		\Big\}
	\text{ .}
\end{align*}
%
Entonces, obtenemos biyecciones
\begin{equation}
	\label{eq:biyeccionesheckedefinida}
	\Theta(\frak p) \,\leftrightarrow\,
		\bigsqcup_{t=1}^{H}\,\mathscr R_{\frak p}(I)_t
			\,\leftrightarrow\,
		\mathscr L_{\frak p}(I)
	\text{ ,}
\end{equation}
%
dadas por
\begin{math}
	\hhat\pi\in\frak I(\frak p)\mapsto\Gamma_t\rho
\end{math} --donde $t$ y $\rho\in B^\times$ cumplen que
\begin{math}
	\rho\hhat\alpha\hhat\pi^{-1}\Idfin{\cal O}=
		\hhat\alpha_t\Idfin{\cal O}
\end{math}-- y por
\begin{math}
	\rho\mapsto L=II_t^{-1}\rho
\end{math}.

Usando las biyecciones \eqref{eq:biyeccionesheckedefinida}, para cada
$t\in[\![1,H]\!]$ podemos reescribir
\begin{align*}
	\sum_i\,f(\hhat\alpha\hhat\pi_i^{-1}\Idfin{\cal O})\,
		[\hhat\alpha_t\Idfin{\cal O}]
			(\hhat\alpha\hhat\pi_i^{-1}\Idfin{\cal O}) & \,=\,
		\sum_{\mathscr R_{\frak p}(I)_t}\,
			f(\hhat\alpha_t\Idfin{\cal O})^{\rho} \,=\,
		\sum_{\mathscr R_{\frak p}(I)_t}\,f(I_t)^\rho
	\text{ .}
\end{align*}
%
Las sumatorias de la derecha, son sobre representantes de
$\mathscr R_{\frak p}(I)_t$ y, como vale que
$f(I_t)^{\lambda\rho}=f(\lambda^{-1} I_t)^\rho=f(I_t)^\rho$, si
$\lambda\in\Gamma_t$, entonces las mismas est\'{a}n bien definidas.
Reemplazando id\`{e}les por ideales, obtenemos la siguiente expresi\'{o}n para
$T_{\frak p}$:
\begin{equation}
	\label{eq:heckeparadefinidosconidealesyelementosglobales}
	\big(T_{\frak p}f\big)(I) \,=\,\sum_{t=1}^{H}\,
		\sum_{\mathscr R_{\frak p}(I)_t}\,f(I_t)^\rho
\end{equation}
%

Buscamos ahora una expresi\'{o}n para $T_{\frak p}$ en t\'{e}rminos de ideales
en $\ideales{\cal O}$. Definimos el siguiente conjunto asociado a
$I\in\ideales{\cal O}$:
\begin{align*}
	\mathscr T_{\frak p}(I) & \,=\,\Big\{J\in\ideales{\cal O}\,:\,
		J\supset I,\,\nrd(I)=\frak p\cdot\nrd(J)\Big\}
	\text{ .}
\end{align*}
%
%
% Si $L\in\mathscr L_{\frak p}(I)$, entonces
% $J=L^{-1}I\in\mathscr T_{\frak p}(I)$: $\Oder(J)=\Oder(I)=\cal O$ y, por
% multiplicatividad, $\nrd(I)=\frak p\cdot\nrd(J)$. Resta ver que $J\supset I$,
% pero
% \begin{align*}
	% J & \,=\,L^{-1}I\,\supset\,\Oizq(L)^{-1}I\,=\,\Oizq(I)I\,=\,I
	% \text{ .}
% \end{align*}
% %
% Rec\'{\i}procamente, si $J\in\mathscr T_{\frak p}(I)$, entonces
% $L=IJ^{-1}\in\mathscr L_{\frak p}(I)$: porque $\Oizq(L)=\Oizq(I)$ y
% $\nrd(L)=\frak p$. Resta ver que $L$ es \'{\i}ntegro, pero
% \begin{align*}
	% L & \,=\,IJ^{-1} \,\subset\,II^{-1}\,=\,\Oizq(L)
	% \text{ .}
% \end{align*}
% %
% En definitiva, obtenemos aplicaciones
Este conjunto est\'{a} en biyecci\'{o}n con $\mathscr L_{\frak p}(I)$ v\'{\i}a:
\begin{math}
	\big(L\mapsto L^{-1}I\big)\,:\,\mathscr L_{\frak p}(I)\rightarrow
		\mathscr T_{\frak p}(I)
\end{math}.
% \begin{equation}
	% \label{eq:biyeccionesheckereticuloseideales}
	% \begin{aligned}
		% \big(L\,\mapsto\,L^{-1}I\big) \,:\,
			% \mathscr L_{\frak p}(I)\,\rightarrow\,
				% \mathscr T_{\frak p}(I) % \quad\text{y}\quad
		% \big(J\,\mapsto\,IJ^{-1}\big) \,:\,
			% \mathscr T_{\frak p}(I)\,\rightarrow\,
				% \mathscr L_{\frak p}(I)
	% \end{aligned}
	% \text{ .}
% \end{equation}
%
% inversas una de la otra.
% 
% Ahora bien, $\mathscr T_{\frak p}(I)$ se parte como uni\'{o}n disjunta en
% funci\'{o}n de las clases en $\lClass{\cal O}$: si definimos
% \begin{math}
	% \mathscr T_{\frak p}(I)_s:=\mathscr T_{\frak p}(I)\cap\{J\sim I_s\}
% \end{math}, entonces
% \begin{align*}
	% \mathscr T_{\frak p}(I) & \,=\,\bigsqcup_{t=1}^{H}\,
		% \mathscr T_{\frak p}(I)_t
	% \text{ .}
% \end{align*}
% %
Por otra parte, la composici\'{o}n
\begin{math}
	\bigsqcup_t\,\mathscr R_{\frak p}(I)_t\rightarrow
		\mathscr L_{\frak p}(I)\rightarrow\mathscr T_{\frak p}(I)
\end{math} est\'{a} dada por
\begin{align*}
	& \Gamma_t\rho\,\mapsto\, II_t^{-1}\rho\,\mapsto\,
		\big(II_t^{-1}\rho\big)^{-1}I\,=\,\rho^{-1}I_t
\end{align*}
%
% e induce una correspondencia
% \begin{math}
	% \mathscr R_{\frak p}(I)_t\leftrightarrow
		% \mathscr T_{\frak p}(I)_t
% \end{math}.
As\'{\i}, concluimos que
\begin{equation}
	\label{eq:heckeparadefinidosconideales}
	\big(T_{\frak p}f\big)(I) \,=\,\sum_{t=1}^{H}\,
		\sum_{\mathscr R_{\frak p}(I)_t}\,f(\rho^{-1}I_t) \,=\,
		% \sum_{t=1}^{H}\,\sum_{J\in\mathscr T_{\frak p}(I)_t}\, f(J)
		\sum_{J\in\mathscr T_{\frak p}(I)}\, f(J)
	\text{ .}
\end{equation}
%

Si $\peso{k}\not =(\lista[\null]{2}{\null})$, toda forma modular para $B$
de peso $\peso{k}$ determina una forma cuspidal v\'{\i}a la correspondencia de
Jacquet-Langlands. En este caso, definimos
$\spitzH[B]{k}{\frak{N}}:=\modularH[B]{k}{\frak{N}}$. En cambio, cuando
$\peso{k}=(\lista[\null]{2}{\null})$, no toda forma
$f\in\modular[B]{(\lista[\null]{2}{\null})}{\frak{N}}$ tiene asociada una forma
de Hilbert cuspidal. A continuaci\'{o}n, definimos el espacio de formas
cuspidales de peso paralelo $2$ para un \'{a}lgebra de cuaterniones totalmente
definida y demostramos que los operadores de Hecke $T_{\frak p}$,
$\frak p\nmid\frak D\frak N$ son normales.
% La expresi\'{o}n \eqref{eq:heckeparadefinidosconideales}
% para los operadores de Hecke es v\'{a}lida, sin embargo, para un peso
% arbitrario. Para terminar con la descripci\'{o}n del m\'{e}todo definido,
% definimos el espacio $\spitz[B]{(\lista[\null]{2}{\null})}{\frak{N}}$ y
% proporcionamos una descripci\'{o}n m\'{a}s detallada de los operadores
% $T_{\frak{p}}$ actuando en este espacio.
% 
