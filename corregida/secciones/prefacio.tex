\section*{\null}
El objetivo de este trabajo ha sido presentar los fundamentos te\'{o}ricos
detr\'{a}s de dos m\'{e}todos utilizados para el c\'{a}lculo de formas de
Hilbert. Estos m\'{e}todos son conocidos como el \emph{m\'{e}todo definido} y
el \emph{m\'{e}todo indefinido}, debido al uso que hacen de las \'{a}lgebras de
cuaterniones denominadas \emph{definidas} e \emph{indefinidas},
respectivamente. Nos hemos enfocado \'{u}nicamente en los puntos que hemos
considerado m\'{a}s importantes para entender la relevancia y el funcionamiento
de dichos m\'{e}todos. Hemos intentado responder a la siguientes preguntas:
?`qu\'{e} es una forma de Hilbert? ?`qu\'{e} quiere decir ``calcular'' en este
contexto? y ?`c\'{o}mo hacemos para calcular formas de Hilbert?

Vale la pena aclarar que, en cuanto a la \'{u}ltima de estas tres preguntas, el
``c\'{o}mo'', la respuesta que damos aqu\'{\i} es una respuesta parcial; nos
hemos restringido a mostrar que nuestros objetos de inter\'{e}s tienen diversas
realizaciones, siendo algunas de ellas m\'{a}s amenas al estudio mediante
m\'{e}todos computacionales. No nos hemos detenido, sin embargo, en
proporcionar una descripci\'{o}n detallada de los algoritmos involucrados en
estos c\'{a}lculos. Los mismos se pueden hallar en las referencias;
mencionamos, particularmente, los trabajos \cite{DembeleExplicitOnQQsqrt5,
DembeleManinSymbols,DembeleDonnellyNontrivial,DembeleVoight,
GreenbergVoight,VoightComputingOverArbitrary}.

En el Cap\'{\i}tulo~\ref{cap:intro}, mencionamos algunas definiciones
b\'{a}sicas e intentamos, dentro de lo posible, ilustrar algunos puntos a
desarrollar en los cap\'{\i}tulos subsiguientes, con el caso del cuerpo de
n\'{u}meros racionales.

El Cap\'{\i}tulo~\ref{cap:preliminares} est\'{a} dedicado a proporcionar
resultados generales de la teor\'{\i}a de las \'{a}lgebras de cuaterniones. Nos
interesar\'{a}, especialmente, entender el caso en el que el cuerpo de base es
un cuerpo de n\'{u}meros. Para poder tratar este caso, repasamos brevemente el
lenguaje de los ad\`{e}les y, utilizando esta herramienta, podremos enunciar
los teoremas de clasificaci\'{o}n (Teorema~\ref{thm:clasificacionglobal}), de
aproximaci\'{o}n fuerte (Teorema~\ref{thm:aproxfuerte}) y de la norma
(Teorema~\ref{thm:eichlernorma}). Dada un \'{a}lgebra de cuaterniones y un
orden en este \'{a}lgebra podemos definir un objeto an\'{a}logo al grupo de
clases de un cuerpo de n\'{u}meros, el \emph{conjunto} de clases del orden. Los
elementos de este conjunto son clases de ideales \emph{invertibles} y, en el
caso de un orden de Eichler, la cantidad de dichas clases es finita. Hacia el
final del cap\'{\i}tulo enunciamos un resultado que relaciona el conjunto de
clases de un orden de Eichler con un grupo de clases del cuerpo de base.

En el Cap\'{\i}tulo~\ref{cap:dehilbert}, introducimos las formas de Hilbert
sobre un cuerpo totalmente real $F$; daremos una definici\'{o}n
``cl\'{a}sica'', como funciones en el producto $\hP^{n}$ de copias del
semiplano complejo superior, y una definici\'{o}n ``ad\'{e}lica'', como
funciones en el grupo $\GL_{2}(\adeles{F})$. Este cap\'{\i}tulo tambi\'{e}n
tiene por ojetivo introducir los operadores de Hecke y mostrar en qu\'{e}
sentido es posible diagonalizarlos simult\'{a}neamente y caracterizarlos por
los sistemas de autovalores asociados. Calcular los espacios de formas
de Hilbert, se reduce entonces a determinar los posibles sistemas de
autovalores para los operadores de Hecke.

Los \'{u}ltimos dos cap\'{\i}tulos est\'{a}n dedicados a las formas modulares
cuaterni\'{o}nicas. En el Cap\'{\i}tulo~\ref{cap:cuaternionicas} definimos
formas modulares para un \'{a}lgebra de cuaterniones arbitraria y enunciamos la
correspondencia de Jacquet-Langlands. Finalmente, en el Cap\'{\i}tulo~%
\ref{cap:metodos}, nos restringimos a las \'{a}lgebras definidas y a las
\'{a}lgebras indefinidas ramificadas en todos salvo un \'{u}nico lugar
arquimediano, describiendo en mayor detalle los objetos utiizados en la
pr\'{a}ctica para calcular efectivamente formas de Hilbert.

\section*{Notaci\'{o}n}
A lo largo del presente trabajo utilizaremos la siguiente notaci\'{o}n sin
nuevas aclaraciones. Los s\'{\i}mbolos $\bb{Z}$, $\bb{Q}$, $\bb{R}$, $\bb{C}$ y
$\bb{H}$ denotan, respectivamente, el anillo de n\'{u}meros enteros racionales,
los cuerpos de n\'{u}meros racionales, reales y complejos y el \'{a}lgebra de
cuaterniones de Hamilton. El semiplano complejo superior lo denotamos $\hP$. Si
$A$ es un anillo, $A^{\times}$ denotar\'{a} su grupo de unidades. Dado un
cuerpo de n\'{u}meros denotado por $K$, utilizamos $\oka{K}$ para referirnos a
su anillo de enteros. En general, letras como $\frak{n}$, $\frak{m}$,
$\frak{N}$ o $\frak{M}$ denotan ideales (\'{\i}ntegros o fraccionarios) en un
cuerpo de n\'{u}meros y $\frak{p}$ o $\frak{q}$ ideales primos. Por \'{u}ltimo,
$\GL_{2}$ denota el grupo de matrices invertibles de tama\~{n}o $2\times 2$.


\section*{Agradecimientos}
A mi madre y a mi padre, por todo su amor y apoyo y por haberme brindado tantas
oportunidades; a mi hermano, por no parar de hacerme preguntas para las cuales
no tengo respuesta; a Rosa, por tus consejos y tu insistencia; a la t\'{\i}a,
por tu cari\~{n}o constante; a Julia, por tu compa\~{n}\'{\i}a y por ``no
hacerme bulla''; a la Sette, por tu fuerza (espero que algo haya llegado hasta
ac\'{a}); a Marian, por el tenis, el truco, las papas a las brasas, pero, por
sobre todo, por tu amistad !`Gracias!

Quisiera agradecer, tambi\'{e}n, a Ariel Pacetti, por haber aceptado guiarme en
la realizaci\'{o}n de esta tesina, por haberme introducido a este \'{a}rea de
la Matem\'{a}tica, por compartir su punto de vista y por seguir
escuch\'{a}ndome y contestando mis preguntas !`Gracias, Ariel!

Finalmente, agradezco a los miembros del jurado haber dedicado su tiempo a leer
y evaluar este trabajo.

