\theoremstyle{plain}
\newtheorem{teoJacquetLanglands}{Teorema}[chapter]
\newtheorem{propoFormasModularesComoFormasAutomorfas}[teoJacquetLanglands]%
	{Proposici\'{o}n}

\theoremstyle{definition}
\newtheorem{obsGruposYOrdenesAsociadosAUnaClaseEstricta}[teoJacquetLanglands]%
	{Observaci\'{o}n}
\newtheorem{obsVariedadDeShimuraEsCompacta}[teoJacquetLanglands]%
	{Observaci\'{o}n}
\newtheorem{defFormaCuaternionicaCasoIndefinido}[teoJacquetLanglands]%
	{Definici\'{o}n}
\newtheorem{defFormaCuaternionicaCasoDefinido}[teoJacquetLanglands]%
	{Definici\'{o}n}
\newtheorem{defFormaAutomorfa}[teoJacquetLanglands]{Definici\'{o}n}


%%%intro
% 
% En esta secci\'{o}n vemos c\'{o}mo asociarle a un \'{a}lgebra de cuaterniones
% sobre un cuerpo de n\'{u}meros totalmente real una variedad de Shimura.
% %Cuando el \'{a}lgebra es un \'{a}lgebra de matrices, recuperamos las
% %variedades $Y(\Gamma)$ definidas como el cociente de un producto de
% %copias del semiplano complejo por un subgrupo discreto de $\GLtp_{2}(F)$
% %(matrices invertibles cuyo determinante es totalmente positivo).
% % Las propiedades de estas variedades son marcadamente distintas, si el
% % \'{a}lgebra es totalmente definida, indefinida o un \'{a}lgebra de matrices.
% Luego definimos, usando el lenguaje ad\'{e}lico, los espacios de formas
% modulares cuaterni\'{o}nicas, objetos an\'{a}logos a las formas modulares
% cl\'{a}sicas, y los operadores de Hecke actuando en los correspondientes
% subespacios de formas cuspidales. El producto interno de Petersson nos
% permite separar los subespacios de formas nuevas y, finalmente, enunciamos
% el resultado que nos garantiza que las formas modulares para un \'{a}lgebra
% de cuaterniones se pueden ver como formas de Hilbert.

Comenzamos este cap\'{\i}tulo asociando un objeto geom\'{e}trico a un orden
de Eichler de nivel $\frak{N}$ en un \'{a}lgebra de cuaterniones $B/F$
sobre un cuerpo de n\'{u}meros totalmente real. Estos objetos se denominan
variedades de Shimura cuaterni\'{o}nicas. Estas variedades son compactas,
si y s\'{o}lo si $B$ es un \'{a}lgebra de divisi\'{o}n y esta propiedad se
ve reflejada en la definici\'{o}n de las formas automorfas (modulares)
correspondientes. Luego de introducir una representaci\'{o}n del grupo de
unidades $B^{\times}$, definimos las formas modulares asociadas a esta
representaci\'{o}n. Como en el caso de las formas de Hilbert, existe una
correspondencia entre formas modulares cuaterni\'{o}nicas y cierto espacio de
formas automorfas.

La correspondencia de Jacquet-Langlands, Teorema~\ref{thm:correspondenciajl},
permite reconstruir el m\'{o}dulo de Hecke $\spitzH{k}{\frak{N}}$ a partir de
espacios de formas cuaterni\'{o}nicas. \'{E}ste es el resultado fundamental en
el que se basan los m\'{e}todos para el c\'{a}lculo de las formas de Hilbert
descriptos en el Cap\'{\i}tulo~\ref{cap:metodos}, ya que reduce el problema
de determinar la acci\'{o}n de los operadores $T_{\frak{p}}$ al c\'{a}lculo de
los mismos en el espacio de formas modulares cuaterni\'{o}nicas. La naturaleza
de la correspondencia impone ciertas restricciones a la aplicabilidad de estos
m\'{e}todos.
% En lo que resta de la secci\'{o}n, se describir\'{a} la acci\'{o}n
% de Hecke en los espacios de formas cuaterni\'{o}nicas.


%%%texto
\section{Variedades de Shimura cuaterni\'{o}nicas}%
	\label{sec:cuaternionicasvariedadesdeshimura}
Sea $F$ un cuerpo de n\'{u}meros totalmente real de grado $[F:\bb{Q}]=n$ y
sea $B/F$ un \'{a}lgebra de cuaterniones. Cada una de las completaciones
$F_{v}$ con $v\in\lugares[\infty]{F}$ se identifica con $\bb{R}$ y $B_{v}$
con un \'{a}lgebra de cuaterniones real. Entonces, o bien
\begin{math}
	B_{v}\simeq\MM_{2\times 2}(\bb{R})
\end{math}
es un \'{a}lgebra de matrices, o bien $B_{v}\simeq\bb{H}$; en el primer caso,
decimos que el lugar \emph{$v$ es no ramificado}, o que $B$ no ramifica en $v$,
y, en el segundo, decimos que \emph{$v$ es ramificado}, o que $B$ ramifica en
$v$. Sean $\lista{v}{n}$ los lugares arquimedianos de $F$ y supongamos que
est\'{a}n ordenados de manera que, entre ellos, los lugares en donde $B$
ramifica son $\lista[r+1]{v}{n}$. Sea $\cal{O}$ un orden de $B$ y sea
$\ideles{B}$ el grupo de id\`{e}les de $B$. Entonces
\begin{math}
	\ideles{B}=\Idinf{B}\times\Idfin{B}
\end{math}~,
donde
\begin{equation}
	\label{eq:cuaternionesinfinitounidades}
	\Idinf{B} \,=\,\GL_{2}(\bb{R})^{r}\times (\bb{H}^{\times})^{n-r}
\end{equation}
%
y $\Idfin{B}$ es el producto restringido de $B_{v}^{\times}$ respecto de
los subgrupos compactos y abiertos $\cal{O}_{v}^{\times}$, donde $v$
recorre el conjunto de lugares finitos de $F$. Notemos que, si
en vez de $\cal{O}$, eligi\'{e}semos otro orden $\cal{O}'$ de $B$ en principio
distinto, por la Proposici\'{o}n \ref{propo:localgloballattices},
$\cal{O}_{v}=\cal{O}'_{v}$ para casi todo $v\in\lugares[f]{F}$, con lo cual
obtendr\'{\i}amos la misma \'{a}lgebra $\Idfin{B}$.

Supongamos que $B/F$ es un \'{a}lgebra indefinida distinta del \'{a}lgebra
de matrices, es decir, $r\geq 1$ y
\begin{math}
	B\not\simeq\MM_{2\times 2}(F)
\end{math}~.
La acci\'{o}n de $\GL_{2}(\bb{R})$ sobre $\hP^{\pm}$ se extiende a una
acci\'{o}n de $\Idinf{B}$ sobre el producto cartesiano $(\hP^{\pm})^{r}$: si
\begin{math}
	g_{\infty}=(\lista[v_{1}]{g}{v_{n}})\in\Idinf{B}
\end{math}
y $z=(\lista{z}{r})\in(\hP^{\pm})^{r}$ definimos
\begin{equation}
	\label{eq:acciondeidinf}
	g_{\infty}\cdot z \,=\, (g_{v_{1}}z_{1},\,\dots,\,g_{v_{r}}z_{r})
	\text{ .}
\end{equation}
%
Sea
\begin{math}
	C^{B}_{\mathbf{i}}\subset\Idinf{B}
\end{math}
el estabilizador del punto
\begin{math}
	\mathbf{i}=(\sqrt{-1},\,\dots,\,\sqrt{-1})\in(\hP^{\pm})^{r}
\end{math}~.
Como la acci\'{o}n \eqref{eq:acciondeidinf} es transitiva, podemos identificar
$(\hP^{\pm})^{r} =\Idinf{B}/C^{B}_{\mathbf{i}}$ v\'{\i}a
\begin{align*}
	& g_{\infty}\in\Idinf{B}\,\mapsto\, g_{\infty}\cdot\mathbf{i}
	\text{ .}
\end{align*}
%
El subgrupo estabilizador del punto $\mathbf{i}$ viene dado por
\begin{align*}
	C^{B}_{\mathbf{i}} & \,=\,\centre(\bb{R})\,
		\big( \SO{2}^{r}\times
		(\bb{H}^{1})^{n-r} \big)
	\text{ ,}
\end{align*}
%
donde
\begin{math}
	\centre(\bb{R})=\bb{R}^{\times}\times\,\cdots\,\times\bb{R}^{\times}
\end{math} --%
un factor por cada lugar arriba de $\infty$-- es el centro de $\Idinf{B}$,
\begin{math}
	\bb{H}^{1}=\{x\in\bb{H}:\conj{x}x=x\conj{x}=1\}
\end{math}~es el grupo de unidades de norma $1$ en el \'{a}lgebra de Hamilton
y $\SO{2}$ es el grupo ortogonal especial.

% Sea $K\subset\Idfin{B}$ un subgrupo compacto abierto de los id\`{e}les finitos.
% Como con ${\GL_{2}}_{/\bb{Q}}$, queremos estudiar un cociente de la forma
% \begin{align*}
	% \centre(\adeles{F})B^{\times}\backslash\ideles{B}/C^{B}_{\mathbf{i}}K
	% \text{ ,}
% \end{align*}
% %
% donde ahora $\ideles{B}$ juega el rol de $\GL_{2}(\adeles{\bb{Q}})$ (los
% puntos ad\'{e}licos del centro, $\centre(\adeles{F})$, se identifican con
% el grupo $\ideles{F}$ de ideles de $F$, siendo $B$ un \'{a}lgebra central).
% 
Sea $\cal{O}\subset B$ un orden de Eichler. Entonces
% Por aproximaci\'{o}n fuerte (Teorema \ref{thm:aproxfuerte}), [[NO]]
\begin{equation}
	\label{eq:descomposicionidelescuaterniones}
	\ideles{B} \,=\,\Idinf{B}\,\times\,\Idfin{B}
		\,=\, \bigsqcup_{g}\,
		B^{\times}g\Idinf{B}\Idfin{\cal{O}}
	\text{ ,}
\end{equation}
%
donde $g=(g_{v})_{v}\in\Idfin{B}$ recorre un sistema de representantes en
correspondencia con el conjunto de clases $\lClass{\cal{O}}$
--que es finito, seg\'{u}n el Corolario \ref{thm:numerodeclasesfinito}.
% Ese corolario s\'{\i} depende de aproximaci\'{o}n fuerte
%
% Pasando al cociente
% \begin{align*}
	% B^{\times}\backslash\Idinf{B}\times\Idfin{B}/
	% C^{B}_{\mathbf{i}}\Idfin{\cal{O}}
	% \,\simeq\, & \bigsqcup_{g}\,\Gamma_{g}\backslash (\hP^{\pm})^{r}
	% \text{ ,}
% \end{align*}
% %
% donde $\Gamma_{g}=g(\Idinf{B}\Idfin{\cal{O}})g^{-1}\cap B^{\times}$ es un
% subgrupo discreto de $\Idinf{B}$ actuando en $(\hP^{\pm})^{r}$. Dado que
% $B^{\times}/B_{+}^{\times}\simeq\{\pm\}^{r}$,
Sea $B_{\infty,+}^{\times}\subset\Idinf{B}$ el subgrupo de elementos cuyas
coordenadas tienen norma reducida positiva, es decir,
\begin{equation}
	\label{eq:cuaternionesinfinitounidadestotalmentepositivas}
	B_{\infty,+}^{\times} \,=\,
		\GLtp_{2}(\bb{R})^{r}\times (\bb{H}^{\times})^{n-r}
		\subset\Idinf{B}
	\text{ .}
\end{equation}
%
Por \eqref{eq:cuaternionesinfinitounidades},
\eqref{eq:cuaternionesinfinitounidadestotalmentepositivas} y la igualdad
\begin{math}
	\nrd(B^{\times})=F_{(+)}^{\times}
\end{math}
(ver el Teorema \ref{thm:eichlernorma}), pasando al cociente en
\eqref{eq:descomposicionidelescuaterniones}, se deduce, reemplazando
$\Idinf{B}$ por $B_{\infty,+}^{\times}$ (y $\hP^{\pm}$ por $\hP$), que
\begin{align*}
	B^{\times}\backslash
		\Idinf{B}\times\Idfin{B}/C^{B}_{\mathbf{i}}\Idfin{\cal{O}}
		& \,\simeq\,B_{+}^{\times}\backslash
		(\hP^{r}\times (\Idfin{B}/\Idfin{\cal{O}})) \\
	& \,=\,\bigsqcup_{g}\,\Gamma_{g}\backslash g\hP^{r}
	\text{ ,}
\end{align*}
%
donde ahora $g$ recorre un sistema de representantes en correspondencia
con $\pClass{F}$ el grupo de clases estrictas de $F$ y
\begin{math}
	\Gamma_{g}=g(\Idinf{B}\Idfin{\cal{O}})g^{-1}\cap B_{+}^{\times}
\end{math}
es un subgrupo discreto de $B_{\infty,+}^{\times}$ actuando en $\hP^{r}$.
Este cociente es la \emph{variedad de Shimura cuaterni\'{o}nica %
(de nivel $\frak{N}$)}\index{variedad de Shimura}
asociada a $B$ y al orden de Eichler $\cal{O}$. Es una variedad compleja
de dimensi\'{o}n $r$ y, salvo que $\#\pClass{F}=1$, no es conexa.
La denotaremos $\shimura[B]{\frak{N}}$. Si el n\'{u}mero de lugares del
infinito no ramificados es $r=1$, entonces se obtiene una curva. En general, si
$B$ es de divisi\'{o}n, $\shimura[B]{\frak{N}}$ es una variedad compacta.

\begin{obsGruposYOrdenesAsociadosAUnaClaseEstricta}%
	\label{obs:gruposyordenesasociadosaunaclaseestricta}
	Dado un ideal de $F$, $\frak{a}$, se puede elegir un elemento
	$\hhat{a}\in\Adfin{\oka{F}}$ tal que
	\begin{equation}
		\label{eq:ideleintegroasociado}
		\frak{a} \,=\,\hhat{a}\Adfin{\oka{F}}\,\cap\,F
		\text{ .}
	\end{equation}
	%
	Este id\`{e}le est\'{a} determinado salvo una unidad en
	$\Idfin{\oka{F}}$. Luego, existe $\hhat{\alpha}\in\Idfin{B}$ tal que
	$\nrd(\hhat{\alpha})=\hhat{a}$, pues la norma reducida es sobreyectiva
	en las componentes no arquimedeanas. A $\hhat{\alpha}$ le asociamos,
	una familia de ret\'{\i}culos locales
	$\Adfin{\cal{O}}_{\hhat{\alpha}}$, un ret\'{\i}culo global
	$\cal{O}_{\hhat{\alpha}}$ y un grupo $\Gamma_{\hhat{\alpha}}$ de la
	siguiente manera:
	\begin{equation}
		\label{eq:ordenasociadocuaterniones}
		\Adfin{\cal{O}}_{\hhat{\alpha}} \,=\,
			\hhat{\alpha}\Adfin{\cal{O}}\hhat{\alpha}^{-1}
			\text{ ,}\quad
		\cal{O}_{\hhat{\alpha}}\,=\,\Adfin{\cal{O}}_{\hhat{\alpha}}
			\,\cap\, B
			\quad\text{y}\quad
		\Gamma_{\hhat{\alpha}}\,=\,
			\cal{O}_{\hhat{\alpha},+}^{\times}\,=\,
			\cal{O}_{\hhat{\alpha}}^{\times}\cap B_{+}^{\times}
		\text{ .}
	\end{equation}
	%
	El ret\'{\i}culo $\cal{O}_{\hhat{\alpha}}$ es un orden de Eichler
	de nivel igual al nivel de $\cal{O}$; ambos \'{o}rdenes son
	\emph{localmente} conjugados, pero no necesariamente conjugados por un
	elemento de $B^{\times}$. El grupo $\cal{O}_{\hhat{\alpha},+}^{\times}$
	es el grupo de unidades ``totalmente positivas'' del orden
	$\cal{O}_{\hhat{\alpha}}$. Cambiando $\hhat{\alpha}$ por otro elemento
	perteneciente a la misma clase en
	\begin{math}
		B_{+}^{\times}\backslash\Idfin{B}/\Idfin{\cal{O}}
	\end{math}~,
	se obtiene un orden de $B$ conjugado a $\cal{O}_{\hhat{\alpha}}$ por un
	elemento de $B_{+}^{\times}$. Por esta raz\'{o}n, utilizaremos,
	principalmente, un sub\'{\i}ndice $\frak{a}$ en lugar de
	$\hhat{\alpha}$.
\end{obsGruposYOrdenesAsociadosAUnaClaseEstricta}

La elecci\'{o}n de los representantes $g$ en la descomposici\'{o}n de
$\shimura[B]{\frak{N}}$ se puede hacer expl\'{\i}cita en el siguiente sentido.
En primer lugar, se determina un sistema de representantes de $\pClass{F}$,
$\{\frak{a}\}_{[\frak{a}]\in\pClass{F}}$. Quedan determinados conjuntos
\begin{math}
	\{\hhat{a}\}_{\frak{a}}\subset\Adfin{\oka{F}}
\end{math}
y
\begin{math}
	\{\hhat{\alpha}\}_{\frak{a}}\subset\Idfin{B}
\end{math}~,
los \'{o}rdenes $\cal{O}_{\frak{a}}$ y los grupos $\Gamma_{\frak{a}}$
seg\'{u}n la Observaci\'{o}n
\ref{obs:gruposyordenesasociadosaunaclaseestricta}. Realizadas estas
elecciones, la variedad de Shimura $\shimura[B]{\frak{N}}$ se identifica con
una uni\'{o}n disjunta de variedades conexas:
\begin{equation}
	\label{eq:descomposicionvariedaddeshimuraindefinida}
\begin{aligned}
	\shimura[B]{\frak{N}} \,=\,
		\bigsqcup_{[\frak{a}]\in\pClass{F}}\,
			B_{+}^{\times}\backslash\big(\hP^{r}\times
				\hhat{\alpha}\Idfin{\cal{O}}\big)
		& \,\simeq\,
		\bigsqcup_{[\frak{a}]\in\pClass{F}}\,
		\Gamma_{\frak{a}}\backslash\hP^{r} \\
	B_{+}^{\times}(z,\hhat{\alpha}\Idfin{\cal{O}}) & \,\mapsto\,
		\Gamma_{\frak{a}}z
	\text{ .}
\end{aligned}
\end{equation}
%
Cada una de estas componentes es compacta, lo que implica que
$\shimura[B]{\frak{N}}$ es compacta.

\begin{obsVariedadDeShimuraEsCompacta}\label{obs:variedaddeshimuraescompacta}
	De hecho, si
	\begin{math}
		\cal{O}_{\hhat{\alpha}}^{1}=
			\{x\in\cal{O}_{\hhat{\alpha}}\,:\,\nrd(x)=1\}
	\end{math}~,
	el cociente
	$\cal{O}_{\hhat{\alpha}}^{1}\backslash\hP^{r}$ es compacto.
	Definimos
	\begin{align*}
		\adeles{B}^{(1)} \,=\,\big\{x\in\ideles{B}\,:\,
			|\nrd(x)|_{\adeles{F}}=1\big\}\text{ ,}\quad
		\adeles{B}^{1} \,=\,\big\{x\in\ideles{B}\,:\,\nrd(x)=1\big\}
	\end{align*}
	%
	y, dado un subgrupo $H\subset\ideles{B}$, subgrupos
	$H^{(1)}=H\cap\adeles{B}^{(1)}$ y $H^{1}=H\cap\adeles{B}^{1}$.
	Recordemos, tambi\'{e}n, el siguiente hecho: dados un grupo
	topol\'{o}gico $G$, un subgrupo abierto $H$ y un subconjunto arbitrario
	$A\subset G$, el producto $A\cdot H$ y su complemento son abiertos,
	puedi\'{e}ndose escribir como uniones de aquellas coclases $x\cdot H$
	que los intersecan:
	\begin{math}
		A\cdot H=\bigcup_{x\in A\cdot H}\,x\cdot H
	\end{math}~y
	\begin{math}
		G\setmin\big(A\cdot H\big)=
			\bigcup_{x\not\in A\cdot H}\,x\cdot H
	\end{math}~. Ahora, dado que $B$ es un \'{a}lgebra indefinida,
	$\lugares[\infty]{F}$ contiene, al menos, un lugar en donde el
	\'{a}lgebra no ramifica. En particular, por aproximaci\'{o}n fuerte
	(Teorema~\ref{thm:aproxfuerte}), el producto $B^{1}\cdot\Adinf{B}^{1}$
	es denso en $\adeles{B}^{1}$ y, puesto que
	\begin{math}
		\Adinf{B}^{1}\,\Adfin{\cal{O}}^{1}_{\hhat{\alpha}}=
			\adeles{B}^{1}\cap\big(B^{\times}_{\infty,+}\,
				\Idfin{\cal{O}}_{\hhat{\alpha}}\big)
	\end{math}
	es un subgrupo abierto de $\adeles{B}^{1}$, el complemento
	\begin{math}
		\adeles{B}^{1}\setmin B^{1}\cdot\big(\Adinf{B}^{1}\,
			\Adfin{\cal{O}}^{1}_{\hhat{\alpha}}\big)
	\end{math}
	es abierto y
	\begin{align*}
		\adeles{B}^{1} & \,=\, B^{1}\Adinf{B}^{1}
			\Adfin{\cal{O}}^{1}_{\hhat{\alpha}}
		\text{ .}
	\end{align*}
	%
	En consecuencia, la aplicaci\'{o}n $\adeles{B}^{1}\rightarrow\hP^{r}$
	dada por $g=(g_{\infty},\hhat{g})\mapsto g_{\infty}\cdot\mathbf{i}$
	induce una correspondencia
	\begin{align*}
		B^{1}\backslash\adeles{B}^{1}/K_{\infty}
			\Adfin{\cal{O}}^{1}_{\hhat{\alpha}} & \,\simeq\,
			\cal{O}^{1}_{\hhat{\alpha}}\backslash\hP^{r}
		\text{ ,}
	\end{align*}
	%
	donde $K_{\infty}=\SO{2}^{r}\times(\bb{H}^{1})^{n-r}$ es el subgrupo
	compacto maximal en $\Adinf{B}^{1}$. Ahora, como $B$ es de
	divisi\'{o}n, el cociente $B^{\times}\backslash\adeles{B}^{(1)}$ es
	compacto.%
	\footnote{
		\cite[Ch.~III, \S~1]{Vigneras}
		%(teorema de Fujisaki)
	}
	El subgrupo
	\begin{math}
		B^{(1)}_{\infty,+}\,\Idfin{\cal{O}}_{\hhat{\alpha}}=
			\adeles{B}^{(1)}\cap\big(B^{\times}_{\infty,+}\,
				\Idfin{\cal{O}}_{\hhat{\alpha}}\big)
	\end{math}
	es abierto, lo que implica que
	\begin{math}
		W=\adeles{B}^{(1)}\setmin B^{\times}\cdot\big(
			B^{(1)}_{\infty,+}\,
				\Idfin{\cal{O}}_{\hhat{\alpha}}\big)
	\end{math}
	sea abierto. En particular, $B^{\times}\backslash W$ es abierto en el
	cociente y su complemento,
	\begin{math}
		B^{\times}\backslash B^{\times}\cdot\big(B^{(1)}_{\infty,+}\,
			\Idfin{\cal{O}}_{\hhat{\alpha}}\big)
	\end{math}
	es compacto. Pero
	\begin{math}
		B_{\infty,+}^{(1)}=
			\big(\centre(\bb{R})\cap\adeles{B}^{(1)}\big)\cdot
				B_{\infty}^{1}
	\end{math}
	y, a fin de cuentas,
	\begin{align*}
		\big(\centre(\bb{R})\cap\adeles{B}^{(1)}\big)\,B^{\times}
			\backslash B^{\times}\cdot\big(B_{\infty,+}^{(1)}\,
				\Idfin{\cal{O}}_{\hhat{\alpha}}\big)/
				K_{\infty}\Idfin{\cal{O}}_{\hhat{\alpha}}
			& \,=\,
			B^{1}\backslash B^{1}\cdot\big(\Adinf{B}^{1}\,
				\Adfin{\cal{O}}^{1}_{\hhat{\alpha}}\big)/
				K_{\infty}\Adfin{\cal{O}}^{1}_{\hhat{\alpha}}
	\end{align*}
	%
	es compacto.
\end{obsVariedadDeShimuraEsCompacta}

Cuando $B/F$ es un \'{a}lgebra definida, $r=0$, no tenemos una acci\'{o}n sobre
$\hP$ y el cociente ad\'{e}lico
\begin{math}
	B^{\times}\backslash\Idinf{B}\times\Idfin{B}/\Idinf{B}\Idfin{\cal{O}}=
	B^{\times}\backslash\Idfin{B}/\Idfin{\cal{O}}
\end{math}
es simplemente un conjunto finito de puntos en correspondencia con el conjunto
$\lClass{\cal{O}}$ de clases de ideales cuyo orden a derecha es $\cal{O}$
(Teorema \ref{thm:numerodeclasesfinito}).
% En este caso, la variedad de Shimura asociada es una variedad compacta de
% dimensi\'{o}n $0$, que seguimos denotando $\shimura[B]{\frak{N}}$.

\section{Un $B^{\times}$-m\'{o}dulo}\label{sec:cuaternionicasunbxmodulo}

Las formas modulares cuaterni\'{o}nicas que vamos a definir est\'{a}n
dadas como funciones en $\ideles{B}$. Para poder tratarlas de manera
homog\'{e}nea, lo primero ser\'{a} definir el espacio de llegada de dichas
funciones.

Dado un entero $w\in\bb{Z}$, consideramos el $\bb{C}$-espacio vectorial de
polinomios homog\'{e}neos de grado $w$ en variables $X,Y$, con una acci\'{o}n a
derecha de $\GL_{2}(\bb{C})$ dada por
% a izquierda en los argumentos, a derecha en las funciones
\begin{equation}
	\label{eq:accionenpolinomioshomogeneos}
\begin{aligned}
	\begin{bmatrix} X \\ Y \end{bmatrix}\,\cdot\,\gamma \,:=\, &
		(\gamma^{\iota})^{t}\begin{bmatrix} X \\ Y \end{bmatrix}
		\,=\, \begin{bmatrix} d & -c \\ -b & a \end{bmatrix}
			\begin{bmatrix} X \\ Y \end{bmatrix} \\
	\,=\, & \begin{bmatrix} d X - c Y \\ -b X + a Y \end{bmatrix}
	\text{ .}
\end{aligned}
\end{equation}
%
En la expresi\'{o}n anterior, $\gamma$ es la matriz
\begin{math}
	\begin{bmatrix} a & b \\ c & d \end{bmatrix}
\end{math}
con coeficientes complejos, $\null^{t}$ denota transposici\'{o}n e
$\null^{\iota}$ denota la adjunta.
% en este caso, para no confundir el conjugado de un elemento del \'{a}lgebra de
% matrices (ejemplo \ref{ejemplo:matrices}) con conjugaci\'{o}n compleja.
Fijando un segundo entero $m\in\bb{Z}$, consideramos una acci\'{o}n modificada
a partir de la anterior: si $p$ es un polinomio homog\'{e}neo de grado $w$ y
$\gamma\in\GL_{2}(\bb{C})$, definimos
\begin{equation}
	\label{eq:accionenpolinomioscontwist}
	p(X,Y)\,\cdot\,\gamma \,=\,\det(\gamma)^{m}\,p(dX-cY,-bX+aY)
	\text{ .}
\end{equation}
%
Denotaremos el espacio de polinomios homog\'{e}neos de grado $w$ con esta
acci\'{o}n de $\GL_{2}(\bb{C})$ por $\wmpoli{w}{m}(\bb{C})$.
% Vale la pena notar que este mismo m\'{o}dulo (pero con una acci\'{o}n
% \emph{a izquierda}) es el descripto en \cite{EichlerBasisProblem}.

Sea $\peso{k}=(\lista{k}{n})$ un peso (ver
\S~\ref{sec:dehilbertformasparacongruencia}) y sean
\begin{align*}
	k_{0} \,=\,\mathrm{max}_{i}\,k_{i} & \text{ ,}\quad
		m_{i} \,=\,\frac{k_{0} - k_{i}}{2}\quad\text{y}\quad
		w_{i} \,=\,k_{i}-2
	\text{ .}
\end{align*}
%
Si
\begin{math}
	B\otimes_{\bb{Q}}\bb{R}=\MM_{2\times 2}(\bb{R})^{r}\times\bb{H}^{n-r}
\end{math}~,
se considera el $\GL_{2}(\bb{C})^{n-r}$-m\'{o}dulo
\begin{equation}
	\label{eq:unbxmodulo}
	W_{\peso{k}}(\bb{C}) \,:=\,
		\wmpoli{w_{r+1}}{m_{r+1}}(\bb{C})\,\otimes\,\cdots\,\otimes\,
		\wmpoli{w_{n}}{m_{n}}(\bb{C})
	\text{ .}
\end{equation}
%
Si $r=n$, se define $W_{\peso{k}}(\bb{C}):=\bb{C}$. Tambi\'{e}n se cumple
$W_{\peso{k}}(\bb{C})\simeq\bb{C}$, si $\peso{k}=(2,\,\dots,\,2)$.

Sean $\lista{v}{n}$ los lugares arquimedianos de $F$. Para cada $v_{j}$ podemos
escoger una inclusi\'{o}n
\begin{math}
	B_{v_{j}}\hookrightarrow\MM_{2\times 2}(\bb{C})
\end{math}
de manera que la imagen de un elemento
\begin{math}
	t\in\centre(B_{v_{j}}^{\times})=\bb{R}^{\times}
\end{math}
sea la matriz escalar
\begin{math}
	\begin{bmatrix} t & \\ & t \end{bmatrix}
\end{math}
y la denotamos $\gamma\mapsto\gamma_{j}$. El morfismo
\begin{math}
	B^{\times}\hookrightarrow\GL_{2}(\bb{C})^{n-r}
\end{math}
dado por
\begin{align*}
	\gamma\,\mapsto\, & (\gamma_{r+1},\,\dots,\,\gamma_{n})
\end{align*}
%
determina una estructura de $B^{\times}$-m\'{o}dulo en $W_{\peso{k}}(\bb{C})$.
Si $r=n$, por ejemplo si $B\simeq\MM_{2\times 2}(F)$,
$W_{\peso{k}}(\bb{C})=\bb{C}$ con la acci\'{o}n trivial de $B^{\times}$.
% Vale la pena observar que este m\'{o}dulo involucra
% \'{u}nicamente los lugares arquimedianos ramificados del \'{a}lgebra.

\section{Formas modulares cuaterni\'{o}nicas}%
	\label{sec:cuaternionicasformasmodulares}
En lo que resta de esta secci\'{o}n, asumimos, salvo que se indique lo
contrario, que $B/F$ es un \'{a}lgebra de cuaterniones \emph{de %
divisi\'{o}n} sobre el cuerpo de n\'{u}meros totalmente real $F$ de grado $n$,
que $r\geq 0$ indica la cantidad de lugares arquimedianos no ramificados y que
los lugares arquimedianos $\lista{v}{n}$ est\'{a}n ordenados de forma tal que
$v_{i}$ sea no ramificado, si y s\'{o}lo si $i\leq r$. Fijamos un peso
$\peso{k}=(\lista{k}{n})\in\bb{Z}^{n}$ y escribimos $W_{\peso{k}}$ en lugar de
$W_{\peso{k}}(\bb{C})$. Fijamos tambi\'{e}n inclusiones
\begin{math}
	(\gamma\mapsto\gamma_{j}):\,
		B_{v_{j}}^{\times}\hookrightarrow\GL_{2}(\bb{C})
\end{math}~para $j\geq r+1$, de manera que quede inducida una acci\'{o}n de
$B^{\times}$ en el $\GL_{2}(\bb{C})^{n-r}$-m\'{o}dulo $W_{\peso{k}}$, que
denotamos $x\cdot\gamma=x^{\gamma}$, si $\gamma\in B^{\times}$ y
$x\in W_{\peso{k}}$.

\subsection{Caso indefinido}%
	% \label{subsec:formascuaternionicascasoindefinidobis}
Supongamos que $B$ es un \'{a}lgebra indefinida (en este caso, $r\geq 1$) y
sea $\cal{O}\subset B$ un orden de Eichler. En analog\'{\i}a con las formas
modulares el\'{\i}pticas y teniendo en cuenta la definici\'{o}n
\eqref{eq:descomposicionvariedaddeshimuraindefinida} de la variedad de Shimura
asociada a $B$ y al orden $\cal{O}$, las formas modulares que definiremos a
continuaci\'{o}n son funciones de la forma
\begin{math}
	f:\,(\hP^{\pm})^{r}\times(\Idfin{B}/\Idfin{\cal{O}})\rightarrow W
\end{math}
que verifican condiciones de regularidad e invarianza con respecto a una
acci\'{o}n del grupo $B^{\times}$.

Para $i\in [\![1,r]\!]$, introducimos un \emph{factor de automorf\'{\i}a},
\index{factor de automorfia@factor de automorf\'{\i}a}
\begin{align*}
	J_{i} & \,:\,B_{v_{i}}^{\times}\times\hP^{\pm}\,\rightarrow\,\bb{C}
	\text{ ,}
\end{align*}
%
por la expresi\'{o}n
\begin{equation}
	\label{eq:factorddeautomorfia}
	J_{i}(\gamma,z) \,=\,\frac{j(\gamma,z)^{k_{i}}}{%
			\det(\gamma)^{m_{i}+k_{i}-1}}
	\text{ ,}
\end{equation}
%
donde
\begin{math}
	j\big(\left[\begin{smallmatrix} a & b \\
		c & d \end{smallmatrix}\right],z\big)=cz+d
\end{math}~.
Dado que $j(\gamma\gamma',z)=j(\gamma,\gamma'z)\cdot j(\gamma',z)$ para
$\gamma,\gamma'\in\GL_{2}(\bb{R})$ y $z\in\hP^{\pm}$, las funciones $J_{i}$
cumple con la propiedad an\'{a}loga
\begin{equation}
	\label{eq:factordeautomorfiacociclo}
	J_{i}(\gamma\gamma',z) \,=\,
		J_{i}(\gamma,\gamma'z)\cdot J_{i}(\gamma',z)
	\text{ .}
\end{equation}
%
Definimos un \emph{operador de peso $\peso{k}$} de la siguiente manera: dados
\index{operador de peso@operador de peso $\peso{k}$}
un elemento $\gamma\in B^{\times}$ y una funci\'{o}n
\begin{math}
	f:\,(\hP^{\pm})^{r}\times(\Idfin{B}/\Idfin{\cal{O}})\rightarrow
		W_{\peso{k}}
\end{math}~,
sea $f\operadormatrices{\peso{k}}{\gamma}$ la funci\'{o}n
\begin{equation}
	\label{eq:indefinidaoperadordepesok}
	\big(f\operadormatrices{\peso{k}}{\gamma}\big)
		(z,\hhat{\alpha}\Idfin{\cal{O}}) \,=\,
		\bigg(\prod_{i=1}^{r}\,J_{i}(\gamma_{i},z_{i})^{-1}\bigg)\,
		f(\gamma z,\gamma\hhat{\alpha}\Idfin{\cal{O}})^{\gamma}
	\text{ .}
\end{equation}
%

\begin{defFormaCuaternionicaCasoIndefinido}%
	\label{def:formacuaternionicacasoindefinido}
	Dada un \'{a}lgebra de cuaterniones indefinida $B$ de divisi\'{o}n,
	sobre un cuerpo de n\'{u}meros totalmente real $F$ y dado un orden
	de Eichler $\cal{O}\subset B$ de nivel $\frak{N}$, una
	\emph{forma modular cuaterni\'{o}nica}
	\index{forma modular!cuaternionica@cuaterni\'{o}nica}
	de peso $\peso{k}$ y nivel $\Idfin{\cal{O}}$ (o, tambi\'{e}n, de nivel
	$\frak{N}$) para $B$ es una funci\'{o}n
	\begin{align*}
		f & \,:\, (\hP^{\pm})^{r}\times (\Idfin{B}/
			\Idfin{\cal{O}})\rightarrow W_{\peso{k}}(\bb{C})
	\end{align*}
	%
	holomorfa en la primera variable y localmente constante en la segunda
	tal que $f\operadormatrices{\peso{k}}{\gamma}=f$ para toda
	$\gamma\in B^{\times}$. Estas funciones constituyen un
	$\bb{C}$-espacio vectorial que denotamos $\modularH[B]{k}{\frak{N}}$.
\end{defFormaCuaternionicaCasoIndefinido}

% \begin{obsFormaCuaternionicaCasoIndefinido}
	% \label{obs:formacuaternionicacasoindefinido}
	% En general, si $f\operadormatrices{\peso{k}}{\gamma}=f$ para toda
	% $\gamma\in B^{\times}$, entonces $f$ queda determinada por su
	% restricci\'{o}n a $\hP^{r}$, en la primera variable (Teorema
	% \ref{thm:eichlernorma}).
% \end{obsFormaCuaternionicaCasoIndefinido}

La descomoposici\'{o}n \eqref{eq:descomposicionvariedaddeshimuraindefinida} de
$\shimura[B]{\frak{N}}$ se ve reflejada en una descomposici\'{o}n an\'{a}loga
del espacio $\modularH[B]{k}{\frak{N}}$. En primer lugar, si $\frak{a}$ es un
representante de las clases estrictas de $F$ y $\Gamma_{\frak{a}}$ es el grupo
de unidades asociado seg\'{u}n \eqref{eq:ordenasociadocuaterniones},
entonces podemos definir una acci\'{o}n de $\Gamma_{\frak{a}}$ en funciones
$f:\,\hP^{r}\rightarrow W_{\peso{k}}$ de la siguiente manera: si
$\gamma\in\Gamma_{\frak{a}}$, sea $f\operadormatrices{\peso{k}}{\gamma}$ la
funci\'{o}n dada por
\begin{align*}
	\big(f\operadormatrices{\peso{k}}{\gamma}\big)(z) & \,=\,
		\bigg(\prod_{i=1}^{r}\,J_{i}(\gamma_{i},z_{i})^{-1}\bigg)\,
			f(\gamma z)^{\gamma}
	\text{ .}
\end{align*}
%
Luego definimos los espacios de funciones holomorfas invariantes por esta
acci\'{o}n de $\Gamma_{\frak{a}}$:
\begin{align*}
	\modularH[B]{k}{\frak{N},\frak{a}} & \,=\,
		\Big\{ f:\,\hP^{r}\rightarrow W_{\peso{k}}(\bb{C}) \,:\,
			f\text{ es holomorfa y }
			f\operadormatrices{\peso{k}}{\gamma}=f\,
			\forall\gamma\in\Gamma_{\frak{a}}
			\Big\}
	\text{ .}
\end{align*}
%
Entonces la aplicaci\'{o}n $f\mapsto (f_{\frak{a}})_{\frak{a}}$, donde
$f_{\frak{a}}(z)=f(z,\hhat{\alpha}\Idfin{\cal{O}})$, determina una
transformaci\'{o}n lineal
\begin{equation}
	\label{eq:descomposicionmodularesindefinida}
	\modularH[B]{k}{\frak{N}}\,\rightarrow\,
		\bigoplus_{\frak{a}}\,\modularH[B]{k}{\frak{N},\frak{a}}
	\text{ ,}
\end{equation}
%
pues, si $\gamma\in\Gamma_{\frak{a}}$,
\begin{align*}
	\big(f_{\frak{a}}\operadormatrices{\peso{k}}{\gamma}\big)(z)
	& \,=\,\bigg(\prod_{i=1}^{r}\,J_{i}(\gamma_{i},z_{i})^{-1}
		\bigg)\,f(\gamma z,\hhat{\alpha}\Idfin{\cal{O}})^{\gamma}
		\,=\, \big(f\operadormatrices{\peso{k}}{\gamma}\big)
			(z,\gamma^{-1}\hhat{\alpha}\Idfin{\cal{O}}) \\
	& \,=\, f(z,\gamma^{-1}\hhat{\alpha}\Idfin{\cal{O}})
		\,=\, f(z,\hhat{\alpha}\Idfin{\cal{O}})
		\,=\, f_{\frak{a}}(z)
	\text{ .}
\end{align*}
%
Esta transformaci\'{o}n es un isomorfismo de $\bb{C}$-espacios vectoriales.
El argumento es an\'{a}logo al de la demostraci\'{o}n de la Proposici\'{o}n
\ref{thm:descomposiciondelespaciodeformasmodularesdehilbert}, con la salvedad
de que, en este caso, hay que tener en cuenta la acci\'{o}n de $B^{\times}$
en $W_{\peso{k}}$.
% y la Observaci\'{o}n \ref{obs:formacuaternionicacasoindefinido}.
El isomorfismo \eqref{eq:descomposicionmodularesindefinida} depende de la
elecci\'{o}n de los representantes $\frak{a}$ de las clases estrictas de $F$,
de los id\`{e}les $\hhat{a}\in\Adfin{\oka{F}}$ que cumplen
\eqref{eq:ideleintegroasociado} y de los elementos $\hhat{\alpha}$ tales que
$\nrd(\hhat{\alpha})=\hhat{a}$.

\subsection{Caso definido}%
	% \label{subsec:formascuaternionicascasodefinido}
Si $B$ es un \'{a}lgebra definida, no hay acci\'{o}n sobre el semiplano
complejo.

\begin{defFormaCuaternionicaCasoDefinido}%
	\label{def:formacuaternionicacasodefinido}
	Sea $B$ un \'{a}lgebra de cuaterniones definida, sobre un cuerpo de
	n\'{u}meros totalmente real $F$ y sea $\cal{O}\subset B$ un orden de
	Eichler de nivel $\frak{N}$. Una \emph{forma modular cuaterni\'{o}nica}
	%\index{forma modular cuaternionica@forma modular cuaterni\'{o}nica}
	de peso $\peso{k}$ y nivel $\Idfin{\cal{O}}$ (o, tambi\'{e}n, de nivel
	$\frak{N}$) para $B$ es una funci\'{o}n
	\begin{align*}
		f & \,:\,\Idfin{B}/\Idfin{\cal{O}}\,\rightarrow\,
			W_{\peso{k}}(\bb{C})
	\end{align*}
	%
	que satisface, para toda $\gamma\in B^{\times}$,
	\begin{equation}
		\label{eq:definidaoperadordepesok}
		\big(f\operadormatrices{\peso{k}}{\gamma}\big)
			(\hhat{\alpha}\Idfin{\cal{O}})
			\,:=\, f(\gamma\hhat{\alpha}\Idfin{\cal{O}})^{\gamma}
			\,=\, f(\hhat{\alpha}\Idfin{\cal{O}})
		\text{ .}
	\end{equation}
	%
	El espacio de formas modulares correspondientes lo denotamos
	$\modularH[B]{k}{\frak{N}}$.
\end{defFormaCuaternionicaCasoDefinido}

El grupo de unidades $B^{\times}$ act\'{u}a en el conjunto de ideales
$I$ de $B$ con $\Oder(I)=\cal{O}$ por multiplicaci\'{o}n a izquierda.
Una forma modular $f$ para $B$ es entonces una funci\'{o}n equivariante
respecto de esta acci\'{o}n. Si $I=\hhat{\alpha}\Idfin{\cal{O}}\cap B$,
el estabilizador de $I$ es el grupo de unidades de su orden a izquierda:
\begin{align*}
	\Gamma_{\hhat{\alpha}} & \,=\,
		(\hhat{\alpha}\Idfin{\cal{O}}\hhat{\alpha}^{-1})
			\,\cap\,B^{\times}
		\,=\, \left\lbrace b\in B^{\times} \,:\, bI=I \right\rbrace
	\text{ .}
\end{align*}
%
De esta manera, tomando un sistema de representantes de las clases en
$\lClass{\cal{O}}$, se obtiene un isomorfismo
\begin{equation}
	\label{eq:descomposicionmodularesdefinida}
	\modularH[B]{k}{\frak{N}}\,\xrightarrow{\sim}\,
		\bigoplus_{[I]\in\lClass{\cal{O}}}\,
			W_{\peso{k}}(\bb{C})^{\Gamma_{\hhat{\alpha}}} \\
\end{equation}
%
dado por
\begin{math}
	f \mapsto(f(\hhat{\alpha}\Idfin{\cal{O}}))_{[I]\in\lClass{\cal{O}}}
\end{math}~.
%
% \begin{align*}
	% \modularH[B]{k}{\frak{N}} \,=\, & \left\lbrace
	% f:\,\Idfin{B}/\Idfin{\cal{O}}\rightarrow W_{\peso{k}}(\bb{C})
	% \,:\,f\barra{\peso{k}}{\gamma}=f\,\forall\gamma\in B^{\times}
	% \right\rbrace \\
	% \left(f\barra{\peso{k}}{\gamma}\right)(\hhat{\alpha}\Idfin{\cal{O}})
	% \,=\, &
	% f(\gamma\hhat{\alpha}\Idfin{\cal{O}})^{\gamma}
% \end{align*}
% %
% Sea $\hhat{\alpha}\in\Idfin{B}$ y sea $I=\hhat{\alpha}\Idfin{\cal{O}}\cap B$
% el ideal de $B$ correspondiente. Evaluando, se obtiene un punto
% $x=f(\hhat{\alpha}\Idfin{\cal{O}})\in W_{\peso{k}}(\bb{C})$ que
% cumple con $x^{\gamma}=f(\gamma^{-1}\hhat{\alpha}\Idfin{\cal{O}})=x$
% para todo $\gamma\in\Gamma_{\hhat{\alpha}}=%
% \hhat{\alpha}\Idfin{\cal{O}}\hhat{\alpha}^{-1}\cap B$.
% Si $(x_{\hhat{\alpha}})_{\hhat{\alpha}}\in%
% \bigoplus_{[I]\in\lClass{\cal{O}}}\,%
% W_{\peso{k}}(\bb{C})^{\Gamma_{\hhat{\alpha}}}$, definimos
% $f(\hhat{\alpha}\Idfin{\cal{O}}):=x_{\hhat{\alpha}}$. Si
% $\hhat{\beta}\in\Idfin{B}$, existen $\hhat{\alpha}$ y $\rho\in B^{\times}$
% tales que $\rho\hhat{\beta}\Idfin{\cal{O}}=\hhat{\alpha}\Idfin{\cal{O}}$.
% Entonces $f(\hhat{\beta}\Idfin{\cal{O}})=%
% f(\rho\hhat{\beta}\Idfin{\cal{O}})^{\rho}=%
% f(\hhat{\alpha}\Idfin{\cal{O}})^{\rho}$ determina
% $f\in\modularH[B]{k}{\frak{N}}$. As\'{\i}, se obtiene el isomorfismo
% de arriba.


\section{Funciones en los ad\`{e}les}%
	\label{sec:cuaternionicasfuncionesenlosadeles}
A continuaci\'{o}n enunciamos la correspondencia entre formas modulares de
Hilbert y formas automorfas en el grupo $\GL_{2}(\adeles{F})$. Su
demostraci\'{o}n es an\'{a}loga a la de la Proposici\'{o}n~%
\ref{propo:introequivformaglq}.
% (ver \cite[\S~3]{GelbartAutomorphicOnAdeles}).
Para cada lugar arquimediano $v\in\lugares[\infty]{F}$, definimos
$K_{v}=\SO{2}$. El producto $K_{\infty}=\prod_{v\in\lugares[\infty]{F}}\,K_{v}$
es el subgrupo compacto y conexo maximal del estabilizador
del punto $\mathbf{i}=(\sqrt{-1},\,\dots,\,\sqrt{-1})\in(\hP^{\pm})^{n}$ por la
acci\'{o}n de $\GL_{2}(\bb{R})^{n}$.

\begin{propoEquivalenciaAutomorfasFormasCuspidales}
	\label{propo:equivalenciaautomorfasformascuspidales}
	La aplicaci\'{o}n $f\mapsto\phi_{f}$ que a una forma de Hilbert
	cuspidal le asigna la funci\'{o}n
	\begin{equation}
		\label{eq:funcionadelicacorrespondientedehilbert}
		\phi_{f}(g_{\infty},\hhat{\alpha}) \,=\,
		\bigg(\prod_{i=1}^{n}\,J_{i}(g_{i},\sqrt{-1})^{-1}\bigg)\,
		f(g_{\infty}\cdot\mathbf{i},\hhat{\alpha}\Idfin{\cal{O}})
	\end{equation}
	%
	determina una correspondencia entre el espacio $\spitzH{k}{\frak{N}}$
	y el espacio de funciones $\phi:\,\GL_{2}(\adeles{F})%
	\rightarrow\bb{C}$ que cumplen con:
	\begin{itemize}
		\item[(i)] $\phi(\gamma g)=\phi(g)$ para toda
			$\gamma\in\GL_{2}(F)$;
		\item[(ii)] $\phi(g\hhat{\beta})=\phi(g)$ si
			\begin{math}
				\hhat{\beta}\in\Idfin{\cal{O}}=
					\prod_{\frak{p}}\,
					\cal{O}_{\frak{p}}^{\times}
			\end{math}~;
		\item[(iii)]
			\begin{math}
				\phi(gh)=\prod_{i=1}^{n}\,
					e^{\sqrt{-1}k_{i}\theta_{i}}\,\phi(g)
			\end{math}
			si
			\begin{math}
				h=(\lista[\theta_{1}]{r}{\theta_{n}})
					\in K_{\infty}
			\end{math}~;
		\item[(iv)]
			\begin{math}
				\phi(\tau_{\infty}g)=\prod_{i=1}^{n}\,
				\tau_{i}^{k_{0}-2}\phi(g)
			\end{math}
			si
			\begin{math}
				\tau_{\infty}=\bigg(
				\left[\begin{matrix} \tau_{1} & \\
					& \tau_{1} \end{matrix}\right],
					\,\dots,\,
				\left[\begin{matrix} \tau_{n} & \\
					& \tau_{n}\end{matrix}\right]
				\bigg)\in\centre(\bb{R})
			\end{math}~;
			% no hace falta el valor absoluto, porque los k_{i}
			% son todos de igual paridad y no hay problema con las
			% potencias.
		% \item[v.b] $\phi(\hhat{\eta}g)=\prod_{i=1}^{n}\,%
			% t_{i}^{-(k_{0}-2)}%
			% \phi(g\hhat{a})$ si $\hhat{\eta}=t\hhat{a}\hhat{r}\in%
			% F_{+}^{\times}\hhat{a}\Idfin{\oka{F}}\subset%
			% \centre(\Idfin{F})$;
		\item[(v)] $\phi$ es de \emph{crecimiento moderado}: si
			\index{crecimiento moderado}
			$\Omega\subset\GL_{2}(\adeles{F})$ es un subconjunto
			compacto y $c>0$, existen constantes $C$ y $N$ tales
			que
			\begin{align*}
				\left|\phi\bigg(
				\begin{bmatrix} a & \\ & 1 \end{bmatrix}
					g\bigg)\right| & \,\leq\,
					C\left|a\right|^{N}
			\end{align*}
			%
			para toda $g\in\Omega$ y $a\in\ideles{F}$ con $|a|>c$;
			y
		\item[(vi)] como funci\'{o}n de $\GLtp_{2}(\bb{R})^{n}$,
			$\phi$ es $C^{\infty}$ y verifica la ecuaci\'{o}n
			diferencial
			\begin{align*}
				\Delta_{i}\phi\,=\, & -\frac{k_{i}}{2}
				\left(\frac{k_{i}}{2}\,-\,1\right)\phi
				\text{ ;}
			\end{align*}
		\item[(vii)] $\phi$ es \emph{cuspidal}, es decir,
			\index{forma automorfa!cuspidal}
			\begin{align*}
				\int_{F\backslash\adeles{F}}\,
				\phi\bigg(
				\begin{bmatrix} 1 & x \\ & 1 \end{bmatrix}
					g\bigg)\,dx & \,=\, 0
			\end{align*}
			%
			para casi todo $g$.
	\end{itemize}
\end{propoEquivalenciaAutomorfasFormasCuspidales}

% \begin{proof}
	% Los puntos \textit{(i)}, \textit{(ii)}, \textit{(iii)} y \textit{(v)}
	% son consecuencia de la invarianza de una forma $f$ respecto de la
	% acci\'{o}n de $\GL_{2}(F)$. Los puntos restantes se demuestran de
	% manera similar al caso $F=\bb{Q}$. Para ver \textit{(iv)}, sea
	% $f\in\spitzH{k}{\frak{N}}$ y sea $\phi_{f}$ la funci\'{o}n ad\'{e}lica
	% correspondiente. Para ver que satisface \emph{(iv)}, sea
	% $g_{\infty}=(\lista{g}{n})\in\GLtp_{2}(\bb{R})^{n}$ y sea
	% $\hhat{\alpha}\in\GL_{2}(\Adfin{F})$ arbitrario. Cada matriz $g_{i}$ se
	% puede escribir como
	% \begin{equation}
		% \label{eq:iwasawa}
		% g_{i} \,=\,
		% \begin{bmatrix} u_{i} & \\ & u_{i} \end{bmatrix}
		% \begin{bmatrix} y^{1/2} & xy^{-1/2} \\ & y^{-1/2} \end{bmatrix}
		% \begin{bmatrix} \cos(\theta_{i}) & \sin(\theta_{i}) \\
			% -\sin(\theta_{i}) & \cos(\theta_{i}) \end{bmatrix}
		% \text{ .}
	% \end{equation}
	% %
	% Entonces, llamando $s=k_{0}-2$,
	% \begin{align*}
		% \phi_{f}(g_{\infty},\hhat{\alpha}) & \,=\,
			% \bigg(\prod_{i=1}^{n}\,
			% \frac{\det(g_{i})^{m_{i}+k_{i}-1}}{%
				% j(g_{i},\sqrt{-1})^{k_{i}}}\bigg)\,
			% f(g_{\infty}\cdot\mathbf{i},
				% \hhat{\alpha}\Idfin{\cal{O}}) \\
		% & \,=\,\bigg(\prod_{i=1}^{n}\,\det(g_{i})^{s/2}\bigg)\,
			% \bigg(\prod_{i=1}^{n}\,
			% \frac{\det(g_{i})^{k_{i}/2}}{%
				% j(g_{i},\sqrt{-1})^{k_{i}}}\bigg)\,
			% f((x_{i}+\sqrt{-1}y_{i})_{i},
				% \hhat{\alpha}\Idfin{\cal{O}}) \\
		% & \,=\, \bigg(\prod_{i=1}^{n}\,u_{i}^{s}\bigg)\,
			% \bigg(\prod_{i=1}^{n}\,y_{i}^{k_{i}/2}\,
				% e^{\sqrt{-1}k_{i}\theta_{i}}\bigg)\,
			% f((x_{i}+\sqrt{-1}y_{i})_{i},
				% \hhat{\alpha}\Idfin{\cal{O}})
		% \text{ .}
	% \end{align*}
	% %
	% El operador de Casimir $\Delta_{j}$ est\'{a} dado por
	% \begin{align*}
		% \Delta_{j} & \,=\,
			% -y_{j}^{2}\left(\frac{\partial^{2}}{\partial x_{j}^{2}}
			% \,+\,\frac{\partial^{2}}{\partial y_{j}^{2}}\right)
			% \,-\, y_{j}
			% \frac{\partial^{2}}{\partial x_{j}\partial\theta_{j}}
		% \text{ .}
	% \end{align*}
	% %
	% En la introducci\'{o}n, mencionamos que el factor de automorf\'{\i}a
	% suele estar definido de una manera levemente distinta (la potencia en
	% el determinante de la matriz involucrada suele ser $k/2$, en lugar de
	% algo de la forma $m+k-1$, como en nuestra definici\'{o}n). Esta
	% diferencia tiene como consecuencia la aparici\'{o}n de un factor
	% correspondiente a la parte central en la descomposici\'{o}n
	% \eqref{eq:descomposiciondegldos} de $\GLtp_{2}(\bb{R})$,
	% $\prod_{i=1}^{n}\,u_{i}^{s}$, que no estar\'{\i}a presente, si
	% hubi\'{e}semos definido $\phi_{f}$ como en
	% \eqref{eq:funcionadelicacorrespondienteelipticas}. Aun as\'{\i}, como
	% se hizo en la introducci\'{o}n, se deduce que $\phi_{f}$ es una
	% autofunci\'{o}n para $\Delta_{j}$ con autovalor
	% $-\frac{k_{j}}{2}\Big(\frac{k_{j}}{2}-1\Big)$.
% \end{proof}
%
% \paragraph{Formas automorfas}
% \input{./secciones/comentarioFormasAutomorfas.tex}

Teniendo en cuenta la correspondencia entre formas cuspidales y formas
automorfas, si $f\in\spitzH{k}{\frak{N}}$, llamaremos tambi\'{e}n forma de
Hilbert cuspidal a la correspondiente funci\'{o}n $\phi_{f}$ dada por
\eqref{eq:funcionadelicacorrespondientedehilbert}.

\subsection{Funciones de cuadrado integrable}%
	\label{subsec:enlosadelescuadradointegrable}
Usando el lenguaje ad\'{e}lico, definimos un producto interno
$\langle f,g\rangle$ para formas cuspidales. Fijamos un sistema de
representantes $\{\frak{a}\}$ del grupo de clases estrictas de $F$ y denotamos
por $\hhat{\alpha}$ las matrices dadas por
\eqref{eq:matrizasociadaaidelematrices}.

Si bien, puede parecer natural definir un producto interno usando el
isomorfismo \eqref{eq:descomposicioncuspidalesmatrices} y la expresi\'{o}n
\eqref{eq:productodepeterssongamacero} para los grupos
$\Gamma_{0}(\frak{N},\frak{a})$, una definici\'{o}n as\'{\i} depender\'{\i}a
del sistema de representantes de $\pClass{F}$ que parametriza las componentes
de la variedad $Y_{0}(\frak{N})$. Para llegar a una definici\'{o}n adecuada,
observamos, en primer lugar, que dicha parametrizaci\'{o}n est\'{a} relacionada
con una descomposici\'{o}n similar del grupo $\GL_{2}(\adeles{F})$. Por
aproximaci\'{o}n fuerte (ver tambi\'{e}n
\S~\ref{sec:cuaternionicasvariedadesdeshimura}),
\begin{align*}
	\GL_{2}(\adeles{F}) & \,=\,\bigsqcup_{\frak{a}}\,\centre(\adeles{F})\,
		\GL_{2}(F)\,\GLtp_{2}(\bb{R})^{n}\,\hhat{\alpha}\Idfin{\cal{O}}
	\text{ .}
\end{align*}
%
Entonces hay una identificaci\'{o}n
\begin{equation}
	\label{eq:variedadmodularycocienteadelico}
	\centre(\adeles{F})\,\GL_{2}(F)\backslash\GL_{2}(\adeles{F})/
		K_{\infty}\Idfin{\cal{O}} \,\simeq\,\bigsqcup_{\frak{a}}\,
			\Gamma_{0}(\frak{N},\frak{a})\backslash\hP^{n}
		\,=\,Y_{0}(\frak{N})
	\text{ .}
\end{equation}
%
En segundo lugar, para cada representante $\frak{a}$, si
$f,g\in\spitzH{k}{\frak{N}}$ el producto interno
$\langle f_{\frak{a}},g_{\frak{a}}\rangle_{\frak{a}}$ se define en t\'{e}rminos
de una integral sobre la componente correspondiente de la variedad
$Y_{0}(\frak{N})$. Del lado ad\'{e}lico, teniendo en cuenta la
identificaci\'{o}n \eqref{eq:variedadmodularycocienteadelico}, debemos poder
integrar sobre el cociente
\begin{math}
	\centre(\adeles{F})\,\GL_{2}(F)\backslash\GL_{2}(\adeles{F})
\end{math}~.
El problema es que, si bien las funciones $\phi_{f}$ y $\phi_{g}$ son
invariantes por $\GL_{2}(F)$, no es obvio c\'{o}mo obtener a partir de ellas un
integrando definido m\'{o}dulo el centro $\centre(\adeles{F})$.

Sea $f\in\spitzH{k}{\frak{N}}$ y sea $\phi=\phi_{f}$ la funci\'{o}n definida
por \eqref{eq:funcionadelicacorrespondientedehilbert}. Seg\'{u}n el \'{\i}tem
(iv) de la Proposici\'{o}n~\ref{propo:equivalenciaautomorfasformascuspidales},
sabemos que
\begin{math}
	\phi(\tau_{\infty}g)=\Big(\prod_{i=1}^{n}\,\tau_{i}^{k_{0}-2}\Big)\,
		\phi(g)
\end{math}~, si $\tau_{\infty}\in\centre(\bb{R})=(\bb{R}^{\times})^{n}$. Pero
no hemos dicho nada acerca de c\'{o}mo afecta multiplicar por un elemento
central de la parte no arquimediana. Para poder describir esta acci\'{o}n,
definimos, para cada $\tau\in\centre(\adeles{F})$ y cada $\phi$, una
funci\'{o}n $\rho(\tau)\phi$ por
\begin{align*}
	\big(\rho(\tau)\phi\big)(g) & \,=\,|\tau|_{\adeles{F}}^{-(k_{0}-2)}\,
		\phi(\tau g)
	\text{ ,}
\end{align*}
%
donde $|\tau|_{\adeles{F}}$ denota el valor absoluto del ad\`{e}le $\tau$.%
\footnote{
	Recordemos que este valor absoluto est\'{a} dado por
	$|\tau|_{\adeles{F}}=\prod_{v}\,|\tau_{v}|_{v}$, donde $|\cdot|_{v}$
	denota el valor absoluto en la completaci\'{o}n $F_{v}$.
}

Sea $\{\hhat{a}\}\subset\Idfin{F}$ un sistema de representantes de las clases
en $\pClass{F}$ y sea $\hhat{\tau}\in\Idfin{F}$. Existen
$\lambda\in F_{+}^{\times}$, $\hhat{r}\in\Idfin{\oka{F}}$ y un representante
$\hhat{a}$ tales que $\hhat{\tau}=\lambda_{0}\hhat{a}\hhat{r}$, donde
$\lambda_{0}$ denota la parte finita de $\lambda$. La parte arquimediana la
denotamos $\lambda_{\infty}$. Entonces, dado que
\begin{math}
	\Idfin{\oka{F}}\subset\Idfin{\cal{O}}\cap\centre(\adeles{F})
\end{math}~, $\phi(\lambda g)=\phi(g)=\phi(g\hhat{r})$,
$|\lambda|_{\adeles{F}}=1=|\hhat{r}|_{\adeles{F}}$ y $\lambda\gg 0$, podemos
deducir que
\begin{align*}
	\big(\rho(\hhat{\tau})\phi\big)(g) & \,=\,
		|\lambda_{0}\hhat{a}\hhat{r}|_{\adeles{F}}^{-(k_{0}-2)}\,
			\phi(\lambda_{0}\hhat{a}\hhat{r}g) \,=\,
		|\lambda_{\infty}|_{\adeles{F}}^{k_{0}-2}
			|\hhat{a}|_{\adeles{F}}^{-(k_{0}-2)}\,
			\phi(\lambda\lambda_{\infty}^{-1}\hhat{a}g\hhat{r}) \\
	& \,=\,|\norma(\lambda)|^{k_{0}-2}
		|\hhat{a}|_{\adeles{F}}^{-(k_{0}-2)}\,
			\phi(\lambda_{\infty}^{-1}\hhat{a}g) \,=\,
		\signo(\norma(\lambda))^{-(k_{0}-2)}
		|\hhat{a}|_{\adeles{F}}^{-(k_{0}-2)}\,\phi(\hhat{a}g) \\
	& \,=\,\big(\rho(\hhat{a})\phi\big)(g)
	\text{ .}
\end{align*}
%
Si llamamos
\begin{math}
	\chi_{\infty}(\tau_{\infty})=\prod_{i=1}^{n}\,
		\signo(\tau_{i})^{k_{0}-2}
\end{math}~, entonces, dado $\tau=(\tau_{\infty},\hhat{\tau})$, vale que
\begin{math}
	\rho(\tau)=\chi_{\infty}(\tau_{\infty})\rho(\hhat{\tau})
\end{math}~. De lo anterior, se deduce que la restricci\'{o}n de $\rho$ a la
parte finita $\Idfin{F}$ es, en otras palabras, una representaci\'{o}n del
grupo abeliano finito $\pClass{F}$. Podemos concluir que $\spitzH{k}{\frak{N}}$
se descompone como suma directa de subrepresentaciones irreducibles, cada una
de grado $1$:%
\footnote{Ver, por ejemplo, \cite{Etingof} o \cite{SerreScott}}
\begin{equation}
	\label{eq:descomposicioncuspidalescuasicaracteres}
	\spitzH{k}{\frak{N}} \,=\,\bigoplus_{\omega}\,
		\spitzH{k}{\frak{N},\omega}
	\text{ ,}
\end{equation}
%
donde
\begin{math}
	\omega:\,\ideles{F}\rightarrow\bb{C}^{\times}
\end{math} es el cuasicar\'{a}cter
\begin{math}
	\omega=|\cdot|_{\adeles{F}}^{k_{0}-2}\,\chi_{\infty}\chi_{0}
\end{math}~, $\chi_{0}$ recorre los caracteres del grupo $\pClass{F}$ y
\begin{align*}
	\spitzH{k}{\frak{N},\omega} & \,=\,\Big\{\phi\in\spitzH{k}{\frak{N}}
		\,:\,\phi(\tau g)=\omega(\tau)\cdot\phi(g)
			\text{ , si }\tau\in\centre(\adeles{F})
		\Big\}
	\text{ .}
\end{align*}
%
Notamos que
\begin{math}
	|\omega|=|\cdot|_{\adeles{F}}^{k_{0}-2}
\end{math} y que
\begin{math}
	|\omega|^{-1}\omega=\chi_{\infty}\chi_{0}:\,
		\ideles{F}\rightarrow S^1
\end{math} es un car\'{a}cter de $\ideles{F}$ trivial en $F^{\times}$ y en
$\Idfin{\oka{F}}$. En general, escribiremos $\chi$ para denotar dicho
car\'{a}cter. Si $\hhat{\tau}\in\Idfin{F}$ y
\begin{math}
	\frak{t}=\hhat{\tau}\Adfin{\oka{F}}\,\cap\,F
\end{math} es el ideal fraccionario determinado por $\hhat{\tau}$, podemos
definir
\begin{align*}
	\chi(\frak{t}) \,:=\,\chi(\hhat{\tau})
		& \quad\text{y}\quad\omega(\frak{t})\,:=\,\omega(\hhat{\tau})
	\text{ .}
\end{align*}
%
El valor de $\chi(\frak{t})$ depende \'{u}nicamente de la clase de $\frak{t}$
en $\pClass{F}$ y
\begin{math}
	|\omega(\frak{t})|=\idnorm(\frak{t})^{-(k_{0}-2)}
\end{math}~, donde $\idnorm(\frak{t})$ denota la norma del ideal.
%%
% \{\{\{
% Observaci\'{o}n al margen. Si $f=(f_{\frak{a}})_{\frak{a}}$ es una forma
% modular,
% \begin{align*}
	% \rho(\hhat{\tau})f(z,\hhat{\alpha}\Idfin{\cal{O}}) & \,:=\,
		% f(z,\hhat{\tau}\hhat{\alpha}\Idfin{\cal{O}})
	% \text{ .}
% \end{align*}
% %
% Entonces $\rho(\hhat{\tau})f$ tambi\'{e}n es una forma modular, del mismo peso
% que $f$. Sea $\frak{t}\subset F$ el ideal correspondiente a $\hhat{\tau}$.
% Tomando norma reducida, el ideal fraccionario determinado por
% $\nrd(\hhat{\tau}\hhat{\alpha}\Idfin{\cal{O}})$ es $\frak{t}^{2}\frak{a}$. Sea
% $\frak{b}$ el \'{u}nico representante de las clases estrictas tal que
% $[\frak{t}^{2}\frak{a}]=[\frak{b}]$. Por aproximaci\'{o}n fuerte, existe
% $\gamma\in\GLtp_{2}(F)$ tal que
% $\gamma\hhat{\tau}\hhat{\alpha}\Idfin{\cal{O}}=\hhat{\beta}\Idfin{\cal{O}}$ y,
% por lo tanto,
% \begin{align*}
	% \big(\rho(\hhat{\tau})f\big)_{\frak{a}}(z) & \,=\,
		% f(z,\gamma^{-1}\hhat{\beta}\Idfin{\cal{O}})\,=\,
		% f_{\frak{b}}\operadormatrices{\peso{k}}{\gamma}
	% \text{ .}
% \end{align*}
% %
% La relaci\'{o}n entre los representantes del grupo de clases estrictas est\'{a}
% dada por
% \begin{align*}
	% \lambda\frak{t}^{2}\frak{a} & \,=\,\frak{b}
	% \text{ ,}
% \end{align*}
% %
% para alg\'{u}n elemento totalmente positivo $\lambda\gg 0$. La
% representaci\'{o}n $\rho$ permuta las componentes de $f$ correspondientes a
% ideales pertenecientes al mismo ``g\'{e}nero''.
% \}\}\}
%%

Dado un cuasicar\'{a}cter $\omega:\,\ideles{F}\rightarrow\bb{C}^{\times}$
(trivial en $F^{\times}$), escribimos $f\in\spitzH{k}{\frak{N},\omega}$, si
$f\in\spitzH{k}{\frak{N}}$ es tal que $\phi_{f}\in\spitzH{k}{\frak{N},\omega}$.
Denotamos por $\cuadradointegrables{\omega}$ el espacio de funciones medibles
$\phi:\,\GL_{2}(\adeles{F})\rightarrow\bb{C}^{\times}$ tales que:%
\footnote{Comparar con \cite[Ch.~3]{Bump}}
\begin{itemize}
	\item[(i)] $\phi(\tau g)=\omega(\tau)\cdot\phi(g)$ para todo
		$\tau\in\centre(\adeles{F})$,
	\item[(ii)] $\phi(\gamma g)=\phi(g)$ para toda $\gamma\in\GL_{2}(F)$ y
	\item[(iii)] $\phi$ es de \emph{cuadrado integrable m\'{o}dulo el %
		centro}, es decir,
		\begin{align*}
			\int_{\centre(\adeles{F})\,\GL_{2}(F)\backslash%
				\GL_{2}(\adeles{F})}\,|\omega(\det\,g)|^{-1}
				|\phi(g)|^{2}\,dg & \,<\,\infty
			\text{ .}
		\end{align*}
		%
\end{itemize}
%
Si $\omega=\psi\cdot\chi$ y si $f\in\spitzH{k}{\frak{N},\omega}$, entonces
$\phi=\phi_{f}$ pertenece a este espacio; lo \'{u}nico que hay que verificar es
la \'{u}ltima condici\'{o}n. Para ver esto, sean $f_{\frak{a}}$ las componentes
dadas por el isomorfismo \eqref{eq:descomposicioncuspidalesmatrices}. El
integrando $|\omega(\det(g))|^{-1}|\phi(g)|^{2}$ es invariante por la
acci\'{o}n del centro, es decir, por $\rho$, con lo cual, la integral anterior
est\'{a} bien definida. Llamemos temporariamente
$\tilde{Y}_{\hhat{\alpha}}(\Idfin{\cal{O}})$ a la componente
\begin{align*}
	\tilde{Y}_{\hhat{\alpha}}(\Idfin{\cal{O}}) & \,=\,
		\centre(\adeles{F})\,\GL_{2}(F)\,\GLtp_{2}(\bb{R})^{n}\,
			\hhat{\alpha}\Idfin{\cal{O}}
\end{align*}
%
de $\GL_{2}(\adeles{F})$. Con $g_{\infty}\in\GLtp_{2}(\bb{R})^{n}$, $\hhat{u}$
variando en $\Idfin{\cal{O}}$, $d\theta$ y $d\hhat{u}$ las medidas que cumplen
$d\theta(\SO{2}^{n})=1$ y $d\hhat{u}(\Idfin{\cal{O}})=1$ y usando la
descomposici\'{o}n de Iwasawa \eqref{eq:descomposiciondegldos},
\begin{align*}
	& \int_{\centre(\adeles{F})\,\GL_{2}(F)\backslash%
		\tilde{Y}_{\hhat{\alpha}}(\Idfin{\cal{O}})}\,
		|\omega(\det(g_{\infty}\hhat{\alpha}\hhat{u}))|^{-1}
		|\phi(g_{\infty},\hhat{\alpha}\hhat{u})|^{2}\,
		dg_{\infty}\,d\hhat{u} \\
	& \qquad\,=\, \frac{1}{|\omega(\frak{a})|}\,
		\int_{\centre(\adeles{F})\,\GL_{2}(F)\backslash%
		\tilde{Y}_{\hhat{\alpha}}(\Idfin{\cal{O}})}\,
			\Big|\phi\Big(
			\left[\begin{smallmatrix}
				y^{1/2} & xy^{-1/2} \\
				& y^{-1/2}
			\end{smallmatrix}\right]\,h(\theta),
			\hhat{\alpha}\hhat{u}
			\Big)\Big|^{2}
		d\mu\,d\theta\,d\hhat{u} \\[5pt]
	& \qquad\,=\,\idnorm(\frak{a})^{k_{0}-2}\,
		\int_{\Gamma_{0}(\frak{N},\frak{a})\backslash%
		\GLtp_{2}(\bb{R})^{n}/\centre(\bb{R})\,\SO{2}^{n}}\,
		\big|f_{\frak{a}}(x+y\mathbf{i})\big|^{2}\,y^{\peso{k}}\,d\mu
	\text{ .}
\end{align*}
%
Pero este \'{u}ltimo t\'{e}rmino es finito, porque la forma cuspidal
$f_{\frak{a}}$ es de cuadrado integrable. Esto demuestra que $\phi_{f}$
pertenece al espacio $\cuadradointegrables{\omega}$. En realidad,
$\phi\in\cuadradocuspidales{\omega}$, el subespacio de funciones
\emph{cuspidales}.%
\footnote{
	Ver (vii) de la Proposici\'{o}n~%
	\ref{propo:equivalenciaautomorfasformascuspidales}
}
El producto interno en este espacio est\'{a} dado por
\begin{equation}
	\label{eq:peterssonenadelesmatrices}
	\langle \phi,\phi'\rangle \,:=\,
	\int_{\centre(\adeles{F})\,\GL_{2}(F)\backslash\GL_{2}(\adeles{F})}\,
		|\omega(\det\,g)|^{-1}\,\phi(g)\lconj{\phi'(g)}\,dg
	\text{ ,}
\end{equation}
%

Finalmente, observamos que
\begin{math}
	\phi(g_{\infty},\hhat{g}) =\sum_{\frak{a}}\,
		\phi(g_{\infty},\hhat{g})\,
		[\hhat{\alpha}\Idfin{\cal{O}}](\hhat{g})
\end{math}~,
donde $[\hhat{\alpha}\Idfin{\cal{O}}]$ denota la funci\'{o}n
caracter\'{\i}stica del conjunto $\hhat{\alpha}\Idfin{\cal{O}}$. De esto
deducimos que, dadas $f,g\in\spitzH{k}{\frak{N},\omega}$, se cumple
\begin{align*}
	\langle f,g\rangle & \,\equiv\,\langle\phi_{f},\phi_{g}\rangle\,=\,
		\sum_{\frak{a}}\,\idnorm(\frak{a})^{k_{0}-2}\,
		\langle f_{\frak{a}},g_{\frak{a}}\rangle_{\frak{a}}
	\text{ .}
\end{align*}
%

\subsection{Una acci\'{o}n a derecha}%
	\label{subsec:operadoresdepesokadelesmatrices}
Ser\'{a} \'{u}til, tambi\'{e}n, traducir y extender la definici\'{o}n de los
operadores de peso $\peso{k}$ de $\modularH{k}{\frak{N}}$ a funciones
ad\'{e}licas. Dada $\phi:\,\GL_{2}(\adeles{F})\rightarrow\bb{C}$ y
$(h,\hhat{\beta})\in\GL_{2}(\adeles{F})$, con $h\in\GL_{2}(\bb{R})^{n}$ y
$\hhat{\beta}\in\GL_{2}(\Adfin{F})$, definimos
\begin{equation}
	\label{eq:operadordepesokenadelesmatrices}
	\big(\phi\operadormatrices{\peso{k}}{(h,\hhat{\beta})}\big)
		(g_{\infty},\hhat{\alpha}) \,=\,
			% \big|\omega(\det(h\hhat{\beta}))\big|^{1/2}\,
		\bigg(\prod_{i=1}^{n}\,J_{i}(h_{i},\sqrt{-1})^{-1}\bigg)\,
		\phi(g_{\infty}h^{-1},\hhat{\alpha}\hhat{\beta}^{-1})
	\text{ .}
\end{equation}
%
Esta expresi\'{o}n define una acci\'{o}n del grupo $\GL_{2}(\Adfin{F})$. La
propiedad (ii) en la Proposici\'{o}n~%
\ref{propo:equivalenciaautomorfasformascuspidales} se puede expresar entonces
\begin{equation}
	\label{eq:matricesinvariantedepesokadelicas}
	\phi\operadormatrices{\peso{k}}(1,\hhat{\beta}) \,=\, \phi
\end{equation}
%
para toda $\hhat{\beta}\in\Idfin{\cal{O}}$.
% (notemos que el peso es irrelevante en esta expresi\'{o}n).
Si una funci\'{o}n $\phi$ cumple \eqref{eq:matricesinvariantedepesokadelicas}
con $\hhat{\beta}$ variando en alg\'{u}n subgrupo $K\subset\GL_{2}(\Adfin{F})$,
decimos que es invariante a derecha por el grupo $K$.

\begin{obsOperadorDePesoKAdjunto}\label{obs:operadordepesokadjunto}
	En relaci\'{o}n con el producto interno, dado un id\`{e}le finito
	$\hhat{\pi}\in\GL_{2}(\Adfin{F})$ y dadas
	$\phi,\phi'\in\cuadradointegrables{\omega}$, se deduce de
	\eqref{eq:peterssonenadelesmatrices} y de
	\eqref{eq:operadordepesokenadelesmatrices} que
	\begin{align*}
		\langle \phi\operadormatrices{\peso{k}}{(1,\hhat{\pi})},
			\phi'\rangle
			& \,=\,|\omega(\det\,\hhat{\pi})|^{-1}\,
				\langle\phi,\phi'\operadormatrices{\peso{k}}{%
				(1,\hhat{\pi}^{-1})}\rangle
			\,=\,\lconj{\chi(\det\,\hhat{\pi})}\,
			\langle\phi,\phi'\operadormatrices{\peso{k}}{%
				(1,\hhat{\pi}^\iota)}\rangle
		\text{ ,}
	\end{align*}
	%
	donde $\chi=|\omega|^{-1}\,\omega$. La \'{u}ltima igualdad es
	consecuencia de que el conjugado del id\`{e}le $\hhat{\pi}$ est\'{a}
	dado por $\hhat{\pi}^\iota=\det(\hhat{\pi})\,\hhat{\pi}^{-1}$.
\end{obsOperadorDePesoKAdjunto}

\section{La correspondencia de Jacquet-Langlands}%
	\label{sec:correspondenciadejl}
La interpretaci\'{o}n automorfa de las formas modulares cuaterni\'{o}nicas
permite definir de manera uniforme los operadores de Hecke para un \'{a}lgebra
definida o indefinida, tal como se hace para el \'{a}lgebra de matrices. Dados,
pues, un \'{a}lgebra de cuaterniones $B/F$, un orden de Eichler $\cal{O}$ en
$B$ y un id\`{e}le $\hhat{\pi}\in\Idfin{B}$, el subgrupo abierto
$\Idfin{\cal{O}}$ act\'{u}a a izquierda sobre el compacto
$\Idfin{\cal{O}}\hhat{\pi}\Idfin{\cal{O}}\subset\Idfin{B}$, de lo que se deduce
que existe una descomposici\'{o}n
\begin{equation}
	\label{eq:cuaternionesdescomposiciondecoclasedobleideles}
	\Idfin{\cal{O}}\hhat{\pi}\Idfin{\cal{O}} \,=\,
		\bigsqcup_{i}\,\Idfin{\cal{O}}\hhat{\pi}_{i}
\end{equation}
%
y el sistema de representantes $\{\hhat{\pi}_{i}\}_{i}$ es un conjunto finito.
Sea $\frak{N}$ el nivel del orden $\cal{O}$. Si $B$ es un \'{a}lgebra
indefinida y $f\in\modularH[B]{k}{\frak{N}}$, la expresi\'{o}n
\begin{equation}
	\label{eq:cuaternionicasoperadorcoclase}
	\big(T_{\hhat{\pi}}f\big)(z,\hhat{\alpha}\Idfin{\cal{O}}) \,=\,
		\sum_{i}\,f(z,\hhat{\alpha}\hhat{\pi}_{i}^{-1}\Idfin{\cal{O}})
\end{equation}
%
define un nuevo elemento de $\modularH[B]{k}{\frak{N}}$; si $B$ es un
\'{a}lgebra definida, la expresi\'{o}n an\'{a}loga, sin el argumento
arquimediano `$z$', tambi\'{e}n define una forma modular del mismo peso y del
mismo nivel que $f$. En todo caso, queda determinado, de esta manera, un
operador
\begin{math}
	T_{\hhat{\pi}}:\,\modularH[B]{k}{\frak{N}}\rightarrow
		\modularH[B]{k}{\frak{N}}
\end{math}
asociado al id\`{e}le $\hhat{\pi}$.

Sea $\frak{D}$ el discriminante de $B$ y sea $\frak{p}$ un ideal primo tal que
$\frak{p}\nmid\frak{D}\frak{N}$. Sea $p$ un uniformizador de
$\oka{F,\frak{p}}$, un generador de su ideal maximal, y sea
$\pi\in\cal{O}_{\frak{p}}$ el elemento dado por la matriz
\begin{align*}
	\pi & \,=\,\begin{bmatrix} p & \\ & 1 \end{bmatrix}
	\text{ .}
\end{align*}
%
Sea $\hhat{\pi}\in\Idfin{B}$ el id\`{e}le dado por $\hhat{\pi}_{v}=1$, si
$v\not=\frak{p}$ y tal que $\hhat{\pi}_{\frak{p}}=\pi$. Llamamos
\emph{operador de Hecke en $\frak{p}$} al operador
$T_{\frak{p}}:=T_{\hhat{\pi}}$, determinado por una descomposici\'{o}n de
$\Idfin{\cal{O}}\hhat{\pi}\Idfin{\cal{O}}$. Definimos, tambi\'{e}n,
\begin{align*}
	\frak{I}(\frak{p}) \,=\,\Big\{\hhat{x}\in\Adfin{\cal{O}}\,:\,
		\nrd(\hhat{x})\in\hhat{p}\Idfin{\oka{F}}\Big\}
		& \quad\text{y} \quad
	\Theta(\frak{p})\,=\,\Idfin{\cal{O}}\backslash\frak{I}(\frak{p})
	\text{ .}
\end{align*}
%
Entonces $\frak{I}(\frak{p})=\Idfin{\cal{O}}\hhat{x}\Idfin{\cal{O}}$ para todo
$\hhat{x}$ perteneciente a este conjunto, pues $(\frak{p},\frak{D}\frak{N})=1$,
y, en particular,
$\frak{I}(\frak{p})=\Idfin{\cal{O}}\hhat{\pi}\Idfin{\cal{O}}$. Adem\'{a}s, la
sumatoria en la definici\'{o}n de $T_{\frak{p}}$ se realiza sobre un sistema de
representantes de $\Theta(\frak{p})$, tanto en el caso indefinido, como en el
definido. Dado que para ideales primos distintos $\frak{p},\frak{q}$ los
operadores correspondientes $T_{\frak{p}}$ y $T_{\frak{q}}$ act\'{u}an en
distintas coordenadas, se deduce que conmutan. El \emph{\'{a}lgebra de Hecke}
actuando en $\modularH[B]{k}{\frak{N}}$ es el \'{a}lgebra de endomorfismos
generada por el conjunto
\begin{math}
	\big\{T_{\frak{p}}\,:\,\frak{p}\nmid\frak{D}\frak{N}\big\}
\end{math}~.
Denotamos por $\spitzH[B]{k}{\frak{N}}$ el espacio de formas cuaterni\'{o}nicas
\emph{cuspidales} de peso $\peso{k}$ y nivel $\frak{N}$ para $B$. En la
siguiente secci\'{o}n se explicar\'{a} lo que se quiere decir por ``cuspidal'';
a los fines de enunciar el siguiente teorema, es suficiente saber que
$\spitzH[B]{k}{\frak{N}}$ es un subespacio de $\modularH[B]{k}{\frak{N}}$, con
la estructura de m\'{o}dulo de Hecke dada por restringir los operadores
$T_{\frak{p}}$.

Sea $\frak{N}\subset\oka{F}$ un ideal \'{\i}ntegro y supongamos que
$\frak{N}=\frak{D}\frak{N}'$ con $\frak{D}$ y $\frak{N}'$ \'{\i}ntegros y
$\frak{D}$ libre de cuadrados. Por el Teorema de clasificaci\'{o}n
\ref{thm:clasificacionglobal}, existe al menos un \'{a}lgebra de cuaterniones
$B/F$ de discriminante $\frak{D}$. Supongamos, adem\'{a}s, que
$(\frak{D},\frak{N}')=1$.

\begin{teoJacquetLanglands}[Jacquet-Langlands]\label{thm:correspondenciajl}
	Existe un morfismo inyectivo
	\begin{math}
		\spitzH[B]{k}{\frak{N}'}
			\hookrightarrow\spitzH{k}{\frak{D}\frak{N}'}
	\end{math}
	que preserva la acci\'{o}n de Hecke y cuya imagen es el subespacio
	$\spitzH{k}{\frak{D}\frak{N}'}^{\frak{D}-\neue}$ de formas
	nuevas en todo ideal primo divisor de $\frak{D}$.
\end{teoJacquetLanglands}

El Teorema \ref{thm:correspondenciajl} nos da un procedimiento para recuperar
$\spitzH{k}{\frak{N}}$ a trav\'{e}s de formas modulares cuaterni\'{o}nicas.
Supongamos que $\frak{l}$ es un ideal primo que divide a $\frak{N}$ pero que al
cuadrado no lo divide. Por el Teorema de clasificaci\'{o}n global para
\'{a}lgebras de cuaterniones \ref{thm:clasificacionglobal}, sabemos que existe
un \'{a}lgebra de cuaterniones $B/F$ tal que
$\Ram(B)\cap\lugares[f]{F}=\{\frak{l}\}$, es decir, $\frak{l}$ es el \'{u}nico
lugar finito de $F$ en donde $B$ ramifica. Si elegimos $B$ de esta manera,
\begin{align*}
	\spitzH{k}{\frak{N}} & \,=\, \spitzH{k}{\frak{N}}^{\frak{l}-\neue}
		\,\oplus\,\spitzH{k}{\frak{N}}^{\frak{l}-\oude} \\
	& \,=\, \spitzH[B]{k}{\frak{N}'}\,\oplus\,
		\iota_{1}\left(\spitzH{k}{\frak{N}'}\right)\,\oplus\,
		\iota_{\frak{l}}\left(\spitzH{k}{\frak{N}'}\right)
	\text{ .}
\end{align*}
%
El problema de la descripci\'{o}n de $\spitzH{k}{\frak{N}}$ se reduce
entonces a poder determinar un subespacio en correspondencia con un espacio
de formas cuaterni\'{o}nicas y un subespacio de formas de Hilbert de nivel
``m\'{a}s bajo''.

Aun asumiendo que podemos \emph{calcular} sin problemas los espacios de formas
cuspidales cuaterni\'{o}nicas $\spitzH[B]{k}{\frak{N}'}$ para un nivel
arbitrario $\frak{N}'$, este procedimiento presenta un inconveniente.
Supongamos primero, para simplificar, que $\frak{N}=\frak{p}\frak{q}$, con
$\frak{p}$ y $\frak{q}$ ideales primos distintos. Entonces la
descomposici\'{o}n de $\spitzH{k}{\frak{p}\frak{q}}$ se puede hacer, en
principio, de varias maneras. Podemos elegir un \'{a}lgebra $B$ con
$\Ram(B)\cap\lugares[f]{F}=\{\frak{p}\}$ para obtener
\begin{align*}
	\spitzH{k}{\frak{p}\frak{q}} & \,=\, \spitzH[B]{k}{\frak{q}}\,\oplus\,
		\spitzH{k}{\frak{p}\frak{q}}^{\frak{p}-\oude}
	\text{ ;}
\end{align*}
%
si $n=[F:\bb{Q}]=1\text{ o }2$, hay una \'{u}nica \'{a}lgebra de tales
caracter\'{\i}sticas, pero si $n>2$ hay m\'{a}s de una elecci\'{o}n posible.
Podemos, tambi\'{e}n, intercambiar los roles de $\frak{p}$ y de $\frak{q}$.
O bien podemos elegir $B$ de manera que
$\Ram(B)\cap\lugares[f]{F}=\{\frak{p},\frak{q}\}$ --de nuevo, hay m\'{a}s
de una manera de hacer esta elecci\'{o}n, si $n\geq 2$-- y obtener as\'{\i}
\begin{align*}
	\spitzH{k}{\frak{p}\frak{q}} & \,=\, \spitzH[B]{k}{1}\,\oplus\,
		\spitzH{k}{\frak{p}\frak{q}}^{\frak{p}\frak{q}-\oude} \\
	& \,=\,\spitzH[B]{k}{1}\,\oplus\, \spitzH{k}{\frak{p}\frak{q}}^{\oude}
	\text{ .}
\end{align*}
%

Pero, si el nivel $\frak{N}$ no es libre de cuadrados, no tenemos tantas
elecciones. Por ejemplo, si $\frak{N}=\frak{p}^{2}$, el discriminante de $B$
debe ser $\frak{D}=1$, es decir, s\'{o}lo podemos permitir ramificaci\'{o}n en
infinito. Si $n=1$ esto es imposible y si $n=2$ hay una \'{u}nica elecci\'{o}n
posible. En general, los m\'{e}todos que describiremos en el
Cap\'{\i}tulo~\ref{cap:metodos} se basan en poder elegir $B$ de manera que $r$,
la cantidad de lugares arquimedianos en donde $B$ no ramifica, sea, o bien $0$,
o bien $1$. En ese caso, si $\frak{N}=\frak{p}^{2}$, hay una \'{u}nica forma de
realizar esa elecci\'{o}n.

En el cap\'{\i}tulo siguiente, describimos en detalle los m\'{o}dulos de Hecke
$\spitzH[B]{k}{\frak{N}}$, manteniendo, como hasta ahora, la distinci\'{o}n
entre las \'{a}lgebras definidas e indefinidas. La descripci\'{o}n en el caso
indefinido es muy similar a la del m\'{o}dulo de formas de Hilbert cuspidales.
V\'{\i}a los isomorfismos de Eichler-Shimura, los espacios de formas
cuatern\'{o}nicas para un \'{a}lgebra indefinida $B$ se realizan en la
cohomolog\'{\i}a de la variedad de Shimura $\shimura[B]{\frak{N}}$. Esta
reinterpretaci\'{o}n da lugar, en el caso particular en que $B$ ramifica en
todos excepto un \'{u}nico lugar arquimediano, a un m\'{e}todo para calcular
formas de Hilbert cuspidales. Por otra parte, la descripci\'{o}n cuando $B$ es
totalmente definida es lo suficientemente expl\'{\i}cita como para ser
implementada y obtener, as\'{\i}, un segundo m\'{e}todo.

% \section{Formas cuaterni\'{o}nicas ``cuspidales''}%
	% \label{sec:cuaternionicasformascuspidales}
% Para definir los subespacios de formas cuspidales sobre un \'{a}lgebra de
cuaterniones arbitraria, distinguimos dos casos. Si $B/F$ es un \'{a}lgebra
indefinida, distinta de matrices, entonces la variedad de Shimura
$\shimura[B]{\frak{N}}$ es compacta y no tiene c\'{u}spides. En este caso,
tiene sentido decir que toda forma modular para $B$ es cuspidal, es decir,
definimos $\spitzH[B]{k}{\frak{N}}:=\modularH[B]{k}{\frak{N}}$.

Cuando $B/F$ es un \'{a}lgebra totalmente definida, la variedad de Shimura
asociada es simplemente un conjunto finito de puntos. Si $\cal{O}\subset B$ un
orden de Eichler, una forma modular cuaterni\'{o}nica para $B$ de nivel
$\cal{O}$ y peso $\peso{k}$ es una funci\'{o}n
$f:\,\Idfin{B}/\Idfin{\cal{O}}\rightarrow W_{\peso{k}}(\bb{C})$ tal que
\begin{math}
	f(\gamma\hhat{\alpha}\Idfin{\cal{O}})=
		f(\hhat{\alpha}\Idfin{\cal{O}})^{\gamma^{-1}}
\end{math}~.
Ac\'{a} tambi\'{e}n distinguimos dos casos. Si $\peso{k}\not =(2,\,\dots,\,2)$,
definimos $\spitzH[B]{k}{\frak{N}}:=\modularH[B]{k}{\frak{N}}$.

% \paragraph{$\peso{k}=(2,\,\dots,\,2)$}
Sea $\ideales{\cal{O}}$ el conjunto de ideales $I$ de $B$ con
$\Oder(I)=\cal{O}$. Si $\peso{k}=(2,\,\dots,\,2)$, el $\GL_{2}(\bb{C})^{n}$%
-m\'{o}dulo $W_{\peso{k}}(\bb{C})$ es trivial y una forma modular
$f\in\modular[B]{(2,\,\dots,\,2)}{\frak{N}}$ es una funci\'{o}n
$f:\,\ideales{\cal{O}}\rightarrow\bb{C}$ constante en clases de isomorfismo
(dos ideales de $B$ son isomorfos, si uno es un m\'{u}ltiplo del otro por un
elemento de $B^{\times}$), es decir, $f(bI)=f(I)$ para todo $b\in B^{\times}$.

Dado un ideal $I\in\ideales{\cal{O}}$, denotamos por $[I]$, tanto la clase
de $I$ en $\lClass{\cal{O}}$, como la funci\'{o}n car\'{a}cteristica de su
clase. Las funciones $[I]$ son formas modulares y, si $\{\lista{I}{H}\}$ es un
sistema de representantes de las clases a izquierda, el conjunto
$\{[I_{1}],\,\dots,\,[I_{H}]\}$ constituye una base de
$\modular[B]{(2,\,\dots,\,2)}{\frak{N}}$. Si $I\in\ideales{\cal{O}}$, existe
$\hhat{\alpha}\in\Idfin{B}$ tal que $I=\hhat{\alpha}\Idfin{\cal{O}}\cap B$. El
grupo
\begin{math}
	\Gamma_{I}:=\Gamma_{\hhat{\alpha}}=
		(\hhat{\alpha}\Idfin{\cal{O}}\hhat{\alpha})\cap B^{\times}
\end{math}
es el estabilizador de $I$ y los elementos centrales son
$\Gamma_{I}\cap F^{\times}=\oka{F}^{\times}$. Las inmersiones reales de $F$
determinan una inclusi\'{o}n discreta en un compacto:
\begin{align*}
	& \Gamma_{I}/\oka{F}^{\times}\,\hookrightarrow\,
		\Idinf{B}/\centre(\bb{R})
		% \,\simeq\,(\bb{H}^{\times}/\bb{R}^{\times})^{n}
		\,\simeq\,(S^{3})^{n}
	\text{ ,}
\end{align*}
%
de donde se deduce que
\begin{align*}
	w_{I} & \,:=\, \left|\Gamma_{I}/\oka{F}^{\times}\right|\,<\,\infty
	\text{ .}
\end{align*}
%

Se define un producto interno (\emph{producto interno de Petersson}) en
$\modular[B]{(2,\,\dots,\,2)}{\frak{N}}$ por
\begin{align*}
	\langle [I],[J]\rangle & \,:=\,
	\begin{cases}
		w_{I} & \quad\text{si } [I] = [J]\text{ ,} \\
		0 & \quad\text{si } [I]\not = [J]\text{ .}
	\end{cases}
\end{align*}
%
Con esta definici\'{o}n, la base $\{[I_{1}],\,\dots,\,[I_{H}]\}$ es una base
ortogonal.

Por otro lado, para cada ideal $\frak{a}\in\ideales{\oka{F}}$ del cuerpo $F$,
denotamos con $[\frak{a}]$ tanto la clase estricta de $\frak{a}$, como la
funci\'{o}n caracter\'{\i}stica de la clase. Como ya mencionamos, la norma
reducida $\nrd:\,B^{\times}\rightarrow F^{\times}$ induce una funci\'{o}n
sobreyectiva
\begin{align*}
	n & \,:\, B^{\times}\backslash\Idfin{B}/\Idfin{\cal{O}}
		\,\twoheadrightarrow\,
		F_{+}^{\times}\backslash\Idfin{F}/\Idfin{\oka{F}}
\end{align*}
%
dada por $n([I])=[\nrd(I)]$. Dado un ideal $\frak{a}$, definimos una
funci\'{o}n $e_{\frak{a}}:\,\ideales{\cal{O}}\rightarrow\bb{C}$ por
\begin{align*}
	e_{\frak{a}}(I) & \,=\,
	\begin{cases}
		\frac{1}{w_{I}} & \quad\text{si } [\nrd(I)]=[\frak{a}]
								\text{ ,}\\
		0 & \quad\text{si } [\nrd(I)]\not =[\frak{a}] \text{ .}
	\end{cases} \\
	%e_{\frak{a}} \,=\, & \bigg(
	%\sum_{i=1}^{H}\,\frac{1}{w_{I_{i}}}[I_{i}]\bigg)\cdot
	%\left([\frak{a}]\circ\nrd\right)
	e_{\frak{a}} & \,=\, \sum_{[\nrd(I_{i})]=[\frak{a}]}\,
				\frac{1}{w_{I_{i}}}\,[I_{i}]
	\text{ .}
\end{align*}
%
Si $\grado[\frak{a}](f):=\langle f,e_{\frak{a}}\rangle$, entonces
\begin{align*}
	\grado[\frak{a}] \,=\, & [\frak{a}]\circ\nrd
	\text{ .}
\end{align*}
%

\begin{defFormaCuspidalDefinida}\label{def:formacuspidaldefinida}
	Una forma modular cuaterni\'{o}nica
	$f\in\modular[B]{(2,\,\dots,\,2)}{\frak{N}}$ se dice \emph{cuspidal},
	\index{forma modular!cuaternionica@cuaterni\'{o}nica!cuspidal}
	si $f$ es ortogonal al subespacio generado por las funciones en
	$\modular[B]{(2,\,\dots,\,2)}{\frak{N}}$ que se factorizan por $\nrd$.
	Denotamos $\spitz[B]{(2,\,\dots,\,2)}{\frak{N}}$ a este subespacio. Es
	decir,
	\begin{align*}
		\spitz[B]{(2,\,\dots,\,2)}{\frak{N}} & \,=\,
			\Big\{ f\in\modular[B]{(2,\,\dots,\,2)}{\frak{N}}
				\,:\,\grado[\frak{a}](f)
						%=\langle f,e_{\frak{a}}\rangle
					=0\,\forall\frak{a}\subset F
					\text{ ideal}%\in\ideales{\oka{F}}
			\Big\}
		\text{ .}
	\end{align*}
	%
\end{defFormaCuspidalDefinida}

% \section{Operadores de Hecke en $\spitzH[B]{k}{\frak{N}}$: caso indefinido}%
	% \label{sec:cuaternionicasheckeindefinida}
% Los operadores de Hecke en los espacios de formas de Hilbert cuspidales se
definen como operadores de coclases dobles. Si $B/F$ es un \'{a}lgebra de
cuaterniones indefinida, la definici\'{o}n de la estructura de m\'{o}dulo de
Hecke en $\spitzH[B]{k}{\frak{N}}$ es similar, teniendo en cuenta los lugares
(finitos) en donde $B$ ramifica. Sea $\frak{D}$ el discriminante de $B$ y sea
$\cal{O}$ un orden de Eichler de nivel $\frak{N}$.

\begin{defHeckeParaIndefinidos}\label{def:heckeparaindefinidos}
	Si $\frak{p}\subset\oka{F}$ es un ideal primo tal que
	$\frak{p}\nmid\frak{D}\frak{N}$, elegimos un generador
	$p\in\oka{F,\frak{p}}$ del ideal maximal en el anillo de enteros de
	la completaci\'{o}n $F_{\frak{p}}$. Sea $\hhat{\pi}\in\Idfin{B}$ el
	id\`{e}le definido por $\hhat{\pi}_{v}=1$, si $v\not =\frak{p}$, y
	\begin{math}
		\hhat{\pi}_{\frak{p}}=\begin{bmatrix} p & \\ & 1 \end{bmatrix}
	\end{math}~.
	Entonces el \emph{operador de Hecke} en $\frak{p}$ es
	$T_{\frak{p}}:=T_{\hhat{\pi}}$, donde
	\begin{align*}
		(T_{\hhat{\pi}}f)(z,\hhat{\alpha}\Idfin{\cal{O}})
			& \,=\,\sum_{i}\,f(z,\hhat{\alpha}\hhat{\pi}_{i}^{-1}
							\Idfin{\cal{O}})
		\text{ ,}
	\end{align*}
	%
	habiendo elegido un conjunto (finito) $\{\hhat{\pi}_{i}\}_{i}$ tal que
	\begin{math}
		\Idfin{\cal{O}}\hhat{\pi}\Idfin{\cal{O}}=
			\bigsqcup_{i}\,\Idfin{\cal{O}}\hhat{\pi}_{i}
	\end{math}~.
\end{defHeckeParaIndefinidos}

Esto equivale a considerar el conjunto
\begin{align*}
	\Theta(\frak{p}) & \,=\, \Idfin{\cal{O}}\backslash
		\Big\{\hhat{\pi}\in\Adfin{\cal{O}}\,:\,
			\nrd(\hhat{\pi})\in\hhat{p}\Idfin{\oka{F}}\Big\}
	\text{ ,}
\end{align*}
%
donde $\hhat{p}\in\Adfin{\oka{F}}$ es tal que
$\hhat{p}\Adfin{\oka{F}}\cap F=\frak{p}$, elegir un conjunto de representantes
y sumar
\begin{math}
	\sum_{\hhat{\pi}\in\Theta(\frak{p})}\,
		f(z,\hhat{\alpha}\hhat{\pi}^{-1}\Idfin{\cal{O}})
\end{math}
sobre dicho conjunto (comparar con la observaci\'{o}n
\ref{obs:idelesdenormap}).
% Notemos que, si $\hhat{\alpha}$ se corresponde con un ideal \'{\i}ntegro
% $\frak{a}$, el producto $\hhat{\alpha}\hhat{\pi}^{-1}$ deber\'{\i}a
% correponderse con el ideal $\frak{a}\frak{p}^{-1}$.
Dado que la sumatoria es finita, $T_{\frak{p}}f$ es una funci\'{o}n holomorfa
en la primera variable, localmente constante en la segunda y se verifica
$(T_{\frak{p}}f)\barra{\peso{k}}{\gamma}=T_{\frak{p}}f$ para todo
$\gamma\in B^{\times}$. Es decir que $T_{\frak{p}}$ define un operador en
$\spitzH[B]{k}{\frak{N}}$. Adem\'{a}s, dado que para distintos ideales primos
$\frak{p},\frak{q}$ los operadores correspondientes $T_{\frak{p}}$ y
$T_{\frak{q}}$ act\'{u}an en distintas coordenadas, se deduce inmediatamente
que conmutan. El \emph{\'{a}lgebra de Hecke} actuando en
$\spitzH[B]{k}{\frak{N}}$ es el \'{a}lgebra de endomorfismos generada por el
conjunto $\big\{ T_{\frak{p}}\,:\,\frak{p}\nmid\frak{D}\frak{N}\big\}$.

El espacio $\spitzH[B]{k}{\frak{N}}$ se descompone como suma directa de
$\spitzH[B]{k}{\frak{N},\frak{a}}$ v\'{\i}a el isomorfismo
$f\mapsto(f_{\frak{a}})_{\frak{a}}$ donde
$f_{\frak{a}}(z)=f(z,\hhat{\alpha}\Idfin{\cal{O}})$ y
$\nrd(\hhat{\alpha})=\hhat{a}\in\Adfin{\oka{F}}$ es tal que
$\hhat{a}\Adfin{\oka{F}}\cap F=\frak{a}$. Los operadores de Hecke act\'{u}an
permutando estos subespacios. Sea $\frak{b}$ un ideal \'{\i}ntegro de $\oka{F}$
cuya clase estricta es $[\frak{b}]=[\frak{a}\frak{p}^{-1}]$, sea
$\hhat{b}\in\Adfin{\oka{F}}$ tal que $\hhat{b}\Adfin{\oka{F}}\cap F=\frak{b}$ y
sea $\hhat{\beta}\in\Idfin{B}$ tal que $\nrd(\hhat{\beta})=\hhat{b}$. Entonces,
para cada representante $\hhat{\pi}$ de las \'{o}rbitas en $\Theta(\frak{p})$,
el ret\'{\i}culo
\begin{math}
	\hhat{\alpha}\hhat{\pi}^{-1}\Adfin{\cal{O}}\hhat{\beta}^{-1}\cap B
\end{math}
es un ideal de $B$ cuyo orden a derecha es
\begin{math}
	\cal{O}_{\frak{b}}=\hhat{\beta}\Adfin{\cal{O}}\hhat{\beta}^{-1}\cap B
\end{math}
y su norma reducida es un ideal en la clase (estricta) principal. En
consecuencia, %por aproximaci\'{o}n fuerte,
existe $\varpi\in B_{+}^{\times}$ tal que
\begin{math}
	\hhat{\alpha}\hhat{\pi}^{-1}\Adfin{\cal{O}}\hhat{\beta}^{-1}\cap B=
		\varpi^{-1}\cal{O}_{\frak{b}}
\end{math}
y una unidad $\hhat{u}\in\Idfin{\cal{O}}$ tal que
\begin{math}
	\hhat{\alpha}\hhat{\pi}^{-1}\hhat{u}\hhat{\beta}^{-1}=\varpi^{-1}
\end{math}~.
Si ahora miramos las componentes de la forma $T_{\frak{p}}f$ se deduce que,
usando la invarianza de $f$,
\begin{align*}
	(T_{\frak{p}}f)_{\frak{a}}(z) & \,=\,
		(T_{\frak{p}}f)(z,\hhat{\alpha}\Idfin{\cal{O}}) \,=\,
		\sum_{\hhat{\pi}\in\Theta(\frak{p})}\,
			f(z,\hhat{\alpha}\hhat{\pi}^{-1}\Idfin{\cal{O}})
		\,=\, \sum_{\varpi}\,
			f(z,\varpi^{-1}\hhat{\beta}\Idfin{\cal{O}}) \\
	& \,=\, \bigg(\prod_{i=1}^{r}\,J_{i}(\varpi_{i},z_{i})^{-1}\bigg)\,
		f(\varpi z,\hhat{\beta}\Idfin{\cal{O}})^{\varpi}
	\,=\, \sum_{\varpi}\,(f_{\frak{b}}\barra{\peso{k}}{\varpi})(z)
	\text{ .}
\end{align*}
%
% As\'{\i}, $T_{\frak{p}}:\,\spitzH[B]{k}{\frak{N},\frak{b}}\rightarrow%
% \spitzH[B]{k}{\frak{N},\frak{a}}$.
Si bien esto muestra que la descripci\'{o}n de $T_{\frak{p}}$ en t\'{e}rminos
de la descomposici\'{o}n de $\spitzH[B]{k}{\frak{N}}$ es sencilla, la utilidad
de la igualdad
\begin{math}
	\big(T_{\frak{p}}f\big)_{\frak{a}}=
		\sum_{\varpi}\,f_{\frak{b}}\barra{\peso{k}}{\varpi}
\end{math}
depende de poder hallar todos aquellos elementos $\varpi$ cuya existencia
est\'{a} garantizada por aproximaci\'{o}n fuerte. Estos elementos se pueden
caracterizar globalmente. La sumatoria se realiza sobre un sistema de
representantes de
\begin{align*}
	\Theta(\frak{p})_{\frak{a},\frak{b}} & \,=\,
		\Gamma_{\frak{b}}\backslash\Big\{\varpi\in
			I_{\frak{b}}I_{\frak{a}}^{-1}\cap B_{+}^{\times}\,:\,
			\nrd(\varpi)\frak{a}\frak{p}^{-1}=\frak{b}
		\Big\}
	\text{ ,}
\end{align*}
%
donde $I_{\frak{a}}=\hhat{\alpha}\Adfin{\cal{O}}\cap B$ e
$I_{\frak{b}}=\hhat{\beta}\Adfin{\cal{O}}\cap B$. El argumento es similar al
dado en la observaci\'{o}n \ref{obs:idelesdenormapporglobales}. El resultado
\ref{propo:descomposicioninducedescomposicion} tambi\'{e}n es v\'{a}lido en
este caso, es decir, siendo $B$ un \'{a}lgebra de divisi\'{o}n indefinida (ver
\cite[Propo.~2.3]{ShimuraDirichletSeriesAndAbelianVarieties}).

\begin{obsHeckeParaIndefinidosPorBloques}%
	\label{obs:heckeparaindefinidosporbloques}
	Los operadores de Hecke act\'{u}an por bloques en
	$\spitzH[B]{k}{\frak{N}}$, permutando en cierto sentido los sumandos
	$\spitzH[B]{k}{\frak{N},\frak{a}}$: dados
	$\frak{p}\nmid\frak{D}\frak{N}$ primo y $\frak{a}$ y $\frak{b}$ tales
	que $[\frak{b}]=[\frak{a}\frak{p}^{-1}]$, definimos
	\begin{align*}
		\big(T_{\frak{p}}\big)_{\frak{a},\frak{b}} & \,:\,
			\spitzH[B]{k}{\frak{N},\frak{b}}\,\rightarrow\,
			\spitzH[B]{k}{\frak{N},\frak{a}}
	\end{align*}
	%
	por
	\begin{align*}
		\big(T_{\frak{p}}\big)_{\frak{a},\frak{b}}f_{\frak{b}} & \,=\,
		\sum_{\varpi\in\Theta(\frak{p})_{\frak{a},\frak{b}}}\,
			f_{\frak{b}}\operadormatrices{\peso{k}}{\varpi}
		\text{ .}
	\end{align*}
	%
	Entonces el operador $T_{\frak{p}}$ act\'{u}a como la matriz de
	operadores
	\begin{math}
		\big[\big(T_{\frak{p}}\big)_{\frak{a},\frak{b}}
			\big]_{\frak{a},\frak{b}}
	\end{math}~,
	donde $\frak{a}$ y $\frak{b}$ recorren los representantes de las
	clases estrictas de $F$ y $(T_{\frak{p}})_{\frak{a},\frak{b}}$ es
	el operador reci\'{e}n definido, si $[\frak{a}]=[\frak{b}\frak{p}]$, y
	es igual a $0$, en caso contrario.
\end{obsHeckeParaIndefinidosPorBloques}



% \section{Operadores de Hecke en $\spitzH[B]{k}{\frak{N}}$: caso definido}%
	% \label{sec:cuaternionicasheckedefinida}
% Supongamos ahora que $B/F$ es un \'{a}lgebra totalmente definida y sea
$\cal{O}\subset B$ un orden de Eichler. La definici\'{o}n de los operadores de
Hecke es an\'{a}loga a la definici\'{o}n para \'{a}lgebras indefinidas en
t\'{e}rminos de operadores de colcases dobles. Dado $\hhat{\pi}\in\Idfin{B}$,
vale que
\begin{math}
	\Idfin{\cal{O}}\hhat{\pi}\Idfin{\cal{O}}=\bigsqcup_{i}\,
		\Idfin{\cal{O}}\hhat{\pi}_{i}
\end{math}~,
donde $\hhat{\pi}_{i}$ recorre un conjunto finito. Dada
$f\in\modularH[B]{k}{\frak{N}}$, la expresi\'{o}n
\begin{equation}
	\label{eq:heckeparadefinidos}
	\big(T_{\hhat{\pi}}f\big)(\hhat{\alpha}\Idfin{\cal{O}}) \,=\,
		\sum_{i}\,f(\hhat{\alpha}\hhat{\pi}_{i}^{-1}\Idfin{\cal{O}})
\end{equation}
%
determina un nuevo elemento de $\modularH[B]{k}{\frak{N}}$.

\begin{defHeckeParaDefinidos}\label{def:heckeparadefinidos}
	Sea $\frak{p}\subset\oka{F}$ un ideal primo que no divide a
	$\frak{D}\frak{N}$. Si $\hhat{\pi}\in\Idfin{B}$ es el id\`{e}le dado
	por
	\begin{math}
		\begin{bmatrix} p & \\ & 1 \end{bmatrix}
	\end{math}
	en el lugar $\frak{p}$ y $1$ en $v\not =\frak{p}$. El
	\emph{operador de Hecke} en $\frak{p}$ es el operador
	$T_{\frak{p}}:=T_{\hhat{\pi}}$ asociado a la coclase doble
	$\Idfin{\cal{O}}\hhat{\pi}\Idfin{\cal{O}}$.
\end{defHeckeParaDefinidos}

Como en la secci\'{o}n \S~\ref{subsec:dehilbertoperadoresdehecke}, sea
$\frak{I}(\frak{p})=\Idfin{\cal{O}}\hhat{\pi}\Idfin{\cal{O}}=\{\}$.

Sea $H=\#\lClass{\cal{O}}$ y fijemos un sistema de representantes
$\{I_{t}\}_{t}$ de las clases a izquierda en $\lClass{\cal{O}}$ y elementos
$\hhat{\alpha}_{t}\in\Idfin{B}$ tales que
$I_{t}=\hhat{\alpha}_{t}\Adfin{\cal{O}}\cap B$. Dado
$\hhat{\alpha}\Idfin{\cal{O}}$, para cada $\hhat{\pi}_{i}$ existe un \'{u}nico
$t\in [\![1,H]\!]$ y un elemento $\rho_{i}\in B^{\times}$ que verifica
\begin{equation}
	\label{eq:cuaternionicadefinidaidealesequivalentes}
	\rho_{i}\hhat{\alpha}\hhat{\pi}_{i}^{-1}\Idfin{\cal{O}} \,=\,
		\hhat{\alpha}_{t}\Idfin{\cal{O}}
	\text{ .}
\end{equation}
%
Dos elementos $\rho_{i},\rho_{i}'$ que cumplen con
\eqref{eq:cuaternionicadefinidaidealesequivalentes} difieren en una
unidad del orden $\cal{O}_{t}=\Oizq(I_{t})$.
Entonces podemos reescribir $T_{\hhat{\pi}}f$ de la siguiente manera:
\begin{align*}
	\big(T_{\hhat{\pi}}f\big)
	(\hhat{\alpha}\Idfin{\cal{O}}) & \,=\,\sum_{i}\,\sum_{t=1}^{H}\,
		f(\hhat{\alpha}\hhat{\pi}_{i}^{-1}\Idfin{\cal{O}})\,
		[\hhat{\alpha}_{t}\Idfin{\cal{O}}]
			(\hhat{\alpha}\hhat{\pi}_{i}^{-1}\Idfin{\cal{O}}) \\
	& \,=\, \sum_{t=1}^{H}\,\sum_{i}\,
		f(\hhat{\alpha}_{t}\Idfin{\cal{O}})^{\rho_{i}}\,
		[\hhat{\alpha}_{t}\Idfin{\cal{O}}]
			(\hhat{\alpha}\hhat{\pi}_{i}^{-1}\Idfin{\cal{O}})
	\text{ ,}
\end{align*}
%
identificando un ret\'{\i}culo $\hhat{\alpha}\Adfin{\cal{O}}\cap B$ con la
coclase $\hhat{\alpha}\Idfin{\cal{O}}$ en los id\`{e}les finitos de $B$.
El t\'{e}rmino
\begin{math}
	[\hhat{\alpha}_{t}\Idfin{\cal{O}}]
			(\hhat{\alpha}\hhat{\pi}_{i}^{-1}\Idfin{\cal{O}})
\end{math}
es igual a $1$ o a $0$, si las clases
\begin{math}
	[\hhat{\alpha}_{t}\Idfin{\cal{O}}]
\end{math}
y
\begin{math}
	[\hhat{\alpha}\hhat{\pi}_{i}^{-1}\Idfin{\cal{O}}]
\end{math}
son iguales o no.

Ahora bien, la igualdad de las clases
\begin{math}
	[\hhat{\alpha}_{t}\Idfin{\cal{O}}]=
		[\hhat{\alpha}\hhat{\pi}^{-1}\Idfin{\cal{O}}]
\end{math}~,
con $\hhat{\pi}\in\Adfin{\cal{O}}$ de norma reducida igual a un id\`{e}le
correspondiente a $\frak{p}$, equivale a la existencia de una unidad
$\rho\in B^{\times}$ tal que
\begin{math}
	\rho\in\hhat{\alpha}_{t}\Idfin{\cal{O}}\hhat{\pi}\hhat{\alpha}^{-1}
\end{math}~;
esto es, a su vez, equivalente, en t\'{e}rminos de ideales de $B$, a que exista
$\rho\in B^{\times}$ tal que $J=II_{t}^{-1}\cal{O}_{t}\rho$ define un ideal con
orden a izquierda $\Oizq(I)=\hhat{\alpha}\Adfin{\cal{O}}\hhat{\alpha}\cap B$,
\'{\i}ntegro y de norma reducida $\frak{p}$. Rec\'{\i}procamente, dado un ideal
(invertible) $J$ de $B$ que cumple $\Oizq(J)=\Oizq(I)$, $J\sim II_{t}^{-1}$ y
$\nrd(J)=\frak{p}$, existe un id\`{e}le $\hhat{\pi}\in\Adfin{\cal{O}}$ tal que
\begin{align*}
	J & \,=\,\hhat{\alpha}\Adfin{\cal{O}}\hhat{\alpha}^{-1}
		(\hhat{\alpha}\hhat{\pi}\hhat{\alpha}^{-1})\,\cap\,B
	\text{ .}
\end{align*}
%
Pero, entonces, existe $\rho\in B^{\times}$ tal que
\begin{math}
	\hhat{\alpha}\hhat{\pi}^{-1}\Idfin{\cal{O}}=
		\rho^{-1}\hhat{\alpha}_{t}\Idfin{\cal{O}}
\end{math}~.
En definitiva,
\begin{equation}
	\label{eq:heckeparadefinidosconideales}
	\big(T_{\hhat{\pi}}f\big)(I) \,=\,\sum_{t=1}^{H}\,\sum_{\rho}\,
		\frac{1}{w_{t}}\,f(I_{t})^{\rho}
	\text{ ,}
\end{equation}
%
donde $w_{t}=\big|\cal{O}_{t}^{\times}/\oka{F}^{\times}\big|$ y $\rho$
recorre un sistema de representantes de
\begin{align*}
	& \Big\{ \rho\in I_{t}I^{-1}\,:\,\nrd(II_{t}^{-1}\,\rho)=\frak{p}\Big\}
		/\oka{F}^{\times}
	\text{ .}
\end{align*}
%

\subsubsection*{Peso paralelo $\peso{k}=(\lista[\null]{2}{\null})$}
Cuando $\peso{k}=(2,\,\dots,\,2)$, los elementos $\rho_{i}$ act\'{u}an de
manera trivial y queda:
\begin{align*}
	\big(T_{\hhat{\pi}}f\big)
	% f\barra{(2,\,\dots,\,2)}{[\Idfin{\cal{O}}\hhat{\pi}\Idfin{\cal{O}}]}
	(\hhat{\alpha}\Idfin{\cal{O}}) & \,=\, \sum_{t}\,
		f(\hhat{\alpha}_{t}\Idfin{\cal{O}})\cdot
		\#\big\{\hhat{\pi}_{i}\,:\,
			[\hhat{\alpha}\hhat{\pi}_{i}^{-1}\Idfin{\cal{O}}]=
			[\hhat{\alpha}_{t}\Idfin{\cal{O}}]
		\big\}
	\text{ .}
\end{align*}
%
Si $f=[I_{s}]$, con $I_{s}=\hhat{\alpha}_{s}\Adfin{\cal{O}}\cap B$ uno de los
representantes de las clases de ideales,
\begin{align*}
	% [I_{s}]\barra{(2,\,\dots,\,2)}%
	% {[\Idfin{\cal{O}}\hhat{\pi}\Idfin{\cal{O}}]}
	% (\hhat{\alpha}\Idfin{\cal{O}}) \,=\, &
	% \sum_{t}\,[\hhat{\alpha}_{s}\Idfin{\cal{O}}]
	% (\hhat{\alpha}_{t}\Idfin{\cal{O}})\cdot
	% \#\left\lbrace\hhat{\pi}_{i}\,:\,[\hhat{\alpha}\hhat{\pi}_{i}^{-1}
	% \Idfin{\cal{O}}]=[\hhat{\alpha}_{t}\Idfin{\cal{O}}]\right\rbrace \\
	% \,=\, & \#\left\lbrace\hhat{\pi}_{i}\,:\,
	% [\hhat{\alpha}\hhat{\pi}_{i}^{-1}\Idfin{\cal{O}}]=
	% [\hhat{\alpha}_{s}\Idfin{\cal{O}}]\right\rbrace
	% \quad\text{e} \\
	T_{\hhat{\pi}}[I_{s}]
	% [I_{s}]
	% \barra{(2,\,\dots,\,2)}{[\Idfin{\cal{O}}\hhat{\pi}\Idfin{\cal{O}}]}
	& \,=\,\sum_{t}\,\#\big\{\hhat{\pi}_{i}\,:\,
		[\hhat{\alpha}_{t}\hhat{\pi}_{i}^{-1}\Idfin{\cal{O}}]=
		[\hhat{\alpha}_{s}\Idfin{\cal{O}}]\big\} [I_{t}]
	\text{ .}
\end{align*}
%

Intentaremos hallar una expresi\'{o}n global para $T_{\frak{p}}$, es decir, en
t\'{e}rminos de ideales del \'{a}lgebra y que no dependa de tener que elegir
representantes $\{\hhat{\pi}_{i}\}_{i}$ de las coclases de $\Idfin{\cal{O}}$.
En primer lugar, podemos suponer que los representantes $\hhat{\pi}_{i}$
cumplen
\begin{align*}
	\Idfin{\cal{O}}\hhat{\pi}\Idfin{\cal{O}} & \,=\,
		\bigsqcup_{i}\,\Idfin{\cal{O}}\hhat{\varpi}_{i}
		\,=\,\bigsqcup_{i}\,\hhat{\varpi}_{i}\Idfin{\cal{O}}
	\text{ .}
\end{align*}
%

Sean $s,t\in[\![1,h]\!]$. La igualdad entre las clases
\begin{math}
	[\hhat{\alpha}_{t}\hhat{\varpi}_{i}^{-1}\Idfin{\cal{O}}]=
		[\hhat{\alpha}_{s}\Idfin{\cal{O}}]
\end{math}
equivale a la existencia de elementos $b\in B^{\times}$ y
$\hhat{u}\in\Idfin{\cal{O}}$ tales que
$b\hhat{\alpha}_{t}\hhat{\varpi}_{i}^{-1}=\hhat{\alpha}\hhat{u}$.
A su vez, usando que $\{\hhat{\varpi}_{i}\}_{i}$ es un conjunto de
representantes de las coclases tanto a izquierda como a derecha en
$\Idfin{\cal{O}}\hhat{\pi}\Idfin{\cal{O}}$, esto equivale a
$b\hhat{\alpha}_{t}=\hhat{\alpha}_{s}\hhat{\varpi}_{j}^{-1}\hhat{v}$ para un
\'{u}nico $j$ (posiblemente distinto de $i$) y alg\'{u}n
$\hhat{v}\in\Idfin{\cal{O}}$. En particular,
\begin{math}
	[\hhat{\alpha}_{t}\Idfin{\cal{O}}]=
		[\hhat{\alpha}_{s}\hhat{\varpi}_{j}\Idfin{\cal{O}}]
\end{math}
y
\begin{align*}
	% T_{\frak{p}}([\hhat{\alpha}_{s}\Idfin{\cal{O}}]) \,=\, &
	T_{\frak{p}}([I_{s}]) & \,=\,
		\sum_{j}\,[\hhat{\alpha}_{s}\hhat{\varpi}_{j}\Idfin{\cal{O}}]
	\text{ .}
\end{align*}
%

Notemos que $\hhat{\pi}$ es un elemento del orden $\Adfin{\cal{O}}$ en los
ad\`{e}les finitos de $B$ y, por lo tanto,
$\hhat{\varpi}_{j}\in\Adfin{\cal{O}}$ tambi\'{e}n. Ahora bien, la coclase
$\hhat{\alpha}_{s}\hhat{\varpi}_{j}\Idfin{\cal{O}}$ se corresponde con un ideal
$I$ de $B$ tal que:
\begin{align*}
	\Oder(I)\,=\,\cal{O}\text{ ,} & \quad
		I\,\subset\, I_{s} \quad\text{y}\quad
		\nrd(I)\,=\,\frak{p}\cdot\nrd(I_{s})
	\text{ .}
\end{align*}
%
Rec\'{\i}procamente, se verifica localmente que, si $I\subset I_{s}$, entonces
la coclase correspondiente en $\Idfin{B}/\Idfin{\cal{O}}$ es de la forma
$\hhat{\alpha}\Idfin{\cal{O}}$ con
$\hhat{\alpha}\Adfin{\cal{O}}\subset\hhat{\alpha}_{s}\Adfin{\cal{O}}$ y que, si
$\nrd(I)=\frak{p}\cdot\nrd(I_{s})$, entonces
$\hhat{\alpha}\Adfin{\cal{O}}=\hhat{\varpi}\hhat{\alpha}_{s}\Adfin{\cal{O}}$
con $\hhat{\varpi}\in\hhat{\alpha}_{s}\Adfin{\cal{O}}\hhat{\alpha}_{s}^{-1}$
y $\nrd(\hhat{\varpi})=\hhat{p}$. En particular,
$\hhat{\alpha}\Adfin{\cal{O}}=\hhat{\alpha}_{s}\hhat{\varpi}'\Adfin{\cal{O}}$
para cierto $\hhat{\varpi}'\in\Adfin{\cal{O}}$ tal que
$\nrd(\hhat{\varpi}')=\hhat{p}$. En definitiva, deducimos que
\begin{align*}
	T_{\frak{p}}([I]) & \,=\,
	% \sum_{t}\,\#\left\lbrace I'\in\mathscr{T}_{\frak{p}}(I)\,:\,
	% [I']=[I_{t}]\right\rbrace [I_{t}] \\
	% \,=\, &
	\sum_{I'\in\mathscr{T}_{\frak{p}}(I)}\,[I']\text{ , donde} \\
	\mathscr{T}_{\frak{p}}(I) & \,=\,
		\Big\{ I'\in\ideales{\cal{O}}\,:\,
			I'\subset I,\,\nrd(I')=\frak{p}\cdot\nrd(I)\Big\}
	\text{ .}
\end{align*}
%

Si $\frak{p},\frak{q}$ son ideales primos distintos en $\oka{F}$,
$T_{\frak{p}}T_{\frak{q}}=T_{\frak{q}}T_{\frak{p}}$. Pero, en general, no son
autoadjuntos respecto del producto interno definido en
$\modular[B]{(2,\,\dots,\,2)}{\frak{N}}$.
\begin{align*}
	\langle T_{\frak{p}}([I]),[J]\rangle & \,=\,
		\#\Big\{ I'\in\mathscr{T}_{\frak{p}}(I)\,:\,
			[I']=[J]\Big\}\cdot w_{J} \\
	& \,=\,\left|\Big\{ b\in B^{\times}\,:\,
		bJ\in\mathscr{T}_{\frak{p}}(I)\Big\}/\Gamma_{J}\right|\cdot
		\left|\Gamma_{J}/\oka{F}^{\times}\right| \\
	& \,=\, \left|\Big\{ b\in B^{\times}\,:\,bJ\in
		\mathscr{T}_{\frak{p}}(I)\Big\}/\oka{F}^{\times}\right|
	\text{ .}
\end{align*}
%
Dado que la norma reducida satisface $\nrd(I_{v})=\nrd(I)_{v}$ para todo ideal
$I$ y todo lugar $v$, los conjuntos $\mathscr{T}_{\frak{p}}(I)$ quedan
determinados localmente. De esto se deduce que, dados dos ideales
$I,I'\subset B$, $I'\in\mathscr{T}_{\frak{p}}(I)$ si y s\'{o}lo si
$\frak{p}I\in\mathscr{T}_{\frak{p}}(I')$. En nuestro caso, si
$b\in B^{\times}$, $bJ\in\mathscr{T}_{\frak{p}}(I)$ si y s\'{o}lo si
$\frak{p}b^{-1}I\in\mathscr{T}_{\frak{p}}(J)$. As\'{\i},
\begin{align*}
	\langle T_{\frak{p}}([I]),[J]\rangle & \,=\,
		\left|\Big\{ b\in B^{\times}\,:\,
			\frak{p}b^{-1}I\in\mathscr{T}_{\frak{p}}(J)\Big\}/
			\oka{F}^{\times}\right|
\end{align*}
%
Si el n\'{u}mero de clases de $F$ fuese uno, entonces podr\'{\i}amos reemplazar
el ideal $\frak{p}$ por un elemento global $p\in\oka{F}$ y
\begin{align*}
	\langle T_{\frak{p}}([I]),[J]\rangle & \,=\,
	\left|\Big\{ b'\in B^{\times}\,:\,b'I\in
		\mathscr{T}_{\frak{p}}(J)\Big\}/\oka{F}^{\times}\right| \\
	& \,=\, \langle [I],T_{\frak{p}}([J])\rangle
	\text{ .}
\end{align*}
%
Pero esto no es posible en general. Entonces, lo que se deduce es:
\begin{align*}
	\langle T_{\frak{p}}([I]),[J]\rangle & \,=\,
		\langle [\frak{p}I],T_{\frak{p}}([J])\rangle
	\text{ .}
\end{align*}
%

Dado un ideal primo $\frak{p}\subset\oka{F}$ tal que
$(\frak{p},\frak{D}\frak{N})=1$, podemos asociarle otro operador: si ahora
$\hhat{\pi}$ es el id\`{e}le dado por
\begin{math}
	\begin{bmatrix} p & \\ & p \end{bmatrix}
\end{math}
en el lugar $\frak{p}$ y $1$ en $v\not =\frak{p}$, donde, como antes,
$p\in\oka{F,\frak{p}}$ es un uniformizador, definimos el \emph{operador %
diamante}\index{operador diamante}
como el operador de coclase doble asociado a $\hhat{\pi}$ y lo denotamos
$\diamante{\frak{p}}$. Como $\hhat{\pi}\in\Adfin{\oka{F}}$, pertenece al centro
de $\Idfin{B}$ y, en un elemento de la base, est\'{a} dado por
\begin{align*}
	\diamante{\frak{p}} [I] (\hhat{\alpha}\Idfin{\cal{O}}) & \,=\,
		T_{\hhat{\pi}}[I]
	% [I]
	% \barra{(2,\,\dots,\,2)}{[\Idfin{\cal{O}}\hhat{\pi}\Idfin{\cal{O}}]}
			(\hhat{\alpha}\Idfin{\cal{O}})
		\,=\,[I](\hhat{\alpha}\hhat{\pi}^{-1}\Idfin{\cal{O}})
	\text{ .}
\end{align*}
%
Pero el ideal correspondiente a $\hhat{\alpha}\hhat{\pi}^{-1}\Idfin{\cal{O}}$
pertenece a la clase de $I$, si y s\'{o}lo si el correspondiente a
$\hhat{\alpha}\Idfin{\cal{O}}$ pertenece a $\frak{p}I$.
% Esto se debe a la forma particular que tiene $\hhat{\pi}$: es,
% esencialmente, el id\`{e}le en $\Idfin{B}$ correspondiente a $\frak{p}$.
Es decir,
\begin{align*}
	\diamante{\frak{p}} [I] & \,=\, [\frak{p}I]
	\text{ .}
\end{align*}
%

\begin{obsDiamanteParaDefinidos}\label{obs:diamanteparadefinidos}
	Recordemos que los operadores diamante act\'{u}an trivialmente en
	formas modulares el\'{\i}pticas para los grupos de congruencia
	$\Gamma_{0}(N)$ y notemos que esto no sigue siendo cierto en la
	situaci\'{o}n an\'{a}loga cuando el n\'{u}mero de clases del cuerpo es
	$\#\Class{F}>1$.
\end{obsDiamanteParaDefinidos}

Se puede hacer lo mismo con cada ideal fraccionario $\frak{d}$ de $F$ coprimo
con $\frak{D}\frak{N}$ eligiendo $\hhat{\pi}$ de manera apropiada y se ve que
\begin{align*}
	\diamante{\frak{d}} \diamante{\frak{e}} & \,=\,
		\diamante{\frak{d}\frak{e}}
		\,=\,\diamante{\frak{e}}\diamante{\frak{d}}
	\text{ .}
\end{align*}
%
Podemos deducir varias cosas: los operadores diamante son multiplicativos y, no
s\'{o}lo conmutan entre s\'{\i}, sino tambi\'{e}n conmutan con los operadores
$T_{\frak{p}}$. Adem\'{a}s, si $T^{*}$ denota el adjunto de un operador $T$
respecto del producto interno de Petersson, entonces
\begin{align*}
	\langle\frak{p}\rangle^{*} & \,=\,
		\diamante{\frak{p}^{-1}} \,=\,\diamante{\frak{p}}^{-1}
			\quad\text{y} \\
	T_{\frak{p}}^{*} & \,=\, \diamante{\frak{p}}^{-1}T_{\frak{p}}
	\text{ .}
\end{align*}
%
Queremos extender la definici\'{o}n de $T_{\frak{p}}$ a ideales
\emph{\'{\i}ntegros} de $F$, de manera que sean multiplicativos.
% (por lo menos entre operadores asociados a ideales coprimos).
%
% Podemos definir inductivamente $T_{\frak{p}^{k}}$ copiando la relaci\'{o}n
% que cumplen los operadores de Hecke en los espacios de formas modulares
% el\'{\i}pticas, o bien podemos extender la expresi\'{o}n para
% $T_{\frak{p}}$ en t\'{e}rminos de ideales del \'{a}lgebra:

Sea $I\subset B$ un ideal con $\Oder(I)=\cal{O}$ y sea $\frak{m}\subset\oka{F}$
un ideal \'{\i}ntegro de $F$ coprimo con $\frak{D}\frak{N}$.
% (en realidad, cuando definimos $T_{\frak{p}}$ no hac\'{\i}a falta
% asumir que $\frak{p}$ fuese coprimo con $\frak{N}$ (si con $\frak{D}$,
% though), entonces podemos no asumir que $\frak{m}$ es coprimo con
% $\frak{N}$).
Sea $\mathscr{T}_{\frak{m}}(I)$ el conjunto de ideales
\begin{align*}
	\mathscr{T}_{\frak{m}}(I) & \,=\,
		\Big\{ I'\in\ideales{\cal{O}}\,:\,
			I'\subset I,\,\nrd(I')=\frak{m}\cdot\nrd(I)\Big\}
	\text{ ,}
\end{align*}
%
y definimos el operador $T_{\frak{m}}$ usando la expresi\'{o}n global para los
operadores de Hecke en los elementos de la base:
\begin{align*}
	T_{\frak{m}}([I]) & \,=\,
		\sum_{I'\in\mathscr{T}_{\frak{m}}(I)}\,[I']
	\text{ .}
\end{align*}
%
Se verifica que $T_{\frak{m}}T_{\frak{m}'}=T_{\frak{m}\frak{m}'}$ para ideales
$\frak{m}$ y $\frak{m}'$ coprimos y que, por lo tanto, conmutan.
% Cuando $\frak{m}=\frak{p}$ un ideal primo, la equivalencia entre esta
% expresi\'{o}n para $T_{\frak{p}}$ y la expresi\'{o}n como operador
% de coclase doble depend\'{\i}a de que el id\`{e}le $\hhat{\pi}$ y su
% conjugado $\lconj{\hhat{\pi}}$ verificaran
% $\hhat{u}\lconj{\hhat{\pi}}\hhat{u}^{-1}=\hhat{\pi}$ para alguna
% unidad $\hhat{u}\in\Idfin{\cal{O}}$. Esto sigue siendo cierto para
% $\frak{m}$, si se elige un id\`{e}le adecuado $\hhat{\mu}$.

\begin{propoHeckeParaDefinidosRecursiva}\label{thm:heckeparadefinidosrecursiva}
	El operador $T_{1}$ act\'{u}a como la identidad de
	$\spitz[B]{(2,\,\dots,\,2)}{\frak{N}}$. Sea $\frak{p}\subset\oka{F}$ un
	ideal primo y sea $k\geq 1$. Entonces
	\begin{equation}
		\label{eq:heckeparadefinidosrecursiva}
		T_{\frak{p}^{k+2}} \,=\,
			T_{\frak{p}^{k+1}}T_{\frak{p}} \,-\,\norma(\frak{p})
			\diamante{\frak{p}} T_{\frak{p}^{k}}
		\text{ .}
	\end{equation}
	%
\end{propoHeckeParaDefinidosRecursiva}

\begin{proof}[Demostraci\'{o}n]
	Demostraremos \eqref{eq:heckeparadefinidosrecursiva} localmente. Es
	decir, sea $F$ un cuerpo local, sea $\frak{o}$ su anillo de enteros,
	sea $p\in\frak{o}$ un generador del ideal maximal, sea $B/F$ el
	\'{a}lgebra de matrices $\MM_{2\times 2}(F)$ y sea $\cal{O}$ el orden
	maximal
	\begin{math}
		\begin{bmatrix} \frak{o} & \frak{o} \\
		\frak{o} & \frak{o} \end{bmatrix}
	\end{math}
	de $B$. Entonces afirmamos que
	\begin{math}
		T_{p^{k+2}}=T_{p^{k+1}}T_{p}-\norma(p\frak{o})
			\diamante{p} T_{p^{k}}
	\end{math}~.

	Por un lado,
	\begin{align*}
		T_{p^{k+2}}([I]) & \,=\,
			\sum_{I'\in\mathscr{T}_{p^{k+2}}(I)}\,[I']
						\quad\text{y} \\
		T_{p^{k+1}}\circ T_{p}([I]) & \,=\,
			\sum_{I'\in\mathscr{T}_{p^{k+1}}(I)}\,
			\sum_{J\in\mathscr{T}_{p}(I')}\,[J]
		\text{ .}
	\end{align*}
	%
	Recordemos que, sobre un cuerpo local, los ideales $I\subset B$ con
	$\Oder(I)=\cal{O}$ son todos principales. Si fijamos un ideal $I$ y
	llamamos $\cal{O}'=\Oizq(I)$ a su orden a izquierda,
	\begin{align*}
		T_{p^{k+2}}([I]) & \,=\,\sum_{
			\begin{smallmatrix}
				\omega\in\cal{O}'/\cal{O}'^{\times} \\
				\nrd(\omega)=p^{k+2}
			\end{smallmatrix}
			}\,[\omega I] \\
		T_{p^{k+1}}\circ T_{p}([I]) & \,=\,\sum_{
			\begin{smallmatrix}
				\omega_{0}\in\cal{O}'/\cal{O}'^{\times} \\
				\nrd(\omega_{0})=p
			\end{smallmatrix}
			}\,\sum_{
			\begin{smallmatrix}
				\omega_{1}\in\omega_{0}\cal{O}'
				\omega_{0}^{-1}/\omega_{0}
				\cal{O}'^{\times}\omega_{0}^{-1} \\
				\nrd(\omega_{1})=p^{k+1}
			\end{smallmatrix}
			}\,[\omega_{1}\omega_{0}I]
		\text{ .}
	\end{align*}
	%
	Cada ideal $J=\omega I$ que aparece en la sumatoria de $T_{p^{k+2}}$
	aparece en la sumatoria de $T_{p^{k+1}}T_{p}$ y, como
	$\nrd(\omega_{1}\omega_{0})=\nrd(\omega_{1})\nrd(\omega_{0})$,
	todos los ideales $\omega_{1}\omega_{0}I$ que aparecen en la sumatoria
	de $T_{p^{k+1}}T_{p}$ aparecen en la sumatoria de $T_{p^{k+2}}$. Pero
	cada $J=\omega I$ con $\nrd(\omega)=p^{k+2}$ es igual a
	$(\omega\omega_{0}^{-1})\omega_{0} I$ para cada $\omega_{0}\in\cal{O}'$
	con $\nrd(\omega_{0})=p$. Entonces cada $J$ que aparece en
	$T_{p^{k+2}}$ aparece $\norma(p\frak{o})+1$ veces en
	$T_{p^{k+1}}T_{p}$, pues un sistema de representantes de los elementos
	en $\cal{O}'$ de norma reducida $p$ m\'{o}dulo unidades en $\cal{O}'$
	est\'{a} en biyecci\'{o}n con un sistema de representantes
	$\{\pi_{i}\}_{i}\subset\cal{O}$ tal que
	\begin{math}
		\cal{O}^{\times}\begin{bmatrix} p & \\ & 1 \end{bmatrix}
			\cal{O}^{\times}=
			\bigsqcup_{i}\,\cal{O}^{\times}\pi_{i}
	\end{math}~,
	y \'{e}stos son precisamente $\norma(p\frak{o})+1$: est\'{a}n dados,
	por ejemplo, por
	\begin{align*}
		\begin{bmatrix} 1 & j \\ & p \end{bmatrix}
			% \quad\text{con } j\in\frak{o}/p\frak{o}\text{ y} \\
			& \quad\text{y}\quad
		\begin{bmatrix} p & \\ & 1 \end{bmatrix}
			\text{ ,}
	\end{align*}
	%
	donde $j$ recorre un sistema de representantes de $\frak{o}/p\frak{o}$.
\end{proof}

% Evaluando la expresi\'{o}n anterior en un elemento $f=[I]$
% %=[\hhat{\alpha}_{t_{0}}\Idfin{\cal{O}}]$
% de la base de $\modular[B]{(2,\,\dots,\,2)}{\frak{N}}$, resulta
% \begin{align*}
	% % [\hhat{\alpha}_{t_{0}}\Idfin{\cal{O}}]
	% T_{\frak{p}}([I])
	% % \barra{(2,\,\dots,\,2)}{[\Idfin{\cal{O}}\hhat{\pi}\Idfin{\cal{O}}]}
	% \,=\, & \sum_{t}\,\#\left\lbrace I'\in\ideales{\cal{O}}\,:\,
	% I'\subset I,\nrd(I')=\frak{p}\cdot\nrd(I),\,
	% [I']=[I_{t}]
	% \right\rbrace [I_{t}]
	% \text{ ,}
% \end{align*}
% %
% %
% La cuenta: sea $\gamma\in B^{\times}$ tal que
% $\gamma (\hhat{\alpha}_{t}\hhat{\pi}_{i}^{-1}\Adfin{\cal{O}}\cap B)=%
% \hhat{\alpha}_{t_{0}}\Adfin{\cal{O}}\cap B$. Existe
% $\hhat{u}\in\Idfin{\cal{O}}$ tal que
% $\gamma\hhat{\alpha}_{t}\hhat{\pi}_{i}^{-1}=\hhat{\alpha}_{t_{0}}\hhat{u}$.
% Entonces
% \begin{align*}
	% (\hhat{\alpha}_{t}\hhat{\pi}_{i}^{-1}\Adfin{\cal{O}}
	% \hhat{\alpha}_{t_{0}})\,\cap\, B \,=\, &
	% \gamma^{-1}\cal{O}_{\hhat{\alpha}_{t_{0}}}\quad\text{y} \\
	% \cal{O}_{\hhat{\alpha}_{t_{0}}}\gamma \,=\, &
	% (\hhat{\alpha}_{t_{0}}\Adfin{\cal{O}}
	% \hhat{\pi}_{i}\hhat{\alpha}_{t}^{-1}) \,\cap\, B \\
	% \,=\, & (\hhat{\alpha}_{t_{0}}\Adfin{\cal{O}}\cap B)
	% (\hhat{\alpha}_{t}\Adfin{\cal{O}}\cap B)^{-1}
	% \cal{O}_{\hhat{\alpha}_{t}}
	% (\hhat{\alpha}_{t}\hhat{\pi}_{i}\hhat{\alpha}_{t}^{-1}) \\
	% \,=\, & I_{t_{0}}I_{t}^{-1}J\,\subset\,I_{t_{0}}I_{t}^{-1}
	% \text{ ,}
% \end{align*}
% %
% donde $J:=\cal{O}_{\hhat{\alpha}_{t}}%
% (\hhat{\alpha}_{t}\hhat{\pi}_{i}\hhat{\alpha}_{t}^{-1})$. Finalmente,
% \begin{align*}
	% \gamma J\gamma^{-1}I_{t_{0}} \,=\, & \gamma I_{t}
	% \qquad\text{(todos los ideales de la clase $[I_{t}]$ %
	% de norma $\frak{p}\cdot\nrd(I_{t_{0}})$).} \\
	% \hhat{\pi}_{i}\hhat{\alpha}_{t}^{-1}\gamma^{-1}\hhat{\alpha}_{t_{0}}
	% \,\in\, & \Idfin{\cal{O}}\text{ ,} \\
	% \Oder(\gamma J\gamma^{-1}) \,=\, &
	% (\hhat{\alpha}_{t}\hhat{\pi}_{i}%
	% \hhat{\alpha}_{t}^{-1}\gamma^{-1})^{-1}
	% \cal{O}_{\hhat{\alpha}_{t}}
	% (\hhat{\alpha}_{t}\hhat{\pi}_{i}\hhat{\alpha}_{t}^{-1}\gamma^{-1}) \\
	% \,=\, & \cal{O}_{\hhat{\alpha}_{t_{0}}}\quad\text{y} \\
	% \gamma J\gamma^{-1}I_{t_{0}} \,=\, &
	% \gamma\cal{O}_{\hhat{\alpha}_{t}}\hhat{\alpha}_{t}
	% \hhat{\pi}_{i}\hhat{\alpha}_{t}^{-1}\gamma^{-1}
	% \qquad\text{(y contenidos en $I_{t_{0}}$).}
% \end{align*}
% %
% o, equivalentemente,
% \begin{align*}
	% T_{\frak{p}}([I]) \,=\, &
	% \sum_{I'\in\mathscr{T}_{\frak{p}}(I)}\,[I']\quad\text{, donde} \\
	% \mathscr{T}_{\frak{p}}(I) \,=\, &
	% \left\lbrace I'\in\ideales{\cal{O}}\,:\,
	% I'\subset I,\,\nrd(I')\,\frak{p}\cdot\nrd(I)
	% \right\rbrace
	% \text{ .}
% \end{align*}
% %
% Dado que la norma reducida satisface $\nrd(I_{v})=\nrd(I)_{v}$ para
% todo ideal $I$ y todo lugar finito $v$, los conjuntos
% $\mathscr{T}_{\frak{p}}(I)$ quedan determinados localmente.
% % En un \'{a}lgebra de cuaterniones sobre un cuerpo local, todo ideal de un
% % orden de Eichler del \'{a}lgebra es principal.
% De esto se deduce que, dados dos ideales $I',I\subset B$,
% $I'\in\mathscr{T}_{\frak{p}}(I)$ si y s\'{o}lo si
% $\frak{p}I\in\mathscr{T}_{\frak{p}}(I')$. Esto tiene como consecuencia
% que los operadores de Hecke $T_{\frak{p}}$ son autoadjuntos respecto del
% producto interno de Petersson:
% \begin{align*}
	% \langle T_{\frak{p}}([I]),[J]\rangle \,=\, &
	% \#\left\lbrace I'\in\mathscr{T}_{\frak{p}}(I)\,:\,
	% [I']=[J]\right\rbrace
	% \cdot w_{J} \\
	% \,=\, &	\#\left(\left\lbrace b\in B^{\times}\,:\,
	% bJ\in\mathscr{T}_{\frak{p}}(I)\right\rbrace/\oka{F}^{\times}\right)
	% \text{ .}
% \end{align*}
% %

% 
