\newtheorem*{teoClassificationSurUnCorpsLocal}{Teorema}
\newtheorem*{teoClassificationSurUnCorpsGlobal}{Teorema}
\newtheorem*{propoMaxOrdInvertible}{Proposici\'{o}n}
\newtheorem*{propoInvertibleImplicaPrincipal}{Proposici\'{o}n}
\newtheorem*{propoLatticesSplitAlg}{Proposici\'{o}n}
\newtheorem*{propoLatticesSplitAlgCont}{Proposici\'{o}n}
\newtheorem*{teoMaxOrdsSplitAlg}{Teorema}
\newtheorem*{propoOrdEichlerSplitAlg}{Proposici\'{o}n}

\theoremstyle{remark}
\newtheorem*{obsConjugadosEInversos}{Observaci\'{o}n}

En esta secci\'{o}n repasamos algunos resultados de la teor\'{\i}a de
\'{a}lgebras de cuaterniones. Tomamos como referencia [Vign\'{e}ras].

Sea $K$ un cuerpo. Un \emph{\'{a}lgebra de cuaterniones sobre $K$} es un
\'{a}lgebra $B$ central de dimensi\'{o}n cuatro sobre $K$ tal que existen
$L\subset B$ un \'{a}lgebra separable de dimensi\'{o}n dos sobre $K$ y
$\theta\in K^{\times}$ de manera que $B=L\oplus uL$, donde $u\in B$ satisface
\begin{align*}
 u^{2} \,=\, & \theta\quad\text{y} \\
 ux \,=\, & \conj{x}u
\end{align*}
%
para todo $x\in L$, donde $x\mapsto \conj{x}$ es el $K$-automorfismo
no trivial de $L$. Equivalentemente, $B/K$ es un \'{a}lgebra central simple
de dimensi\'{o}n cuatro.

Manteniendo la notaci\'{o}n del p\'{a}rrafo anterior, la
\emph{conjugaci\'{o}n} en $B$ se define como el $K$-morfismo de $B$ que
extiende el automorfismo de $L$ determinado por
\begin{align*}
 \conj{\conj{b}} \,=\, & b\text{ ,} \\
 \conj{bb_{1}} \,=\, & \conj{b_{1}}\conj{b}\quad\text{y} \\
 \conj{u} \,=\, & -u\text{ .}
\end{align*}
%
para $b\in B$. Definimos la \emph{traza reducida} y la \emph{norma reducida}
de $b$ por, respectivamente,
\begin{align*}
 \trd(b)\,=\, & b+\conj{b}\text{ ,}\\
 \nrd(b)\,=\, & b\conj{b}\text{ .}
\end{align*}
%
Se verifica que $\trd(b),\nrd(b)\in K$ para todo $b$ y que $b$ es
soluci\'{o}n del polinomio $X^{2}-\trd(b)X+\nrd(b)\in K[X]$, denominado
\emph{polinomio minimal (reducido) de $b$}. La norma reducida es
multiplicativa y un elemento $b\in B$ es invertible, si y s\'{o}lo si
$\nrd(b)\not =0$. En particular, la norma reducida induce un morfismo de
grupos multiplicativos
\begin{align*}
 \nrd\,:\, & B^{\times}\rightarrow K^{\times}
 \text{ .}
\end{align*}
%
Adem\'{a}s,
\begin{align*}
 \conj{b} \,=\, & \trd(b)\,-\,b\quad\text{y}\\
 b^{-1} \,=\, & \conj{b}/\nrd(b)\text{ .}
\end{align*}
%
La traza reducida es una transformaci\'{o}n $K$-lineal $\trd :\,B\rightarrow K$
y la aplicaci\'{o}n $(b,b_{1})\mapsto \trd(bb_{1})$ es una forma bilineal
no degenerada.

\paragraph{Ejemplo}
El \'{a}lgebra de matrices $B=\MM_{2\times 2}(K)$ con coeficientes en un
cuerpo $K$ es un \'{a}lgebra de cuaterniones sobre $K$. El cuerpo $K$ se
identifica con
\begin{math}
 I\cdot K=\left\lbrace \begin{bmatrix} a & \\ & a \end{bmatrix}\,:\,
 a\in K\right\rbrace
\end{math}~.
Elegimos $\alpha=\begin{bmatrix} a_{1} & \\ & a_{2} \end{bmatrix}$ con
$a_{1}\not = a_{2}$ y definimos $L:=K(\alpha)$. El $K$-automorfismo no trivial
de $L$ est\'{a} dado por el automorfismo interior $x\mapsto uxu^{-1}$,
donde $u=\begin{bmatrix} & 1 \\ 1 & \end{bmatrix}$. Eligiendo $L$ y $u$
de esta manera, vemos que $B$ es un \'{a}lgebra de cuaterniones. Si
$\gamma=\begin{bmatrix} a & b \\ c & d \end{bmatrix}\in\MM_{2\times 2}(K)$,
entonces
\begin{align*}
 \conj{\gamma} \,=\, & \begin{bmatrix} d & -b \\ -c & a \end{bmatrix} \\
 \trd(\gamma) \,=\, & a\,+\,d \\
 \nrd(\gamma) \,=\, & ad\,-\,bc\text{ .}
\end{align*}
%
En particular, la traza y la norma reducidas est\'{a}n dadas por la traza
y el determinante usuales.


Si $K$ es un cuerpo contenido en otro cuerpo $K'$ y $B$ es un \'{a}lgebra
de cuaterniones sobre $K$, el producto tensorial (sobre $K$)
$K'\otimes B$ es un \'{a}lgebra de cuaterniones sobre el cuerpo m\'{a}s
grande que se obtiene por extensi\'{o}n de escalares.

Si la caracter\'{\i}stica del cuerpo $K$ es distinta de $2$, entonces
el \'{a}lgebra $B$ se puede definir como la $K$-\'{a}lgebra con base
$\{1,i,j,k=ij\}$, donde los elementos $i,j\in B$ verifican
\begin{align*}
 i^{2} \,=\, & a \text{ ,} \\
 j^{2} \,=\, & b \quad\text{y} \\
 ji \,=\, & -ij
\end{align*}
%
con $a,b\in K$. Recuperamos la definici\'{o}n anterior tomando
$L=K(i)$, $u=j$ y $\theta=b$. Cuando un \'{a}lgebra de cuaterniones
es definida de esta manera, escribimos $B=(a,b)_{K}$.
Si $a$ es un cuadrado en $K^{\times}$, queda $B\simeq\MM_{2\times 2}(K)$.
Si no, tomamos $L=K(\alpha)$ con $\alpha^{2}=a$ y $(a,b)_{K}$ es
una sub\'{a}lgebra de $(a,b)_{L}=L\otimes (a,b)_{K}$. Entonces podemos
describir a $(a,b)_{K}$ de la siguiente manera:
\begin{align*}
 (a,b)_{K} \,\simeq\, & \left\lbrace \begin{bmatrix} u & v \\
					b\conj{v} & \conj{u} \end{bmatrix}
	\,:\, u,v\in L\right\rbrace
	\text{ .}
\end{align*}
%

Si $x=x_{0}+x_{1}i+x_{2}j+x_{3}k\in B$, entonces
\begin{align*}
 \conj{x} \,=\, & x_{0} - x_{1} i - x_{2} j - x_{3} k \text{ ,} \\
 \trd(x) \,=\, & 2x_{0} \quad\text{y} \\
 \nrd(x) \,=\, & x_{0}^{2} - a x_{1}^{2} - b x_{2}^{2} + ab x_{3}^{2}
 \text{ .}
\end{align*}
%

\paragraph{Ejemplo}
Los cuaterniones de Hamilton, $\bb{H}$, constituyen un \'{a}lgebra de
cuaterniones sobre $\bb{R}$: $\bb{H}\simeq (-1,-1)_{\bb{R}}$. Sobre $\bb{C}$
se obtiene una representaci\'{o}n como sub\'{a}lgebra de un \'{a}lgebra de
marices:
\begin{align*}
 \bb{H}\,\simeq\, & \left\lbrace \begin{bmatrix} z & w \\
					-\conj{w} & \conj{z} \end{bmatrix}
	\,:\,z,w\in\bb{C}\right\rbrace\text{ .}
\end{align*}
%

\subsection{Clasificaci\'{o}n de las \'{a}lgebras de cuaterniones}
En general, un \'{a}lgebra central simple de dimensi\'{o}n $4$ sobre
un cuerpo $K$, o bien es isomorfa a $\MM_{2\times 2}(K)$, o bien es de
divisi\'{o}n.

Dado que todo $(\bb{C}^{\times})^{2}=\bb{C}^{\times}$, toda \'{a}lgebra de
cuaterniones sobre $\bb{C}$ es isomorfa al \'{a}lgebra de matrices
$\MM_{2\times 2}(\bb{C})$. Sobre $\bb{R}$ hay s\'{o}lo dos clases de
isomorfismo de \'{a}lgebras de cuaterniones: $\bb{H}$ y
$\MM_{2\times 2}(\bb{R})$. Notemos que $\bb{H}^{1}$, el subgrupo de unidades
de norma reducida $1$ en $\bb{H}$, es compacto, mientras que
el grupo correspondiente en matrices, $\SL_{2}(\bb{R})$, no lo es.

Sobre los cuerpos locales $\bb{Q}_{p}$, como en el caso real, hay
exactamente dos clases de isomorfismo de \'{a}lgebras de cuaterniones:
una correspondiente a matrices y otra correspondiente a un \'{a}lgebra de
divisi\'{o}n sobre $\bb{Q}_{p}$. Esto es cierto bien en general:
un \emph{cuerpo local} es un cuerpo (conmutativo) localmente compacto
(y no discreto), es decir, $\bb{R}$, $\bb{C}$ o una extensi\'{o}n
finita de $\bb{Q}_{p}$ (si la caracter\'{\i}stica es $0$), o $\bb{F}_{q}[[T]]$,
si la caracter\'{\i}stica es positiva.

\begin{teoClassificationSurUnCorpsLocal}[Clasificaci\'{o}n local]
 Sea $K$ un cuerpo local distinto de $\bb{C}$. Entonces existe una \'{u}nica
 \'{a}lgebra de cuaterniones de divisi\'{o}n, salvo isomorfismo.
\end{teoClassificationSurUnCorpsLocal}

En adelante, consideraremos \'{a}lgebras de cuaterniones sobre un cuerpo
global, m\'{a}s precisamente, sobre una extensi\'{o}n finita de $\bb{Q}$.
Sea $K/\bb{Q}$ un cuerpo de n\'{u}meros, sea $v$ un lugar finito o infinito
de $K$ y sea $K_{v}$ la completaci\'{o}n correspondiente.
Decimos que un \'{a}lgebra de cuaterniones $B$ sobre $K$ \emph{ramifica}
en $v$, si $B_{v}=B\otimes K_{v}$ es un \'{a}lgebra de divis\'{o}n.
Sea $\mathrm{Ram}(B)$ el \emph{conjunto de ramificaci\'{o}n de $B$},
conjunto de lugares de $K$ en donde $B$ ramifica.

\begin{teoClassificationSurUnCorpsGlobal}[Clasificaci\'{o}n global]
 El cardinal $\#\mathrm{Ram}(B)$ es par. Si $S$ es un subconjunto de lugares
 de $K$ de cardinal par, existe una \'{u}nica \'{a}lgebra de cuaterniones
 cuyo conjunto de ramificaci\'{o}n es $S$, salvo isomorfismo.
\end{teoClassificationSurUnCorpsGlobal}

En particular, el teorema de clasificaci\'{o}n nos dice que dos \'{a}lgebras
de cuaterniones sobre un cuerpo de n\'{u}meros son isomorfas, si y s\'{o}lo
si son isomorfas localmente.

\subsection{\'{O}rdenes e ideales}
Sean $R$ un anillo de Dedekind, $K$ su cuerpo de fracciones y $B$ un
\'{a}lgebra de cuaterniones sobre $K$.

Dado un espacio vectorial $V$ sobre $K$, un \emph{ret\'{\i}culo} en $V$
un $R$-m\'{o}dulo $L\subset V$ finitamente generado. Si $L$ contiene una
base de $V$, es decir, si $L\otimes_{R}K=V$, decimos que $L$ es un
ret\'{\i}culo \emph{completo}. (Si es necesario enfatizar el rol del anillo
$R$ diremos que $L$ es un $R$-lattice o un ret\'{\i}culo sobre $R$).

Un elemento $b\in B$ es \emph{entero} (o \emph{entero sobre $R$}),
si el $R$-subm\'{o}dulo $R[b]\subset B$ es un ret\'{\i}culo en $B$.
Un elemento $b$ es entero, si y s\'{o}lo si su traza reducida y su norma
reducida pertenecen a $R$. A diferencia de lo que ocurre en un cuerpo de
n\'{u}meros, el conjunto de enteros de $B$ no constituye un anillo.

Un \emph{ideal} ($R$-ideal) de $B$ es un ret\'{\i}culo (sobre $R$) completo
en $B$. Un \emph{orden} ($R$-orden) de $B$ es un ideal que es un anillo
(con unidad), o, equivalentemente, un anillo de enteros $\cal{O}\subset B$
que contiene a $R$ y tal que $\cal{O}K=B$. Un \emph{orden maximal} es un
orden que no est\'{a} contenido propiamente en otro orden. Un
\emph{orden de Eichler} es una intersecci\'{o}n de dos \'{o}rdenes maximales.

Dado $I$ un ideal de $B$, le asociamos los conjuntos
\begin{align*}
	\Oizq(I) \,=\, & \left\lbrace b\in B\,:\,bI\subset I\right\rbrace
	\quad\text{y}\\
	\Oder(I) \,=\, & \left\lbrace b\in B\,:\,Ib\subset I\right\rbrace
\end{align*}
%
denominados, respectivamente, \emph{orden a izquierda} y \emph{orden a derecha}
de $I$. Son \'{o}rdenes de $B$. Un ideal se dice \emph{\'{\i}ntegro}, si sus
elementos son \'{\i}ntegros. Esta condici\'{o}n es equivalente a
$I\subset\Oizq(I)$ (o, lo que es lo mismo, a $I\subset\Oder(I)$).
La \emph{norma reducida de $I$}, denotada $\nrd(I)$ es el ideal fraccionario
de $R$ generado por $\{\nrd(x)\,:\,x\in I\}$.

Definimos el \emph{producto de dos ideales $I,J$} como el conjunto de
sumas finitas de elementos de la forma $ab$, con $a\in I$ y $b\in J$. Es
un ideal de $B$ y lo denotamos $IJ$. Pero, a diferencia de lo que ocurre en
un cuerpo de n\'{u}meros, si multiplicamos dos ideales cuyo orden, digamos,
a izquierda es $\cal{O}$, no es cierto que de esta manera se obtiene un
ideal cuyo orden a izquierda es $\cal{O}$.

Si $I$ es un ideal de $B$, definimos el \emph{inverso} de $I$ como
\begin{align*}
 I^{-1} \,=\, & \left\lbrace b\in B\,:\,
	IbI\subset I\right\rbrace
	\text{ .}
\end{align*}
%
El inverso $I^{-1}$ es un ideal, pero no es cierto que $II^{-1}$ o que
$I^{-1}I$ sean iguales a alguno de los \'{o}rdenes de $I$, como ocurre
con los ideales fraccionarios del cuerpo $K$, por ejemplo. S\'{o}lo
contamos con las inclusiones
\begin{align*}
 II^{-1}\,\subset\, & \Oizq(I)\,\subset\,\Oder(I^{-1})\text{ ,}\\
 I^{-1}I\,\subset\, & \Oder(I)\,\subset\,\Oizq(I^{-1})\text{ .}
\end{align*}
%
Decimos que \emph{$I$ es invertible}, si todas estas inclusiones son
igualdades.

Si $b\in B$ e $I\subset B$ es un ideal, $b\Oder(I)=\Oizq(I)b$. El ideal $I$
se dice \emph{principal}, si es de la forma $\Oizq(I)b$ para alg\'{u}n
elemento $b$. Si $I=\cal{O}b$ es un ideal principal, su orden a izquierda
es $\cal{O}$ y su orden a derecha es el conjugado $b^{-1}\cal{O}b$. Notemos
que $I=\cal{O}b\cal{O}'$. Vale tambi\'{e}n que $I^{-1}$ es un ideal principal
($I^{-1}=b^{-1}\cal{O}$). En particular, todo ideal principal es invertible.
Los ideales principales tambi\'{e}n satisfacen $\nrd(IJ)=\nrd(I)\nrd(J)$.

Dado un ret\'{\i}culo $I\subset B$ (un $R$-subm\'{o}dulo finitamente
generado), el \emph{dual} de $I$, definido como
\begin{align*}
	I^{*} \,:=\, & \left\lbrace b\in B\,:\,\trd(bI)\subset R\right\rbrace
	\text{ ,}
\end{align*}
%
es un ret\'{\i}culo tambi\'{e}n y vale que $(I^{*})^{*}=I$.
Como la traza define una forma bilineal no degenerada, toda $R$-base de
$I$ determina \'{u}nivocamente una \emph{base dual} para $I^{*}$. En
particular, $I^{*}$ es un ideal, si $I$ lo es. Si $\cal{O}$ es un orden en
$B$, definimos su \emph{diferente} como $(\cal{O}^{*})^{-1}$ y su
\emph{discriminante reducido} como $\drd{\cal{O}}:=\nrd((\cal{O}^{*})^{-1})$.
%Si $\cal{O}\subset\cal{O}'$ son \'{o}rdenes de $B$, los ideales diferentes
%satifacen $\drd{\cal{O}}\mid\drd{\cal{O}'}$ y son iguales, si y s\'{o}lo si
%los \'{o}rdenes son iguales.

Dado un ret\'{\i}culo $I$, la presencia de la involuci\'{o}n
$x\mapsto\conj{x}$ nos permite definir su \emph{conjugado} como
$\conj{I}=\{\conj{x}\,:\,x\in I\}$. El conjugado de un ret\'{\i}culo es
un ret\'{\i}culo y la aplicaci\'{o}n $I\mapsto\conj{I}$ satisface
$\lconj{IJ}=\conj{J}\conj{I}$. Si $I$ es un ideal, $\conj{I}$ tambi\'{e}n
lo es y vale que $\Oizq(\conj{I})=\Oder(I)$ y $\Oder(\conj{I})=\Oizq(I)$.
Si $I$ es un ret\'{\i}culo con $1\in I$ y tal que $\trd(I)\subset R$, como
$\conj{x}=\trd(x)-x$, se deduce que $\conj{I}=I$ y, en ese caso, los
\'{o}rdenes a derecha y a izquierda son iguales.

Decimos que \emph{$I$ es compatible con $J$}, o que el producto
$IJ$ es \emph{compatible}, si $\Oder(I)=\Oizq(J)$. Si $IJ$ es un producto
compatible, y, adem\'{a}s, $J$ es invertible, vale que
\begin{align*}
	\Oizq(IJ)\cdot IJ\,\subset\, & IJ\quad\text{y que} \\
	\Oizq(IJ)\cdot IJJ^{-1} \,\subset\, & I\cdot\Oizq(J)
	\,=\,I\cdot\Oder(I)\,=\,I
	\text{ .}
\end{align*}
%
Como siempre vale la inclusi\'{o}n $\Oizq(I)\subset\Oizq(IJ)$, tenemos que
\begin{align*}
	\Oizq(IJ) \,=\, & \Oizq(I)
	\text{ .}
\end{align*}
%
Si suponemos que $I$ tambi\'{e}n es invertible, podemos concluir que
$\Oder(IJ)=\Oder(J)$, tambi\'{e}n. En definitiva, si $\cal{O}$ y $\cal{O}'$
son dos \'{o}rdenes de $B$, si tomamos un producto compatible $IJ$
con $I$ y $J$ invertibles, $\Oizq(I)=\cal{O}$ y $\Oder(J)=\cal{O}'$,
entonces obtenemos un ideal cuyo orden a izquierda es $\cal{O}$ y
cuyo orden a derecha es $\cal{O}'$. Si $\cal{O}=\cal{O}'$, entonces
podemos definir un grupo, el \emph{grupo de $\cal{O}$-ideales bil\'{a}teros %
e invertibles}. Denotamos este grupo por $\ideales{\cal{O}}$. Como
ya mencionamos, los ideales principales son invertibles y el producto de
dos ideales principales es principal, con lo que podemos definir el
\emph{subgrupo de ideales principales} $\ppales{\cal{O}}\leq\ideales{\cal{O}}$.
Estos son los ideales de la forma $\cal{O}b\cal{O}$.
En general, si $\cal{O}\not =\cal{O}'$, no hay estructura de grupo
y obtenemos un \emph{grupoide} considerando todos los conjuntos
$\ideales{\cal{O},\cal{O}'}$ de ideales invertibles cuyo orden a izquierda
es $\cal{O}$ y $\cal{O}'$ a derecha.

\subsection{Clases}
Dos ideales $I,J$ de $B$ se dicen \emph{equivalentes a izquierda}, si
existe $b\in B^{\times}$ tal que $I=bJ$.
Si $\cal{O}$ es un orden de $B$, esto determina una relaci\'{o}n de
equivalencia en el conjunto de ideales cuyo orden \emph{a derecha} es
$\cal{O}$ y dos $\cal{O}$-ideales a derecha son equivalentes a izquierda,
si y s\'{o}lo si son isomorfos como $\cal{O}$-m\'{o}dulos a derecha.
Notemos que si $I$ es un ideal invertible con $\Oder(I)=\cal{O}$, todo
ideal $J=bI$ equivalente a izquierda cn $I$ es invertible, tambi\'{e}n.
En tal caso, decimos que la \emph{clase} correspondiente a $I$ es
\emph{invertible}. El \emph{conjunto de clases a izquierda} del orden
$\cal{O}$, denotado $\lClass{\cal{O}}$, es el conjunto de ideales
$I$ tales que $\Oder(I)=\cal{O}$, m\'{o}dulo la relaci\'{o}n
$I\sim J\Leftrightarrow I=bJ$ para alg\'{u}n $b\in B^{\times}$, y cuya clase
es invertible. An\'{a}logamente, podemos definir las clases a derecha de
$\cal{O}$ (y denotamos el conjunto de dichas clases por $\rClass{\cal{O}}$).
Notemos que $\rClass{\cal{O}}$ y $\lClass{\cal{O}}$ son \emph{a priori}
s\'{o}lo \emph{conjuntos}.
% El conjunto $\lClass{\cal{O}}$ tiene un elemento distinguido: la clase de
% $\cal{O}$.
Como un \'{a}lgebra de cuaterniones cuenta con una involuci\'{o}n
$x\mapsto\conj{x}$, la aplicaci\'{o}n $I\mapsto\conj{I}$ da una biyecci\'{o}n
entre las clases a izquierda y las clases a derecha de un orden $\cal{O}$.

Decimos que dos \'{o}rdenes $\cal{O},\cal{O}'$ \emph{est\'{a}n conectados},
si existe un ideal $I$ de $B$ cuyo orden a izquierda es $\cal{O}$ y cuyo
orden a derecha es $\cal{O}'$ y se dicen \emph{del mismo tipo}, si son
conjugados por un elemento de $B^{\times}$ (o, equivalentemente, si son
isomorfos como $R$-\'{a}lgebras). Los \'{o}rdenes $\cal{O}$ y $\cal{O}'$ son
del mismo tipo, si y s\'{o}lo si est\'{a}n conectados por un ideal principal.
Un isomorfismo $\cal{O}\simeq\cal{O}'$ (dado por conjugaci\'{o}n) induce una
biyecci\'{o}n $\lClass{\cal{O}}\rightarrow\lClass{\cal{O}'}$.
La existencia de una involuci\'{o}n en $B$ nos permite concluir
que ``estar conectados'' es una relaci\'{o}n de equivalencia entre los
\'{o}rdenes de $B$. El \emph{g\'{e}nero} de $\cal{O}$ es el conjunto de
\'{o}rdenes conectados con $\cal{O}$, es decir, su clase respecto de
esta relaci\'{o}n, lo denotamos $\Genus{\cal{O}}$.
Si $\cal{O}$ y $\cal{O}'$ son \'{o}rdenes conectados
y $J$ es un ideal que los conecta, la aplicaci\'{o}n $I\mapsto IJ$ determina
una biyecci\'{o}n $\lClass{\cal{O}}\rightarrow\lClass{\cal{O}'}$;
su inversa est\'{a} dada por $I'\mapsto I'J^{-1}$.

% Se puede ver que, sobre un cuerpo global, dos \'{o}rdenes est\'{a}n
% conectados, si y s\'{o}lo si son isomorfos \emph{localmente} (pero
% no necesariamente globalmente); dos \'{o}rdenes pueden pertenecer
% al mismo g\'{e}nero pero no ser del mismo tipo.

%%---

\subsection{Grupoide de Brandt}
Describimos la construcci\'{o}n an\'{a}loga al grupo de clases de ideales
de un cuerpo de n\'{u}meros para un \'{a}lgebra de cuaterniones.

Dados $\cal{O},\cal{O}'$ dos \'{o}rdenes de $B$, sea
$\ideales{\cal{O},\cal{O}'}$ el conjunto de ideales invertibles cuyo
orden a izquierda es $\cal{O}$ y cuyo orden a derecha es $\cal{O}'$.
Este conjunto es no vac\'{\i}o precisamente cuando $\cal{O}$ y $\cal{O}'$
pertenecen al mismo g\'{e}nero. El conjunto de ideales invertibles de $B$,
con el producto dado por el producto compatible de ideales y las
unidades los \'{o}rdenes de $B$, tiene estructura de grupoide.
El producto $IJ$ est\'{a} definido \'{u}nicamente cuando el producto de
ideales es compatible. \'{O}rdenes $\cal{O},\cal{O}',\cal{O}''$ pertenecen al
mismo g\'{e}nero, si y s\'{o}lo si los conjuntos
$\ideales{\cal{O},\cal{O}'}$ e $\ideales{\cal{O}',\cal{O}''}$ son no
vac\'{\i}os, y, en tal caso, hay una aplicaci\'{o}n
\begin{align*}
	\ideales{\cal{O},\cal{O}'}\,\times\,\ideales{\cal{O}',\cal{O}''}
	& \,\rightarrow\,\ideales{\cal{O},\cal{O}''} \\
	(I,J) & \,\mapsto\,IJ
	\text{ .}
\end{align*}
%
Entonces, dado un g\'{e}nero $\cal{G}=\Genus{\cal{O}}$, definimos
\begin{align*}
	\Brandt{\cal{G}} \,:=\, & \left\lbrace
	\text{Ideales invertibles } I\text{ de } B\text{ con }
	\Oizq(I),\Oder(I)\in\cal{G}\right\rbrace
	\text{ .}
\end{align*}
%
Este conjunto tiene estructura de grupoide tambi\'{e}n y adem\'{a}s es conexo.

Si $I,J\in\ideales{\cal{O},\cal{O}'}$, entonces $I$ y $J$ son isomorfos
como $\cal{O},\cal{O}'$-bim\'{o}dulos, si y s\'{o}lo si existe
$a\in K^{\times}$ tal que $J=aI$, es decir, son homot\'{e}ticos
(extendiendo escalares %tensorizando con $K$
se ve que todo isomorfismo de bim\'{o}dulos est\'{a} dado por multiplicar
por un elemento de $B$. Necesariamente este elemento est\'{a}en el centro que
es $K$). Esto determina una relaci\'{o}n de equivalencia y cada clase de
homotecia tiene \'{o}rdenes a izquierda y a derecha bien definidos.
De esta manera, las clases m\'{o}dulo homotecias de ideales invertibles de
$B$ tambi\'{e}n constituyen un grupoide con el producto compatible de
(clases de) ideales. Este grupoide admite un morfismo sobreyectivo del
grupoide de los ideales invertibles de $B$, y una componente conexa
$\Brandt{\cal{G}}$ determina un grupoide conexo asociado al g\'{e}nero
$\cal{G}$ formado por las im\'{a}genes de los conjuntos
$\ideales{\cal{O},\cal{O}'}$ con $\cal{O},\cal{O}'\in\cal{G}$.
Definimos
\begin{align*}
 \Pic{\cal{O},\cal{O}'} \,:=\, & \ideales{\cal{O},\cal{O}'}/\sim \\
	\,=\, &
	\left\lbrace [I]\,:\,I\in\ideales{\cal{O},\cal{O}'}\right\rbrace
	\text{ ,}
\end{align*}
%
donde $[I]$ denota la clase de homotecia de $I$. En general, como
$\ideales{\cal{O},\cal{O}'}$ no es un grupo, la suryecci\'{o}n
$\ideales{\cal{O},\cal{O}'}\rightarrow\Pic{\cal{O},\cal{O}'}$ determina un
morfismo
\begin{align*}
	\Brandt{\cal{G}}\,=\, & \bigcup_{\cal{O},\cal{O}'\in\cal{G}}\,
	\ideales{\cal{O},\cal{O}'}
	\,\rightarrow\,\bigcup_{\cal{O},\cal{O}'\in\cal{G}}\,
	\Pic{\cal{O},\cal{O}'}
	\text{ .}
\end{align*}
%
Pero cuando $\cal{O}=\cal{O}'$, la relaci\'{o}n de homotecia est\'{a}
dada por el subgrupo (normal) $\ppales{R}$ de ideales fraccionarios
principales de $R$, que se identifica con los ideales principales de la
forma $\cal{O}a\cal{O}\in\ideales{\cal{O}}$, $a\in K^{\times}$ v\'{\i}a
$aR\mapsto\cal{O}a\cal{O}$. En este caso, tenemos un morfismo de grupos
\begin{align*}
 \ideales{\cal{O}} & \,\rightarrow\,\Pic{\cal{O}}\,:=\,
	\ideales{\cal{O}}/\ppales{R}
	\text{ .}
\end{align*}
%
El grupo $\Pic{\cal{O}}$ es el \emph{grupo de Picard} del orden $\cal{O}$.
Notemos que la inclusi\'{o}n $\ppales{R}\hookrightarrow\ideales{\cal{O}}$
es la restricci\'{o}n de la inclusi\'{o}n de todos los ideales $\ideales{R}$,
no s\'{o}lo los principales, por $\frak{a}\mapsto\cal{O}\frak{a}\cal{O}$.

% Dentro del conjunto $\ideales{\cal{O},\cal{O}'}$ encontramos el subconjunto
% $\ppales{\cal{O},\cal{O}'}=\{\cal{O}b\cal{O}'\,:\,b\in B^{\times}\}$ de
% ideales principales. Si
% \begin{align*}
 % \normalizador{\cal{O},\cal{O}'}\,:=\, &
	% \left\lbrace \alpha\in B^{\times}\,:\,
	% \alpha^{-1}\cal{O}\alpha=\cal{O}'\right\rbrace
	% \text{ ,}
% \end{align*}
% %
% tenemos una suryecci\'{o}n
% $\normalizador{\cal{O},\cal{O}'}\rightarrow\ppales{\cal{O},\cal{O}'}$
% dada por $\alpha\mapsto\cal{O}\alpha\cal{O}'$. Si $\alpha,\alpha_{1}$
% pertenecen a $\normalizador{\cal{O},\cal{O}'}$,
% \begin{align*}
	% \cal{O}\alpha\cal{O}'\,=\,\cal{O}\alpha_{1}\cal{O}'
	% \,\Leftrightarrow\, & \cal{O}\alpha\,=\,\cal{O}\alpha_{1}
	% \,\Leftrightarrow\,\alpha\cal{O}'\,=\,\alpha_{1}\cal{O}' \\
	% \,\Leftrightarrow\, & \alpha_{1}^{-1}\alpha\in\cal{O}^{\times}
	% \,\Leftrightarrow\, \alpha\alpha_{1}^{-1}\in\cal{O}'^{\times}
	% \text{ .}
% \end{align*}
% %
Volvamos al caso especial en que $\cal{O}'=\cal{O}$. Sea
\begin{align*}
 \cal{N}\,:=\,\normalizador[B^{\times}]{\cal{O}} \,=\, &
	\left\lbrace b\in B^{\times}\,:\,b^{-1}\cal{O}b=\cal{O}
	\right\rbrace
\end{align*}
%
el normalizador de $\cal{O}$ en $B^{\times}$. %Tenemos una aplicaci\'{o}n
%sobreyectiva $\normalizador[B^{\times}]{\cal{O}}\rightarrow\ppales{\cal{O}}$
%dada por $b\mapsto\cal{O}b\cal{O}$, de la que se deduce que
Existe una sucesi\'{o}n exacta corta de grupos
\begin{center}
\begin{tikzcd}
	1\arrow{r} & \cal{O}^{\times}\arrow{r} &
	% \normalizador[B^{\times}]{\cal{O}}\arrow{r} &
	\cal{N}\arrow{r} &
	\ppales{\cal{O}}\arrow{r} & 1
\end{tikzcd}\quad\text{,}
\end{center}
%
donde %$\normalizador[B^{\times}]{\cal{O}}=\normalizador{\cal{O},\cal{O}}$,
la flecha de la izquierda es la inclusi\'{o}n y la de la derecha est\'{a}
dada por $b\mapsto\cal{O}b\cal{O}$. Notemos que
$K^{\times}\subset\cal{N}$. Para un elemento
$a\in K^{\times}$, vale que $\cal{O}a\cal{O}=\cal{O}$, si y s\'{o}lo si
$a$ es una unidad del orden $\cal{O}$ y esto equivale a que $a\in R^{\times}$.
Es decir, la imagen de $K^{\times}$ por el morfismo de la derecha es igual
a $\ppales{R}$. En particular, hay un isomorfismo
\begin{align*}
	K^{\times}\backslash \cal{N}/\cal{O}^{\times}
	\,\simeq\, & \ppales{\cal{O}}/\ppales{R}
	\text{ .}
\end{align*}
%

\subsection{\'{O}rdenes e ideales sobre cuerpos locales no arquimedianos}
Dado que estaremos interesados principalmente en \'{a}lgebras de cuaterniones
sobre cuerpos de n\'{u}meros, ser\'{a} fundamental entender las \'{a}lgebras
sobre cuerpos locales. Si $B/K$ es un \'{a}lgebra de cuaterniones
sobre un cuerpo de n\'{u}meros $K$, entender los \'{o}rdenes y los ideales
en las \'{a}lgebras $B_{v}=B\otimes_{K}K_{v}$ nos ayudar\'{a} a entender
los \'{o}rdenes e ideales de $B$. Dividiremos la descripci\'{o}n en dos
casos: $B$ es isomorfa a un \'{a}lgebra de matrices, o bien $B$ es de
divisi\'{o}n. Pero antes hacemos una observaci\'{o}n v\'{a}lida en ambos.

Sabemos que, si un ideal $I$ es principal, entonces es invertible.

\begin{propoInvertibleImplicaPrincipal}
	Sea $B$ un \'{a}lgebra de cuaterniones sobre una extensi\'{o}n finita
	$K$ de $\bb{Q}_{p}$ con anillo de enteros $R$. Sea $I$ un ideal
	invertible de $B$. Entonces $I$ es principal.
\end{propoInvertibleImplicaPrincipal}

% Dados dos ret\'{\i}culos completos $L,L'$ de $B=\MM_{2\times 2}(D)$,
% $D$ una $K$-\'{a}lgebra de divisi\'{o}n, existe $h\in D^{\times}$ tal
% que $hL'\subset L$ [Weil, p.30]. Si $B$ es de matrices, $D=K$ y
% $h\in K^{\times}$; si es de divisi\'{o}n, $B=D$ y s\'{o}lo se tiene
% $h\in B^{\times}$.

\begin{proof}[Demostraci\'{o}n]
	Sea $\cal{O}=\Oizq(I)$. Multiplicando a derecha por un elemento
	$b\in I^{-1}\cap B^{\times}$, $I':=I\cdot b\subset\cal{O}$
	y $\nrd(I')\subset R$.
	% Si $B=\MM_{2\times 2}(K)$, se puede tomar $b\in K^{\times}$.
	Si $1\not\in I'$,
	elegimos $\alpha\in I'\cap B^{\times}$ tal que la
	valuaci\'{o}n de $\nrd(\alpha)$ sea m\'{\i}nima y tomamos
	$J:=I'\cdot\alpha^{-1}$. Como la norma de $\alpha$ era de
	valuaci\'{o}n m\'{\i}nima, $\nrd(J)\subset R$ y $1\in J$.
	La igualdad $\nrd(x+1)=\nrd(x)+\trd(x)+1$ implica que esto es
	equivalente a que $1\in J$ y todo elemento de $J$ es \'{\i}ntegro.
	% Adem\'{a}s, $\Oizq(J)=\cal{O}$.
	% En particular,
	% \begin{align*}
		% J\,\subset\, & JJ^{-1}\,\subset\,\cal{O}\quad\text{y} \\
		% \cal{O}\,\subset\, & J\text{ .}
	% \end{align*}
	% !`No! Esto no tiene por qu\'{e} ser as\'{\i}: $1\in J^{-1}$, sii
	% $JJ\subset J$, sii $J\subset\Oizq(J)=\cal{O}$. !`Y esto \'{u}ltimo
	% es justamente lo que queremos probar! No es cierto, a priori.
	%
	En particular, $\trd(J)\subset R$. Esto implica que $\conj{J}=J$,
	pues $1\in J$ y $R\subset J$. Veamos que dos elementos de $J$
	(anti) conmutan m\'{o}dulo $J$: en primer lugar, si $x,y\in B$
	\begin{align*}
		xy\,+\,yx \,=\, & \trd(y)x\,-\,\trd(x)\conj{y}\,+\,
		\trd(yx)
	\end{align*}
	%
	y, por otro lado, para $x,y\in J$ vale que $\nrd(x),\nrd(y)\in R$,
	entonces
	$\trd(yx)\in R$ si y s\'{o}lo si $\trd(yx)+\nrd(y)+\nrd(x)\in R$,
	pero
	\begin{align*}
		\trd(yx)\,+\,\nrd(y)\,+\,\nrd(x)\,=\, & \nrd(x+\conj{y})
	\end{align*}
	y, como $\conj{J}=J$, $x+\conj{y}\in J$ y su norma es un entero.
	% En definitiva, $J=\cal{O}$ e $I=\cal{O}\alpha b^{-1}$ es principal.
	Ahora bien, como $1\in J$, tenemos inclusiones $J^{t}\subset J^{t+1}$
	para todo $t\geq 0$. Como $J$ es finitamente generado en tanto
	$R$-m\'{o}dulo, para $t$ suficientemente grande, el producto de
	$t+1$ elementos en un conjunto de generadores contiene elementos
	repetidos. Dado que $xy=-yx\,(\modulo J)$, si $x,y\in J$, podemos
	suponer que, en el producto de $t+1$ generadores, un elemento
	aparece repetido s\'{o}lo si alg\'{u}n par de elementos adyacentes
	son iguales (m\'{o}dulo $J^{t}$). Entonces, como
	$x^{2}=\trd(x)x-\nrd(x)$ para todo $x\in B$, deducimos que
	$J^{t+1}=J^{t}$.

	Finalmente, como $J$ es invertible, $J^{t+1}=J^{t}$ implica
	$J=\Oizq(J)=\Oizq(I)=\cal{O}$, de lo que se deduce que
	$I=\cal{O}\alpha b^{-1}$ es principal.
\end{proof}

Es decir, en las \'{a}lgebras de cuaterniones sobre un cuerpo local,
$\ideales{\cal{O},\cal{O}'}=\ppales{\cal{O},\cal{O}'}$ para todo par
de \'{o}rdenes. En particular, dos \'{o}rdenes est\'{a}n conectados,
si y s\'{o}lo si est\'{a}n conectados por un ideal principal, si y
s\'{o}lo si son conjugados.

Podemos deducir las siguientes propiedades de los ideales invertibles%
%de un \'{a}lgebra de cuaterniones sobre un cuerpo local%
. En primer lugar, si $I,J$ son invertibles y el producto $IJ$ es compatible,
entonces $\nrd(IJ)=\nrd(I)\nrd(J)$, pues $I$ y $J$ son principales.
En segundo lugar, si $I=\cal{O}\alpha=\alpha\cal{O}'$,
\begin{align*}
	I\conj{I} \,=\, & \cal{O}\alpha\conj{\alpha}\cal{O}\,=\,
	\nrd(\alpha)\cal{O}\,=\,\nrd(I)\cal{O}\quad\text{y} \\
	\conj{I}I \,=\, & \nrd(I)\cal{O}'
	\text{ .}
\end{align*}
%
El inverso de $I$ est\'{a} dado por $I^{-1}=\nrd(I)^{-1}\conj{I}$, cuando
$I$ es invertible (principal).

\begin{obsConjugadosEInversos}
Si $I$ es un ideal arbitrario de $B$, el ideal $\nrd(I)$ de $R$ es principal.
Si $\alpha\in I$ es tal que $\nrd(\alpha)R=\nrd(I)$, tomamos $J=I\alpha^{-1}$.
Entonces $1\in J$ y $\nrd(J)=R$. El argumento dado en la demostraci\'{o}n
de la proposici\'{o}n ??? muestra que $J^{4}$ es un orden. Adem\'{a}s,
$\conj{J}=J$ implica que $J$ los \'{o}rdenes a izquierda y a derecha de $J$
coinciden y son iguales al orden $\Oizq(I)$. Por otro lado,
\begin{align*}
 (J^{2})^{2} \,=\, & J^{4}
	\,=\,\Oizq(J)\,=\,\Oizq(J^{2}) \,=\, \Oder(J^{2})
	\text{ .}
\end{align*}
%
Entonces $J^{2}$ es invertible, $1\in J^{2}$ y $J^{2}\subset J^{4}$
implica $\nrd(J^{2})\subset R$. De esto se deduce que $J^{2}$ es un orden
que llamamos $\cal{O}_{i}$. Ese orden satisface
\begin{align*}
 \cal{O}_{i} \,=\, & I\conj{I}\nrd(\alpha)^{-1}
	\,=\, I\conj{I}\nrd(I)^{-1}\quad\text{y} \\
	\cal{O}_{i}\,\supset\, & \Oizq(I)
	\text{ .}
\end{align*}
%
De manera an\'{a}loga, hallamos un orden $\cal{O}_{d}$ que cumple
\begin{align*}
 \cal{O}_{d} \,=\, & \conj{I}I\nrd(I)^{-1}\quad\text{y} \\
	\cal{O}_{d} \,\supset\, & \Oder(I)
	\text{ .}
\end{align*}
%
Si $\Oizq(I)=\cal{O}_{i}$, se deduce que $\conj{I}\nrd(I)^{-1}\subset I^{-1}$.
Entonces deben ser iguales, pues la inclusi\'{o}n
$I^{-1}\subset\conj{I}\nrd(I)^{-1}$ es cierta en general.
% se deduce de $I^{-1}I\subset\Oder(I)=\Oizq(\conj{I})$.
De esta igualdad, se deduce que
% $\Oder(I)\supset I^{-1}I=\conj{I}I\nrd(I)=\cal{O}_{d}$ y que
$\Oder(I)=\cal{O}_{d}$, tambi\'{e}n. En particular, $I$ es invertible
si $\cal{O}_{i}$ es igual a su orden a izquierda. De la misma manera,
si $\cal{O}_{d}$ es igual al orden a derecha, entonces el ideal es invertible.
\end{obsConjugadosEInversos}

\begin{propoMaxOrdInvertible}
 Sea $I$ un ideal. Si alguno de $\Oizq(I)$ u $\Oder(I)$ es maximal,
 entonces $I$ es invertible.
\end{propoMaxOrdInvertible}

Fijamos $K$ una extensi\'{o}n finita de $\bb{Q}_{p}$, $R$ su anillo de
enteros, $\pi\in R$ un generador del ideal maximal $P=\pi R$ y $B$ un
\'{a}lgebra de cuaterniones sobre $K$.

\subsubsection{$B\simeq\MM_{2\times 2}(K)$}
Sea $V/K$ un espacio vectorial de dimensi\'{o}n $2$. Eligiendo una base
de $V$, $B$ se identifica con el \'{a}lgebra de endomorfismos de $V$,
$\Endo(V)$.

\begin{propoLatticesSplitAlg}
 \textit{(i)} Los \'{o}rdenes maximales de $\Endo(V)$ son los anillos
 $\Endo_{R}(L)$, donde $L\subset V$ es un ret\'{\i}culo completo
 ($R$-subm\'{o}dulo finitamente generado que contiene una base de $V$
 como $K$-e.v).
 \textit{(ii)} Los ideales de $\Endo(V)$ cuyos \'{o}rdenes a izquierda y a
 derecha son maximales son los ideales $\Endo_{R}(L,M)$, donde $L$ y $M$ son
 ret\'{\i}culos completos de $V$.
 \textit{(iii)} Sean $L,M$ ret\'{\i}culos completos de $V$. Existen elementos
 $f_{1},f_{2}\in L$ tales que $\{f_{1},f_{2}\}$ es una $R$-base de $L$ y
 $\{f_{1}\pi^{a_{1}},f_{2}\pi^{a_{2}}\}$ es una $R$-base de $M$ para
 ciertos enteros $a_{1},a_{2}$ un\'{\i}vocamente determinados.
\end{propoLatticesSplitAlg}

Ver [Weil, Ch. \dots].

\begin{propoLatticesSplitAlgCont}
 Sean $L\subset M$ ret\'{\i}culos completos de $V$. Si $\{f_{1},f_{2}\}$ es
 una base dada de $L$, existen enteros $n,m\in\bb{Z}$ y $r\in R$,
 representante de las clases en $R/\pi^{m}R$ tales que
 $\{f_{1}\pi^{n},r+f_{2}\pi^{m}\}$ es base de $M$.
\end{propoLatticesSplitAlgCont}

Ver [Vign\'{e}ras, Ch. II].

\begin{teoMaxOrdsSplitAlg}
 \textit{(i)} Todos los \'{o}rdenes maximales de $\MM_{2\times 2}(K)$ son
 conjugados (en $\MM_{2\times 2}(K)$) a
 $\begin{bmatrix} R & R \\ R & R \end{bmatrix}=\MM_{2\times 2}(R)$.
 \textit{(ii)} Los ideales \'{\i}ntegros cuyo orden a izquierda es
 $\MM_{2\times 2}(R)$ son todos de la forma
 \begin{align*}
	& \MM_{2\times 2}(R)
	\begin{bmatrix} \pi^{n} & r \\ & \pi^{m} \end{bmatrix}
 \end{align*}
 %
 con $n,m\geq 0$ y $r\in R$ representante de $R/\pi^{m}R$.
\end{teoMaxOrdsSplitAlg}

% \begin{paragraph}{El nivel de un orden de Eichler}
\paragraph{El nivel de un orden de Eichler}
Dos ret\'{\i}culos $L$ y $L'$ tienen igual anillo de endomorfismos, si
y s\'{o}lo si son homot\'{e}ticos. Dados dos \'{o}rdenes $\cal{O},\cal{O}'$
de $B=\Endo(V)$, $\cal{O}=\Endo(L)$ y $\cal{O}'=\Endo(M)$ para ciertos
ret\'{\i}culos completos $L,M$. De los resultados citados, podemos deducir
que existe una base $\{f_{1},f_{2}\}$ de $L$ y enteros $a_{1},a_{2}$ tales
que $\{f_{1}\pi^{a_{1}},f_{2}^{a_{2}}\}$ es base de $M$. La \emph{distancia}
entre los \'{o}rdenes $\cal{O}$ y $\cal{O}'$ es
$\mathrm{dist}(\cal{O},\cal{O}'):=|a_{1}-a_{2}|\in\bb{Z}_{\geq 0}$.
Este n\'{u}mero no depende de los ret\'{\i}culos elegidos para representar
a $\cal{O}$ y a $\cal{O}'$. Un orden de Eichler es la intersecci\'{o}n
de dos \'{o}rdenes maximales. Si $\cal{O},\cal{O}'$ son maximales,
el \emph{nivel} del orden de Eichler $\cal{O}\cap\cal{O}'$ es la
distancia $\mathrm{dist}(\cal{O},\cal{O}')$ entre los correspondientes
\'{o}rdenes maximales.

\begin{propoOrdEichlerSplitAlg}
 Sea $\cal{O}$ un orden de $\MM_{2\times 2}(K)$. Las propiedades
 siguientes son equivalentes:
 \textit{(i)} existe un \'{u}nico par $\cal{O}_{1},\cal{O}_{2}$ de
 \'{o}rdenes maximales tales que $\cal{O}=\cal{O}_{1}\cap\cal{O}_{2}$;
 \textit{(ii)} $\cal{O}$ es un orden de Eichler;
 \textit{(iii)} existe un \'{u}nico entero $e\geq 0$ tal que $\cal{O}$
	es conjugado a $\begin{bmatrix} R & R \\ \pi^{e}R & R \end{bmatrix}$.
\end{propoOrdEichlerSplitAlg}

El entero $e$ en la proposici\'{o}n anterior es igual al nivel del orden
de Eichler $\cal{O}$. Decimos tambi\'{e}n que el nivel de $\cal{O}$ es
$\pi^{e}R$. Se verifica que el discriminante reducido del orden
$\begin{bmatrix} R & R \\ \pi^{e}R & R \end{bmatrix}$ es $\pi^{e}R$.
A este orden $\MM_{2\times 2}(K)$ lo llamamos
\emph{orden de Eichler est\'{a}ndar de nivel $\pi^{e}R$} y lo denotamos
$\cal{O}_{0}(\pi^{e}R)$. En general, el discriminante reducido de un orden
de Eichler de nivel $\pi^{e}R$ en $B$ es $\pi^{e}R$.
% \end{paragraph}

Para resumir, si $B/K$ es un \'{a}lgebra de matrices, hay infinitos
\'{o}rdenes maximales, pero son todos conjugados, y todo ideal para
un orden maximal es principal. En otras palabras, hay un \'{u}nico tipo
de orden maximal y el n\'{u}mero de clases (a izquierda o a derecha,
en este caso no hay distinci\'{o}n) de ideales de un orden maximal es uno.
La clase de conjugaci\'{o}n de un orden de Eichler --su tipo-- est\'{a}
determinada por su nivel:
\begin{align*}
 \cal{O}\sim\begin{bmatrix} R & R \\ \pi^{e}R & R \end{bmatrix}
	\,\Leftrightarrow\, &
 \text{nivel}\,\cal{O}=\pi^{e}R\,\Leftrightarrow\,
	\drd{\cal{O}}=\pi^{e}R
	\text{ .}
\end{align*}
%

\paragraph{Ideales de un orden de Eichler}
Queremos ver c\'{o}mo son los ideales de un orden de Eichler. Sea
$\cal{O}:=\cal{O}_{0}(\pi^{e}R)$ el orden de Eichler est\'{a}ndar de nivel
$\pi^{e}R$.
%, como ya mencionamos, el discriminante reducido de $\cal{O}$
%es $\drd{\cal{O}}=\pi^{e}R$ igual al nivel.
Recordando que estamos sobre un cuerpo local,
$\ideales{\cal{O}}=\ppales{\cal{O}}$, con lo cual
\begin{align*}
 K^{\times}\backslash\normalizador[B^{\times}]{\cal{O}}/\cal{O}^{\times}
	\,\simeq\, & \ppales{\cal{O}}/\ppales{R}
	\,=\, \ideales{\cal{O}}/\ppales{R}
	\text{ .}
\end{align*}
%
Definimos un elemento $\varpi\in B$ por
\begin{align*}
	\varpi\,:=\, & \begin{bmatrix} & 1 \\ \pi^{e} & \end{bmatrix}
		\text{ .}
\end{align*}
%
El orden de Eichler es igual a la intersecci\'{o}n
\begin{align*}
 \cal{O} \,=\, & \MM_{2\times 2}(R)\,\cap\,
	\varpi^{-1}\MM_{2\times 2}(R)\varpi
	\text{ .}
\end{align*}
%
Identificando $K$ con las matrices escalares, vemos que $\varpi^{2}=\pi^{e}$
y que $\varpi$ pertenece a $\cal{N}:=\normalizador[B^{\times}]{\cal{O}}$.
Sea $I:=\cal{O}\varpi$. Es decir, $\cal{O}\varpi=\varpi\cal{O}$ e $I$ es un
ideal principal cuyos \'{o}rdenes a izquierda y a derecha coinciden y son
iguales a $\cal{O}$. Si $e=0$, $I=\cal{O}$. Supongamos que $e\geq 1$.

Sea $\alpha\in\cal{N}$. De acuerdo con la proposici\'{o}n
???, los \'{o}rdenes $\MM_{2\times 2}(R)$ y su conjugado constituyen el
\'{u}nico par de \'{o}rdenes maximales cuya intersecci\'{o}n es $\cal{O}$.
Como conjugar por $\alpha$ estabiliza $\cal{O}$, los elementos del grupo
$\cal{N}$ permutan los dos \'{o}rdenes y queda determinado un morfismo
sobreyectivo $\cal{N}\rightarrow C_{2}$ en el grupo c\'{\i}clico de orden
$2$. El n\'{u}cleo del morfismo est\'{a} incluido en el normalizador
$\normalizador[B^{\times}]{\MM_{2\times 2}(R)}$, que es igual a
$K^{\times}\GL_{2}(R)$. Entonces
\begin{align*}
 \ker(\cal{N}\rightarrow C_{2}) \,=\, &
	\normalizador[B^{\times}]{\MM_{2\times 2}(R)}\,\cap\,
	\varpi^{-1}\normalizador[B^{\times}]{\MM_{2\times 2}(R)}\varpi \\
 \,=\, & K^{\times}
	\left(\GL_{2}(R)\,\cap\,\varpi^{-1}\GL_{2}(R)\varpi\right) \\
 \,=\, & K^{\times}\cal{O}^{\times}
	\quad\text{y} \\
	\ideales{\cal{O}}/\ppales{R}\,=\, & \langle\varpi\rangle\,\simeq\,
	C_{2}
	\text{ .}
\end{align*}
%
Concluimos, as\'{\i}, que el grupo $\ideales{\cal{O}}$ es abeliano,
generado por $I=\varpi\cal{O}$ y $P=\pi\cal{O}$ con la \'{u}nica relaci\'{o}n
$I^{2}=\pi^{e}\cal{O}=P^{e}$. En particular, si $e=0$,
$\cal{O}=\MM_{2\times 2}(R)$ es maximal y los ideales bil\'{a}teros
forman un grupo libre generado por $P$.

Esta descripci\'{o}n se simplifica cuando $B$ es de divisi\'{o}n.

\subsubsection{$B\not\simeq\MM_{2\times 2}(K)$}
La valuaci\'{o}n $v$ en $K$ se extiende a una valuaci\'{o}n en $B$
componiendo con la norma reducida. Si $v:\,K^{\times}\rightarrow\bb{Z}$,
definimos $w:\,B^{\times}\rightarrow\bb{Z}$ por
\begin{align*}
	w\,=\,v\circ\nrd
\end{align*}
%
que es una valuaci\'{o}n discreta en $B$. Como
$\nrd:\,B^{\times}\rightarrow K^{\times}$ es sobreyectiva (ver [Weil, Ch???]),
existe $u\in B^{\times}$ cuya norma reducida es $\nrd(u)=\pi$ y
$w(u)=1$. Entonces, si
\begin{align*}
	\cal{O}\,:=\, & \left\lbrace w\geq 0\right\rbrace\quad\text{y}\\
	\cal{P}\,:=\, & \left\lbrace w>0 \right\rbrace\text{ ,}
\end{align*}
%
el \'{u}nico orden maximal de $B$ es $\cal{O}$, cuyo \'{u}nico ideal
maximal es $\cal{P}$, que es principal ($\cal{P}=\cal{O}u=u\cal{O}$).
La norma reducida de $\cal{P}$ es $\nrd(\cal{P})=\pi R$ y los ideales
de $\cal{O}$ son todos de la forma $\cal{P}^{i}$ con $i\in\bb{Z}$.
Adem\'{a}s, $(\cal{O}^{*})^{-1}=\cal{P}$ y $\drd{\cal{O}}=\pi R$.
En particular, en este caso tambi\'{e}n, el n\'{u}mero de clases de $\cal{O}$
es uno (todos sus ideales a derecha y a izquierda son principales).
Hay un \'{u}nico orden de Eichler y es igual a $\cal{O}$.

Con esto terminamos la descripci\'{o}n de los \'{o}rdenes e ideales en
un \'{a}lgebra sobre una extensi\'{o}n finita de $\bb{Q}_{p}$.
