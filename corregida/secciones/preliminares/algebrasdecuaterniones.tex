Sea $K$ un cuerpo. Un \emph{\'{a}lgebra de cuaterniones sobre $K$}
\index{algebra de cuaterniones@\'{a}lgebra de cuaterniones}
es un \'{a}lgebra central de dimensi\'{o}n cuatro sobre $K$ con la siguiente
propiedad:
\begin{quote}
	existen $L\subset B$ un $K$-\'{a}lgebra separable de dimensi\'{o}n dos
	sobre $K$, $\theta\in K^{\times}$ y $u\in B$ de manera que
	$B=L\oplus uL$, $u^{2}=\theta$ y $mu=u\conj{m}$, para todo $m\in L$,
\end{quote}
donde $\iota:\,m\mapsto \conj{m}$ es el $K$-automorfismo no trivial de $L/K$.
Escribimos $B=\quatalg{L,\theta}$ para indicar que $B$ se obtiene a partir de
$L$ y de $\theta$ de esta manera. Toda \'{a}lgebra de cuaterniones sobre un
cuerpo $K$ es un \'{a}lgebra central simple (sobre $K$); rec\'{\i}procamente,
toda \'{a}lgebra central simple de dimensi\'{o}n cuatro sobre $K$ es un
\'{a}lgebra de cuaterniones.

Sea $B=\quatalg{L,\theta}$ un \'{a}lgebra de cuaterniones sobre el cuerpo $K$.
La \emph{conjugaci\'{o}n} en $B$\index{conjugacion@conjugaci\'{o}n}
es la transformaci\'{o}n $K$-lineal $\iota$ de $B$ determinada por
\begin{equation}
	\label{eq:conjugacioncuaterniones}
	\iota(\iota(b))\,=\,b \quad\text{,}\quad
	\iota(bb_{1}) \,=\,\iota(b_{1})\iota(b) \quad\text{y}\quad
	\iota(u)\,=\,-u
	\text{ .}
\end{equation}
%
Denotamos el \emph{conjugado}\index{conjugado}
$\iota(b)$ de $b$ en $B$ por $\conj{b}$ y se verifica que
$\conj{b}=\conj{m}-un$, si $b=m+un$ con $m,n\in L$.

El producto de dos elementos $b=m+un$, $b_{1}=m_{1}+un_{1}$ est\'{a} dado por
\begin{equation}
	\label{eq:productocuaterniones}
	bb_{1} \,=\,(m+un)\cdot (m_{1}+un_{1}) \,=\,
		(\theta\conj{n}n_{1}+mm_{1})\,+\,u\,(\conj{m}n_{1}+nm_{1})
	\text{ .}
\end{equation}
%
La \emph{norma reducida} de un elemento $b\in B$ se define como
$\nrd(b)=b\conj{b}$ y su \emph{traza reducida} es $\trd(b)=b+\conj{b}$.
\index{norma reducida}\index{traza reducida}
Si $b=m+un$, entonces
\begin{equation}
	\label{eq:normaytrazacuaterniones}
	\nrd(b) \,=\,-\theta\conj{n}n+m\conj{m}	\quad\text{y}\quad
		\trd(b) \,=\,m+\conj{m}
	\text{ .}
\end{equation}
%
En particular, $\nrd(b)$ y $\trd(b)$ son elementos del cuerpo $K$. Todo
elemento $b\in B$ es soluci\'{o}n de un polinomio cuadr\'{a}tico con
coeficientes en $K$:
\begin{align*}
	\mathsf{m}_{b} & \,=\,X^{2}-\trd(b)\,X+\nrd(b) \,=\,(X-b)\,(X-\conj{b})
	\text{ ,}
\end{align*}
%
su \emph{polinomio minimal}.\index{polinomio minimal}

\begin{lemaNormaYTrazaCuaterniones}\label{lema:normaytrazacuaterniones}
	Las unidades del \'{a}lgebra $B$ son, exactamente, los elementos de
	norma reducida no nula:
	\begin{align*}
		B^{\times} & \,=\,\big\{\nrd\not=0\big\}
		\text{ .}
	\end{align*}
	%
	Adem\'{a}s, la restricci\'{o}n a $B^{\times}$ determina un morfismo de
	grupos $\nrd :\,B^{\times}\rightarrow K^{\times}$, al que tambi\'{e}n
	se denomina \emph{norma reducida}. Por otra parte, la traza reducida es
	una transformaci\'{o}n $K$-lineal $\trd:\,B\rightarrow K$ y la
	aplicaci\'{o}n
	\begin{align*}
		\phi(b,b_{1}) & \,=\,\trd(bb_{1})
	\end{align*}
	%
	es una forma $K$-bilineal no degenerada.
\end{lemaNormaYTrazaCuaterniones}

Sea $H/K$ un espacio vectorial de dimensi\'{o}n cuatro con base
$\{1,i,j,k\}$ y sean $a,b\in K\setmin \{0\}$. Bajo las relaciones
$ji=k$, $1\cdot h=h$ para todo $h\in H$,
\begin{align*}
	i^2 \,=\,1\cdot a & \quad\text{,}\quad
		j^2 \,=\,1\cdot b\quad\text{y}\quad
		ij \,=\,-ji
	\text{ ,}
\end{align*}
%
el espacio $H$ se vuelve una $K$-\'{a}lgebra. Si la caracter\'{\i}stica de $K$
es distinta de $2$, $H$ tiene estructura de \'{a}lgebra de cuaterniones sobre
$K$: si $L=K(i)$, $\theta=b$ y $u=j$, entonces $H=\quatalg{L,\theta}$.
% Si la caracter\'{\i}stica de $K$ es $2$, $H$ es conmutativa (en particular,
% todo ideal es bil\'{a}tero y el centro contiene propiamente a $K$
Si $h=x_{1}+ix_{2}+jx_{3}+kx_{4}\in H$, entonces $h=m+un$, con
$m=x_{1}+ix_{2}$ y $n=x_{3}+ix_{4}$ en $L$. El conjugado, la norma reducida y
la traza reducida de $h$ est\'{a}n dados por
\begin{align*}
	\conj{h} & \,=\,\conj{m}-un\,=\,
		% (x_{1}-ix_{2})\,-\,j\,(x_{3}+ix_{4})\,=\,
		x_{1}-ix_{2}-jx_{3}-kx_{4} \text{ ,} \\
	\nrd(h) & \,=\,-\theta\conj{n}n+m\conj{m} \,=\,
		% -b(x_{3}^{2}-ax_{4})\,+\,(x_{1}^{2}-ax_{2}^{2}) \,=\,
		x_{1}^{2}-ax_{2}^{2}-bx_{3}^{2}+abx_{4}^{2} \quad\text{y} \\
	\trd(h) & \,=\, m+\conj{m} \,=\,
		% (x_{1}+ix_{2})\,+\,(x_{1}-ix_{2}) \,=\,
		2x_{1}
	\text{ .}
\end{align*}
%
La aplicaci\'{o}n forma bilineal $\phi(h,h_{1})=\trd(hh_{1})$ restringida a $L$
es no degenerada. Si $m=x+iy,m_{1}=x_{1}+iy_{1}\in L$, entonces
$\trd(mm_{1})=mm_{1}+\lconj{mm_{1}}=2(xx_{1}+ayy_{1})$. Eligiendo $m=x_{1}$,
si $x_{1}\not =0$, o bien $m=iy_{1}$, si $x_{1}=0$, se comprueba que
$\trd(mm_{1})$ es no nula.
%
Si la caracter\'{\i}stica de $K$ es $2$, se obtiene un \'{a}lgebra de
cuaterniones imponiendo las relaciones
\begin{align*}
	i^2 + i\,=\, 1\cdot a & \quad\text{,}\quad
	j^2\,=\,1\cdot b \quad\text{y}\quad
	ij\,=\,j\,(i+1)
	\text{ .}
\end{align*}
%
Si un \'{a}lgebra de cuaterniones $H/K$ est\'{a} dada de esta manera,
escribimos $H=\varquatalg[K]{a,b}$.

%\paragraph{Ejemplo}
\begin{ejemploMatrices}\label{ejemplo:matrices}
	El \'{a}lgebra de matrices $B=\MM_{2\times 2}(K)$ con coeficientes en
	un cuerpo $K$ es un \'{a}lgebra de cuaterniones sobre $K$. El cuerpo
	$K$ se identifica con las matrices escalares $I\cdot a$. Dada un matriz
	\begin{math}
		\gamma=\begin{bmatrix} a & b \\ c & d \end{bmatrix}
	\end{math}~,
	la adjunta de $\gamma$ es la matriz
	\begin{align*}
		\gamma^\iota & \,=\,
			\begin{bmatrix} d & -b \\
			-c & a \end{bmatrix}
		\text{ .}
	\end{align*}
	%
	Vales que $\gamma\gamma^\iota=\det(\gamma)$ y que
	$\gamma+\gamma^\iota=\traza(\gamma)$. El polinomio caracter\'{\i}stico
	de $\gamma$ est\'{a} dado por
	\begin{math}
		(X-\gamma)\,(X-\gamma^\iota)=
			X^{2}-\traza(\gamma)\,X+\det(\gamma)
	\end{math}~.
	En particular, $\gamma$ y $\gamma^\iota$ poseen los mismos autovalores
	y son similares.
	Si la caracter\'{\i}stica de $K$ es distinta de $2$, tomamos
	\begin{align*}
		i\,=\,\begin{bmatrix} -1 & \\ & 1 \end{bmatrix} &
			\quad\text{y}\quad
		j \,=\,\begin{bmatrix} & 1 \\ 1 & \end{bmatrix}
		\text{ .}
	\end{align*}
	%
	Si la caracter\'{\i}stica de $K$ es $2$, tomamos
	\begin{align*}
		i\,=\,\begin{bmatrix} 1 & 1 \\ 1 & \end{bmatrix}
			& \quad\text{y}\quad
		j\,=\,\begin{bmatrix} & 1 \\ 1 & \end{bmatrix}
		\text{ .}
	\end{align*}
	%
	En ambos casos, se obtiene un \'{a}lgebra de cuaterniones. Notamos que
	el conjugado, la norma reducida y la traza reducida se realizan,
	respectivamente, como la matriz adjunta, el determinante y la traza
	usuales. El $K$-automorfismo no trivial de $L=K(i)$ est\'{a} dado por
	el automorfismo interior $x\mapsto jxj^{-1}$.

	Si la caracter\'{\i}stica es distinta de $2$, tambi\'{e}n podemos
	elegir una matriz $m$ de autovalores distintos (no nulos) y
	$u\in\GL_{2}(K)$ tal que $um^\iota u^{-1}=m$. En particular, si
	definimos $L:=K(m)$, vale $nu=un^\iota$ para toda $n\in L$. La igualdad
	$\traza(un)=un+n^\iota u^\iota=n^\iota\traza(u)$, v\'{a}lida para
	$n\in L$, implica $\traza(u)=0$ y $u^\iota =-u$. Como
	$uu^\iota=\det(u)\in K^{\times}$, se deduce que
	$\theta=u^2\in K^{\times}$.
\end{ejemploMatrices}

Sea $F/K$ una extensi\'{o}n finita de cuerpos y sea $B=\quatalg{L,\theta}$ un
\'{a}legbra de cuaterniones sobre $K$. Tomando producto tensorial sobre $K$, se
obtiene un \'{a}lgebra de cuaterniones sobre $F$:
$B\tensor[K]F=\quatalg{L,\theta}\otimes_{K}F$. Denotamos este \'{a}lgebra por
$B_{F}$. Se cumple que
\begin{align*}
	B_{F} & \,=\,\quatalg{L\tensor[K] F,\theta}
	\text{ .}
\end{align*}
%
% Se dice que $F$ es un cuerpo de descomposici\'{o}n para $B$, si
% $B_{F}\simeq\MM_{2\times 2}(F)$; los posibles isomorfismos se denominan
% $F$-representaciones de $B$

% Sea $B=\varquatalg[K]{a,b}$ un \'{a}lgebra de cuaterniones sobre $K$. Si $a$
% es un cuadrado en $K^{\times}$, queda $B\simeq\MM_{2\times 2}(K)$. Si no,
% tomamos $L=K(i)$ con $i^{2}=a$ (si $\mathrm{char}(K)\not=2$).
% y $(a,b)_{K}$ es una sub\'{a}lgebra de $(a,b)_{L}=(a,b)_{K}\otimes L$.
% Entonces podemos describir a $(a,b)_{K}$ de la siguiente manera:
% \begin{align*}
	% (a,b)_{K} & \,\simeq\, \left\lbrace \begin{bmatrix} u & v \\
			% b\conj{v} & \conj{u} \end{bmatrix}
		% \,:\, u,v\in L\right\rbrace
% \end{align*}
% %
% en tanto $K$-\'{a}lgebras. Entonces $B=L\oplus jL$. Como en el Ejemplo
% \ref{ejemplo:matrices}, $K$ se identifica con las matrices escalares
% $I\cdot x_{0}$, $x_{0}\in K$, la extensi\'{o}n $L/K$ es una $K$-\'{a}lgebra de
% divisi\'{o}n conmutativa maximal contenida en $B$ y se identifica con las
% matrices de la forma
% \begin{math}
	% \begin{bmatrix} u & \\ & \conj{u}\end{bmatrix}
% \end{math}
% con $u\in L$ y
% \begin{math}
	% j=\begin{bmatrix} & 1 \\ b & \end{bmatrix}
% \end{math}~.
% Si $x=x_{0}+x_{1}i+x_{2}j+x_{3}k=u+j\conj{v}\in B$, entonces
% \begin{align*}
	% \conj{x} & \,=\, x_{0} - x_{1} i - x_{2} j - x_{3} k
		% \,=\, \conj{u} - j\conj{v}\text{ ,} \\
	% \trd(x) & \,=\, 2x_{0}
		% \,=\, u + \conj{u} \quad\text{y} \\
	% \nrd(x) & \,=\, x_{0}^{2} - a x_{1}^{2} - b x_{2}^{2} + ab x_{3}^{2}
		% \,=\,u\conj u - b v\conj v
	% \text{ .}
% \end{align*}
% %
