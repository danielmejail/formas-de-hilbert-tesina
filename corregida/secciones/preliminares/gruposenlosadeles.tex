En esta secci\'{o}n repasamos las definiciones del anillo de ad\`{e}les y del
grupo de id\`{e}les de un cuerpo de n\'{u}meros y algunas construcciones
relacionadas. Aprovechamos tambi\'{e}n para indicar la notaci\'{o}n que se
utilizar\'{a} en el resto del trabajo.

Sea $V$ un conjunto de \'{\i}ndices y, para cada $v\in V$, sea $G_{v}$ un
grupo. Si adem\'{a}s contamos con un subconjunto finito $S\subset V$ y, para
cada $v\not\in S$, con un subgrupo $C_{v}\leq G_{v}$, podemos definir el
\emph{producto restringido} de los $G_{v}$ respecto de los $C_{v}$:
\index{producto restringido}
\begin{align*}
	G & \,=\, \Big\{ (x_{v})_{v\in V}\,\in\,\prod_{v\in V}\,G_{v}\,:\,
		x_{v}\in C_{v}\text{ para casi todo }v\in V\setmin S\Big\}
	\text{ .}
\end{align*}
%
Denotaremos este grupo por $\prod_{v}'\,G_{v}$, cuando no sea necesario
mencionar expl\'{\i}citamente el conjunto de lugares $S$ y la familia de
subgrupos $C_v$.
Decimos que una propiedad se cumple para casi todo elemento de un conjunto, si
se cumple para todos los elementos salvo posiblemente finitos de ellos. En los
casos que nos interesan, los grupos $G_{v}$ son grupos topol\'{o}gicos
localmente compactos y, cuando est\'{a}n dados, los subgrupos $C_{v}$ son
abiertos y compactos. En este caso, el producto restringido $G$ tiene
estructura de grupo topol\'{o}gico localmente compacto. Una base de entornos
del elemento neutro est\'{a} dada por los subconjuntos de la forma
$\prod_{v\in V}\,U_{v}$, donde cada $U_{v}$ es abierto con clausura compacta y
$U_{v}=C_{v}$, para casi todo $v$. Los conjuntos de \'{\i}ndices $V$ y $S$ son,
m\'{a}s aun, $V=\lugares{K}$, el conjunto de lugares de un cuerpo de
n\'{u}meros $K/\bb{Q}$, y $S\subset\lugares{K}$ un subconjunto finito de
lugares que contiene los lugares arquimedianos.

Fijemos $K/\bb{Q}$ un cuerpo de n\'{u}meros. Denotamos por $\oka{K}$ su anillo
de enteros y por $\lugares{K}$ el conjunto de lugares de $K$. Escribimos
$\lugares[f]{K}$ y $\lugares[\infty]{K}$ para referirnos, respectivamente, al
subconjunto de lugares finitos, o no arquimedianos, correspondientes a los
primos de $\oka{K}$, y al subconjunto de lugares infinitos, o arquimedianos. Si
$v\in\lugares{K}$, denotamos la completaci\'{o}n de $K$ en $v$ por $K_{v}$. Si
$v\in\lugares[f]{K}$, denotamos el orden maximal en $K_{v}$ por $\oka{K,v}$,
compuesto por los elementos de m\'{o}dulo menor o igual a $1$. Si $S$ es un
subconjunto finito de lugares de $K$ que contiene a $\lugares[\infty]{K}$, el
\emph{anillo de $S$-enteros} est\'{a} definido por
\index{Senteros@$S$-enteros}
\begin{align*}
	\oka{K}(S) & \,=\, \bigcap_{v\not\in S}\,\big(K\cap\oka{K,v}\big)
	\text{ .}
\end{align*}
%
Este conjunto es un subanillo de $K$ y un dominio de Dedekind. Si
$S=\lugares[\infty]{K}$, entonces $\oka{K}(S)=\oka{K}$.

\begin{defAdelesIdelesNumeros}\label{def:adelesidelesnumeros}
	El \emph{anillo de ad\`{e}les de $K$}, denotado $\adeles{K}$, es el
	producto restringido determinado por $S=\lugares[\infty]{K}$,
	$G_{v}=K_{v}$ y $C_{v}=\oka{K,v}$. Es un anillo topol\'{o}gico.
	El \emph{grupo de id\`{e}les de $K$}, denotado $\ideles{K}$, es el
	grupo topol\'{o}gico que se obtiene tomando $S=\lugares[\infty]{K}$,
	$G_{v}=K_{v}^{\times}$ y $C_{v}=\oka{K,v}^{\times}$. Es igual al grupo
	de unidades de $\adeles{K}$ (como conjunto).
	\index{anillo de adeles@anillo de ad\`{e}les}\index{adeles@ad\`{e}les}
	\index{grupo de ideles@grupo de id\`{e}les}\index{ideles@id\`{e}les}
\end{defAdelesIdelesNumeros}

Sea $B/K$ un \'{a}lgebra de cuaterniones sobre $K$. Dado $v\in\lugares{K}$, se
obtiene un \'{a}lgebra de cuaterniones sobre la completaci\'{o}n $K_{v}$
extendiendo escalares:
\begin{align*}
	B_{v} & \,=\,B\tensor[K]K_{v}
	\text{ .}
\end{align*}
%
Sea $S\subset\lugares{K}$ un subconjunto finito de lugares de $K$ que contiene
$\lugares[\infty]{K}$ y sea $\cal{O}\subset B$ un $\oka{K}(S)$-orden del
\'{a}lgebra. Si $v\in\lugares[f]{K}\setmin S$, entonces
\begin{align*}
	\cal{O}_{v} & \,=\,\cal{O}\tensor[\oka{K}(S)]\oka{K,v}
\end{align*}
%
define un ($\oka{K,v}$-)orden en $B_{v}$.

\begin{defAdelesIdelesCuaterniones}\label{def:adelesidelescuaterniones}
	El \emph{anillo de ad\`{e}les de $B$}, denotado $\adeles{B}$, es el
	producto restringido de $G_{v}=B_{v}$ respecto de $C_{v}=\cal{O}_{v}$
	(definidos para $v\not\in S$). Esta definici\'{o}n no depende del
	conjunto finito $S$ (siempre que contenga $\lugares[\infty]{K}$), ni
	del orden $\cal{O}$ elegidos; se verifica que
	\begin{align*}
		\adeles{B} & \,=\, B\tensor[K]\adeles{K}
		\text{ .}
	\end{align*}
	%
	El \emph{grupo de id\`{e}les de $B$}, denotado $\ideles{B}$, es el
	producto restringido que se obtiene eligiendo $G_{v}=B_{v}^{\times}$ y
	$C_{v}=\cal{O}_{v}^{\times}$. Es un grupo topol\'{o}gico y el grupo de
	unidades de $\adeles{B}$.
\end{defAdelesIdelesCuaterniones}

En general, si $V=\lugares{K}$, $S\supset\lugares[\infty]{K}$ es un subconjunto
finito, $\{G_{v}\,:\,v\in\lugares{K}\}$ es una familia de grupos indexada
por los lugares de $K$ y $\{C_{v}\,:\,v\not\in S\}$ es una familia de
subgrupos, entonces el producto restringido se descompone como producto de una
parte arquimediana y una parte no arquimediana:
\begin{align*}
	G & \,=\,\Adinf{G}\times \Adfin{G}
	\text{ ,}
\end{align*}
%
donde $\Adinf{G}=\prod_{v\in\lugares[\infty]{K}}\,G_{v}$ y
$\Adfin{G}=\prod_{v\in\lugares[f]{F}}'\,G_{v}$ es el producto restringido de
$\{G_{v}\,:\,v\in\lugares[f]{K}\}$ respecto de
$\{C_{v}\,:\,v\in\lugares[f]{K}\setmin S\}$. Con respecto a los ejemplos
anteriores, usaremos la siguiente notaci\'{o}n:
\begin{align*}
	\adeles{X}\,=\,\Adinf{X}\times\Adfin{X} & \qquad\text{y}\qquad
		\ideles{X}\,=\,\Idinf{X}\times\Idfin{X}
	\text{ ,}
\end{align*}
%
donde $X=K\text{ o }B$. Si $S\subset\lugares{K}$ es un subconjunto finito que
contiene los lugares arquimedianos, entonces
\begin{align*}
	\Adfin{\oka{K}(S)}\,=\,\prod_{v\in\lugares[f]{K}\setmin S}\,
		\oka{K,v} & \qquad\text{y}\qquad
	\Idfin{\oka{K}(S)}\,=\,\prod_{v\in\lugares[f]{K}\setmin S}\,
		\oka{K,v}^{\times}
\end{align*}
%
definen, respectivamente, un subanillo compacto de $\Adfin{K}$ y su grupo
de unidades, tambi\'{e}n un subgrupo compacto de $\Idfin{K}$ con la
topolog\'{\i}a heredada de los id\`{e}les. An\'{a}logamente,
\begin{align*}
	\Adfin{\cal{O}} \,=\,\prod_{v\in\lugares[f]{K}\setmin S}\,
		\cal{O}_{v} \,=\,\cal{O}\tensor[\oka{K}(S)]\Adfin{\oka{K}(S)}
		& \qquad\text{y}\qquad
	\Idfin{\cal{O}} \,=\,\prod_{v\in\lugares[f]{K}\setmin S}\,
		\cal{O}_{v}^{\times}
\end{align*}
%
definen un subanillo compacto de $\Adfin{B}$ y su grupo de unidades, que es un
subgrupo compacto en $\Idfin{B}$.
%
% Por \'{u}ltimo, asociado al \'{a}lgebra $B$, introducimos el grupo ad\'{e}lico
% $\adeles{B}^{1}$ dado por el producto restringido de $G_{v}=B_{v}^{1}$, el
% n\'{u}cleo de la norma reducida, respecto de $C_{v}=\cal{O}_{v,1}$, los
% elementos del orden $\cal{O}_{v}$ de norma reducida $1$.
%% ?`y el n\'{u}cleo del m\'{o}dulo c\'{o}mo se denota?
