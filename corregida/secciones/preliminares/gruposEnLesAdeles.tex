% \theoremstyle{plain}
% \newtheorem*{propoLocalGlobalLattices}{Proposici\'{o}n}
% \newtheorem*{teoEichlerNorma}{Teorema}
% \newtheorem*{teoAproxFuerte}{Teorema}
% \newtheorem*{teoEichlerNormaClases}{Teorema}
% \newtheorem*{coroFinClassNum}{Corolario}
% 
% \theoremstyle{remark}

\subsection{Grupos en los ad\`{e}les}
En esta secci\'{o}n repasamos definiciones y algunos resultados sobre el
anillo de ad\`{e}les de un cuerpo de n\'{u}meros y algunas construcciones
relacionadas.

Empezamos recordando la definici\'{o}n de \emph{producto restringido}.
\index{producto restringido}
Sea $I$ un conjunto de \'{\i}ndices y supongamos que, para cada $v\in I$,
tenemos un grupo $G_{v}$. Si adem\'{a}s tenemos, para casi todo $v$
(es decir, todos salvo finitos) un subgrupo $C_{v}$ de $G_{v}$, podemos
definir el producto restringido de los $G_{v}$ respecto de los $C_{v}$:
\begin{align*}
 G \,=\, & \left\lbrace (x_{v})_{v\in I}\,\in\,\prod_{v\in I}\,G_{v}\,:\,
	x_{v}\in C_{v}\text{ para casi todo }v\in I\right\rbrace
	\text{ .}
\end{align*}
%
En el caso que nos interesa, los grupos $G_{v}$ son grupos topol\'{o}gicos
localmente compactos y, cuando est\'{a}n dados, los subgrupos $C_{v}$ son
compactos. En este caso, el producto restringido $G$ tiene estructura de
grupo topol\'{o}gico localmente compacto. Una base de entornos del elemento
neutro est\'{a} dada por los subconjuntos $\prod_{v\in I}\,U_{v}$, donde
cada $U_{v}$ es un abierto con clausura compacta en $G_{v}$ y, para casi todo
$v$, $U_{v}=C_{v}$. Se demuestra que este grupo topol\'{o}gico no depende
de la elecci\'{o}n de conjunto finito $S$.

Fijemos un cuerpo de n\'{u}meros $K$ y sea $\oka{K}$ el anillo de enteros.
El conunto de lugares de $K$ lo denotamos $\lugares{K}$ y tenemos
$\lugares{K}=\lugares[\infty]{K}\cup\lugares[f]{K}$, donde $\lugares[f]{K}$
es el conjunto de lugares finitos o no arquimedianos correspondientes a los
primos de $\oka{K}$ y $\lugares[\infty]{K}$ es el conjunto de lugares
infinitos o arquimedianos (escribimos tambi\'{e}n $v\in\lugares[\infty]{K}$
en lugar de $v\mid\infty$,
%y $v<\infty$ en lugar de $v\in\lugares[f]{K}$,
ocasionalmente).
Para cada $v$, $K_{v}$ denota la completaci\'{o}n
de $K$ en el lugar $v$. Si $v$ es un lugar finito, $\oka{K,v}$ denota el
anillo de enteros, el orden maximal, de $K_{v}$.
En general, si $S$ es un conjunto de lugares
de $K$ que contiene a $\lugares[\infty]{K}$, definimos $\oka{K}(S)$
como el anillo de $S$-enteros:
\begin{align*}
 \oka{K}(S) \,=\, & \bigcap_{v\not\in S}\,(\oka{K,v}\cap K)
\end{align*}
%

El \emph{anillo de ad\`{e}les de $K$}
\index{anillo de adeles@anillo de ad\`{e}les}
es el anillo topol\'{o}gico que se
obtiene tomando el producto restringido de las completaciones $K_{v}$
respecto de $\oka{K,v}$ ($v\in\lugares[f]{K}$) y el
\emph{grupo de id\`{e}les de $K$}
\index{grupo de ideles@grupo de id\`{e}les}
es el producto restringido de
$K_{v}^{\times}$ respecto de $\oka{K,v}^{\times}$ ($v\in\lugares[f]{K}$).
Los denotamos, respecivamente, $\adeles{K}$ y $\ideles{K}$.

% El anillo de matrices $\MM_{n\times n}(\adeles{K})$ y el grupo de matrices
% invertibles $\GL_{n}(\adeles{K})$ se obtienen como productos restringidos
% de los respectivos grupos asociados a las completaciones de $K$.
%
Supongamos que $K$ es un cuerpo de n\'{u}meros totalmente real y sea $B$
un \'{a}lgebra de cuaterniones definida sobre $K$. Sea
$S\supset\lugares[\infty]{K}$ un conjunto finito de lugares y sea
$\cal{O}\subset B$ un $\oka{K}(S)$-orden fijo
(no necesariamente un orden maximal).
%Definiciones en Vign\'{e}ras
Para cada lugar $v$, tomamos $G_{v}$ el \'{a}lgebra de cuaterniones
$B_{v}=B\otimes_{K}K_{v}$ sobre el cuerpo local $K_{v}$ y, para cada lugar
finito $v\in\lugares[f]{K}\setmin S$, definimos el orden
$\cal{O}_{v}=\cal{O}\otimes_{\oka{K}(S)}\oka{K,v}$ y tomamos
$C_{v}=\cal{O}_{v}$. El producto restringido se denomina anillo de ad\`{e}les
de $B$ y es igual a tomar el producto tensorial sobre $K$ de $B$ con
$\adeles{K}$. Podemos descomponer este anillo de la siguiente manera:
\begin{align*}
 \adeles{B} \,:=\,B\,\otimes_{K}\,\adeles{K}\,=\, &
	\Adinf{B}\,\times\,\hhat{B}
	\text{ ,}
\end{align*}
%
donde $\Adinf{B}$ es el producto $\prod_{v\in\lugares[\infty]{K}}\,B_{v}$ con
$v\in\lugares[\infty]{K}$ y $\hhat{B}$ es el anillo de ad\`{e}les finitos:
con la notaci\'{o}n anterior, el producto restringido de $G_{v}$ respecto
de $C_{v}$, pero donde $v$ recorre los lugares finitos \'{u}nicamente.
Si $\Adfin{K}$ es el anillo de ad\`{e}les finitos de $K$ y
$\Adinf{K}=\prod_{v\in\lugares[\infty]{K}}\,K_{v}$, tenemos que
$\adeles{K}=\Adinf{K}\times\Adfin{K}$ y, en general, si
$G$ es un grupo algebraico af\'{\i}n definido sobre $K$,
\begin{align*}
	G(\adeles{K}) \,=\, & G(\Adinf{K})\,\times\,G(\Adfin{K})
	\text{ ,}
\end{align*}
%
identificando $\Adinf{K}$ con los ad\`{e}les $(x_{v})_{v\in\lugares{K}}$
tales que $x_{v}=1$, si $v\nmid\infty$ e identificando
$\Adfin{K}$ con aquellos tales que $x_{v}=1$ para $v\in\lugares[\infty]{K}$.
Como hicimos con los id\`{e}les, definimos el grupo de unidades de
$\adeles{B}$ como el producto restringido de $B_{v}^{\times}$
($v\in\lugares{K}$) respecto de $\cal{O}_{v}^{\times}$ ($v\not\in S$).
Lo denotamos $\ideles{B}$.
Por \'{u}ltimo, asociado al \'{a}lgebra $B$, introducimos el grupo
ad\'{e}lico $\adeles{B}^{1}$ dado por el producto restringido de
$G_{v}=B_{v}^{1}$, el n\'{u}cleo de la norma reducida, respecto de
$C_{v}=\cal{O}_{v,1}$, los enteros de norma reducida uno.

\subsection{\'{A}lgebras de cuaterniones sobre un cuerpo de n\'{u}meros}
Pasamos ahora a describir algunos aspectos de los \'{o}rdenes y de los
ideales de un \'{a}lgebra de cuaterniones sobre un cuerpo global, teniendo
en cuenta lo que sucede en el caso local. Sea $K/\bb{Q}$ una extensi\'{o}n
finita, $\oka{K}\subset K$ su anillo de enteros y $B/K$ un \'{a}lgebra de
cuaterniones. Dado un ret\'{\i}culo $L\subset B$, llamamos
$L_{v}=L\otimes_{\oka{K}}\oka{K,v}$ a su clausura en $B_{v}$.

\begin{propoLocalGlobalLattices}
 Sean $L_{0},L\subset B$ ret\'{\i}culos completos.
	\begin{itemize}
		\item[i] $L=\bigcap_{v\in\lugares[f]{K}}\,B\cap L_{v}$,
		\item[ii] $L_{v}=L_{0,v}$ para casi todo $v\in\lugares[f]{K}$ y
		\item[iii] Si $\{L'_{v}\}_{v\in\lugares[f]{K}}$ es una familia
			de ret\'{\i}culos locales completos en $B_{v}$,
			existe un (\'{u}nico) ret\'{\i}culo global completo
			$M\subset B$ tal que $L'_{v}=M_{v}$ para todo
			$v\in\lugares[f]{K}$.
	\end{itemize}
\end{propoLocalGlobalLattices}

El \'{a}lgebra $B/K$ se dice \emph{indefinida},
\index{algebra indefinida@\'{a}lgebra indefinida}
si $B$ es no ramificada
\index{algebra ramificada@\'{a}lgebra ramificada}
en alg\'{u}n lugar arquimediano, es decir, si existe $v\in\lugares[\infty]{K}$
tal que $B_{v}\simeq\MM_{2\times 2}(K_{v})$. Si $B_{v}$ es de divisi\'{o}n
para todo lugar infinito $v$, decimos que $B$ es \emph{(totalmente) definida}.
\index{algebra definida@\'{a}lgebra (totalmente) definida}
Esto \'{u}ltimo solamente es posible cuando $K/\bb{Q}$ es totalmente real.

Sea $K_{(+)}^{\times}\subset K^{\times}$ el subconjunto de
elementos cuya imagen en $K_{v}$ es positiva para todo lugar real $v$
en donde $B$ ramifica.

\begin{teoEichlerNorma}[de la imagen de la norma]
	La imagen del morfismo $\nrd:\,B^{\times}\rightarrow K^{\times}$ es
	\begin{align*}
		\nrd(B^{\times}) \,=\, & K_{(+)}^{\times}
		\text{ .}
	\end{align*}
	%
\end{teoEichlerNorma}

% \begin{proof}[Demostraci\'{o}n]
%  Ver [Weil, XI \S 3]
% \end{proof}

En particular, si $K/\bb{Q}$ es una extensi\'{o}n totalmente real de
grado $n$ y $B$ es un \'{a}lgebra que ramifica en $n-r$ lugares arquimedianos,
es decir,
\begin{align*}
 B_{\infty} \,\simeq\, & \MM_{2\times 2}(\bb{R})^{r} \,\times\,
	\bb{H}^{n-r}
	\text{ ,}
\end{align*}
%
y si
$B_{+}^{\times}=\{\gamma\in B^{\times}\,:\,\nrd(\gamma)\in K_{+}^{\times}\}$
denota el subgrupo de $B^{\times}$ de elementos cuya norma reducida es
totalmente positiva, entonces
\begin{align*}
	B^{\times}/B_{+}^{\times} \,\simeq\, & (\bb{Z}/2\bb{Z})^{r}
	\text{ .}
\end{align*}
%

Sea $S\subset\lugares{K}$ un subconjunto finito. Sea $B^{1}$
el subgrupo de $B^{\times}$ de elementos de norma reducida uno, sea
$B_{v}^{1}$ el subgrupo correspondiente en $B_{v}^{\times}$ y sea
$B_{S}^{1}:=\prod_{v\in S}\,B_{v}^{1}$. Cada $B_{v}^{1}$ es compacto, si
y s\'{o}lo si $B$ ramifica en $v$ y $B_{S}^{1}$ es compacto, si y s\'{o}lo
si $S\subset\Ram(B)$.

\begin{teoAproxFuerte}[de aproximaci\'{o}n fuerte]
	Sea $S\subset\lugares{K}$ un subconjunto finito que contiene al menos
	un lugar arquimediano. Entonces, si $B_{S}^{1}$ no es compacto,
	$B^{1}\cdot B_{S}^{1}$ es denso en $\adeles{B}^{1}$.
\end{teoAproxFuerte}

De ahora en adelante, tomamos $S=\lugares[\infty]{K}$. Entonces
\begin{align*}
	B_{S}^{1}\text{ es compacto}\,\Leftrightarrow\, &
	B\text{ es totalmente definida.}
\end{align*}
%

\subsection{Propiedades locales}
Un ret\'{\i}culo $L\subset B$ es un orden o un ideal, si y s\'{o}lo si
$L_{v}$ es un orden o un ideal, respectivamente, para todo lugar
finito $v$. Dado $X\subset B$, decimos que $X$ satisface una propiedad
\emph{localmente},
\index{propiedad local}
si todas las clausuras $X_{v}$, $v\in\lugares[f]{K}$
tienen la propiedad correspondiente. Un orden de $B$ es maximal, si y
s\'{o}lo si es localmente maximal y es de Eichler, si y s\'{o}lo si
lo es localmente. Si $v\in\lugares[f]{K}$ es un lugar finito e $I\subset B$
es un ideal,
\begin{align*}
	\Oizq(I_{v}) \,=\, & \Oizq(I)_{v}\text{ ,} \\
	\Oder(I_{v}) \,=\, & \Oder(I)_{v}\text{ ,} \\
	(I_{v})^{*} \,=\, & (I^{*})_{v}\text{ y} \\
	\nrd(I_{v}) \,=\, & \nrd(I)_{v}\text{ .}
\end{align*}
%
El ideal $I$ es invertible, si y s\'{o}lo si es localmente invertible, si
y s\'{o}lo si es localmente principal. En particular, todos los ideales
de un orden de Eichler de $B$ son localmente principales y, por lo tanto,
invertibles. Si $\cal{O}$ es un orden de $B$,
\begin{align*}
	\drd{\cal{O}_{v}} \,=\, & (\drd{\cal{O}})_{v}\text{ .}
\end{align*}
%
El orden $\cal{O}$ es maximal, si y s\'{o}lo si
$\drd{\cal{O}}=\prod_{\frak{p}\in\Ram(B)\setmin S}\,\frak{p}$,
el producto de los primos (finitos)
%fuera de $S$
en donde $B$ ramifica. A este producto lo llamamos el
\emph{discriminante de $B$}
\index{discriminante}
y lo denotaremos $\frak{D}$.
Si $\cal{O}$ es de Eichler, su \emph{nivel}
\index{nivel de un orden de Eichler}
es el ideal $\frak{N}\subset\oka{K}$ tal que para cada $v\in\lugares[f]{K}$
$\frak{N}_{v}\subset\oka{K,v}$ es el nivel del orden $\cal{O}_{v}$
($v$ debe ser no ramificado). El ideal $\frak{N}$ es tal que
\begin{align*}
	\cal{O}_{v}\,\sim\, & \begin{bmatrix} \oka{K,v} & \oka{K,v} \\
				\frak{N}\oka{K,v} & \oka{K,v}
				\end{bmatrix}
\end{align*}
%
en $B_{v}=\MM_{2\times 2}(K_{v})$ para
$v\in\lugares[f]{K}\setmin\Ram(B)$. El discriminante reducido de
un orden de Eichler $\cal{O}$ de nivel $\frak{N}$ es igual a
$\drd{\cal{O}}=\frak{D}\cdot\frak{N}$. Notemos que $(\frak{N},\frak{D})=1$.

Sea $\cal{O}$ un orden de Eichler de nivel $\frak{N}$ y sea
\begin{align*}
	\Idfin{\cal{O}}\,:=\, &
	(\cal{O}\otimes_{\bb{Z}}\Adfin{\bb{Z}})^{\times}
	\,=\, \prod_{v\in\lugares[f]{K}}\,\cal{O}_{v}^{\times}
	\text{ .}
\end{align*}
%
Sea $(g_{v})_{v}\in\Idfin{B}$, los id\`{e}les finitos en
$\ideles{B}=\Idinf{B}\times\Idfin{B}$. La familia de ret\'{\i}culos
$\{g_{v}\cal{O}_{v}\}_{v}$ determina un ideal de $B$
cuyo orden a derecha es $\cal{O}$.
% La estructura de $B$ en tanto producto restringido depende,
% \textit{a priori}, de elegir un orden previamente en $B$ y de un conjunto
% finito $S$ de lugares. Pero como $S$ es finito y todos los \'{o}rdenes
% $\cal{O},\cal{O}'$ (ret\'{\i}culos completos) de $B$ satisfacen
% $\cal{O}_{v}=\cal{O}'_{v}$ para casi todo $v$, no hay diferencia.
Como todo ideal $I$ de $\cal{O}$ es localmente principal, la aplicaci\'{o}n
$(g_{v})_{v}\mapsto I$ as\'{\i} definida es sobreyectiva. Si
$(g_{v})_{v},(g'_{v})_{v}\in\Idfin{B}$, vale
$g_{v}\cal{O}_{v}=g'_{v}\cal{O}_{v}$ para todo $v$, si y s\'{o}lo si
$g'_{v}\in g_{v}\cal{O}_{v}^{\times}$ para todo $v$. Hay una biyecci\'{o}n
\begin{align*}
	\Idfin{B}/\Idfin{\cal{O}}\,\xrightarrow{\sim}\, &
	\left\lbrace
	\begin{array}{c}
		\text{Ideales }I\subset B\text{ con} \\
		\Oder(I)=\cal{O}
	\end{array}
	\right\rbrace
	\text{ ,}
\end{align*}
%
que determina una correspondencia entre
$B^{\times}\backslash\Idfin{B}/\Idfin{\cal{O}}$ y $\lClass{\cal{O}}$.

\begin{teoEichlerNormaClases}[Eichler]
	Si $B/K$ es un \'{a}lgebra de cuaterniones sobre un cuerpo de
	n\'{u}meros y $\cal{O}\subset B$ es un orden de Eichler, la
	norma reducida induce una suryecci\'{o}n del conjunto de clases
	$\lClass{\cal{O}}$ de ideales cuyo orden a derecha es $\cal{O}$
	m\'{o}dulo la relaci\'{o}n: $I\sim J$ si existe $b\in B^{\times}$
	con $J=bI$, en un grupo de clases asociado al cuerpo $K$ y al
	\'{a}lgebra $B$:
	\begin{align*}
		n\,:\, & B^{\times}\backslash \Idfin{B}/\Idfin{\cal{O}}
		\,\twoheadrightarrow\,
		K_{(+)}^{\times}\backslash\Idfin{K}/\Idfin{\oka{K}}
		\text{ .}
	\end{align*}
	%
	Si, adem\'{a}s, $B/K$ es indefinida, esta aplicaci\'{o}n es biyectiva.
\end{teoEichlerNormaClases}

\begin{proof}[Demostraci\'{o}n]
	La norma reducida $B^{\times}\rightarrow K^{\times}$ induce morfismos
	en las completaciones $B_{v}^{\times}\rightarrow K_{v}^{\times}$ y
	una aplicaci\'{o}n $\ideles{B}\rightarrow\ideles{K}$ evaluando en
	los puntos ad\'{e}licos. Por el teorema ???,
	$\nrd(B^{\times})=K_{(+)}^{\times}$ y para $v\in\lugares[f]{K}$
	finito $\nrd(\cal{O}_{v}^{\times})=\oka{K,v}^{\times}$. Con lo
	cual se obtiene una aplicaci\'{o}n bien definida y sobreyectiva
	\begin{align*}
		n\,:\, & B^{\times}\backslash
		\Idinf{B}\times\Idfin{B}/\Idinf{B}\Idfin{\cal{O}}
		\,\twoheadrightarrow\,
		K_{(+)}^{\times}\backslash
		\Idinf{K}\times\Idfin{K}/\Idinf{K}\Idfin{\oka{K}}
		\text{ ,}\quad\text{o bien} \\
		n\,:\, & B^{\times}\backslash\Idfin{B}/\Idfin{\cal{O}}
		\,\twoheadrightarrow\,
		K_{(+)}^{\times}\backslash\Idfin{K}/\Idfin{\oka{K}}
		\text{ .}
	\end{align*}
	%
	El cociente de la izquierda est\'{a} en correspondencia con el
	conjunto $\lClass{\cal{O}}$ y el de la derecha con el grupo de
	clases $\cal{I}/\cal{P}_{B}$, donde $\cal{I}$
	es el grupo de ideales fraccionarios de $K$ y $\cal{P}_{B}$ es el
	subgrupo de ideales principales con generadores en
	$K_{(+)}^{\times}=\nrd(B^{\times})$.
	Si $B/K$ es indefinida, el grupo $B_{S}^{1}$ contiene a
	$\SL_{2}(\bb{R})$ como alguno de sus factores y, por lo tanto,
	no es compacto. Por el teorema ??? $B^{1}B_{S}^{1}$ es denso
	en $\adeles{B}^{1}$. Como $\Idfin{\cal{O}}$ es abierto, el producto
	$\Idfin{\cal{O}}\Idinf{B}$ es abierto y contiene a $B_{S}^{1}$.
	Entonces
	$\adeles{B}^{1}\subset B^{\times}\cdot\Idfin{\cal{O}}\Idinf{B}$
	y $n$ resulta ser inyectiva.
\end{proof}

Expresado de otra manera, el n\'{u}cleo del morfismo suryectivo $n$ es,
en general,
\begin{align*}
	& B^{\times}\backslash
	\adeles{B}^{1}/(\Idinf{B}\Idfin{\cal{O}}\cap\adeles{B}^{1})
	\text{ .}
\end{align*}
%
Si $B\not\simeq\MM_{2\times 2}(K)$, es decir, si $B$ es de divisi\'{o}n,
este cociente es finito: en primer lugar, al ser
$\Idinf{B}\Idfin{\cal{O}}\cap\adeles{B}^{1}$ abierto en $\adeles{B}^{1}$,
el cociente del segundo por el primero es discreto, y, en segundo lugar,
como sucede con un cuerpo de n\'{u}meros, $B^{\times}$ es discreto en
$\adeles{B}^{1}$ y el cociente $B^{\times}\backslash\adeles{B}^{1}$ es
compacto. El n\'{u}cleo de $n$ es compacto y discreto y, por lo tanto,
finito. cuando $B/K$ es indefinida,
$\adeles{B}^{1}\subset B^{\times}\cdot\Idinf{B}\Idfin{\cal{O}}$, con lo
cual el n\'{u}cleo es trivial. Por otro lado, la imagen de $n$ es
$K_{(+)}^{\times}\backslash\Idfin{K}/\Idfin{\oka{K}}$, un grupo de clases
de $K$ y, en consecuencia, finito.

\begin{coroFinClassNum}
	Si $B/K$ es un \'{a}lgebra de cuaterniones sobre un cuerpo de
	n\'{u}meros (definida o indefinida, de matrices o de divisi\'{o}n)
	y $\cal{O}\subset B$ es un orden de Eichler, el cardinal del conjunto
	de clases $\lClass{\cal{O}}$, el \emph{n\'{u}mero de clases}
	\index{numero de clases@n\'{u}mero de clases}
	de $\cal{O}$, es finito.
\end{coroFinClassNum}

Otra consecuencia del teorema es que, cuando $B$ es totalmente definida,
$K_{(+)}^{\times}=K_{+}^{\times}$, el conjunto de elementos totalmente
positivos y obtenemos una suryecci\'{o}n del cociente
$B^{\times}\backslash\Idfin{B}/\Idfin{\cal{O}}$ en $\pClass{K}$,
el grupo de clases estrictas de $K$. Si $B$ es indefinida se obtiene
una biyecci\'{o}n con un cociente de este grupo. Pero, de la demostraci\'{o}n
del teorema, se puede ver que $\pClass{K}$ aparece si modificamos el grupo
actuando a izquierda: si
$B_{+}^{\times}=\{\gamma\in B^{\times}:\nrd(\gamma)\gg 0\}$,
se obtiene una
\begin{align*}
	\widetilde{n}\,:\, &
	B_{+}^{\times}\backslash\Idfin{B}/\Idfin{\cal{O}}
	\,\twoheadrightarrow\,
	K_{+}^{\times}\backslash\Idfin{K}/\Idfin{\oka{K}}
\end{align*}
%
suryectiva, independientemente de $B$, que, si $B$ es indefinida, resutla
una biyecci\'{o}n.

% Las demostraciones de estos resultados se pueden encontrar en [Weil],
% [Hida], [Vign\'{e}ras], \dots
