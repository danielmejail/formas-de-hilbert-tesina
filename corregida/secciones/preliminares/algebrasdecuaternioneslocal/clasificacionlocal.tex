En general, un \'{a}lgebra de cuaterniones sobre un cuerpo, o bien es isomorfa
a un \'{a}lgebra de matrices, o bien es de divisi\'{o}n. Todo anillo de
divisi\'{o}n con una cantidad finita de elementos es conmutativo; sobre un
cuerpo finito, entonces, toda \'{a}lgebra de cuaterniones es isomorfa al
\'{a}lgebra de matrices con coeficientes en el cuerpo. Sobre el cuerpo de
n\'{u}meros complejos --o, en general, sobre un cuerpo separablemente cerrado--
existe, tambi\'{e}n, una \'{u}nica clase de isomorfismo de \'{a}lgebras de
cuaterniones.

Toda \'{a}lgebra de divisi\'{o}n real, no conmutativa y de dimensi\'{o}n finita
es isomorfa al \'{a}lgebra de cuaterniones de Hamilton,
$\bb{H}=\varquatalg[\bb{R}]{-1,-1}$. Hay, entonces, exactamente dos clases de
isomorfismo de \'{a}lgebras de cuaterniones reales: la clase del \'{a}lgebra de
matrices $\MM_{2\times 2}(\bb{R})$ y la clase de las \'{a}lgebras de
divisi\'{o}n, representadas por $\bb{H}$.
%
% Sobre los cuerpos locales $\bb{Q}_{p}$, como en el caso real, hay exactamente
% dos clases de isomorfismo de \'{a}lgebras de cuaterniones: una correspondiente
% a matrices y otra correspondiente a un \'{a}lgebra de divisi\'{o}n sobre
% $\bb{Q}_{p}$.
%%
Esto es cierto bien en general.
% : un \emph{cuerpo local}
% \index{cuerpo local}
% es un cuerpo (conmutativo) localmente compacto (y no discreto), es decir,
% $\bb{R}$, $\bb{C}$ o una extensi\'{o}n finita de $\bb{Q}_{p}$ (si la
% caracter\'{\i}stica es $0$), o $\bb{F}_{q}[[T]]$, si la caracter\'{\i}stica es
% positiva.

\begin{teoClassificationSurUnCorpsLocal}[Clasificaci\'{o}n local]
 \label{thm:clasificacionlocal}
 Sea $K$ un cuerpo local distinto de $\bb{C}$. Entonces existe una \'{u}nica
 \'{a}lgebra de cuaterniones de divisi\'{o}n, salvo isomorfismo.
\end{teoClassificationSurUnCorpsLocal}
