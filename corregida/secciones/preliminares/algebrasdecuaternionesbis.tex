Sea $K$ un cuerpo. Sea $B/K$ un espacio vectorial de dimensi\'{o}n cuatro con
base $\{1,i,j,k\}$ y sean $a,b\in K\setmin \{0\}$. Bajo las relaciones $ji=k$,
$1\cdot h=h$ para todo $h\in B$,
\begin{align*}
	i^2 \,=\,1\cdot a & \quad\text{,}\quad
		j^2 \,=\,1\cdot b\quad\text{y}\quad
		ij \,=\,-ji
	\text{ ,}
\end{align*}
%
el espacio $B$ se vuelve una $K$-\'{a}lgebra. Si la caracter\'{\i}stica de $K$
es distinta de $2$, un \'{a}lgebra definida de esta manera es un
\emph{\'{a}lgebra de cuaterniones sobre $K$}. En tal caso escribimos
$B=\varquatalg[K]{a,b}$. Toda \'{a}lgebra de cuaterniones sobre un cuerpo $K$
es un \'{a}lgebra central simple (sobre $K$); rec\'{\i}procamente, toda
\'{a}lgebra central simple de dimensi\'{o}n cuatro sobre $K$ es un \'{a}lgebra
de cuaterniones. De ahora en m\'{a}s, todo cuerpo ser\'{a} de
caracter\'{\i}stica cero, y esta descripci\'{o}n de las \'{a}lgebras de
cuaterniones ser\'{a} suficiente.%
\footnote{
	Si la caracter\'{\i}stica de $K$ es $2$, se obtiene un \'{a}lgebra de
	cuaterniones imponiendo las relaciones
	\begin{align*}
		i^2 + i\,=\, 1\cdot a & \quad\text{,}\quad
		j^2\,=\,1\cdot b \quad\text{y}\quad
		ij\,=\,j\,(i+1)
		\text{ .}
	\end{align*}
	%
}
\index{algebra de cuaterniones@\'{a}lgebra de cuaterniones}

Sea $B=\varquatalg[K]{a,b}$ un \'{a}lgebra de cuaterniones sobre el cuerpo $K$.
La \emph{conjugaci\'{o}n} en $B$ es la transformaci\'{o}n $K$-lineal $\iota$ de
$B$ determinada por
\index{conjugacion@conjugaci\'{o}n}\index{conjugado}
\begin{equation}
	\label{eq:conjugacioncuaterniones}
	\iota(x_{1}+i\,x_{2}+j\,x_{3}+k\,x_{4}) \,=\,
		x_{1}-i\,x_{2}-j\,x_{3}-k\,x_{4}
	\text{ .}
\end{equation}
%
Denotamos el \emph{conjugado} $\iota(h)$ de $h$ en $B$ por $\conj{h}$. La
\emph{norma reducida} de un elemento $b\in B$ se define como
$\nrd(h)=h\conj{h}$ y su \emph{traza reducida} es $\trd(h)=h+\conj{h}$.
\index{norma reducida}\index{traza reducida}
En coordenadas,
% Si $h=x_{1}+ix_{2}+jx_{3}+kx_{4}$, entonces
\begin{equation}
	\label{eq:normaytrazacuaterniones}
		\nrd(h) \,=\,x_{1}^{2}- ax_{2}^{2}-bx_{3}^{2}+abx_{4}^{2}
			\quad\text{y}\quad
		\trd(h) \,=\,2x_{1}
		\text{ .}
\end{equation}
%
En particular, $\nrd(h)$ y $\trd(h)$ son elementos del cuerpo $K$. Todo
elemento $h\in B$ es soluci\'{o}n de un polinomio cuadr\'{a}tico con
coeficientes en $K$:
\begin{align*}
	\mathsf{m}_{h} & \,=\,X^{2}-\trd(h)\,X+\nrd(h) \,=\,(X-h)\,(X-\conj{h})
	\text{ ,}
\end{align*}
%
su \emph{polinomio minimal}.\index{polinomio minimal}

\begin{lemaNormaYTrazaCuaterniones}\label{lema:normaytrazacuaterniones}
	Las unidades del \'{a}lgebra $B$ son, exactamente, los elementos de
	norma reducida no nula:
	\begin{align*}
		B^{\times} & \,=\,\big\{\nrd\not=0\big\}
		\text{ .}
	\end{align*}
	%
	Adem\'{a}s, la restricci\'{o}n a $B^{\times}$ determina un morfismo de
	grupos $\nrd :\,B^{\times}\rightarrow K^{\times}$, al que tambi\'{e}n
	se denomina \emph{norma reducida}. Por otra parte, la traza reducida es
	una transformaci\'{o}n $K$-lineal $\trd:\,B\rightarrow K$ y la
	aplicaci\'{o}n
	\begin{align*}
		\phi(h,h_{1}) & \,=\,\trd(hh_{1})
	\end{align*}
	%
	es una forma $K$-bilineal no degenerada.
\end{lemaNormaYTrazaCuaterniones}

%\paragraph{Ejemplo}
\begin{ejemploMatrices}\label{ejemplo:matrices}
	El \'{a}lgebra de matrices $B=\MM_{2\times 2}(K)$ con coeficientes en
	un cuerpo $K$ es un \'{a}lgebra de cuaterniones sobre $K$. El cuerpo
	$K$ se identifica con las matrices escalares $I\cdot a$. Dada un matriz
	\begin{math}
		\gamma=\begin{bmatrix} a & b \\ c & d \end{bmatrix}
	\end{math}~,
	la adjunta de $\gamma$ es la matriz
	\begin{align*}
		\gamma^\iota & \,=\,
			\begin{bmatrix} d & -b \\
			-c & a \end{bmatrix}
		\text{ .}
	\end{align*}
	%
	Vale que $\gamma\gamma^\iota=\det(\gamma)$ y que
	$\gamma+\gamma^\iota=\traza(\gamma)$. El polinomio caracter\'{\i}stico
	de $\gamma$ est\'{a} dado por
	\begin{math}
		(X-\gamma)\,(X-\gamma^\iota)=
			X^{2}-\traza(\gamma)\,X+\det(\gamma)
	\end{math}~. En particular, $\gamma$ y $\gamma^\iota$ poseen los mismos
	autovalores y son similares. Tomando
	\begin{align*}
		i\,=\,\begin{bmatrix} -1 & \\ & 1 \end{bmatrix} &
			\quad\text{y}\quad
		j \,=\,\begin{bmatrix} & 1 \\ 1 & \end{bmatrix}
		\text{ ,}
	\end{align*}
	%
	se obtiene una estructura de \'{a}lgebra de cuaterniones en el espacio
	de matrices.%
	\footnote{
		Si la caracter\'{\i}stica de $K$ es $2$, tomamos
		\begin{align*}
			i\,=\,\begin{bmatrix} 1 & 1 \\ 1 & \end{bmatrix}
				& \quad\text{y}\quad
			j\,=\,\begin{bmatrix} & 1 \\ 1 & \end{bmatrix}
			\text{ .}
		\end{align*}
		%
	}
	Notamos que el conjugado, la norma reducida y la traza reducida se
	realizan, respectivamente, como la matriz adjunta, el determinante y la
	traza usuales.
\end{ejemploMatrices}

Sea $F/K$ una extensi\'{o}n finita de cuerpos y sea $B=\varquatalg[K]{a,b}$ un
\'{a}legbra de cuaterniones sobre $K$. Tomando producto tensorial sobre $K$, se
obtiene un \'{a}lgebra de cuaterniones sobre $F$:
$B\tensor[K]F=\varquatalg[K]{a,b}\otimes_{K}F$. Denotamos este \'{a}lgebra por
$B_{F}$. Se cumple que
\begin{math}
	B_{F}=\varquatalg[F]{a,b}
\end{math}~.
