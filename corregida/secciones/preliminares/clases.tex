
Dos ideales $I,J$ de $B$ se dicen \emph{equivalentes a izquierda},
\index{ideales equivalentes a izquierda}
si existe $b\in B^{\times}$ tal que $I=bJ$.
Si $\cal{O}$ es un orden de $B$, esto determina una relaci\'{o}n de
equivalencia en el conjunto de ideales cuyo orden \emph{a derecha} es
$\cal{O}$ y dos $\cal{O}$-ideales a derecha son equivalentes a izquierda,
si y s\'{o}lo si son isomorfos como $\cal{O}$-m\'{o}dulos a derecha.
\index{clase a izquierda de ideales}
Notemos que si $I$ es un ideal invertible con $\Oder(I)=\cal{O}$, todo
ideal $J=bI$ equivalente a izquierda a $I$ es invertible, tambi\'{e}n.
En tal caso, decimos que la \emph{clase} correspondiente a $I$ es
\emph{invertible}.\index{clase a izquierda de ideales!invertible}
El \emph{conjunto de clases a izquierda}\index{conjunto de clases a izquierda}
del orden $\cal{O}$, denotado $\lClass{\cal{O}}$, es el conjunto de
ideales $I$ tales que $\Oder(I)=\cal{O}$, m\'{o}dulo la relaci\'{o}n
$I\sim J\Leftrightarrow I=bJ$ para alg\'{u}n $b\in B^{\times}$, y cuya clase
es invertible. An\'{a}logamente, podemos definir las clases a derecha de
$\cal{O}$ (y denotamos el conjunto de dichas clases por $\rClass{\cal{O}}$).
Notemos que $\rClass{\cal{O}}$ y $\lClass{\cal{O}}$ son \emph{a priori}
s\'{o}lo \emph{conjuntos}.
% El conjunto $\lClass{\cal{O}}$ tiene un elemento distinguido: la clase de
% $\cal{O}$.
Como un \'{a}lgebra de cuaterniones cuenta con una involuci\'{o}n
$x\mapsto\conj{x}$, la aplicaci\'{o}n $I\mapsto\conj{I}$ da una biyecci\'{o}n
entre las clases a izquierda y las clases a derecha de un orden $\cal{O}$.

Decimos que dos \'{o}rdenes $\cal{O},\cal{O}'$ \emph{est\'{a}n conectados},
\index{ordenes@\'{o}rdenes!conectados}
si existe un ideal $I$ de $B$ cuyo orden a izquierda es $\cal{O}$ y cuyo
orden a derecha es $\cal{O}'$ y se dicen \emph{del mismo tipo},
\index{ordenes@\'{o}rdenes!del mismo tipo}
si son conjugados por un elemento de $B^{\times}$ (o, equivalentemente, si
son isomorfos como $R$-\'{a}lgebras). Los \'{o}rdenes $\cal{O}$ y $\cal{O}'$
son del mismo tipo, si y s\'{o}lo si est\'{a}n conectados por un ideal
principal. Un isomorfismo $\cal{O}\simeq\cal{O}'$ (dado por conjugaci\'{o}n)
induce una biyecci\'{o}n $\lClass{\cal{O}}\rightarrow\lClass{\cal{O}'}$.
La existencia de una involuci\'{o}n en $B$ nos permite concluir
que ``estar conectados'' es una relaci\'{o}n de equivalencia entre los
\'{o}rdenes de $B$. El \emph{g\'{e}nero}
\index{orden!genero de un@g\'{e}nero de un}
de $\cal{O}$ es el conjunto de \'{o}rdenes conectados con $\cal{O}$, es
decir, su clase respecto de esta relaci\'{o}n, lo denotamos $\Genus{\cal{O}}$.
Si $\cal{O}$ y $\cal{O}'$ son \'{o}rdenes conectados
y $J$ es un ideal que los conecta, la aplicaci\'{o}n $I\mapsto IJ$ determina
una biyecci\'{o}n $\lClass{\cal{O}}\rightarrow\lClass{\cal{O}'}$;
su inversa est\'{a} dada por $I'\mapsto I'J^{-1}$.
%
% Se puede ver que, sobre un cuerpo global, dos \'{o}rdenes est\'{a}n
% conectados, si y s\'{o}lo si son isomorfos \emph{localmente} (pero
% no necesariamente globalmente); dos \'{o}rdenes pueden pertenecer
% al mismo g\'{e}nero pero no ser del mismo tipo.
%
%%---
%
% \subsection{Grupoide de Brandt}
% Describimos la construcci\'{o}n an\'{a}loga al grupo de clases de ideales
de un cuerpo de n\'{u}meros para un \'{a}lgebra de cuaterniones.

Dados $\cal{O},\cal{O}'$ dos \'{o}rdenes de $B$, sea
$\ideales{\cal{O},\cal{O}'}$ el conjunto de ideales invertibles cuyo
orden a izquierda es $\cal{O}$ y cuyo orden a derecha es $\cal{O}'$.
Este conjunto es no vac\'{\i}o precisamente cuando $\cal{O}$ y $\cal{O}'$
pertenecen al mismo g\'{e}nero. El conjunto de ideales invertibles de $B$,
con el producto dado por el producto compatible de ideales y las
unidades los \'{o}rdenes de $B$, tiene estructura de grupoide.
El producto $IJ$ est\'{a} definido \'{u}nicamente cuando el producto de
ideales es compatible. \'{O}rdenes $\cal{O},\cal{O}',\cal{O}''$ pertenecen al
mismo g\'{e}nero, si y s\'{o}lo si los conjuntos
$\ideales{\cal{O},\cal{O}'}$ e $\ideales{\cal{O}',\cal{O}''}$ son no
vac\'{\i}os, y, en tal caso, hay una aplicaci\'{o}n
\begin{align*}
	\ideales{\cal{O},\cal{O}'}\,\times\,\ideales{\cal{O}',\cal{O}''}
	& \,\rightarrow\,\ideales{\cal{O},\cal{O}''} \\
	(I,J) & \,\mapsto\,IJ
	\text{ .}
\end{align*}
%
Entonces, dado un g\'{e}nero $\cal{G}=\Genus{\cal{O}}$, definimos
\begin{align*}
	\Brandt{\cal{G}} \,:=\, & \left\lbrace
	\text{Ideales invertibles } I\text{ de } B\text{ con }
	\Oizq(I),\Oder(I)\in\cal{G}\right\rbrace
	\text{ .}
\end{align*}
%
Este conjunto tiene estructura de grupoide tambi\'{e}n y adem\'{a}s es conexo.

Si $I,J\in\ideales{\cal{O},\cal{O}'}$, entonces $I$ y $J$ son isomorfos
como $\cal{O},\cal{O}'$-bim\'{o}dulos, si y s\'{o}lo si existe
$a\in K^{\times}$ tal que $J=aI$, es decir, son homot\'{e}ticos
(extendiendo escalares %tensorizando con $K$
se ve que todo isomorfismo de bim\'{o}dulos est\'{a} dado por multiplicar
por un elemento de $B$. Necesariamente este elemento est\'{a}en el centro que
es $K$). Esto determina una relaci\'{o}n de equivalencia y cada clase de
homotecia tiene \'{o}rdenes a izquierda y a derecha bien definidos.
De esta manera, las clases m\'{o}dulo homotecias de ideales invertibles de
$B$ tambi\'{e}n constituyen un grupoide con el producto compatible de
(clases de) ideales. Este grupoide admite un morfismo sobreyectivo del
grupoide de los ideales invertibles de $B$, y una componente conexa
$\Brandt{\cal{G}}$ determina un grupoide conexo asociado al g\'{e}nero
$\cal{G}$ formado por las im\'{a}genes de los conjuntos
$\ideales{\cal{O},\cal{O}'}$ con $\cal{O},\cal{O}'\in\cal{G}$.
Definimos
\begin{align*}
 \Pic{\cal{O},\cal{O}'} \,:=\, & \ideales{\cal{O},\cal{O}'}/\sim \\
	\,=\, &
	\left\lbrace [I]\,:\,I\in\ideales{\cal{O},\cal{O}'}\right\rbrace
	\text{ ,}
\end{align*}
%
donde $[I]$ denota la clase de homotecia de $I$. En general, como
$\ideales{\cal{O},\cal{O}'}$ no es un grupo, la suryecci\'{o}n
$\ideales{\cal{O},\cal{O}'}\rightarrow\Pic{\cal{O},\cal{O}'}$ determina un
morfismo
\begin{align*}
	\Brandt{\cal{G}}\,=\, & \bigcup_{\cal{O},\cal{O}'\in\cal{G}}\,
	\ideales{\cal{O},\cal{O}'}
	\,\rightarrow\,\bigcup_{\cal{O},\cal{O}'\in\cal{G}}\,
	\Pic{\cal{O},\cal{O}'}
	\text{ .}
\end{align*}
%
Pero cuando $\cal{O}=\cal{O}'$, la relaci\'{o}n de homotecia est\'{a}
dada por el subgrupo (normal) $\ppales{R}$ de ideales fraccionarios
principales de $R$, que se identifica con los ideales principales de la
forma $\cal{O}a\cal{O}\in\ideales{\cal{O}}$, $a\in K^{\times}$ v\'{\i}a
$aR\mapsto\cal{O}a\cal{O}$. En este caso, tenemos un morfismo de grupos
\begin{align*}
 \ideales{\cal{O}} & \,\rightarrow\,\Pic{\cal{O}}\,:=\,
	\ideales{\cal{O}}/\ppales{R}
	\text{ .}
\end{align*}
%
El grupo $\Pic{\cal{O}}$ es el \emph{grupo de Picard} del orden $\cal{O}$.
Notemos que la inclusi\'{o}n $\ppales{R}\hookrightarrow\ideales{\cal{O}}$
es la restricci\'{o}n de la inclusi\'{o}n de todos los ideales $\ideales{R}$,
no s\'{o}lo los principales, por $\frak{a}\mapsto\cal{O}\frak{a}\cal{O}$.

% Dentro del conjunto $\ideales{\cal{O},\cal{O}'}$ encontramos el subconjunto
% $\ppales{\cal{O},\cal{O}'}=\{\cal{O}b\cal{O}'\,:\,b\in B^{\times}\}$ de
% ideales principales. Si
% \begin{align*}
 % \normalizador{\cal{O},\cal{O}'}\,:=\, &
	% \left\lbrace \alpha\in B^{\times}\,:\,
	% \alpha^{-1}\cal{O}\alpha=\cal{O}'\right\rbrace
	% \text{ ,}
% \end{align*}
% %
% tenemos una suryecci\'{o}n
% $\normalizador{\cal{O},\cal{O}'}\rightarrow\ppales{\cal{O},\cal{O}'}$
% dada por $\alpha\mapsto\cal{O}\alpha\cal{O}'$. Si $\alpha,\alpha_{1}$
% pertenecen a $\normalizador{\cal{O},\cal{O}'}$,
% \begin{align*}
	% \cal{O}\alpha\cal{O}'\,=\,\cal{O}\alpha_{1}\cal{O}'
	% \,\Leftrightarrow\, & \cal{O}\alpha\,=\,\cal{O}\alpha_{1}
	% \,\Leftrightarrow\,\alpha\cal{O}'\,=\,\alpha_{1}\cal{O}' \\
	% \,\Leftrightarrow\, & \alpha_{1}^{-1}\alpha\in\cal{O}^{\times}
	% \,\Leftrightarrow\, \alpha\alpha_{1}^{-1}\in\cal{O}'^{\times}
	% \text{ .}
% \end{align*}
% %
Volvamos al caso especial en que $\cal{O}'=\cal{O}$. Sea
\begin{align*}
 \cal{N}\,:=\,\normalizador[B^{\times}]{\cal{O}} \,=\, &
	\left\lbrace b\in B^{\times}\,:\,b^{-1}\cal{O}b=\cal{O}
	\right\rbrace
\end{align*}
%
el normalizador de $\cal{O}$ en $B^{\times}$. %Tenemos una aplicaci\'{o}n
%sobreyectiva $\normalizador[B^{\times}]{\cal{O}}\rightarrow\ppales{\cal{O}}$
%dada por $b\mapsto\cal{O}b\cal{O}$, de la que se deduce que
Existe una sucesi\'{o}n exacta corta de grupos
\begin{center}
\begin{tikzcd}
	1\arrow{r} & \cal{O}^{\times}\arrow{r} &
	% \normalizador[B^{\times}]{\cal{O}}\arrow{r} &
	\cal{N}\arrow{r} &
	\ppales{\cal{O}}\arrow{r} & 1
\end{tikzcd}\quad\text{,}
\end{center}
%
donde %$\normalizador[B^{\times}]{\cal{O}}=\normalizador{\cal{O},\cal{O}}$,
la flecha de la izquierda es la inclusi\'{o}n y la de la derecha est\'{a}
dada por $b\mapsto\cal{O}b\cal{O}$. Notemos que
$K^{\times}\subset\cal{N}$. Para un elemento
$a\in K^{\times}$, vale que $\cal{O}a\cal{O}=\cal{O}$, si y s\'{o}lo si
$a$ es una unidad del orden $\cal{O}$ y esto equivale a que $a\in R^{\times}$.
Es decir, la imagen de $K^{\times}$ por el morfismo de la derecha es igual
a $\ppales{R}$. En particular, hay un isomorfismo
\begin{align*}
	K^{\times}\backslash \cal{N}/\cal{O}^{\times}
	\,\simeq\, & \ppales{\cal{O}}/\ppales{R}
	\text{ .}
\end{align*}
%


