Sean $I,J$ dos ideales de $B$. Decimos que $I$ y $J$ son \emph{equivalentes a %
izquierda}, si existe $b\in B^{\times}$ tal que $I=bJ$. Dado un orden $\cal{O}$
en $B$, el \emph{conjunto de clases a izquierda de $\cal{O}$}, denotado
$\lClass{\cal{O}}$, es el conjunto de clases de ideales invertibles $I$ de $B$
tales que $\Oder(I)=\cal{O}$ m\'{o}dulo la relaci\'{o}n de equivalencia a
izquierda. A diferencia de lo que ocurre en el caso de un cuerpo de
n\'{u}meros, este conjunto es, simplemente, un conjunto.

Sea $\hhat{g}=(g_{v})_{v}\in\Idfin{B}$. La familia de ret\'{\i}culos
$\{g_{v}\cal{O}_{v}\}_{v}$ determina un ideal de $B$ cuyo orden a derecha es
$\cal{O}$. Dicho ideal es, por definici\'{o}n, localmente principal y, por lo
tanto, invertible. Rec\'{\i}procamente, como todo ideal invertible es
localmente principal, la aplicaci\'{o}n $(g_{v})_{v}\mapsto I$ as\'{\i}
definida es sobreyectiva. Si $(g_{v})_{v},(g'_{v})_{v}\in\Idfin{B}$, vale
$g_{v}\cal{O}_{v}=g'_{v}\cal{O}_{v}$ para todo $v$, si y s\'{o}lo si
$g'_{v}\in g_{v}\cal{O}_{v}^{\times}$ para todo $v$. Hay una biyecci\'{o}n
\begin{align*}
	\Idfin{B}/\Idfin{\cal{O}} & \,\xrightarrow{\sim}\,
		\bigg\{\begin{array}{c}
			\text{Ideales invertibles }I\subset B\text{ con} \\
			\Oder(I)=\cal{O}
		\end{array}\bigg\} \,=\,
			\bigcup_{\cal{O}'\subset B}\,\ideales{\cal{O}',\cal{O}}
	\text{ ,}
\end{align*}
%
que determina una correspondencia entre
$B^{\times}\backslash\Idfin{B}/\Idfin{\cal{O}}$ y $\lClass{\cal{O}}$.
%
% La estructura de $\ideles{B}$ en tanto producto restringido depende,
% \textit{a priori}, de elegir un orden previamente en $B$ y de un conjunto
% finito $S$ de lugares. Pero como $S$ es finito y todos los \'{o}rdenes
% $\cal{O},\cal{O}'$ (ret\'{\i}culos completos) de $B$ satisfacen
% $\cal{O}_{v}=\cal{O}'_{v}$ para casi todo $v$, no hay diferencia.

\begin{teoFinClassNum}\label{thm:numerodeclasesfinito}
	Si $B/K$ es un \'{a}lgebra de cuaterniones sobre un cuerpo de
	n\'{u}meros y $\cal{O}\subset B$ es un orden de Eichler, el cardinal
	del conjunto $\lClass{\cal{O}}$, el \emph{n\'{u}mero de clases} de
	$\cal{O}$, es finito.\index{numero de clases@n\'{u}mero de clases}
\end{teoFinClassNum}

\begin{teoEichlerNormaClases}[Eichler]\label{thm:eichlernormaclases}
	% Si $B/K$ es un \'{a}lgebra de cuaterniones sobre un cuerpo de
	% n\'{u}meros y $\cal{O}\subset B$ es un orden de Eichler,
	La norma reducida induce una suryecci\'{o}n
	\begin{align*}
		n & \,:\, B^{\times}\backslash \Idfin{B}/\Idfin{\cal{O}}
			\,\twoheadrightarrow\,
			K_{(+)}^{\times}\backslash\Idfin{K}/\Idfin{\oka{K}}
	\end{align*}
	%
	del conjunto de clases $\lClass{\cal{O}}$, en un grupo de clases
	asociado al cuerpo $K$ y al \'{a}lgebra $B$. Si, adem\'{a}s, $B/K$ es
	indefinida, esta aplicaci\'{o}n es biyectiva.
\end{teoEichlerNormaClases}
%
% \begin{proof}
	% La norma reducida $B^{\times}\rightarrow K^{\times}$ induce morfismos
	% en las completaciones $B_{v}^{\times}\rightarrow K_{v}^{\times}$ y
	% una aplicaci\'{o}n $\ideles{B}\rightarrow\ideles{K}$ evaluando en
	% los puntos ad\'{e}licos. Por el Teorema \ref{thm:eichlernorma},
	% $\nrd(B^{\times})=K_{(+)}^{\times}$ y para $v\in\lugares[f]{K}$
	% finito $\nrd(\cal{O}_{v}^{\times})=\oka{K,v}^{\times}$. Con lo
	% cual se obtiene una aplicaci\'{o}n bien definida y sobreyectiva
	% \begin{align*}
		% n\,:\, & B^{\times}\backslash
		% \Idinf{B}\times\Idfin{B}/\Idinf{B}\Idfin{\cal{O}}
		% \,\twoheadrightarrow\,
		% K_{(+)}^{\times}\backslash
		% \Idinf{K}\times\Idfin{K}/\Idinf{K}\Idfin{\oka{K}}
		% \text{ ,}\quad\text{o bien} \\
		% n\,:\, & B^{\times}\backslash\Idfin{B}/\Idfin{\cal{O}}
		% \,\twoheadrightarrow\,
		% K_{(+)}^{\times}\backslash\Idfin{K}/\Idfin{\oka{K}}
		% \text{ .}
	% \end{align*}
	% %
	% El cociente de la izquierda est\'{a} en correspondencia con el
	% conjunto $\lClass{\cal{O}}$ y el de la derecha con el grupo de
	% clases $\cal{I}/\cal{P}_{B}$, donde $\cal{I}$
	% es el grupo de ideales fraccionarios de $K$ y $\cal{P}_{B}$ es el
	% subgrupo de ideales principales con generadores en
	% $K_{(+)}^{\times}=\nrd(B^{\times})$.
	% Si $B/K$ es indefinida, el grupo $B_{S}^{1}$ contiene a
	% $\SL_{2}(\bb{R})$ como alguno de sus factores y, por lo tanto,
	% no es compacto. Por el Teorema \ref{thm:aproxfuerte},
	% $B^{1}B_{S}^{1}$ es denso en $\adeles{B}^{1}$. Como
	% $\Idfin{\cal{O}}$ es abierto, el producto $\Idfin{\cal{O}}\Idinf{B}$
	% es abierto y contiene a $B_{S}^{1}$. Entonces
	% \begin{math}
		% \adeles{B}^{1}\subset
			% B^{\times}\cdot\Idfin{\cal{O}}\Idinf{B}
	% \end{math}
	% y $n$ resulta ser inyectiva.
% \end{proof}
%
% Expresado de otra manera, el n\'{u}cleo del morfismo (!`No es morfismo!)
% suryectivo $n$ es, en general,
% \begin{align*}
	% & B^{\times}\backslash
	% \adeles{B}^{1}/(\Idinf{B}\Idfin{\cal{O}}\cap\adeles{B}^{1})
	% \text{ .}
% \end{align*}
% %
% Si $B\not\simeq\MM_{2\times 2}(K)$, es decir, si $B$ es de divisi\'{o}n,
% este cociente es finito: en primer lugar, al ser
% $\Idinf{B}\Idfin{\cal{O}}\cap\adeles{B}^{1}$ abierto en $\adeles{B}^{1}$,
% el cociente del segundo por el primero es discreto, y, en segundo lugar,
% como sucede con un cuerpo de n\'{u}meros, $B^{\times}$ es discreto en
% $\adeles{B}^{1}$ y el cociente $B^{\times}\backslash\adeles{B}^{1}$ es
% compacto. El n\'{u}cleo de $n$ es compacto y discreto y, por lo tanto,
% finito. cuando $B/K$ es indefinida,
% $\adeles{B}^{1}\subset B^{\times}\cdot\Idinf{B}\Idfin{\cal{O}}$, con lo
% cual el n\'{u}cleo es trivial. Por otro lado, la imagen de $n$ es
% $K_{(+)}^{\times}\backslash\Idfin{K}/\Idfin{\oka{K}}$, un grupo de clases
% de $K$ y, en consecuencia, finito.
%
% Las inclusiones $B_{+}^{\times}\subset B^{\times}$ y
% $K_{+}^{\times}\subset K_{(+)}^{\times}$ determinan un diagrama conmutativo:
% \begin{center}
	% \begin{tikzcd}
		% B_{+}^{\times}\backslash\Idfin{B}/\Idfin{\cal{O}}
			% \arrow[d] \arrow[r,two heads,"\widetilde{n}"] &
		% K_{+}^{\times}\backslash\Idfin{K}/\Idfin{\oka{K}}
			% \simeq\pClass{K}
			% \arrow[d] \\
		% B^{\times}\backslash\Idfin{B}/\Idfin{\cal{O}}
			% \arrow[r,two heads,"n"] &
		% K_{(+)}^{\times}\backslash\Idfin{K}/\Idfin{\oka{K}}
			% \simeq\Class[(+)]{K}
	% \end{tikzcd}
% \end{center}
% cuyas flechas representan funciones entre conjuntos, salvo la de la derecha,
% que representa el morfismo de grupos de clases.
% % En general, todas ellas son suryecciones.

Cuando $B$ es totalmente definida, $K_{(+)}^{\times}=K_{+}^{\times}$ y la
aplicaci\'{o}n $n$ del Teorema \ref{thm:eichlernormaclases} es una
suryecci\'{o}n en el grupo de clases estrictas $\pClass{K}$. Si $B$ es
indefinida, se obtiene una biyecci\'{o}n con un cociente de este grupo,
denotado $\Class[(+)]{K}$. De todas maneras, $\pClass{K}$ aparece, si
modificamos el grupo actuando a izquierda.

\begin{coroNormaReducidaClasesEstrictas}\label{coro:normaclasesestrictas}
	La norma reducida induce una aplicaci\'{o}n sobreyectiva
	\begin{align*}
		\widetilde{n} & \,:\,
			B_{+}^{\times}\backslash\Idfin{B}/\Idfin{\cal{O}}
			\,\twoheadrightarrow\,
			K_{+}^{\times}\backslash\Idfin{K}/\Idfin{\oka{K}}
		\text{ .}
	\end{align*}
	%
	Si $B$ es indefinida, $\widetilde{n}$ es una biyecci\'{o}n.
\end{coroNormaReducidaClasesEstrictas}
%
% Las demostraciones de estos resultados se pueden encontrar en [Weil],
% [Hida], [Vign\'{e}ras], \dots
