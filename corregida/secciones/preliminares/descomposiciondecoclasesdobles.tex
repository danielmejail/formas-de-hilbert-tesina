En esta secci\'{o}n volvemos nuevamente a un \'{a}lgebra de matrices
$B=\MM_{2\times 2}(K)$ sobre un cuerpo local $K$ no arquimediano (m\'{a}s
precisamente una extensi\'{o}n finita de $\bb{Q}_{p}$). Sea $R\subset K$ su
anillo de enteros y sea $\pi\in R$ un elemento primo, un generador del
\'{u}nico ideal maximal $P=\pi R$. Si $x\in K$, sea $\theta(x)\in\bb{Z}$ tal
que existe $u\in R^{\times}$ con $x=u\pi^{\theta(x)}$. Denotamos por $\varpi$
la matriz
\begin{math}
	\begin{bmatrix} \pi & \\ & 1 \end{bmatrix}
\end{math}~.
Dado $e\geq 0$ un entero no negativo, $\cal{O}_{0}(\pi^{e}R)$ denota el orden
de Eichler est\'{a}ndar de nivel $\pi^{e}R$, es decir,
\begin{math}
	\cal{O}_{0}(\pi^{e}R)=
		\begin{bmatrix} R & R \\ \pi^{e}R & R \end{bmatrix}
\end{math}~.
Si $e=0$, entonces
\begin{align*}
	\cal{O}_{0}(R)^{\times} & \,=\,\GL_{2}(R) \,=\,
		\MM_{2\times 2}(R)\cap\big\{\det\in R^{\times}\big\}
	\text{ .}
\end{align*}
%
En general, si $e\geq 1$,
\begin{align*}
	\cal{O}_{0}(\pi^{e}R)^{\times} & \,=\,\cal{O}_{0}(\pi^{e}R)\cap
		\big\{\det\in R^{\times}\big\}
	\text{ .}
\end{align*}
%

\begin{propoDescomposicionEnCoclasesADerecha}
	\label{thm:descomposicionencoclasesaderecha}
	Sea $e\geq 0$ y sean $\Gamma=\cal{O}_{0}(\pi^{e}R)^{\times}$ y
	$\Gamma'=(\varpi^{-1}\Gamma\varpi)\cap\Gamma$. Entonces, si $e=0$,
	existe un sistema de representantes $c_{i}\in R$ las clases $i$ en el
	cuerpo residual $R/(\pi)$ tal que
	\begin{align*}
		\Gamma & \,=\, \bigsqcup_{i\in R/(\pi)}\,
			\Gamma'\begin{bmatrix} 1 & \\ c_{i} & 1 \end{bmatrix}
			\,\sqcup\,
			\Gamma'\begin{bmatrix} & 1 \\ 1 & \end{bmatrix}
		\text{ .}
	\end{align*}
	%
	Si $e\geq 1$, entonces
	\begin{align*}
		\Gamma & \,=\, \bigsqcup_{i\in R/(\pi)}\,
			\Gamma'\begin{bmatrix} 1 & \\
				c_{i}\pi^{e} & 1 \end{bmatrix}
		\text{ ,}
	\end{align*}
	%
	con el mismo sistema de representantes $\{c_{i}\}_{i\in R/(\pi)}$.
\end{propoDescomposicionEnCoclasesADerecha}

\begin{proof}
	En primer lugar,
	\begin{math}
		\Gamma'=\big\{\gamma\in\Gamma\,:\,
			\varpi\gamma\varpi^{-1}\in\Gamma\big\}
	\end{math}~.
	Pero
	\begin{align*}
		\begin{bmatrix} \pi & \\ & 1 \end{bmatrix}
			\begin{bmatrix} a & b \\ c & d \end{bmatrix}
			\begin{bmatrix} \pi & \\ & 1 \end{bmatrix}^{-1}
		& \,=\,\begin{bmatrix} a & \pi b \\ \pi^{-1}c & d \end{bmatrix}
		\text{ .}
	\end{align*}
	%
	Entonces
	\begin{align*}
		\Gamma' & \,=\,\bigg\{
			\begin{bmatrix} a & b \\ c & d \end{bmatrix}
			\in\cal{O}_{0}(\pi^{e}R)\,:\,
			ad-bc\in R^{\times},\,c\in (\pi^{e+1})\bigg\}
		\text{ .}
	\end{align*}
	%
	Sea
	\begin{math}
		\begin{bmatrix} a & b \\ c & d \end{bmatrix}\in\Gamma
	\end{math}
	y sea $u=ad-bc\in R^{\times}$ su determinante. Supongamos primero que
	$e=0$. Entonces podemos distinguir dos casos: $\theta(d)=0$ y $d$ es
	invertible en $R$, o $\theta(d)>0$. En el primer caso,
	\begin{align*}
		\begin{bmatrix} u^{-1}d & -u^{-1}b \\ & d^{-1} \end{bmatrix}
			\begin{bmatrix} a & b \\ c & d \end{bmatrix}
		& \,=\, \begin{bmatrix} 1 & \\ d^{-1}c & 1 \end{bmatrix}
		\text{ ,}
	\end{align*}
	%
	donde $c\in R$ es, en principio, un entero arbitrario. Dos matrices
	de este tipo pertenecen a la misma coclase a derecha de $\Gamma$ con
	respecto a $\Gamma'$, si y s\'{o}lo si
	\begin{align*}
		\begin{bmatrix} 1 & \\ c' & 1 \end{bmatrix}
			\begin{bmatrix} 1 & \\ c & 1 \end{bmatrix}^{-1}
		& \,=\,\begin{bmatrix} 1 & \\ c'-c & 1 \end{bmatrix}
			\,\in\,\Gamma'
		\text{ ,}
	\end{align*}
	%
	o, lo que es lo mismo, $c'-c\in (\pi)$. En el segundo caso,
	\begin{align*}
		\begin{bmatrix} -u^{-1}c & u^{-1}a \\ & c^{-1} \end{bmatrix}
			\begin{bmatrix} a & b \\ c & d \end{bmatrix}
		& \,=\, \begin{bmatrix} & 1 \\ 1 & c^{-1}d \end{bmatrix}
		\text{ .}
	\end{align*}
	%
	Como antes, si $d$ fuese un entero arbitrario, dos matrices de este
	tipo ser\'{\i}an $\Gamma'$-equivalentes, si y s\'{o}lo si
	$d'-d\in (\pi)$. Pero, asumiendo $\theta(d)>0$, el coeficiente
	$d$ de cualquiera de estas matrices pertenece al ideal maximal y, en
	consecuencia, se obtienen matrices pertenencientes a la misma coclase.
	En definitiva, fijando un sistema de representantes
	$\{c_{i}\in R\}_{i\in R/(\pi)}$ de las clases en el cuerpo residual,
	se deduce la primera de las igualdades en el enunciado. Si asumimos
	que $e\geq 1$, entonces el argumento dado en el primero de los dos
	casos anteriores sigue siendo v\'{a}lido, reemplazando la condici\'{o}n
	$c'-c\in (\pi)$ por $c'-c\in (\pi^{e})$. Por otro lado,
	$\theta(c)\geq e\geq 1$ implica $\theta(d)$, con lo que se deduce que
	el segundo caso es trivial. En definitiva, las coclases a derecha
	de $\Gamma$ con respecto a $\Gamma'$ est\'{a}n indexadas por
	representantes del cociente $(\pi^{e})/(\pi^{e+1})$, que,
	v\'{\i}a $c\mapsto c\pi^{e}$, est\'{a}n en correspondencia con el
	sistema $\{c_{i}\}_{i\in R/(\pi)}$.
\end{proof}

En el caso $e=0$, el mismo sistema de representantes de las coclases a derecha
con respecto a $\Gamma'$ determina un sistema de representantes de las coclases
a derecha con respecto al subgrupo conjugado $\varpi\Gamma'\varpi^{-1}$.

\begin{propoDescomposicionEnCoclasesADerechaSubgrupoConjugado}
	\label{thm:descomposicionencoclasesaderechasubgrupoconjugado}
	Sea
	\begin{math}
		\Gamma''=\varpi\Gamma'\varpi^{-1}=
			(\varpi\Gamma\varpi^{-1})\cap\Gamma
	\end{math}~.
	Si $e=0$, entonces
	\begin{align*}
		\Gamma & \,=\,\bigsqcup_{i\in R/(\pi)}\,
			\Gamma'\begin{bmatrix} & 1 \\ 1 & d_{i} \end{bmatrix}
				\,\sqcup\,
			\Gamma'\begin{bmatrix} 1 & \\ & 1 \end{bmatrix}
			\,=\,\bigsqcup_{i\in R/(\pi)}\,
			\Gamma''\begin{bmatrix} a_{i} & 1 \\ 1 & \end{bmatrix}
				\,\sqcup\,
			\Gamma''\begin{bmatrix} 1 & \\ & 1 \end{bmatrix}
		\text{ ,}
	\end{align*}
	%
	donde los coeficientes $a_{i}$ y los coeficientes $d_{i}$ forman
	sistemas de representantes de las clases en $R/(\pi)$.
\end{propoDescomposicionEnCoclasesADerechaSubgrupoConjugado}

\begin{proof}
	Una matriz $\gamma$ pertenece al subgrupo $\Gamma''$, si y s\'{o}lo si
	$\gamma\in\Gamma$ y $\varpi^{-1}\gamma\varpi\in\Gamma$, tambi\'{e}n.
	Entonces $\Gamma''$ es igual a
	\begin{align*}
		\Gamma'' & \,=\,\bigg\{
			\begin{bmatrix} a & b \\ c & d \end{bmatrix}
			\in\cal{O}_{0}(\pi^{e}R)\,:\,ad-bc\in R^{\times},\,
				b\in (\pi)\bigg\}
		\text{ .}
	\end{align*}
	%
	Supongamos que $e=0$ y sea
	\begin{math}
		\begin{bmatrix} a & b \\ c & d \end{bmatrix}
			\in\Gamma
	\end{math}~.
	Este elemento pertenece a $\Gamma''$, si y s\'{o}lo si $\theta(b)>0$.
	En otro caso, $b\in R^{\times}$ y
	\begin{align*}
		\begin{bmatrix} b^{-1} & \\ u^{-1}d & -u^{-1}b \end{bmatrix}
			\begin{bmatrix} a & b \\ c & d \end{bmatrix}
		& \,=\, \begin{bmatrix} b^{-1}a & 1 \\ 1 & \end{bmatrix}
		\text{ .}
	\end{align*}
	%
	% La matriz multiplicando a izquierda, pertenece a $\Gamma''$, porque
	% $e=0$; si, en cambio, $e\geq 1$, esta matriz no pertenece a
	% $\Gamma''$, porque el coeficiente $d$ est\'{a} forzado a ser una
	% unidad.
	Dos matrices de este tipo pertenecen a la misma coclase a derecha con
	respecto a $\Gamma''$, si y s\'{o}lo si
	\begin{align*}
		\begin{bmatrix} a' & 1 \\ 1 & \end{bmatrix}
			\begin{bmatrix} a & 1 \\ 1 & \end{bmatrix}^{-1}
		& \,=\,\begin{bmatrix} 1 & a'-a \\ & 1 \end{bmatrix}
	\end{align*}
	%
	pertenece a $\Gamma''$, es decir, si y s\'{o}lo si $a'-a\in (\pi)$.
	La descomposici\'{o}n con respecto al subgrupo $\Gamma'$ es v\'{a}lida,
	pues, para $c\in R^{\times}$,
	\begin{align*}
		\begin{bmatrix} -c & 1 \\ & c^{-1} \end{bmatrix}
			\begin{bmatrix} 1 & \\ c & 1 \end{bmatrix}
		& \,=\, \begin{bmatrix} & 1 \\ 1 & c^{-1} \end{bmatrix}
	\end{align*}
	%
	y dos matrices de este tipo pertenecen a la misma coclase con respecto
	a $\Gamma'$, si y s\'{o}lo si los coeficientes de la esquina inferior
	derecha son congruentes m\'{o}dulo $(\pi)$.
\end{proof}

\begin{coroDescomposicionEnCoclasesADerechaSubgrupoConjugado}%
	\label{coro:descomposicionencoclasesaderechasubgrupoconjugado}
	Sea $\{d_{i}\}_{i\in R/(\pi)}$ un sistema de representantes de las
	clases en $R/(\pi)$ y, para cada clase $i$, sea
	\begin{math}
		\gamma_{i}=\begin{bmatrix} & 1 \\ 1 & d_{i} \end{bmatrix}
	\end{math}~.
	Sea $I$ la matriz identidad. Entonces el conjunto
	$\{\gamma_{i}\}\cup\{I\}$ constituye un sistema de representantes de
	las coclases a derecha con respecto a $\Gamma'$ en $\Gamma$ y el
	conjunto $\{\gamma_{i}^{-1}\}\cup\{I\}$ constituye un sistema de
	representantes con respecto a $\Gamma''$.
\end{coroDescomposicionEnCoclasesADerechaSubgrupoConjugado}

\begin{proof}
	La primera afirmaci\'{o}n es parte del enunciado de la proposici\'{o}n
	\ref{thm:descomposicionencoclasesaderechasubgrupoconjugado}. La
	segunda afirmaci\'{o}n se deduce de que $-I\in\Gamma$.
	% y de que $-I$ es central.
\end{proof}

Pasamos, ahora, a la descomposici\'{o}n de coclases dobles.

\begin{propoDescomposicionDeLaCoclaseDobleMatrices}%
	\label{thm:descomposiciondelacoclasedoblematrices}
	Las descomposiciones de $\Gamma$ en coclases a derecha con respecto a
	$\Gamma'$ mencionadas en las propocisiones
	\ref{thm:descomposicionencoclasesaderecha} y
	\ref{thm:descomposicionencoclasesaderechasubgrupoconjugado} inducen,
	por multiplicaci\'{o}n a izquierda por $\varpi$, descomposiciones
	\begin{equation}
		\label{eq:descomposiciondelacoclasedoblematrices}
		\begin{aligned}
			& \GL_{2}(R)\begin{bmatrix} \pi & \\
				& 1 \end{bmatrix}\GL_{2}(R)
			\,=\,\bigsqcup_{i\in R/(\pi)}\,\GL_{2}(R)
				\begin{bmatrix} \pi & \\c_{i} & 1\end{bmatrix}
				\,\sqcup\,\GL_{2}(R)
				\begin{bmatrix} 1 & \\ & \pi \end{bmatrix} \\
			& \qquad\qquad\,=\,
			\bigsqcup_{i\in R/(\pi)}\,\GL_{2}(R)
				\begin{bmatrix} 1 & b_{i} \\ & \pi\end{bmatrix}
				\,\sqcup\,\GL_{2}(R)
				\begin{bmatrix} \pi & \\ & 1 \end{bmatrix} 
		\text{ ,}
		\end{aligned}
	\end{equation}
	%
	donde los coeficientes $c_{i}$ y los coeficientes $b_{i}$ forman
	sistemas de representantes de las clases en $R/(\pi)$. Si $e\geq 1$,
	entonces
	\begin{equation}
		\label{eq:descomposiciondelacoclasedoblematricesbis}
		\cal{O}_{0}(\pi^{e}R)^{\times}
			\begin{bmatrix} \pi & \\ & 1 \end{bmatrix}
				\cal{O}_{0}(\pi^{e}R)^{\times}
			\,=\,\bigsqcup_{i\in R/(\pi)}\,
				\cal{O}_{0}(\pi^{e}R)^{\times}
			\begin{bmatrix} \pi & \\
				c_{i}\pi^{e} & 1\end{bmatrix}
		\text{ .}
	\end{equation}
\end{propoDescomposicionDeLaCoclaseDobleMatrices}

\begin{proof}
	Sea $e\geq 0$ y sea $\Gamma=\cal{O}_{0}(\pi^{e}R)^{\times}$. Para
	obtener un sistema de representantes de
	\begin{math}
		\Gamma\backslash\Gamma\varpi\Gamma
	\end{math}
	basta con multiplicar a izquierda por $\varpi$ los elementos en un
	sistema de representantes de
	\begin{math}
		(\varpi^{-1}\Gamma\varpi)\cap\Gamma\backslash\Gamma
	\end{math}~.
	En particular, partiendo de los representantes de la proposici\'{o}n
	\ref{thm:descomposicionencoclasesaderecha},
	\begin{align*}
		\begin{bmatrix} \pi & \\ & 1 \end{bmatrix}
			\begin{bmatrix} 1 & \\ c_{i}\pi^{e} & 1 \end{bmatrix}
		& \,=\,\begin{bmatrix} \pi & \\ c\pi^{e} & 1 \end{bmatrix}
		\text{ .}
	\end{align*}
	%
	En el caso $e=0$,
	\begin{align*}
		\begin{bmatrix} & 1 \\ 1 & \end{bmatrix}
			\begin{bmatrix} \pi & \\ & 1 \end{bmatrix}
		\begin{bmatrix} & 1 \\ 1 & \end{bmatrix} & \,=\,
			\begin{bmatrix} 1 & \\ & \pi \end{bmatrix}
		\text{ .}
	\end{align*}
	%
	La descomposici\'{o}n \eqref{eq:descomposiciondelacoclasedoblematrices}
	se deduce de que
	\begin{math}
		\left[\begin{smallmatrix} & 1 \\ 1 & \end{smallmatrix}\right]
			\in\GL_{2}(R)
	\end{math}~.
	Para obtener la segunda descomposici\'{o}n en el caso $e=0$, se parte
	del conjunto de representantes de la proposici\'{o}n
	\ref{thm:descomposicionencoclasesaderechasubgrupoconjugado}.
\end{proof}

\begin{propoDescomposicionDeLaCoclaseDobleAIzqYADer}%
	\label{thm:descomposiciondelacoclasedobleaizyader}
	Sea $\{d_{i}\}_{i}$ un sistema de representantes de las clases en
	$R/(\pi)$, con $d_{i}$ perteneneciente a la clase $i$. Entonces
	\begin{equation}
		\label{eq:descomposiciondelacoclasedobleaizqyader}
		\begin{aligned}
			& \GL_{2}(R)\begin{bmatrix} \pi & \\ & 1 \end{bmatrix}
				\GL_{2}(R) \\
			& \qquad\qquad\,=\,
			\bigsqcup_{i\in R/(\pi)}\,\GL_{2}(R)
				\begin{bmatrix} 1 & d_{i} \\
					d_{i} & \pi + d_{i}^{2} \end{bmatrix}
			\,\sqcup\,\GL_{2}(R)
				\begin{bmatrix} \pi & \\ & 1 \end{bmatrix} \\
			& \qquad\qquad\,=\,
			\bigsqcup_{i\in R/(\pi)}\,
				\begin{bmatrix} 1 & d_{i} \\
					d_{i} & \pi + d_{i}^{2} \end{bmatrix}
						\GL_{2}(R) \,\sqcup\,
				\begin{bmatrix} \pi & \\ & 1 \end{bmatrix}
					\GL_{2}(R)
			\text{ .}
		\end{aligned}
	\end{equation}
	%
\end{propoDescomposicionDeLaCoclaseDobleAIzqYADer}

\begin{proof}
	Sea $\Gamma=\GL_{2}(R)$ y, para cada $i\in R/(\pi)$, sea
	\begin{math}
		\gamma_{i}=\begin{bmatrix} & 1 \\ 1 & d_{i} \end{bmatrix}
	\end{math}~.
	Por el corolario
	\ref{coro:descomposicionencoclasesaderechasubgrupoconjugado},
	\begin{align*}
		\Gamma\varpi\Gamma & \,=\,\bigsqcup_{i\in R/(\pi)}\,
			\Gamma\varpi\gamma_{i}
				\,\sqcup\,\Gamma\varpi
			\quad\text{y} \\
		\Gamma\varpi^{-1}\Gamma & \,=\,\bigsqcup_{i\in R/(\pi)}\,
			\Gamma\varpi^{-1}\gamma_{i}^{-1}
				\,\sqcup\,\Gamma\varpi^{-1}
		\text{ .}
	\end{align*}
	%
	De la segunda descomposici\'{o}n, se deduce
	\begin{align*}
		\Gamma\varpi\Gamma & \,=\,\bigsqcup_{i\in R/(\pi)}\,
			\gamma_{i}\varpi\Gamma
				\,\sqcup\,\varpi\Gamma
		\text{ .}
	\end{align*}
	%
	Ahora bien, $\varpi\in\Gamma\varpi\cap\varpi\Gamma$ y, por lo tanto,
	\begin{math}
		\gamma_{j}\varpi\gamma_{j}\in\Gamma\varpi\gamma_{j}\cap
			\gamma_{j}\varpi\Gamma
	\end{math}~.
	Este producto es igual a
	\begin{align*}
		\begin{bmatrix} & 1 \\ 1 & d_{j} \end{bmatrix}
			\begin{bmatrix} \pi & \\ & 1 \end{bmatrix}
		\begin{bmatrix} & 1 \\ 1 & d_{j} \end{bmatrix} & \,=\,
			\begin{bmatrix} 1 & d_{j} \\
				d_{j} & \pi + d_{j}^{2}\end{bmatrix}
		\text{ .}
	\end{align*}
	%
\end{proof}
