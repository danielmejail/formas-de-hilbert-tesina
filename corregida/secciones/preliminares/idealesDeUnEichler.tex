
\paragraph{Ideales de un orden de Eichler}
Queremos ver c\'{o}mo son los ideales de un orden de Eichler. Sea
$\cal{O}:=\cal{O}_{0}(\pi^{e}R)$ el orden de Eichler est\'{a}ndar de nivel
$\pi^{e}R$.
%, como ya mencionamos, el discriminante reducido de $\cal{O}$
%es $\drd{\cal{O}}=\pi^{e}R$ igual al nivel.
Recordando que estamos sobre un cuerpo local,
$\ideales{\cal{O}}=\ppales{\cal{O}}$, con lo cual
\begin{align*}
 K^{\times}\backslash\normalizador[B^{\times}]{\cal{O}}/\cal{O}^{\times}
	\,\simeq\, & \ppales{\cal{O}}/\ppales{R}
	\,=\, \ideales{\cal{O}}/\ppales{R}
	\text{ .}
\end{align*}
%
Definimos un elemento $\varpi\in B$ por
\begin{align*}
	\varpi\,:=\, & \begin{bmatrix} & 1 \\ \pi^{e} & \end{bmatrix}
		\text{ .}
\end{align*}
%
El orden de Eichler es igual a la intersecci\'{o}n
\begin{align*}
 \cal{O} \,=\, & \MM_{2\times 2}(R)\,\cap\,
	\varpi^{-1}\MM_{2\times 2}(R)\varpi
	\text{ .}
\end{align*}
%
Identificando $K$ con las matrices escalares, vemos que $\varpi^{2}=\pi^{e}$
y que $\varpi$ pertenece a $\cal{N}:=\normalizador[B^{\times}]{\cal{O}}$.
Sea $I:=\cal{O}\varpi$. Es decir, $\cal{O}\varpi=\varpi\cal{O}$ e $I$ es un
ideal principal cuyos \'{o}rdenes a izquierda y a derecha coinciden y son
iguales a $\cal{O}$. Si $e=0$, $I=\cal{O}$. Supongamos que $e\geq 1$.

Sea $\alpha\in\cal{N}$. De acuerdo con la proposici\'{o}n
???, los \'{o}rdenes $\MM_{2\times 2}(R)$ y su conjugado constituyen el
\'{u}nico par de \'{o}rdenes maximales cuya intersecci\'{o}n es $\cal{O}$.
Como conjugar por $\alpha$ estabiliza $\cal{O}$, los elementos del grupo
$\cal{N}$ permutan los dos \'{o}rdenes y queda determinado un morfismo
sobreyectivo $\cal{N}\rightarrow C_{2}$ en el grupo c\'{\i}clico de orden
$2$. El n\'{u}cleo del morfismo est\'{a} incluido en el normalizador
$\normalizador[B^{\times}]{\MM_{2\times 2}(R)}$, que es igual a
$K^{\times}\GL_{2}(R)$. Entonces
\begin{align*}
 \ker(\cal{N}\rightarrow C_{2}) \,=\, &
	\normalizador[B^{\times}]{\MM_{2\times 2}(R)}\,\cap\,
	\varpi^{-1}\normalizador[B^{\times}]{\MM_{2\times 2}(R)}\varpi \\
 \,=\, & K^{\times}
	\left(\GL_{2}(R)\,\cap\,\varpi^{-1}\GL_{2}(R)\varpi\right) \\
 \,=\, & K^{\times}\cal{O}^{\times}
	\quad\text{y} \\
	\ideales{\cal{O}}/\ppales{R}\,=\, & \langle\varpi\rangle\,\simeq\,
	C_{2}
	\text{ .}
\end{align*}
%
Concluimos, as\'{\i}, que el grupo $\ideales{\cal{O}}$ es abeliano,
generado por $I=\varpi\cal{O}$ y $P=\pi\cal{O}$ con la \'{u}nica relaci\'{o}n
$I^{2}=\pi^{e}\cal{O}=P^{e}$. En particular, si $e=0$,
$\cal{O}=\MM_{2\times 2}(R)$ es maximal y los ideales bil\'{a}teros
forman un grupo libre generado por $P$.