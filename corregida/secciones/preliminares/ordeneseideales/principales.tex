Sea $I=\cal{O}h$ un ideal principal de $B$ y sea $\cal{O}'=h^{-1}\cal{O}h$ su
orden a derecha. Entonces
\begin{align*}
	I^{-1} & \,=\,h^{-1}\cal{O}\,=\,\cal{O}'h^{-1}
	\text{ .}
\end{align*}
%
En particular, $I$ es invertible. Si $I'=\cal{O}'h'$, entonces
\begin{align*}
	I\cdot I' & \,=\,\cal{O}hh'\,=\,hh'\cal{O}''
	\text{ ,}
\end{align*}
%
donde $\cal{O}''=(h')^{-1}\cal{O}'h'$. En general, el producto coherente de
ideales principales es principal y sea cumple
\begin{align*}
	\Oizq(IJ) \,=\,\Oizq(I) & \quad\text{y}\quad
		\Oder(IJ) \,=\,\Oder(J)
	\text{ .}
\end{align*}
%
El pseudoinverso de un ideal principal tambi\'{e}n es principal y es un
verdadero inverso.

La norma reducida de un ideal principal $I=\cal{O}h$ es el ideal principal
\begin{align*}
	\nrd(I) & \,=\,R\,\nrd(h)
\end{align*}
%
en el cuerpo $K$. El inverso de un ideal principal $I$ se puede escribir en
t\'{e}rminos de su norma reducida:
\begin{align*}
	I^{-1} & \,=\,\lconj{I}\,\nrd(I)^{-1}
	\text{ ,}
\end{align*}
%
donde $\lconj{I}=\big\{\conj{x}\,:\,x\in I\big\}$.
% (no es necesariamente cierto que $\nrd(\cal{O})=R$).
Si $J=\cal{O}'h'$, de manera que $IJ$ sea coherente, entonces
\begin{align*}
	\nrd(IJ) & \,=\,\nrd(I)\,\nrd(J)
	\text{ .}
\end{align*}
%

