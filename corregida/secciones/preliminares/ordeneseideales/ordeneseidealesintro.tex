Sean $R$ un anillo de Dedekind, $K$ su cuerpo de fracciones, $B/K$ un
\'{a}lgebra de cuaterniones sobre $K$ y $V/K$ un espacio vectorial sobre $K$.
Un \emph{($R$-)ret\'{\i}culo} en $V$ es un\index{reticulo@ret\'{\i}culo}
$R$-m\'{o}dulo $L\subset V$ finitamente generado. Un elemento $x\in V$ se dice
\emph{\'{\i}ntegro}, si $R[x]$ es un ret\'{\i}culo.\index{integro@\'{\i}ntegro}
Un ret\'{\i}culo $L\subset V$ se dice \emph{completo}, si
$L\tensor[R]K=V$, si $L$ contiene una $K$-base de $V$.

\begin{propoEquivalenciaIntegro}\label{propo:equivalenciaintegro}
	Si $V=B$, $x\in B$ es \'{\i}ntegro, si y s\'{o}lo si
	$\trd(x),\nrd(x)\in R$.
\end{propoEquivalenciaIntegro}

Un ret\'{\i}culo completo en $B$ se denomina \emph{ideal}. Un\index{ideal}
\emph{orden} en $B$ es un ideal $I\subset B$ que cumple\index{orden}
$I\cdot I\subset I$ y $1\in I$, es decir, un ideal que es, adem\'{a}s, un
subanillo (con unidad) de $B$. Todo elemento perteneciente a un orden es
\'{\i}ntegro ($R$ es noetheriano y los \'{o}rdenes son finitamente
generados).

\begin{propoEquivalenciaOrden}\label{propo:equivalenciaorden}
	Un subconjunto $\cal{O}\subset B$ es un orden, si y s\'{o}lo si
	$\cal{O}$ es subanillo de $B$, $R\subset\cal{O}$, los elementos de
	$\cal{O}$ son \'{\i}ntegros y $\cal{O}\cdot K=B$.
	% ($\cal{O}$ contiene una $K$-base de $B$).
\end{propoEquivalenciaOrden}

\begin{obsUnidadesDeUnOrden}\label{obs:unidadesdeunorden}
	Sea $x\in\cal{O}$, un orden de $B$. En general,
	$x^{-1}=\conj{x}\,nrd(x)^{-1}$ y $\conj{x}\in\cal{O}$ (por
	\ref{propo:equivalenciaintegro}). Si $x\in\cal{O}^{\times}$,
	$x^{-1}\in\cal{O}$ y $\nrd(x)^{-1}\in R$. Rec\'{\i}procamente, si
	$\nrd(x^{-1})\in R$, $x^{-1}\in\cal{O}$. As\'{\i}, se deduce que
	$x\in\cal{O}^{\times}$, si y s\'{o}lo si $\nrd(x)\in R^{\times}$.
\end{obsUnidadesDeUnOrden}

Si $L,L'\subset V$ son ret\'{\i}culos, $L\cap L'$ es un ret\'{\i}culo. Si
$L'$ es completo,
\begin{math}
	\dim(L\cap L'\cdot K) =\dim(L\cdot K)
\end{math}
%
(basta expresar los elementos de $L$ como combinaciones lineales de una
$K$-base de $V$ contenida en $L'$).
% $\{\beta_{i}\}_{i}\subset L$ algo, $\{\beta_{k}'\}_{k}\subset L'$ base de $V$.
% $\beta_{i}=b^{k}_{i}\beta_{k}'$,
% $b^{k}_{i}=\gamma^{k}_{i}/\delta^{k}_{i}\in K$,
% $\tilde{\delta}_{i}=\prod_{k}\,\delta^{k}_{i}\in R$,
% $\tilde{\gamma}^{k}_{i}=a^{k}_{i}\tilde{\delta}_{i}\in R$.
% Entonces
% $\tilde{\delta}_{i}\beta_{i}=\tilde{\gamma}^{k}_{i}\beta_{k}'$.
% El lado izquierdo pertenece a $L$ y el lado derecho a $L'$.
%
En particular, la intersecci\'{o}n de dos
ideales es un ideal y la intersecci\'{o}n de dos \'{o}rdenes es un orden. Un
orden se dice \emph{maximal}, si no est\'{a} contenido\index{orden!maximal}
propiamente en otro orden de $B$. Todo orden est\'{a} contenido en alg\'{u}n
orden maximal (por \ref{propo:equivalenciaorden}). Un orden de la forma
$\cal{O}\cap\cal{O}'$ con $\cal{O},\cal{O}'$ maximales en $B$ se denomina
\emph{orden de Eichler}.\index{orden!de Eichler}

