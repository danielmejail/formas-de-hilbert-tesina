Se dice que $B$ es \emph{indefinida}, si $B$ ramifica en alg\'{u}n lugar
% si NO ramifica en alg\'{u}n lugar arquimediano
arquimediano. Si $B_{v}$ es de divisi\'{o}n para todo
$v\in\lugares[\infty]{K}$, se dice que $B$ es \emph{(totalmente) definida}
(esto \'{u}ltimo solamente es posible cuando $K/\bb{Q}$ es totalmente real).
\index{algebra definida@\'{a}lgebra (totalmente) definida}
\index{algebra indefinida@\'{a}lgebra indefinida}\index{totalmente real}
% % Sea $v\in\lugares[\infty]{K}$ un lugar arquimediano de $K$. Decimos que $v$
% % es real, si $K_{v}\simeq\bb{R}$.
% Sea $K_{(+)}^{\times}\subset K^{\times}$ el subconjunto de elementos cuya
% imagen en $K_{v}$ es positiva para todo lugar arquimediano $v$ en donde $B$
% ramifica (necesariamente $K_{v}\simeq\bb{R}$).
%
\begin{teoEichlerNorma}%[de la imagen de la norma]%
	\label{thm:eichlernorma}
	La imagen del morfismo $\nrd:\,B^{\times}\rightarrow K^{\times}$ es
	\begin{align*}
		\nrd(B^{\times}) & \,=\, K_{(+)}^{\times}\,=\,
			\Big\{x\in K\,:\,x_{v}>0\text{ , si }
				v\in\lugares[\infty]{K}\cap\Ram(B)\Big\}
		\text{ .}
	\end{align*}
	%
\end{teoEichlerNorma}

% \begin{proof}[Demostraci\'{o}n]
%  Ver [Weil, XI \S 3]
% \end{proof}

Si $K/\bb{Q}$ es una extensi\'{o}n totalmente real, un elemento $x\in K$ es
\emph{totalmente positivo}, si $x_{v}>0$ para todo lugar arquimediano
$v\in\lugares[\infty]{K}$. Escribimos $x\gg 0$ para indicar que $x$ es
totalmente positivo y denotamos por
\begin{math}
	K_{+}^{\times}=\{x\in K^{\times}\,:\, x \gg 0\}
\end{math}
el subgrupo de elementos totalmente positivos de $K$. Dada un \'{a}lgebra de
cuaterniones $B/K$, los elementos cuya norma reducida es totalmente positiva
conforman un subgrupo que denotamos por
\begin{math}
	B_{+}^{\times}=\{\gamma\in B^{\times}\,:\,
		\nrd(\gamma)\gg 0\}
\end{math}~.

\begin{obsEichlerNorma}\label{obs:eichlernorma}
	Si el grado de la extensi\'{o}n $K/\bb{Q}$ es igual a $n=[K:\bb{Q}]$ y
	$B/K$ ramifica en $n-r$ lugares arquimedianos, entonces
	\begin{math}
		B_{\infty}\simeq
			\MM_{2\times 2}(\bb{R})^{r}\times\bb{H}^{n-r}
	\end{math}
	y
	\begin{align*}
		B^{\times}/B_{+}^{\times} & \,\simeq\,
			K_{(+)}^{\times}/K_{+}^{\times} \,\simeq\,
			(\bb{Z}/2\bb{Z})^{r}
		\text{ ,}
	\end{align*}
	%
	v\'{\i}a la norma reducida.
\end{obsEichlerNorma}

