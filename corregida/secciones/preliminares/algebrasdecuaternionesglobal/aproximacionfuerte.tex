Sea $S\subset\lugares{K}$ un subconjunto finito. Sea $B^{1}$
el subgrupo de $B^{\times}$ de elementos de norma reducida uno, sea
$B_{v}^{1}$ el subgrupo correspondiente en $B_{v}^{\times}$ y sea
$B_{S}^{1}:=\prod_{v\in S}\,B_{v}^{1}$. Cada $B_{v}^{1}$ es compacto, si
y s\'{o}lo si $B$ ramifica en $v$ y $B_{S}^{1}$ es compacto, si y s\'{o}lo
si $S\subset\Ram(B)$. Dado un orden $\cal{O}\subset B$, sean
$\cal{O}^{1}=\cal{O}^{\times}\cap B^{1}$,
$\cal{O}_{v}^{1}=\cal{O}_{v}^{\times}\cap B_{v}^{1}$ y $\adeles{B}^{1}$ el
producto restringido de los grupos $B_{v}^{1}$ respecto de los subgrupos
compactos $\cal{O}_{v}^{1}$ (siempre que est\'{e}n definidos).

\begin{teoAproxFuerte}[de aproximaci\'{o}n fuerte, %
	{\cite[Th\'{e}or\`{e}me~III.4.3]{Vigneras}}]%
	\label{thm:aproxfuerte}
	Sea $S\subset\lugares{K}$ un subconjunto finito que contiene al menos
	un lugar arquimediano. Entonces, si $B_{S}^{1}$ no es compacto,
	$B^{1}\cdot B_{S}^{1}$ es denso en $\adeles{B}^{1}$.
\end{teoAproxFuerte}

De ahora en adelante, tomamos $S=\lugares[\infty]{K}$. Entonces $B_{S}^{1}$ es
compacto, si y s\'{o}lo si $B$ es totalmente definida.

