Sea $B/K$ un \'{a}lgebra de cuaterniones sobre $K$. Dado $v\in\lugares{K}$, se
dice que \emph{$B$ ramifica en $v$}, si el \'{a}lgebra de cuaterniones
$B_{v}/K_{v}$ pertenece a la clase de las \'{a}lgebras de divisi\'{o}n. El
\emph{conjunto de ramificaci\'{o}n de $B$} es el conjunto de lugares $v$ de $K$
en donde $B$ ramifica. Denotamos este conjunto por $\Ram(B)$.
\index{ramificacion@ramificaci\'{o}n}
\index{ramificacion@ramificaci\'{o}n!conjunto de}

\begin{obsRamEsFinita}\label{obs:ramesfinita}
	El conjunto $\Ram(B)$ es un conjunto finito. Sea $L$ el ret\'{\i}culo
	en $B$ generado por una $K$-base. Entonces $L$ es un ret\'{\i}culo
	completo. Por la Proposici\'{o}n~\ref{propo:localgloballattices}, para
	casi todo lugar finito $v$, el ret\'{\i}culo local $L_{v}$ debe ser
	igual a un orden maximal en la correspondiente \'{a}lgebra $B_{v}$. Por
	otro lado, el ideal discriminante reducido $\drd{L}\subset K$ es un
	ret\'{\i}culo completo en $K$ y, por lo tanto,
	$(\drd{L})_{v}=\oka{K,v}$ (el orden maximal) para casi todo $v$.
	Entonces, el discriminante del \'{a}lgebra $B_{v}$ es
	$\drd{B_{v}}=\drd{L_{v}}=\oka{K,v}$ para casi todo $v$. En definitiva,
	por lo visto en la Secci\'{o}n~\ref{subsec:ordeneseidealeslocal},
	$B_{v}\simeq\MM_{2\times 2}(K_{v})$, salvo, posiblemente, en finitos
	lugares.
\end{obsRamEsFinita}

El \emph{discriminante reducido de $B/K$}, denotado por $\drd{B}$, se define
como el producto de los ideales primos correspondientes a los lugares finitos
en donde $B$ ramifica. Se cumple
\index{discriminante reducido!global!de un algebra@de un \'{a}lgebra}
\begin{align*}
	\drd{B} & \,=\,\prod_{v\in\Ram(B)\cap\lugares[f]{K}}\,\frak{p}_{v}
		\,=\,\bigcap_{v\in\lugares[f]{K}}\,\big(K\cap\drd{B_{v}}\big)
	\text{ ,}
\end{align*}
%
o, lo que es lo mismo, $(\drd{B})_{v}=\drd{B_{v}}$.

\begin{teoClassificationSurUnCorpsGlobal}[Clasificaci\'{o}n global]%
	\label{thm:clasificacionglobal}
	El cardinal del conjunto $\mathrm{Ram}(B)$ es par. Si $S$ es un
	subconjunto de lugares de $K$ de cardinal par, existe, salvo
	isomorfismo, una \'{u}nica \'{a}lgebra de cuaterniones cuyo conjunto de
	ramificaci\'{o}n es $S$.
\end{teoClassificationSurUnCorpsGlobal}

En particular, el teorema de clasificaci\'{o}n nos dice que dos \'{a}lgebras de
cuaterniones sobre un cuerpo de n\'{u}meros son isomorfas, si y s\'{o}lo si son
isomorfas localmente.

