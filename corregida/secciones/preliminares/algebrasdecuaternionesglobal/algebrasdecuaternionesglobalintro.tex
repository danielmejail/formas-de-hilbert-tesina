% Pasamos ahora a describir algunos aspectos de los \'{o}rdenes y de los ideales
% de un \'{a}lgebra de cuaterniones sobre un cuerpo global, teniendo en cuenta lo
% que sucede en el caso local.
% 
Sean $K/\bb{Q}$ un cuerpo de n\'{u}meros, $S\subset\lugares{K}$ un subconjunto
finito de lugares que contiene $\lugares[\infty]{K}$, $E/K$ un $K$-espacio
vectorial de dimensi\'{o}n finita y $L\subset E$ un $\oka{K}(S)$-ret\'{\i}culo.
Para cada lugar $v\in\lugares{K}$, sean
\begin{align*}
	E_{v}\,=\,E\tensor[K]{K_{v}} & \qquad\text{y}\qquad
		L_{v}\,=\,L\otimes_{\oka{K}(S)}\oka{K,v}
\end{align*}
%
el espacio vectorial sobre la completaci\'{o}n $K_{v}$ determinado por
extensi\'{o}n de escalares y, respectivamente, el $\oka{K,v}$-ret\'{\i}culo en
$E_{v}$ generado por $L$, su clausura ($L_{v}$ s\'{o}lo est\'{a} definido para
$v\in\lugares[f]{K}\setmin S$). Si $M_{v}$ es un ret\'{\i}culo en $E_{v}$,
decimos que $M_{v}$ es un \emph{ret\'{\i}culo local}. Para enfatizar que $L$ es
un ret\'{\i}culo en el espacio sobre $K$, decimos que $L$ es un
\emph{ret\'{\i}culo global}. Si $x\in E$, denotamos por $x_{v}$ su imagen en
$E_{v}$, es decir, $x_{v}=x\otimes 1$.

\begin{propoLocalGlobalLattices}\label{propo:localgloballattices}
	Sea $E/K$ un espacio vectorial sobre $K$ y sean $L_{0}$ y $L$
	ret\'{\i}culos en $E$. Entonces
	\begin{itemize}
		\item[(i)]
			\begin{math}
				L=\bigcap_{v\in\lugares[f]{K}\setmin S}\,
					E\cap L_{v}
			\end{math}
			y
		\item[(ii)] $L_{v}=L_{0,v}$ para casi todo
			$v\in\lugares[f]{K}\setmin S$.
	\end{itemize}
	%
	Adem\'{a}s, (iii) si $\{M_{v}\}_{v\in\lugares[f]{K}\setmin S}$ es una
	familia de ret\'{\i}culos locales tales que $M_{v}=L_{0,v}$ para casi
	todo $v$, existe un (\'{u}nico) ret\'{\i}culo global $L'\subset E$ tal
	que $L_{v}'=M_{v}$ para todo $v\in\lugares[f]{K}\setmin S$.
\end{propoLocalGlobalLattices}
% 
% \begin{proof}
	% Ver \cite[Ch.~V, \S~2, Thm.~2]{WeilBasic}.
% \end{proof}

