%
% %\paragraph{Ejemplo}
% \begin{ejemploReal}\label{ejemplo:real}
	% Complexificando,
	% \begin{align*}
		% & \MM_{2\times 2}(\bb{R}),\bb{H} \,\hookrightarrow\,
			% \MM_{2\times 2}(\bb{C})
		% \text{ .}
	% \end{align*}
	% %
	% Sea $B=\varquatalg[\bb{R}]{a,b}$ un \'{a}lgebra real de cuaterniones.
	% Si $a$ es un cuadrado, queda $B\simeq\MM_{2\times 2}(\bb{R})$. Si no,
	% dada que $(\bb{R}^{\times})^{2}=\bb{R}^{\times}_{+}$, podemos
	% reemplazar $a$ por $-1$, y tomar $L=\bb{R}(i)$ con $i\in B$ tal que
	% $i^2=-1$. Entonces $L\simeq\bb{C}$
	% Si $K=\bb{R}$ es el cuerpo de n\'{u}meros reales, entonces
	% el grupo de unidades del \'{a}lgebra, $B^{\times}=\GL_{2}(\bb{R})$,
	% se puede descomponer en la uni\'{o}n de dos coclases:
	% \begin{align*}
		% B^{\times} & \,=\,\GLtp_{2}(\bb{R})\,\sqcup\,
					% u\cdot\GLtp_{2}(\bb{R})
		% \text{ ,}
	% \end{align*}
	% %
	% por ejemplo, donde $\GLtp_{2}(\bb{R})$ es el subgrupo de matrices de
	% determinante positivo. En particular, el morfismo
	% $\nrd:\,B^{\times}\rightarrow\bb{R}^{\times}$ es sobreyectivo.
% 
	% Los cuaterniones de Hamilton, $\bb{H}$, constituyen un \'{a}lgebra de
	% cuaterniones sobre $\bb{R}$: $\bb{H}\simeq (-1,-1)_{\bb{R}}$. Sobre
	% $\bb{C}$ se obtiene una representaci\'{o}n como sub\'{a}lgebra de un
	% \'{a}lgebra de marices:
	% \begin{align*}
		% \bb{H} & \,\simeq\, \left\lbrace
			% \begin{bmatrix} z & w \\
				% -\conj{w} & \conj{z} \end{bmatrix}
			% \,:\,z,w\in\bb{C}\right\rbrace
		% \text{ .}
	% \end{align*}
	% %
	% Notamos que el centro $\bb{R}$ de $\bb{H}$ se identifica con las
	% matrices escalares
	% \begin{math}
		% \begin{bmatrix} t & \\ & t \end{bmatrix}
	% \end{math}
	% con $t\in\bb{R}$. A diferencia de lo que ocurre en el Ejemplo
	% \ref{ejemplo:matrices}, $\bb{H}^{\times}=\bb{H}\setmin\{0\}$
	% y toda unidad tiene norma positiva. Adem\'{a}s, la imagen del
	% morfismo
	% \begin{math}
		% \nrd:\,\bb{H}^{\times}\rightarrow\bb{R}^{\times}
	% \end{math}
	% tiene \'{\i}ndice $2$.
% 
	% Notemos que $\bb{H}^{1}$, el subgrupo de unidades de
	% norma reducida $1$ en $\bb{H}$, es compacto, mientras que el grupo
	% correspondiente en matrices, $\SL_{2}(\bb{R})$, no lo es.
% \end{ejemploReal}
