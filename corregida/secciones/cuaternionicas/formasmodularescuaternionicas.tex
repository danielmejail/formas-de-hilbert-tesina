En lo que resta de esta secci\'{o}n, asumimos, salvo que se indique lo
contrario, que $B/F$ es un \'{a}lgebra de cuaterniones \emph{de %
divisi\'{o}n} sobre el cuerpo de n\'{u}meros totalmente real $F$ de grado $n$,
que $r\geq 0$ indica la cantidad de lugares arquimedianos no ramificados y que
los lugares arquimedianos $\lista{v}{n}$ est\'{a}n ordenados de forma tal que
$v_{i}$ sea no ramificado, si y s\'{o}lo si $i\leq r$. Fijamos un peso
$\peso{k}=(\lista{k}{n})\in\bb{Z}^{n}$ y escribimos $W_{\peso{k}}$ en lugar de
$W_{\peso{k}}(\bb{C})$. Fijamos tambi\'{e}n inclusiones
\begin{math}
	(\gamma\mapsto\gamma_{j}):\,
		B_{v_{j}}^{\times}\hookrightarrow\GL_{2}(\bb{C})
\end{math}~para $j\geq r+1$, de manera que quede inducida una acci\'{o}n de
$B^{\times}$ en el $\GL_{2}(\bb{C})^{n-r}$-m\'{o}dulo $W_{\peso{k}}$, que
denotamos $x\cdot\gamma=x^{\gamma}$, si $\gamma\in B^{\times}$ y
$x\in W_{\peso{k}}$.

\subsection{Caso indefinido}%
	% \label{subsec:formascuaternionicascasoindefinidobis}
Supongamos que $B$ es un \'{a}lgebra indefinida (en este caso, $r\geq 1$) y
sea $\cal{O}\subset B$ un orden de Eichler. En analog\'{\i}a con las formas
modulares el\'{\i}pticas y teniendo en cuenta la definici\'{o}n
\eqref{eq:descomposicionvariedaddeshimuraindefinida} de la variedad de Shimura
asociada a $B$ y al orden $\cal{O}$, las formas modulares que definiremos a
continuaci\'{o}n son funciones de la forma
\begin{math}
	f:\,(\hP^{\pm})^{r}\times(\Idfin{B}/\Idfin{\cal{O}})\rightarrow W
\end{math}
que verifican condiciones de regularidad e invarianza con respecto a una
acci\'{o}n del grupo $B^{\times}$.

Para $i\in [\![1,r]\!]$, introducimos un \emph{factor de automorf\'{\i}a},
\index{factor de automorfia@factor de automorf\'{\i}a}
\begin{align*}
	J_{i} & \,:\,B_{v_{i}}^{\times}\times\hP^{\pm}\,\rightarrow\,\bb{C}
	\text{ ,}
\end{align*}
%
por la expresi\'{o}n
\begin{equation}
	\label{eq:factorddeautomorfia}
	J_{i}(\gamma,z) \,=\,\frac{j(\gamma,z)^{k_{i}}}{%
			\det(\gamma)^{m_{i}+k_{i}-1}}
	\text{ ,}
\end{equation}
%
donde
\begin{math}
	j\big(\left[\begin{smallmatrix} a & b \\
		c & d \end{smallmatrix}\right],z\big)=cz+d
\end{math}~.
Dado que $j(\gamma\gamma',z)=j(\gamma,\gamma'z)\cdot j(\gamma',z)$ para
$\gamma,\gamma'\in\GL_{2}(\bb{R})$ y $z\in\hP^{\pm}$, las funciones $J_{i}$
cumple con la propiedad an\'{a}loga
\begin{equation}
	\label{eq:factordeautomorfiacociclo}
	J_{i}(\gamma\gamma',z) \,=\,
		J_{i}(\gamma,\gamma'z)\cdot J_{i}(\gamma',z)
	\text{ .}
\end{equation}
%
Definimos un \emph{operador de peso $\peso{k}$} de la siguiente manera: dados
\index{operador de peso@operador de peso $\peso{k}$}
un elemento $\gamma\in B^{\times}$ y una funci\'{o}n
\begin{math}
	f:\,(\hP^{\pm})^{r}\times(\Idfin{B}/\Idfin{\cal{O}})\rightarrow
		W_{\peso{k}}
\end{math}~,
sea $f\operadormatrices{\peso{k}}{\gamma}$ la funci\'{o}n
\begin{equation}
	\label{eq:indefinidaoperadordepesok}
	\big(f\operadormatrices{\peso{k}}{\gamma}\big)
		(z,\hhat{\alpha}\Idfin{\cal{O}}) \,=\,
		\bigg(\prod_{i=1}^{r}\,J_{i}(\gamma_{i},z_{i})^{-1}\bigg)\,
		f(\gamma z,\gamma\hhat{\alpha}\Idfin{\cal{O}})^{\gamma}
	\text{ .}
\end{equation}
%

\begin{defFormaCuaternionicaCasoIndefinido}%
	\label{def:formacuaternionicacasoindefinido}
	Dada un \'{a}lgebra de cuaterniones indefinida $B$ de divisi\'{o}n,
	sobre un cuerpo de n\'{u}meros totalmente real $F$ y dado un orden
	de Eichler $\cal{O}\subset B$ de nivel $\frak{N}$, una
	\emph{forma modular cuaterni\'{o}nica}
	\index{forma modular!cuaternionica@cuaterni\'{o}nica}
	de peso $\peso{k}$ y nivel $\Idfin{\cal{O}}$ (o, tambi\'{e}n, de nivel
	$\frak{N}$) para $B$ es una funci\'{o}n
	\begin{align*}
		f & \,:\, (\hP^{\pm})^{r}\times (\Idfin{B}/
			\Idfin{\cal{O}})\rightarrow W_{\peso{k}}(\bb{C})
	\end{align*}
	%
	holomorfa en la primera variable y localmente constante en la segunda
	tal que $f\operadormatrices{\peso{k}}{\gamma}=f$ para toda
	$\gamma\in B^{\times}$. Estas funciones constituyen un
	$\bb{C}$-espacio vectorial que denotamos $\modularH[B]{k}{\frak{N}}$.
\end{defFormaCuaternionicaCasoIndefinido}

% \begin{obsFormaCuaternionicaCasoIndefinido}
	% \label{obs:formacuaternionicacasoindefinido}
	% En general, si $f\operadormatrices{\peso{k}}{\gamma}=f$ para toda
	% $\gamma\in B^{\times}$, entonces $f$ queda determinada por su
	% restricci\'{o}n a $\hP^{r}$, en la primera variable (Teorema
	% \ref{thm:eichlernorma}).
% \end{obsFormaCuaternionicaCasoIndefinido}

La descomoposici\'{o}n \eqref{eq:descomposicionvariedaddeshimuraindefinida} de
$\shimura[B]{\frak{N}}$ se ve reflejada en una descomposici\'{o}n an\'{a}loga
del espacio $\modularH[B]{k}{\frak{N}}$. En primer lugar, si $\frak{a}$ es un
representante de las clases estrictas de $F$ y $\Gamma_{\frak{a}}$ es el grupo
de unidades asociado seg\'{u}n \eqref{eq:ordenasociadocuaterniones},
entonces podemos definir una acci\'{o}n de $\Gamma_{\frak{a}}$ en funciones
$f:\,\hP^{r}\rightarrow W_{\peso{k}}$ de la siguiente manera: si
$\gamma\in\Gamma_{\frak{a}}$, sea $f\operadormatrices{\peso{k}}{\gamma}$ la
funci\'{o}n dada por
\begin{align*}
	\big(f\operadormatrices{\peso{k}}{\gamma}\big)(z) & \,=\,
		\bigg(\prod_{i=1}^{r}\,J_{i}(\gamma_{i},z_{i})^{-1}\bigg)\,
			f(\gamma z)^{\gamma}
	\text{ .}
\end{align*}
%
Luego definimos los espacios de funciones holomorfas invariantes por esta
acci\'{o}n de $\Gamma_{\frak{a}}$:
\begin{align*}
	\modularH[B]{k}{\frak{N},\frak{a}} & \,=\,
		\Big\{ f:\,\hP^{r}\rightarrow W_{\peso{k}}(\bb{C}) \,:\,
			f\text{ es holomorfa y }
			f\operadormatrices{\peso{k}}{\gamma}=f\,
			\forall\gamma\in\Gamma_{\frak{a}}
			\Big\}
	\text{ .}
\end{align*}
%
Entonces la aplicaci\'{o}n $f\mapsto (f_{\frak{a}})_{\frak{a}}$, donde
$f_{\frak{a}}(z)=f(z,\hhat{\alpha}\Idfin{\cal{O}})$, determina una
transformaci\'{o}n lineal
\begin{equation}
	\label{eq:descomposicionmodularesindefinida}
	\modularH[B]{k}{\frak{N}}\,\rightarrow\,
		\bigoplus_{\frak{a}}\,\modularH[B]{k}{\frak{N},\frak{a}}
	\text{ ,}
\end{equation}
%
pues, si $\gamma\in\Gamma_{\frak{a}}$,
\begin{align*}
	\big(f_{\frak{a}}\operadormatrices{\peso{k}}{\gamma}\big)(z)
	& \,=\,\bigg(\prod_{i=1}^{r}\,J_{i}(\gamma_{i},z_{i})^{-1}
		\bigg)\,f(\gamma z,\hhat{\alpha}\Idfin{\cal{O}})^{\gamma}
		\,=\, \big(f\operadormatrices{\peso{k}}{\gamma}\big)
			(z,\gamma^{-1}\hhat{\alpha}\Idfin{\cal{O}}) \\
	& \,=\, f(z,\gamma^{-1}\hhat{\alpha}\Idfin{\cal{O}})
		\,=\, f(z,\hhat{\alpha}\Idfin{\cal{O}})
		\,=\, f_{\frak{a}}(z)
	\text{ .}
\end{align*}
%
Esta transformaci\'{o}n es un isomorfismo de $\bb{C}$-espacios vectoriales.
El argumento es an\'{a}logo al de la demostraci\'{o}n de la Proposici\'{o}n
\ref{thm:descomposiciondelespaciodeformasmodularesdehilbert}, con la salvedad
de que, en este caso, hay que tener en cuenta la acci\'{o}n de $B^{\times}$
en $W_{\peso{k}}$.
% y la Observaci\'{o}n \ref{obs:formacuaternionicacasoindefinido}.
El isomorfismo \eqref{eq:descomposicionmodularesindefinida} depende de la
elecci\'{o}n de los representantes $\frak{a}$ de las clases estrictas de $F$,
de los id\`{e}les $\hhat{a}\in\Adfin{\oka{F}}$ que cumplen
\eqref{eq:ideleintegroasociado} y de los elementos $\hhat{\alpha}$ tales que
$\nrd(\hhat{\alpha})=\hhat{a}$.

\subsection{Caso definido}%
	% \label{subsec:formascuaternionicascasodefinido}
Si $B$ es un \'{a}lgebra definida, no hay acci\'{o}n sobre el semiplano
complejo.

\begin{defFormaCuaternionicaCasoDefinido}%
	\label{def:formacuaternionicacasodefinido}
	Sea $B$ un \'{a}lgebra de cuaterniones definida, sobre un cuerpo de
	n\'{u}meros totalmente real $F$ y sea $\cal{O}\subset B$ un orden de
	Eichler de nivel $\frak{N}$. Una \emph{forma modular cuaterni\'{o}nica}
	%\index{forma modular cuaternionica@forma modular cuaterni\'{o}nica}
	de peso $\peso{k}$ y nivel $\Idfin{\cal{O}}$ (o, tambi\'{e}n, de nivel
	$\frak{N}$) para $B$ es una funci\'{o}n
	\begin{align*}
		f & \,:\,\Idfin{B}/\Idfin{\cal{O}}\,\rightarrow\,
			W_{\peso{k}}(\bb{C})
	\end{align*}
	%
	que satisface, para toda $\gamma\in B^{\times}$,
	\begin{equation}
		\label{eq:definidaoperadordepesok}
		\big(f\operadormatrices{\peso{k}}{\gamma}\big)
			(\hhat{\alpha}\Idfin{\cal{O}})
			\,:=\, f(\gamma\hhat{\alpha}\Idfin{\cal{O}})^{\gamma}
			\,=\, f(\hhat{\alpha}\Idfin{\cal{O}})
		\text{ .}
	\end{equation}
	%
	El espacio de formas modulares correspondientes lo denotamos
	$\modularH[B]{k}{\frak{N}}$.
\end{defFormaCuaternionicaCasoDefinido}

El grupo de unidades $B^{\times}$ act\'{u}a en el conjunto de ideales
$I$ de $B$ con $\Oder(I)=\cal{O}$ por multiplicaci\'{o}n a izquierda.
Una forma modular $f$ para $B$ es entonces una funci\'{o}n equivariante
respecto de esta acci\'{o}n. Si $I=\hhat{\alpha}\Idfin{\cal{O}}\cap B$,
el estabilizador de $I$ es el grupo de unidades de su orden a izquierda:
\begin{align*}
	\Gamma_{\hhat{\alpha}} & \,=\,
		(\hhat{\alpha}\Idfin{\cal{O}}\hhat{\alpha}^{-1})
			\,\cap\,B^{\times}
		\,=\, \left\lbrace b\in B^{\times} \,:\, bI=I \right\rbrace
	\text{ .}
\end{align*}
%
De esta manera, tomando un sistema de representantes de las clases en
$\lClass{\cal{O}}$, se obtiene un isomorfismo
\begin{equation}
	\label{eq:descomposicionmodularesdefinida}
	\modularH[B]{k}{\frak{N}}\,\xrightarrow{\sim}\,
		\bigoplus_{[I]\in\lClass{\cal{O}}}\,
			W_{\peso{k}}(\bb{C})^{\Gamma_{\hhat{\alpha}}} \\
\end{equation}
%
dado por
\begin{math}
	f \mapsto(f(\hhat{\alpha}\Idfin{\cal{O}}))_{[I]\in\lClass{\cal{O}}}
\end{math}~.
%
% \begin{align*}
	% \modularH[B]{k}{\frak{N}} \,=\, & \left\lbrace
	% f:\,\Idfin{B}/\Idfin{\cal{O}}\rightarrow W_{\peso{k}}(\bb{C})
	% \,:\,f\barra{\peso{k}}{\gamma}=f\,\forall\gamma\in B^{\times}
	% \right\rbrace \\
	% \left(f\barra{\peso{k}}{\gamma}\right)(\hhat{\alpha}\Idfin{\cal{O}})
	% \,=\, &
	% f(\gamma\hhat{\alpha}\Idfin{\cal{O}})^{\gamma}
% \end{align*}
% %
% Sea $\hhat{\alpha}\in\Idfin{B}$ y sea $I=\hhat{\alpha}\Idfin{\cal{O}}\cap B$
% el ideal de $B$ correspondiente. Evaluando, se obtiene un punto
% $x=f(\hhat{\alpha}\Idfin{\cal{O}})\in W_{\peso{k}}(\bb{C})$ que
% cumple con $x^{\gamma}=f(\gamma^{-1}\hhat{\alpha}\Idfin{\cal{O}})=x$
% para todo $\gamma\in\Gamma_{\hhat{\alpha}}=%
% \hhat{\alpha}\Idfin{\cal{O}}\hhat{\alpha}^{-1}\cap B$.
% Si $(x_{\hhat{\alpha}})_{\hhat{\alpha}}\in%
% \bigoplus_{[I]\in\lClass{\cal{O}}}\,%
% W_{\peso{k}}(\bb{C})^{\Gamma_{\hhat{\alpha}}}$, definimos
% $f(\hhat{\alpha}\Idfin{\cal{O}}):=x_{\hhat{\alpha}}$. Si
% $\hhat{\beta}\in\Idfin{B}$, existen $\hhat{\alpha}$ y $\rho\in B^{\times}$
% tales que $\rho\hhat{\beta}\Idfin{\cal{O}}=\hhat{\alpha}\Idfin{\cal{O}}$.
% Entonces $f(\hhat{\beta}\Idfin{\cal{O}})=%
% f(\rho\hhat{\beta}\Idfin{\cal{O}})^{\rho}=%
% f(\hhat{\alpha}\Idfin{\cal{O}})^{\rho}$ determina
% $f\in\modularH[B]{k}{\frak{N}}$. As\'{\i}, se obtiene el isomorfismo
% de arriba.

