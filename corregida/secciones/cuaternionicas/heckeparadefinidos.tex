Supongamos ahora que $B/F$ es un \'{a}lgebra totalmente definida y sea
$\cal{O}\subset B$ un orden de Eichler. La definici\'{o}n de los operadores de
Hecke es an\'{a}loga a la definici\'{o}n para \'{a}lgebras indefinidas en
t\'{e}rminos de operadores de colcases dobles. Dado $\hhat{\pi}\in\Idfin{B}$,
vale que
\begin{math}
	\Idfin{\cal{O}}\hhat{\pi}\Idfin{\cal{O}}=\bigsqcup_{i}\,
		\Idfin{\cal{O}}\hhat{\pi}_{i}
\end{math}~,
donde $\hhat{\pi}_{i}$ recorre un conjunto finito. Dada
$f\in\modularH[B]{k}{\frak{N}}$, la expresi\'{o}n
\begin{equation}
	\label{eq:heckeparadefinidos}
	\big(T_{\hhat{\pi}}f\big)(\hhat{\alpha}\Idfin{\cal{O}}) \,=\,
		\sum_{i}\,f(\hhat{\alpha}\hhat{\pi}_{i}^{-1}\Idfin{\cal{O}})
\end{equation}
%
determina un nuevo elemento de $\modularH[B]{k}{\frak{N}}$.

\begin{defHeckeParaDefinidos}\label{def:heckeparadefinidos}
	Sea $\frak{p}\subset\oka{F}$ un ideal primo que no divide a
	$\frak{D}\frak{N}$. Si $\hhat{\pi}\in\Idfin{B}$ es el id\`{e}le dado
	por
	\begin{math}
		\begin{bmatrix} p & \\ & 1 \end{bmatrix}
	\end{math}
	en el lugar $\frak{p}$ y $1$ en $v\not =\frak{p}$. El
	\emph{operador de Hecke} en $\frak{p}$ es el operador
	$T_{\frak{p}}:=T_{\hhat{\pi}}$ asociado a la coclase doble
	$\Idfin{\cal{O}}\hhat{\pi}\Idfin{\cal{O}}$.
\end{defHeckeParaDefinidos}

Como en la secci\'{o}n \S~\ref{subsec:dehilbertoperadoresdehecke}, sea
$\frak{I}(\frak{p})=\Idfin{\cal{O}}\hhat{\pi}\Idfin{\cal{O}}=\{\}$.

Sea $H=\#\lClass{\cal{O}}$ y fijemos un sistema de representantes
$\{I_{t}\}_{t}$ de las clases a izquierda en $\lClass{\cal{O}}$ y elementos
$\hhat{\alpha}_{t}\in\Idfin{B}$ tales que
$I_{t}=\hhat{\alpha}_{t}\Adfin{\cal{O}}\cap B$. Dado
$\hhat{\alpha}\Idfin{\cal{O}}$, para cada $\hhat{\pi}_{i}$ existe un \'{u}nico
$t\in [\![1,H]\!]$ y un elemento $\rho_{i}\in B^{\times}$ que verifica
\begin{equation}
	\label{eq:cuaternionicadefinidaidealesequivalentes}
	\rho_{i}\hhat{\alpha}\hhat{\pi}_{i}^{-1}\Idfin{\cal{O}} \,=\,
		\hhat{\alpha}_{t}\Idfin{\cal{O}}
	\text{ .}
\end{equation}
%
Dos elementos $\rho_{i},\rho_{i}'$ que cumplen con
\eqref{eq:cuaternionicadefinidaidealesequivalentes} difieren en una
unidad del orden $\cal{O}_{t}=\Oizq(I_{t})$.
Entonces podemos reescribir $T_{\hhat{\pi}}f$ de la siguiente manera:
\begin{align*}
	\big(T_{\hhat{\pi}}f\big)
	(\hhat{\alpha}\Idfin{\cal{O}}) & \,=\,\sum_{i}\,\sum_{t=1}^{H}\,
		f(\hhat{\alpha}\hhat{\pi}_{i}^{-1}\Idfin{\cal{O}})\,
		[\hhat{\alpha}_{t}\Idfin{\cal{O}}]
			(\hhat{\alpha}\hhat{\pi}_{i}^{-1}\Idfin{\cal{O}}) \\
	& \,=\, \sum_{t=1}^{H}\,\sum_{i}\,
		f(\hhat{\alpha}_{t}\Idfin{\cal{O}})^{\rho_{i}}\,
		[\hhat{\alpha}_{t}\Idfin{\cal{O}}]
			(\hhat{\alpha}\hhat{\pi}_{i}^{-1}\Idfin{\cal{O}})
	\text{ ,}
\end{align*}
%
identificando un ret\'{\i}culo $\hhat{\alpha}\Adfin{\cal{O}}\cap B$ con la
coclase $\hhat{\alpha}\Idfin{\cal{O}}$ en los id\`{e}les finitos de $B$.
El t\'{e}rmino
\begin{math}
	[\hhat{\alpha}_{t}\Idfin{\cal{O}}]
			(\hhat{\alpha}\hhat{\pi}_{i}^{-1}\Idfin{\cal{O}})
\end{math}
es igual a $1$ o a $0$, si las clases
\begin{math}
	[\hhat{\alpha}_{t}\Idfin{\cal{O}}]
\end{math}
y
\begin{math}
	[\hhat{\alpha}\hhat{\pi}_{i}^{-1}\Idfin{\cal{O}}]
\end{math}
son iguales o no.

Ahora bien, la igualdad de las clases
\begin{math}
	[\hhat{\alpha}_{t}\Idfin{\cal{O}}]=
		[\hhat{\alpha}\hhat{\pi}^{-1}\Idfin{\cal{O}}]
\end{math}~,
con $\hhat{\pi}\in\Adfin{\cal{O}}$ de norma reducida igual a un id\`{e}le
correspondiente a $\frak{p}$, equivale a la existencia de una unidad
$\rho\in B^{\times}$ tal que
\begin{math}
	\rho\in\hhat{\alpha}_{t}\Idfin{\cal{O}}\hhat{\pi}\hhat{\alpha}^{-1}
\end{math}~;
esto es, a su vez, equivalente, en t\'{e}rminos de ideales de $B$, a que exista
$\rho\in B^{\times}$ tal que $J=II_{t}^{-1}\cal{O}_{t}\rho$ define un ideal con
orden a izquierda $\Oizq(I)=\hhat{\alpha}\Adfin{\cal{O}}\hhat{\alpha}\cap B$,
\'{\i}ntegro y de norma reducida $\frak{p}$. Rec\'{\i}procamente, dado un ideal
(invertible) $J$ de $B$ que cumple $\Oizq(J)=\Oizq(I)$, $J\sim II_{t}^{-1}$ y
$\nrd(J)=\frak{p}$, existe un id\`{e}le $\hhat{\pi}\in\Adfin{\cal{O}}$ tal que
\begin{align*}
	J & \,=\,\hhat{\alpha}\Adfin{\cal{O}}\hhat{\alpha}^{-1}
		(\hhat{\alpha}\hhat{\pi}\hhat{\alpha}^{-1})\,\cap\,B
	\text{ .}
\end{align*}
%
Pero, entonces, existe $\rho\in B^{\times}$ tal que
\begin{math}
	\hhat{\alpha}\hhat{\pi}^{-1}\Idfin{\cal{O}}=
		\rho^{-1}\hhat{\alpha}_{t}\Idfin{\cal{O}}
\end{math}~.
En definitiva,
\begin{equation}
	\label{eq:heckeparadefinidosconideales}
	\big(T_{\hhat{\pi}}f\big)(I) \,=\,\sum_{t=1}^{H}\,\sum_{\rho}\,
		\frac{1}{w_{t}}\,f(I_{t})^{\rho}
	\text{ ,}
\end{equation}
%
donde $w_{t}=\big|\cal{O}_{t}^{\times}/\oka{F}^{\times}\big|$ y $\rho$
recorre un sistema de representantes de
\begin{align*}
	& \Big\{ \rho\in I_{t}I^{-1}\,:\,\nrd(II_{t}^{-1}\,\rho)=\frak{p}\Big\}
		/\oka{F}^{\times}
	\text{ .}
\end{align*}
%

\subsubsection*{Peso paralelo $\peso{k}=(\lista[\null]{2}{\null})$}
Cuando $\peso{k}=(2,\,\dots,\,2)$, los elementos $\rho_{i}$ act\'{u}an de
manera trivial y queda:
\begin{align*}
	\big(T_{\hhat{\pi}}f\big)
	% f\barra{(2,\,\dots,\,2)}{[\Idfin{\cal{O}}\hhat{\pi}\Idfin{\cal{O}}]}
	(\hhat{\alpha}\Idfin{\cal{O}}) & \,=\, \sum_{t}\,
		f(\hhat{\alpha}_{t}\Idfin{\cal{O}})\cdot
		\#\big\{\hhat{\pi}_{i}\,:\,
			[\hhat{\alpha}\hhat{\pi}_{i}^{-1}\Idfin{\cal{O}}]=
			[\hhat{\alpha}_{t}\Idfin{\cal{O}}]
		\big\}
	\text{ .}
\end{align*}
%
Si $f=[I_{s}]$, con $I_{s}=\hhat{\alpha}_{s}\Adfin{\cal{O}}\cap B$ uno de los
representantes de las clases de ideales,
\begin{align*}
	% [I_{s}]\barra{(2,\,\dots,\,2)}%
	% {[\Idfin{\cal{O}}\hhat{\pi}\Idfin{\cal{O}}]}
	% (\hhat{\alpha}\Idfin{\cal{O}}) \,=\, &
	% \sum_{t}\,[\hhat{\alpha}_{s}\Idfin{\cal{O}}]
	% (\hhat{\alpha}_{t}\Idfin{\cal{O}})\cdot
	% \#\left\lbrace\hhat{\pi}_{i}\,:\,[\hhat{\alpha}\hhat{\pi}_{i}^{-1}
	% \Idfin{\cal{O}}]=[\hhat{\alpha}_{t}\Idfin{\cal{O}}]\right\rbrace \\
	% \,=\, & \#\left\lbrace\hhat{\pi}_{i}\,:\,
	% [\hhat{\alpha}\hhat{\pi}_{i}^{-1}\Idfin{\cal{O}}]=
	% [\hhat{\alpha}_{s}\Idfin{\cal{O}}]\right\rbrace
	% \quad\text{e} \\
	T_{\hhat{\pi}}[I_{s}]
	% [I_{s}]
	% \barra{(2,\,\dots,\,2)}{[\Idfin{\cal{O}}\hhat{\pi}\Idfin{\cal{O}}]}
	& \,=\,\sum_{t}\,\#\big\{\hhat{\pi}_{i}\,:\,
		[\hhat{\alpha}_{t}\hhat{\pi}_{i}^{-1}\Idfin{\cal{O}}]=
		[\hhat{\alpha}_{s}\Idfin{\cal{O}}]\big\} [I_{t}]
	\text{ .}
\end{align*}
%

Intentaremos hallar una expresi\'{o}n global para $T_{\frak{p}}$, es decir, en
t\'{e}rminos de ideales del \'{a}lgebra y que no dependa de tener que elegir
representantes $\{\hhat{\pi}_{i}\}_{i}$ de las coclases de $\Idfin{\cal{O}}$.
En primer lugar, podemos suponer que los representantes $\hhat{\pi}_{i}$
cumplen
\begin{align*}
	\Idfin{\cal{O}}\hhat{\pi}\Idfin{\cal{O}} & \,=\,
		\bigsqcup_{i}\,\Idfin{\cal{O}}\hhat{\varpi}_{i}
		\,=\,\bigsqcup_{i}\,\hhat{\varpi}_{i}\Idfin{\cal{O}}
	\text{ .}
\end{align*}
%

Sean $s,t\in[\![1,h]\!]$. La igualdad entre las clases
\begin{math}
	[\hhat{\alpha}_{t}\hhat{\varpi}_{i}^{-1}\Idfin{\cal{O}}]=
		[\hhat{\alpha}_{s}\Idfin{\cal{O}}]
\end{math}
equivale a la existencia de elementos $b\in B^{\times}$ y
$\hhat{u}\in\Idfin{\cal{O}}$ tales que
$b\hhat{\alpha}_{t}\hhat{\varpi}_{i}^{-1}=\hhat{\alpha}\hhat{u}$.
A su vez, usando que $\{\hhat{\varpi}_{i}\}_{i}$ es un conjunto de
representantes de las coclases tanto a izquierda como a derecha en
$\Idfin{\cal{O}}\hhat{\pi}\Idfin{\cal{O}}$, esto equivale a
$b\hhat{\alpha}_{t}=\hhat{\alpha}_{s}\hhat{\varpi}_{j}^{-1}\hhat{v}$ para un
\'{u}nico $j$ (posiblemente distinto de $i$) y alg\'{u}n
$\hhat{v}\in\Idfin{\cal{O}}$. En particular,
\begin{math}
	[\hhat{\alpha}_{t}\Idfin{\cal{O}}]=
		[\hhat{\alpha}_{s}\hhat{\varpi}_{j}\Idfin{\cal{O}}]
\end{math}
y
\begin{align*}
	% T_{\frak{p}}([\hhat{\alpha}_{s}\Idfin{\cal{O}}]) \,=\, &
	T_{\frak{p}}([I_{s}]) & \,=\,
		\sum_{j}\,[\hhat{\alpha}_{s}\hhat{\varpi}_{j}\Idfin{\cal{O}}]
	\text{ .}
\end{align*}
%

Notemos que $\hhat{\pi}$ es un elemento del orden $\Adfin{\cal{O}}$ en los
ad\`{e}les finitos de $B$ y, por lo tanto,
$\hhat{\varpi}_{j}\in\Adfin{\cal{O}}$ tambi\'{e}n. Ahora bien, la coclase
$\hhat{\alpha}_{s}\hhat{\varpi}_{j}\Idfin{\cal{O}}$ se corresponde con un ideal
$I$ de $B$ tal que:
\begin{align*}
	\Oder(I)\,=\,\cal{O}\text{ ,} & \quad
		I\,\subset\, I_{s} \quad\text{y}\quad
		\nrd(I)\,=\,\frak{p}\cdot\nrd(I_{s})
	\text{ .}
\end{align*}
%
Rec\'{\i}procamente, se verifica localmente que, si $I\subset I_{s}$, entonces
la coclase correspondiente en $\Idfin{B}/\Idfin{\cal{O}}$ es de la forma
$\hhat{\alpha}\Idfin{\cal{O}}$ con
$\hhat{\alpha}\Adfin{\cal{O}}\subset\hhat{\alpha}_{s}\Adfin{\cal{O}}$ y que, si
$\nrd(I)=\frak{p}\cdot\nrd(I_{s})$, entonces
$\hhat{\alpha}\Adfin{\cal{O}}=\hhat{\varpi}\hhat{\alpha}_{s}\Adfin{\cal{O}}$
con $\hhat{\varpi}\in\hhat{\alpha}_{s}\Adfin{\cal{O}}\hhat{\alpha}_{s}^{-1}$
y $\nrd(\hhat{\varpi})=\hhat{p}$. En particular,
$\hhat{\alpha}\Adfin{\cal{O}}=\hhat{\alpha}_{s}\hhat{\varpi}'\Adfin{\cal{O}}$
para cierto $\hhat{\varpi}'\in\Adfin{\cal{O}}$ tal que
$\nrd(\hhat{\varpi}')=\hhat{p}$. En definitiva, deducimos que
\begin{align*}
	T_{\frak{p}}([I]) & \,=\,
	% \sum_{t}\,\#\left\lbrace I'\in\mathscr{T}_{\frak{p}}(I)\,:\,
	% [I']=[I_{t}]\right\rbrace [I_{t}] \\
	% \,=\, &
	\sum_{I'\in\mathscr{T}_{\frak{p}}(I)}\,[I']\text{ , donde} \\
	\mathscr{T}_{\frak{p}}(I) & \,=\,
		\Big\{ I'\in\ideales{\cal{O}}\,:\,
			I'\subset I,\,\nrd(I')=\frak{p}\cdot\nrd(I)\Big\}
	\text{ .}
\end{align*}
%

Si $\frak{p},\frak{q}$ son ideales primos distintos en $\oka{F}$,
$T_{\frak{p}}T_{\frak{q}}=T_{\frak{q}}T_{\frak{p}}$. Pero, en general, no son
autoadjuntos respecto del producto interno definido en
$\modular[B]{(2,\,\dots,\,2)}{\frak{N}}$.
\begin{align*}
	\langle T_{\frak{p}}([I]),[J]\rangle & \,=\,
		\#\Big\{ I'\in\mathscr{T}_{\frak{p}}(I)\,:\,
			[I']=[J]\Big\}\cdot w_{J} \\
	& \,=\,\left|\Big\{ b\in B^{\times}\,:\,
		bJ\in\mathscr{T}_{\frak{p}}(I)\Big\}/\Gamma_{J}\right|\cdot
		\left|\Gamma_{J}/\oka{F}^{\times}\right| \\
	& \,=\, \left|\Big\{ b\in B^{\times}\,:\,bJ\in
		\mathscr{T}_{\frak{p}}(I)\Big\}/\oka{F}^{\times}\right|
	\text{ .}
\end{align*}
%
Dado que la norma reducida satisface $\nrd(I_{v})=\nrd(I)_{v}$ para todo ideal
$I$ y todo lugar $v$, los conjuntos $\mathscr{T}_{\frak{p}}(I)$ quedan
determinados localmente. De esto se deduce que, dados dos ideales
$I,I'\subset B$, $I'\in\mathscr{T}_{\frak{p}}(I)$ si y s\'{o}lo si
$\frak{p}I\in\mathscr{T}_{\frak{p}}(I')$. En nuestro caso, si
$b\in B^{\times}$, $bJ\in\mathscr{T}_{\frak{p}}(I)$ si y s\'{o}lo si
$\frak{p}b^{-1}I\in\mathscr{T}_{\frak{p}}(J)$. As\'{\i},
\begin{align*}
	\langle T_{\frak{p}}([I]),[J]\rangle & \,=\,
		\left|\Big\{ b\in B^{\times}\,:\,
			\frak{p}b^{-1}I\in\mathscr{T}_{\frak{p}}(J)\Big\}/
			\oka{F}^{\times}\right|
\end{align*}
%
Si el n\'{u}mero de clases de $F$ fuese uno, entonces podr\'{\i}amos reemplazar
el ideal $\frak{p}$ por un elemento global $p\in\oka{F}$ y
\begin{align*}
	\langle T_{\frak{p}}([I]),[J]\rangle & \,=\,
	\left|\Big\{ b'\in B^{\times}\,:\,b'I\in
		\mathscr{T}_{\frak{p}}(J)\Big\}/\oka{F}^{\times}\right| \\
	& \,=\, \langle [I],T_{\frak{p}}([J])\rangle
	\text{ .}
\end{align*}
%
Pero esto no es posible en general. Entonces, lo que se deduce es:
\begin{align*}
	\langle T_{\frak{p}}([I]),[J]\rangle & \,=\,
		\langle [\frak{p}I],T_{\frak{p}}([J])\rangle
	\text{ .}
\end{align*}
%

Dado un ideal primo $\frak{p}\subset\oka{F}$ tal que
$(\frak{p},\frak{D}\frak{N})=1$, podemos asociarle otro operador: si ahora
$\hhat{\pi}$ es el id\`{e}le dado por
\begin{math}
	\begin{bmatrix} p & \\ & p \end{bmatrix}
\end{math}
en el lugar $\frak{p}$ y $1$ en $v\not =\frak{p}$, donde, como antes,
$p\in\oka{F,\frak{p}}$ es un uniformizador, definimos el \emph{operador %
diamante}\index{operador diamante}
como el operador de coclase doble asociado a $\hhat{\pi}$ y lo denotamos
$\diamante{\frak{p}}$. Como $\hhat{\pi}\in\Adfin{\oka{F}}$, pertenece al centro
de $\Idfin{B}$ y, en un elemento de la base, est\'{a} dado por
\begin{align*}
	\diamante{\frak{p}} [I] (\hhat{\alpha}\Idfin{\cal{O}}) & \,=\,
		T_{\hhat{\pi}}[I]
	% [I]
	% \barra{(2,\,\dots,\,2)}{[\Idfin{\cal{O}}\hhat{\pi}\Idfin{\cal{O}}]}
			(\hhat{\alpha}\Idfin{\cal{O}})
		\,=\,[I](\hhat{\alpha}\hhat{\pi}^{-1}\Idfin{\cal{O}})
	\text{ .}
\end{align*}
%
Pero el ideal correspondiente a $\hhat{\alpha}\hhat{\pi}^{-1}\Idfin{\cal{O}}$
pertenece a la clase de $I$, si y s\'{o}lo si el correspondiente a
$\hhat{\alpha}\Idfin{\cal{O}}$ pertenece a $\frak{p}I$.
% Esto se debe a la forma particular que tiene $\hhat{\pi}$: es,
% esencialmente, el id\`{e}le en $\Idfin{B}$ correspondiente a $\frak{p}$.
Es decir,
\begin{align*}
	\diamante{\frak{p}} [I] & \,=\, [\frak{p}I]
	\text{ .}
\end{align*}
%

\begin{obsDiamanteParaDefinidos}\label{obs:diamanteparadefinidos}
	Recordemos que los operadores diamante act\'{u}an trivialmente en
	formas modulares el\'{\i}pticas para los grupos de congruencia
	$\Gamma_{0}(N)$ y notemos que esto no sigue siendo cierto en la
	situaci\'{o}n an\'{a}loga cuando el n\'{u}mero de clases del cuerpo es
	$\#\Class{F}>1$.
\end{obsDiamanteParaDefinidos}

Se puede hacer lo mismo con cada ideal fraccionario $\frak{d}$ de $F$ coprimo
con $\frak{D}\frak{N}$ eligiendo $\hhat{\pi}$ de manera apropiada y se ve que
\begin{align*}
	\diamante{\frak{d}} \diamante{\frak{e}} & \,=\,
		\diamante{\frak{d}\frak{e}}
		\,=\,\diamante{\frak{e}}\diamante{\frak{d}}
	\text{ .}
\end{align*}
%
Podemos deducir varias cosas: los operadores diamante son multiplicativos y, no
s\'{o}lo conmutan entre s\'{\i}, sino tambi\'{e}n conmutan con los operadores
$T_{\frak{p}}$. Adem\'{a}s, si $T^{*}$ denota el adjunto de un operador $T$
respecto del producto interno de Petersson, entonces
\begin{align*}
	\langle\frak{p}\rangle^{*} & \,=\,
		\diamante{\frak{p}^{-1}} \,=\,\diamante{\frak{p}}^{-1}
			\quad\text{y} \\
	T_{\frak{p}}^{*} & \,=\, \diamante{\frak{p}}^{-1}T_{\frak{p}}
	\text{ .}
\end{align*}
%
Queremos extender la definici\'{o}n de $T_{\frak{p}}$ a ideales
\emph{\'{\i}ntegros} de $F$, de manera que sean multiplicativos.
% (por lo menos entre operadores asociados a ideales coprimos).
%
% Podemos definir inductivamente $T_{\frak{p}^{k}}$ copiando la relaci\'{o}n
% que cumplen los operadores de Hecke en los espacios de formas modulares
% el\'{\i}pticas, o bien podemos extender la expresi\'{o}n para
% $T_{\frak{p}}$ en t\'{e}rminos de ideales del \'{a}lgebra:

Sea $I\subset B$ un ideal con $\Oder(I)=\cal{O}$ y sea $\frak{m}\subset\oka{F}$
un ideal \'{\i}ntegro de $F$ coprimo con $\frak{D}\frak{N}$.
% (en realidad, cuando definimos $T_{\frak{p}}$ no hac\'{\i}a falta
% asumir que $\frak{p}$ fuese coprimo con $\frak{N}$ (si con $\frak{D}$,
% though), entonces podemos no asumir que $\frak{m}$ es coprimo con
% $\frak{N}$).
Sea $\mathscr{T}_{\frak{m}}(I)$ el conjunto de ideales
\begin{align*}
	\mathscr{T}_{\frak{m}}(I) & \,=\,
		\Big\{ I'\in\ideales{\cal{O}}\,:\,
			I'\subset I,\,\nrd(I')=\frak{m}\cdot\nrd(I)\Big\}
	\text{ ,}
\end{align*}
%
y definimos el operador $T_{\frak{m}}$ usando la expresi\'{o}n global para los
operadores de Hecke en los elementos de la base:
\begin{align*}
	T_{\frak{m}}([I]) & \,=\,
		\sum_{I'\in\mathscr{T}_{\frak{m}}(I)}\,[I']
	\text{ .}
\end{align*}
%
Se verifica que $T_{\frak{m}}T_{\frak{m}'}=T_{\frak{m}\frak{m}'}$ para ideales
$\frak{m}$ y $\frak{m}'$ coprimos y que, por lo tanto, conmutan.
% Cuando $\frak{m}=\frak{p}$ un ideal primo, la equivalencia entre esta
% expresi\'{o}n para $T_{\frak{p}}$ y la expresi\'{o}n como operador
% de coclase doble depend\'{\i}a de que el id\`{e}le $\hhat{\pi}$ y su
% conjugado $\lconj{\hhat{\pi}}$ verificaran
% $\hhat{u}\lconj{\hhat{\pi}}\hhat{u}^{-1}=\hhat{\pi}$ para alguna
% unidad $\hhat{u}\in\Idfin{\cal{O}}$. Esto sigue siendo cierto para
% $\frak{m}$, si se elige un id\`{e}le adecuado $\hhat{\mu}$.

\begin{propoHeckeParaDefinidosRecursiva}\label{thm:heckeparadefinidosrecursiva}
	El operador $T_{1}$ act\'{u}a como la identidad de
	$\spitz[B]{(2,\,\dots,\,2)}{\frak{N}}$. Sea $\frak{p}\subset\oka{F}$ un
	ideal primo y sea $k\geq 1$. Entonces
	\begin{equation}
		\label{eq:heckeparadefinidosrecursiva}
		T_{\frak{p}^{k+2}} \,=\,
			T_{\frak{p}^{k+1}}T_{\frak{p}} \,-\,\norma(\frak{p})
			\diamante{\frak{p}} T_{\frak{p}^{k}}
		\text{ .}
	\end{equation}
	%
\end{propoHeckeParaDefinidosRecursiva}

\begin{proof}[Demostraci\'{o}n]
	Demostraremos \eqref{eq:heckeparadefinidosrecursiva} localmente. Es
	decir, sea $F$ un cuerpo local, sea $\frak{o}$ su anillo de enteros,
	sea $p\in\frak{o}$ un generador del ideal maximal, sea $B/F$ el
	\'{a}lgebra de matrices $\MM_{2\times 2}(F)$ y sea $\cal{O}$ el orden
	maximal
	\begin{math}
		\begin{bmatrix} \frak{o} & \frak{o} \\
		\frak{o} & \frak{o} \end{bmatrix}
	\end{math}
	de $B$. Entonces afirmamos que
	\begin{math}
		T_{p^{k+2}}=T_{p^{k+1}}T_{p}-\norma(p\frak{o})
			\diamante{p} T_{p^{k}}
	\end{math}~.

	Por un lado,
	\begin{align*}
		T_{p^{k+2}}([I]) & \,=\,
			\sum_{I'\in\mathscr{T}_{p^{k+2}}(I)}\,[I']
						\quad\text{y} \\
		T_{p^{k+1}}\circ T_{p}([I]) & \,=\,
			\sum_{I'\in\mathscr{T}_{p^{k+1}}(I)}\,
			\sum_{J\in\mathscr{T}_{p}(I')}\,[J]
		\text{ .}
	\end{align*}
	%
	Recordemos que, sobre un cuerpo local, los ideales $I\subset B$ con
	$\Oder(I)=\cal{O}$ son todos principales. Si fijamos un ideal $I$ y
	llamamos $\cal{O}'=\Oizq(I)$ a su orden a izquierda,
	\begin{align*}
		T_{p^{k+2}}([I]) & \,=\,\sum_{
			\begin{smallmatrix}
				\omega\in\cal{O}'/\cal{O}'^{\times} \\
				\nrd(\omega)=p^{k+2}
			\end{smallmatrix}
			}\,[\omega I] \\
		T_{p^{k+1}}\circ T_{p}([I]) & \,=\,\sum_{
			\begin{smallmatrix}
				\omega_{0}\in\cal{O}'/\cal{O}'^{\times} \\
				\nrd(\omega_{0})=p
			\end{smallmatrix}
			}\,\sum_{
			\begin{smallmatrix}
				\omega_{1}\in\omega_{0}\cal{O}'
				\omega_{0}^{-1}/\omega_{0}
				\cal{O}'^{\times}\omega_{0}^{-1} \\
				\nrd(\omega_{1})=p^{k+1}
			\end{smallmatrix}
			}\,[\omega_{1}\omega_{0}I]
		\text{ .}
	\end{align*}
	%
	Cada ideal $J=\omega I$ que aparece en la sumatoria de $T_{p^{k+2}}$
	aparece en la sumatoria de $T_{p^{k+1}}T_{p}$ y, como
	$\nrd(\omega_{1}\omega_{0})=\nrd(\omega_{1})\nrd(\omega_{0})$,
	todos los ideales $\omega_{1}\omega_{0}I$ que aparecen en la sumatoria
	de $T_{p^{k+1}}T_{p}$ aparecen en la sumatoria de $T_{p^{k+2}}$. Pero
	cada $J=\omega I$ con $\nrd(\omega)=p^{k+2}$ es igual a
	$(\omega\omega_{0}^{-1})\omega_{0} I$ para cada $\omega_{0}\in\cal{O}'$
	con $\nrd(\omega_{0})=p$. Entonces cada $J$ que aparece en
	$T_{p^{k+2}}$ aparece $\norma(p\frak{o})+1$ veces en
	$T_{p^{k+1}}T_{p}$, pues un sistema de representantes de los elementos
	en $\cal{O}'$ de norma reducida $p$ m\'{o}dulo unidades en $\cal{O}'$
	est\'{a} en biyecci\'{o}n con un sistema de representantes
	$\{\pi_{i}\}_{i}\subset\cal{O}$ tal que
	\begin{math}
		\cal{O}^{\times}\begin{bmatrix} p & \\ & 1 \end{bmatrix}
			\cal{O}^{\times}=
			\bigsqcup_{i}\,\cal{O}^{\times}\pi_{i}
	\end{math}~,
	y \'{e}stos son precisamente $\norma(p\frak{o})+1$: est\'{a}n dados,
	por ejemplo, por
	\begin{align*}
		\begin{bmatrix} 1 & j \\ & p \end{bmatrix}
			% \quad\text{con } j\in\frak{o}/p\frak{o}\text{ y} \\
			& \quad\text{y}\quad
		\begin{bmatrix} p & \\ & 1 \end{bmatrix}
			\text{ ,}
	\end{align*}
	%
	donde $j$ recorre un sistema de representantes de $\frak{o}/p\frak{o}$.
\end{proof}

% Evaluando la expresi\'{o}n anterior en un elemento $f=[I]$
% %=[\hhat{\alpha}_{t_{0}}\Idfin{\cal{O}}]$
% de la base de $\modular[B]{(2,\,\dots,\,2)}{\frak{N}}$, resulta
% \begin{align*}
	% % [\hhat{\alpha}_{t_{0}}\Idfin{\cal{O}}]
	% T_{\frak{p}}([I])
	% % \barra{(2,\,\dots,\,2)}{[\Idfin{\cal{O}}\hhat{\pi}\Idfin{\cal{O}}]}
	% \,=\, & \sum_{t}\,\#\left\lbrace I'\in\ideales{\cal{O}}\,:\,
	% I'\subset I,\nrd(I')=\frak{p}\cdot\nrd(I),\,
	% [I']=[I_{t}]
	% \right\rbrace [I_{t}]
	% \text{ ,}
% \end{align*}
% %
% %
% La cuenta: sea $\gamma\in B^{\times}$ tal que
% $\gamma (\hhat{\alpha}_{t}\hhat{\pi}_{i}^{-1}\Adfin{\cal{O}}\cap B)=%
% \hhat{\alpha}_{t_{0}}\Adfin{\cal{O}}\cap B$. Existe
% $\hhat{u}\in\Idfin{\cal{O}}$ tal que
% $\gamma\hhat{\alpha}_{t}\hhat{\pi}_{i}^{-1}=\hhat{\alpha}_{t_{0}}\hhat{u}$.
% Entonces
% \begin{align*}
	% (\hhat{\alpha}_{t}\hhat{\pi}_{i}^{-1}\Adfin{\cal{O}}
	% \hhat{\alpha}_{t_{0}})\,\cap\, B \,=\, &
	% \gamma^{-1}\cal{O}_{\hhat{\alpha}_{t_{0}}}\quad\text{y} \\
	% \cal{O}_{\hhat{\alpha}_{t_{0}}}\gamma \,=\, &
	% (\hhat{\alpha}_{t_{0}}\Adfin{\cal{O}}
	% \hhat{\pi}_{i}\hhat{\alpha}_{t}^{-1}) \,\cap\, B \\
	% \,=\, & (\hhat{\alpha}_{t_{0}}\Adfin{\cal{O}}\cap B)
	% (\hhat{\alpha}_{t}\Adfin{\cal{O}}\cap B)^{-1}
	% \cal{O}_{\hhat{\alpha}_{t}}
	% (\hhat{\alpha}_{t}\hhat{\pi}_{i}\hhat{\alpha}_{t}^{-1}) \\
	% \,=\, & I_{t_{0}}I_{t}^{-1}J\,\subset\,I_{t_{0}}I_{t}^{-1}
	% \text{ ,}
% \end{align*}
% %
% donde $J:=\cal{O}_{\hhat{\alpha}_{t}}%
% (\hhat{\alpha}_{t}\hhat{\pi}_{i}\hhat{\alpha}_{t}^{-1})$. Finalmente,
% \begin{align*}
	% \gamma J\gamma^{-1}I_{t_{0}} \,=\, & \gamma I_{t}
	% \qquad\text{(todos los ideales de la clase $[I_{t}]$ %
	% de norma $\frak{p}\cdot\nrd(I_{t_{0}})$).} \\
	% \hhat{\pi}_{i}\hhat{\alpha}_{t}^{-1}\gamma^{-1}\hhat{\alpha}_{t_{0}}
	% \,\in\, & \Idfin{\cal{O}}\text{ ,} \\
	% \Oder(\gamma J\gamma^{-1}) \,=\, &
	% (\hhat{\alpha}_{t}\hhat{\pi}_{i}%
	% \hhat{\alpha}_{t}^{-1}\gamma^{-1})^{-1}
	% \cal{O}_{\hhat{\alpha}_{t}}
	% (\hhat{\alpha}_{t}\hhat{\pi}_{i}\hhat{\alpha}_{t}^{-1}\gamma^{-1}) \\
	% \,=\, & \cal{O}_{\hhat{\alpha}_{t_{0}}}\quad\text{y} \\
	% \gamma J\gamma^{-1}I_{t_{0}} \,=\, &
	% \gamma\cal{O}_{\hhat{\alpha}_{t}}\hhat{\alpha}_{t}
	% \hhat{\pi}_{i}\hhat{\alpha}_{t}^{-1}\gamma^{-1}
	% \qquad\text{(y contenidos en $I_{t_{0}}$).}
% \end{align*}
% %
% o, equivalentemente,
% \begin{align*}
	% T_{\frak{p}}([I]) \,=\, &
	% \sum_{I'\in\mathscr{T}_{\frak{p}}(I)}\,[I']\quad\text{, donde} \\
	% \mathscr{T}_{\frak{p}}(I) \,=\, &
	% \left\lbrace I'\in\ideales{\cal{O}}\,:\,
	% I'\subset I,\,\nrd(I')\,\frak{p}\cdot\nrd(I)
	% \right\rbrace
	% \text{ .}
% \end{align*}
% %
% Dado que la norma reducida satisface $\nrd(I_{v})=\nrd(I)_{v}$ para
% todo ideal $I$ y todo lugar finito $v$, los conjuntos
% $\mathscr{T}_{\frak{p}}(I)$ quedan determinados localmente.
% % En un \'{a}lgebra de cuaterniones sobre un cuerpo local, todo ideal de un
% % orden de Eichler del \'{a}lgebra es principal.
% De esto se deduce que, dados dos ideales $I',I\subset B$,
% $I'\in\mathscr{T}_{\frak{p}}(I)$ si y s\'{o}lo si
% $\frak{p}I\in\mathscr{T}_{\frak{p}}(I')$. Esto tiene como consecuencia
% que los operadores de Hecke $T_{\frak{p}}$ son autoadjuntos respecto del
% producto interno de Petersson:
% \begin{align*}
	% \langle T_{\frak{p}}([I]),[J]\rangle \,=\, &
	% \#\left\lbrace I'\in\mathscr{T}_{\frak{p}}(I)\,:\,
	% [I']=[J]\right\rbrace
	% \cdot w_{J} \\
	% \,=\, &	\#\left(\left\lbrace b\in B^{\times}\,:\,
	% bJ\in\mathscr{T}_{\frak{p}}(I)\right\rbrace/\oka{F}^{\times}\right)
	% \text{ .}
% \end{align*}
% %
