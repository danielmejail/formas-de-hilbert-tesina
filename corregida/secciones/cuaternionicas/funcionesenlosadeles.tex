A continuaci\'{o}n mostramos que las formas modulares cuaterni\'{o}nicas, como
las formas modulares de Hilbert, tienen una contraparte ad\'{e}lica. El rol que
ocupaba el grupo ${\GL_{2}}_{/F}$ es en este caso ocupado por el grupo de
unidades de un \'{a}lgebra de cuaterniones de divisi\'{o}n.

\subsection{Caso indefinido}
Supongamos que $B$ es un \'{a}lgebra indefinida y sea
\begin{math}
	f :\,(\hP^{\pm})^{r}\times(\Idfin{B}/\Idfin{\cal{O}})\rightarrow
		W_{\peso{k}}
\end{math} tal que $f\operadormatrices{\peso{k}}{\gamma}=f$ para toda
$\gamma\in B^{\times}$. Definimos una funci\'{o}n
\begin{math}
	\phi_{f} :\,\Idinf{B}\times\Idfin{B}\rightarrow W_{\peso{k}}
\end{math}
por
\begin{equation}
	\label{eq:funcionadelicacorrespondienteindefinida}
	\phi_{f}(g,\hhat{\alpha}) \,=\,
		\bigg(\prod_{i=1}^{r}\,J_{i}(g_{i},\sqrt{-1})^{-1}\bigg)\,
		f(g\cdot\mathbf{i},\hhat{\alpha}\Idfin{\cal{O}})^{g}
	\text{ .}
\end{equation}
%

Dada $\phi:\,\Idinf{B}\times\Idfin{B}\rightarrow W_{\peso{k}}$ y un elemento
$(h,\hhat{\beta})\in\Idinf{B}\times\Idfin{B}$, definimos
\begin{equation}
	\label{eq:adelicaindefinidaoperadordepesok}
	\big(\phi\operadormatrices{\peso{k}}{(h,\hhat{\beta})}\big)
		(g,\hhat{\alpha}) \,=\,
		\bigg(\prod_{i=1}^{r}\,J_{i}(h_{i},\sqrt{-1})^{-1}\bigg)\,
		\phi(gh^{-1},\hhat{\alpha}\hhat{\beta}^{-1})^{h}
	\text{ .}
\end{equation}
%
Entonces, la funci\'{o}n $\phi_{f}$ satisface
\begin{itemize}
	\item[I]
		\begin{math}
			\phi_{f}\operadormatrices{\peso{k}}{(h,\hhat{\beta})}=
				\phi_{f}
		\end{math}
		para toda $h\in  C^{B}_{\mathbf{i}}$ y
		$\hhat{\beta}\in\Idfin{\cal{O}}$ y
	\item[II]
		\begin{math}
			\phi_{f}(\gamma g,\gamma\hhat{\alpha})=
				\phi_{f}(g,\hhat{\alpha})
		\end{math}
		para toda $\gamma\in B^{\times}$.
\end{itemize}
%
En cuanto a I, por \eqref{eq:funcionadelicacorrespondienteindefinida} y
\eqref{eq:adelicaindefinidaoperadordepesok},
\begin{align*}
	& \big(\phi_{f}\operadormatrices{\peso{k}}{(h,\hhat{\beta})}\big)
		(g,\hhat{\alpha}) \,=\,
		\bigg(\prod_{i=1}^{r}\,J_{i}(h_{i},\sqrt{-1})^{-1}\bigg)\,
			\phi_{f}(gh^{-1},\hhat{\alpha}\hhat{\beta}^{-1})^{h} \\
	& \qquad\qquad\,=\,
		\bigg(\prod_{i=1}^{r}\,J_{i}(h_{i},\sqrt{-1})^{-1}
		J_{i}(g_{i}h_{i}^{-1},\sqrt{-1})^{-1}\bigg)\,
		f(gh^{-1}\cdot\mathbf{i},
		\hhat{\alpha}\hhat{\beta}^{-1}\Idfin{\cal{O}})^{(gh^{-1})\,h}
	\text{ .}
\end{align*}
%
Si $h\in C^{B}_{\mathbf{i}}$ y $\hhat{\beta}\in\Idfin{\cal{O}}$, esto es
igual a $\phi_{f}(g,\hhat{\alpha})$ de lo que se deduce I. En cuanto a II,
\begin{align*}
	\phi_{f}(\gamma g,\gamma\hhat{\alpha})
	% & \,=\,\bigg(\prod_{i=1}^{r}\,
		% J_{i}(\gamma_{i}g_{i},\sqrt{-1})^{-1}\bigg)\,
		% f(\gamma g\cdot\mathbf{i},
			% \gamma\hhat{\alpha}\Idfin{\cal{O}})^{\gamma g} \\
	& \,=\,\bigg(\prod_{i=1}^{r}\,J_{i}(g_{i},\sqrt{-1})^{-1}\bigg)\,
		\bigg\{\bigg(\prod_{i=1}^{r}\,J_{i}(\gamma_{i},z_{i})^{-1}
			\bigg)\,f(\gamma z,\gamma\hhat{\alpha}\Idfin{\cal{O}}
				)^{\gamma}\bigg\}^{g}
	\text{ ,}
\end{align*}
%
donde $z=g\cdot\mathbf{i}$. Si $\gamma\in B^{\times}$, por
\eqref{eq:indefinidaoperadordepesok} y
$f\operadormatrices{\peso{k}}{\gamma}=f$, esta expresi\'{o}n es igual a
$\phi_{f}(g,\hhat{\alpha})$.

Rec\'{\i}procamente, si
\begin{math}
	\phi:\,\Idinf{B}\times\Idfin{B}\rightarrow W_{\peso{k}}
\end{math}
es una funci\'{o}n que cumple con I y II, entonces la expresi\'{o}n
\begin{align*}
	f_{\phi}(z,\hhat{\alpha}\Idfin{\cal{O}}) & \,=\,
		\bigg(\prod_{i=1}^{r}\,J_{i}(g_{i},\sqrt{-1})\bigg)\,
			\phi(g,\hhat{\alpha})^{g^{-1}}
\end{align*}
%
define una funci\'{o}n
\begin{math}
	f_{\phi}:\,(\hP^{\pm})^{r}\times(\Idfin{B}/\Idfin{\cal{O}})
		\rightarrow W_{\peso{k}}
\end{math}
(por I, $f_{\phi}$ est\'{a} bien definida) que verifica
$f_{\phi}\operadormatrices{\peso{k}}{\gamma}=f_{\phi}$ para
$\gamma\in B^{\times}$ (por II).

\subsection{Caso definido}
Si $B$ es un \'{a}lgebra definida, al no haber acci\'{o}n en el semiplano, la
construcci\'{o}n anterior se simplifica. Sea
 $f:\,\Idfin{B}/\Idfin{\cal{O}}\rightarrow W_{\peso{k}}$ una funci\'{o}n tal
que $f\operadormatrices{\peso{k}}{\gamma}=f$ para $\gamma\in B^{\times}$ (la
acci\'{o}n est\'{a} dada por la expresi\'{o}n
\eqref{eq:definidaoperadordepesok}) y sea
$\phi_{f}:\,\Idinf{B}\times\Idfin{B}\rightarrow W_{\peso{k}}$ la funci\'{o}n
dada por
\begin{equation}
	\label{eq:funcionadelicacorrespondientedefinida}
	\phi_{f}(g,\hhat{\alpha}) \,=\,f(\hhat{\alpha}\Idfin{\cal{O}})^{g}
	\text{ .}
\end{equation}
%
Como en el caso indefinido introducimos un operador de peso $\peso{k}$. Dada
una funci\'{o}n $\phi:\,\Idinf{B}\times\Idfin{B}\rightarrow W_{\peso{k}}$ y
$(h,\hhat{\beta})\in\Idinf{B}\times\Idfin{B}$, sea
$\phi\operadormatrices{\peso{k}}{(h,\hhat{\beta})}$ la funci\'{o}n dada por
\begin{equation}
	\label{eq:adelicadefinidaoperadordepesok}
	\big(\phi\operadormatrices{\peso{k}}{(h,\hhat{\beta})}\big)
		(g,\hhat{\alpha}) \,=\,
		\phi(gh^{-1},\hhat{\alpha}\hhat{\beta}^{-1})^{h}
	\text{ .}
\end{equation}
%
Con estas definiciones, si $f\operadormatrices{\peso{k}}{\gamma}=f$ para
$\gamma\in B^{\times}$, vale tambi\'{e}n que
\begin{itemize}
	\item[I]
		\begin{math}
			\phi_{f}\operadormatrices{\peso{k}}{(h,\hhat{\beta})}=
				\phi_{f}
		\end{math}
		para toda $h\in\Idinf{B}$ y $\hhat{\beta}\in\Idfin{\cal{O}}$ y
	\item[II]
		\begin{math}
			\phi_{f}(\gamma g,\gamma\hhat{\alpha})=
				\phi_{f}(g,\hhat{\alpha})
		\end{math}
		para toda $\gamma\in B^{\times}$.
\end{itemize}
%
Equivalentemente, $\phi_{f}$ satisface
\begin{align*}
	\phi_{f}(gh,\hhat{\alpha}) \,=\, \phi_{f}(g,\hhat{\alpha})^{h}
		& \quad\text{,}\quad
	\phi_{f}(g,\hhat{\alpha}\hhat{\beta})\,=\,\phi_{f}(g,\hhat{\alpha})
		\quad\text{y} \quad
	\phi_{f}(\gamma g,\gamma\hhat{\alpha}) \,=\,\phi_{f}(g,\hhat{\alpha})
\end{align*}
%
para $h\in\Idinf{B}$, $\hhat{\beta}\in\Idfin{\cal{O}}$ y
$\gamma\in B^{\times}$. Rec\'{\i}procamente, si
$\phi:\,\Idinf{B}\times\Idfin{B}\rightarrow W_{\peso{k}}$ satisface estas
condiciones, entonces la expresi\'{o}n
\begin{align*}
	f_{\phi}(\hhat{\alpha}\Idfin{\cal{O}}) & \,=\,
		\phi(g,\hhat{\alpha})^{g^{-1}}
\end{align*}
%
define una funci\'{o}n
\begin{math}
	f_{\phi}:\,\Idfin{B}/\Idfin{\cal{O}}\rightarrow W_{\peso{k}}
\end{math}~que cumple que
\begin{math}
	f_{\phi}\operadormatrices{\peso{k}}{\gamma}=f_{\phi}
\end{math}~para $\gamma\in B^{\times}$. En particular, para cada
$\hhat{\alpha}\in\Idfin{B}$ y cada
\begin{math}
	\gamma\in\Gamma_{\hhat{\alpha}}=
		(\hhat{\alpha}\Idfin{\cal{O}}\hhat{\alpha}^{-1})\cap B^{\times}
\end{math}~,
\begin{align*}
	\phi(1,\hhat{\alpha})^{\gamma} & \,=\,
		\phi(\gamma,\gamma\hhat{\alpha}) \,=\,\phi(1,\hhat{\alpha})
	\text{ .}
\end{align*}
%

\subsection{Formas automorfas}%\label{subsec:formasautomorfascuaternionicas}
Sea, ahora, $B/F$ un \'{a}lgebra de cuaterniones de divisi\'{o}n, definida o
indefinida. Sea $\cal{O}\subset B$ un orden de Eichler. A cada lugar
arquimediano $v\in\lugares[\infty]{F}$ le asociamos un grupo compacto $K_{v}$
de la siguiente manera:
\begin{align*}
	K_{v} & \,=\,
		\begin{cases}
			\SO{2} & \quad\text{si }v
				\text{ es no ramificado} \\
			\bb{H}^{1} & \quad\text{si }v
				\text{ es ramificado}
		\end{cases}
	\text{ .}
\end{align*}
%
Sea $K_{\infty}=\prod_{v\in\lugares[\infty]{F}}\,K_{v}$ el producto de estos
grupos. Entonces $K_{\infty}$ es un subgrupo compacto y conexo de la parte
arquimediana $\Idinf{B}$ de $\ideles{B}$ (si $B$ es indefinida, es el subgrupo
compacto conexo maximal del estabilizador
$C^{B}_{\mathbf{i}}=\centre(\bb{R})\cdot K_{\infty}$).

\begin{defFormaAutomorfa}[\cite{HarrisAGrA}]\label{def:formaautomorfa}
	Una funci\'{o}n
	\begin{math}
		\phi:\,\ideles{B}=\Idinf{B}\times\Idfin{B}\rightarrow\bb{C}
	\end{math} es una \emph{forma automorfa de nivel $\Idfin{\cal{O}}$}, si
	es $C^{\infty}$ en la primera variable y localmente constante en la
	segunda y:\index{forma automorfa}
	\begin{itemize}
		\item[(i)] $\phi(\gamma g)=\phi(g)$ para todo
			$\gamma\in B^{\times}$,
		\item[(ii)] $\phi(g\hhat{\beta})=\phi(g)$ si
			$\hhat{\beta}\in\Idfin{\cal{O}}$,
		\item[(iii)] $\phi$ es $K_{\infty}$-finita y
			$\centre$-finita, es decir, el espacio generado por
			las funciones $g\mapsto\phi(\eta gh)$
			con $h\in K_{\infty}$ y $\eta\in\centre(\adeles{F})$,
			es de dimensi\'{o}n finita,
			% \item[iv)] $\phi$ es soluci\'{o}n de las ecuaciones
			% diferenciales
			% \begin{align*}
				% \Delta_{i}\phi=-\frac{k_{i}}{2}
				% \left(\frac{k_{i}}{2}-1\right)\phi
				% \text{ ,}
			% \end{align*}
			% %
			% donde $\Delta_{i}$ es el elemento de Casimir en
			% $U(\mathit{Lie}(B_{v_{i}}^{\times}))$, (la
			% complexificaci\'{o}n de) el \'{a}lgebra envolvente
			% del \'{a}lgebra de Lie de
			% $B_{v_{i}}^{\times}\simeq\GL_{2}(\bb{R})$ para
			% cada lugar $v_{i}$ que no ramifica
			% ($i=1,\,\dots,\,r$).
		\item[(iv)] si
			\begin{math}
				\Delta_{i}\in U(\mathit{Lie}(
					B_{v_{i}}^{\times}))
			\end{math}
			es el elemento de Casimir en (la complexificaci\'{o}n
			de) el \'{a}lgebra envolvente del \'{a}lgebra de Lie de
			$B_{v_{i}}^{\times}\simeq\GL_{2}(\bb{R})$ para cada
			lugar $v_{i}$ que no ramifica ($i=1,\,\dots,\,r$),
			entonces la \'{o}rbita de $\phi$ por la acci\'{o}n de
			estos operadores est\'{a} contenida en un subespacio de
			dimensi\'{o}n finita de $C(\ideles{B})$.
			% el espacio generado por las funciones que se obtienen
			% aplicando los operadores $\Delta_{i}$ a $\phi$ es
			% de dimensi\'{o}n finita.
	\end{itemize}
	M\'{a}s generalmente, dado un espacio vectorial complejo $W$,
	una funci\'{o}n $\phi:\,\ideles{B}\rightarrow W$ se dice
	forma automorfa de nivel $\Idfin{\cal{O}}$, si para toda funcional
	lineal $\lambda:\,W\rightarrow\bb{C}$, $\lambda\circ\phi$ es una
	forma de ese tipo.
\end{defFormaAutomorfa}

% La condi\'{o}n \textit{(iv)} se puede expresar en t\'{e}rminos un poco
% m\'{a}s generales diciendo que la \'{o}rbita de $\phi$ por la acci\'{o}n
% del centro del \'{a}lgebra envolvente genera un espacio de dimensi\'{o}n
% finita.
%
% Si $B\simeq\MM_{2\times 2}(F)$ hay que agregar una condici\'{o}n de
% crecimiento moderado.
% Cuando $B\not\simeq\MM_{2\times 2}(F)$, es decir, cuando $B$ es de divisi\'{o}n
% (indefinida o no), la condici\'{o}n de crecimiento moderado es redundante
% pues la variedad de Shimura asociada es compacta.

\begin{propoFormasModularesComoFormasAutomorfas}[\cite{HarrisAGrA}]%
	\label{thm:formasmodularescomoformasautomorfas}
	La aplicaci\'{o}n $f\mapsto\phi_{f}$ determina una correspondencia
	entre formas modulares de peso $\peso{k}$ y nivel $\Idfin{\cal{O}}$
	y formas automorfas
	\begin{math}
		\phi:\,\ideles{B}\rightarrow W_{\peso{k}}(\bb{C})
	\end{math}
	que satisfacen \textrm{I} y \textrm{II} y, para cada lugar arquimediano
	no ramificado $v_{i}$,
	\begin{align*}
		d\rho X^{-}_{i}\phi & \,=\, 0
		\text{ ,}
	\end{align*}
	%
	donde
	\begin{math}
		\rho:\,\Idinf{B}\rightarrow\mathrm{Aut}(C^{\infty}(\Idinf{B}))
	\end{math}
	es la representaci\'{o}n regular a derecha y $X^{-}_{i}$ es el
	elemento de
	\begin{math}
		\frak{gl}(2,\bb{C})\simeq
			\mathit{Lie}(B_{v_{i}}^{\times})\otimes_{\bb{R}}\bb{C}
	\end{math}
	dado por la matriz
	\begin{math}
		\begin{bmatrix} 1 & -i \\ -i & -1 \end{bmatrix}
	\end{math}~.
\end{propoFormasModularesComoFormasAutomorfas}

