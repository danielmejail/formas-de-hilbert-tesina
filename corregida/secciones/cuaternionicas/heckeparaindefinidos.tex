Los operadores de Hecke en los espacios de formas de Hilbert cuspidales se
definen como operadores de coclases dobles. Si $B/F$ es un \'{a}lgebra de
cuaterniones indefinida, la definici\'{o}n de la estructura de m\'{o}dulo de
Hecke en $\spitzH[B]{k}{\frak{N}}$ es similar, teniendo en cuenta los lugares
(finitos) en donde $B$ ramifica. Sea $\frak{D}$ el discriminante de $B$ y sea
$\cal{O}$ un orden de Eichler de nivel $\frak{N}$.

\begin{defHeckeParaIndefinidos}\label{def:heckeparaindefinidos}
	Si $\frak{p}\subset\oka{F}$ es un ideal primo tal que
	$\frak{p}\nmid\frak{D}\frak{N}$, elegimos un generador
	$p\in\oka{F,\frak{p}}$ del ideal maximal en el anillo de enteros de
	la completaci\'{o}n $F_{\frak{p}}$. Sea $\hhat{\pi}\in\Idfin{B}$ el
	id\`{e}le definido por $\hhat{\pi}_{v}=1$, si $v\not =\frak{p}$, y
	\begin{math}
		\hhat{\pi}_{\frak{p}}=\begin{bmatrix} p & \\ & 1 \end{bmatrix}
	\end{math}~.
	Entonces el \emph{operador de Hecke} en $\frak{p}$ es
	$T_{\frak{p}}:=T_{\hhat{\pi}}$, donde
	\begin{align*}
		(T_{\hhat{\pi}}f)(z,\hhat{\alpha}\Idfin{\cal{O}})
			& \,=\,\sum_{i}\,f(z,\hhat{\alpha}\hhat{\pi}_{i}^{-1}
							\Idfin{\cal{O}})
		\text{ ,}
	\end{align*}
	%
	habiendo elegido un conjunto (finito) $\{\hhat{\pi}_{i}\}_{i}$ tal que
	\begin{math}
		\Idfin{\cal{O}}\hhat{\pi}\Idfin{\cal{O}}=
			\bigsqcup_{i}\,\Idfin{\cal{O}}\hhat{\pi}_{i}
	\end{math}~.
\end{defHeckeParaIndefinidos}

Esto equivale a considerar el conjunto
\begin{align*}
	\Theta(\frak{p}) & \,=\, \Idfin{\cal{O}}\backslash
		\Big\{\hhat{\pi}\in\Adfin{\cal{O}}\,:\,
			\nrd(\hhat{\pi})\in\hhat{p}\Idfin{\oka{F}}\Big\}
	\text{ ,}
\end{align*}
%
donde $\hhat{p}\in\Adfin{\oka{F}}$ es tal que
$\hhat{p}\Adfin{\oka{F}}\cap F=\frak{p}$, elegir un conjunto de representantes
y sumar
\begin{math}
	\sum_{\hhat{\pi}\in\Theta(\frak{p})}\,
		f(z,\hhat{\alpha}\hhat{\pi}^{-1}\Idfin{\cal{O}})
\end{math}
sobre dicho conjunto (comparar con la observaci\'{o}n
\ref{obs:idelesdenormap}).
% Notemos que, si $\hhat{\alpha}$ se corresponde con un ideal \'{\i}ntegro
% $\frak{a}$, el producto $\hhat{\alpha}\hhat{\pi}^{-1}$ deber\'{\i}a
% correponderse con el ideal $\frak{a}\frak{p}^{-1}$.
Dado que la sumatoria es finita, $T_{\frak{p}}f$ es una funci\'{o}n holomorfa
en la primera variable, localmente constante en la segunda y se verifica
$(T_{\frak{p}}f)\barra{\peso{k}}{\gamma}=T_{\frak{p}}f$ para todo
$\gamma\in B^{\times}$. Es decir que $T_{\frak{p}}$ define un operador en
$\spitzH[B]{k}{\frak{N}}$. Adem\'{a}s, dado que para distintos ideales primos
$\frak{p},\frak{q}$ los operadores correspondientes $T_{\frak{p}}$ y
$T_{\frak{q}}$ act\'{u}an en distintas coordenadas, se deduce inmediatamente
que conmutan. El \emph{\'{a}lgebra de Hecke} actuando en
$\spitzH[B]{k}{\frak{N}}$ es el \'{a}lgebra de endomorfismos generada por el
conjunto $\big\{ T_{\frak{p}}\,:\,\frak{p}\nmid\frak{D}\frak{N}\big\}$.

El espacio $\spitzH[B]{k}{\frak{N}}$ se descompone como suma directa de
$\spitzH[B]{k}{\frak{N},\frak{a}}$ v\'{\i}a el isomorfismo
$f\mapsto(f_{\frak{a}})_{\frak{a}}$ donde
$f_{\frak{a}}(z)=f(z,\hhat{\alpha}\Idfin{\cal{O}})$ y
$\nrd(\hhat{\alpha})=\hhat{a}\in\Adfin{\oka{F}}$ es tal que
$\hhat{a}\Adfin{\oka{F}}\cap F=\frak{a}$. Los operadores de Hecke act\'{u}an
permutando estos subespacios. Sea $\frak{b}$ un ideal \'{\i}ntegro de $\oka{F}$
cuya clase estricta es $[\frak{b}]=[\frak{a}\frak{p}^{-1}]$, sea
$\hhat{b}\in\Adfin{\oka{F}}$ tal que $\hhat{b}\Adfin{\oka{F}}\cap F=\frak{b}$ y
sea $\hhat{\beta}\in\Idfin{B}$ tal que $\nrd(\hhat{\beta})=\hhat{b}$. Entonces,
para cada representante $\hhat{\pi}$ de las \'{o}rbitas en $\Theta(\frak{p})$,
el ret\'{\i}culo
\begin{math}
	\hhat{\alpha}\hhat{\pi}^{-1}\Adfin{\cal{O}}\hhat{\beta}^{-1}\cap B
\end{math}
es un ideal de $B$ cuyo orden a derecha es
\begin{math}
	\cal{O}_{\frak{b}}=\hhat{\beta}\Adfin{\cal{O}}\hhat{\beta}^{-1}\cap B
\end{math}
y su norma reducida es un ideal en la clase (estricta) principal. En
consecuencia, %por aproximaci\'{o}n fuerte,
existe $\varpi\in B_{+}^{\times}$ tal que
\begin{math}
	\hhat{\alpha}\hhat{\pi}^{-1}\Adfin{\cal{O}}\hhat{\beta}^{-1}\cap B=
		\varpi^{-1}\cal{O}_{\frak{b}}
\end{math}
y una unidad $\hhat{u}\in\Idfin{\cal{O}}$ tal que
\begin{math}
	\hhat{\alpha}\hhat{\pi}^{-1}\hhat{u}\hhat{\beta}^{-1}=\varpi^{-1}
\end{math}~.
Si ahora miramos las componentes de la forma $T_{\frak{p}}f$ se deduce que,
usando la invarianza de $f$,
\begin{align*}
	(T_{\frak{p}}f)_{\frak{a}}(z) & \,=\,
		(T_{\frak{p}}f)(z,\hhat{\alpha}\Idfin{\cal{O}}) \,=\,
		\sum_{\hhat{\pi}\in\Theta(\frak{p})}\,
			f(z,\hhat{\alpha}\hhat{\pi}^{-1}\Idfin{\cal{O}})
		\,=\, \sum_{\varpi}\,
			f(z,\varpi^{-1}\hhat{\beta}\Idfin{\cal{O}}) \\
	& \,=\, \bigg(\prod_{i=1}^{r}\,J_{i}(\varpi_{i},z_{i})^{-1}\bigg)\,
		f(\varpi z,\hhat{\beta}\Idfin{\cal{O}})^{\varpi}
	\,=\, \sum_{\varpi}\,(f_{\frak{b}}\barra{\peso{k}}{\varpi})(z)
	\text{ .}
\end{align*}
%
% As\'{\i}, $T_{\frak{p}}:\,\spitzH[B]{k}{\frak{N},\frak{b}}\rightarrow%
% \spitzH[B]{k}{\frak{N},\frak{a}}$.
Si bien esto muestra que la descripci\'{o}n de $T_{\frak{p}}$ en t\'{e}rminos
de la descomposici\'{o}n de $\spitzH[B]{k}{\frak{N}}$ es sencilla, la utilidad
de la igualdad
\begin{math}
	\big(T_{\frak{p}}f\big)_{\frak{a}}=
		\sum_{\varpi}\,f_{\frak{b}}\barra{\peso{k}}{\varpi}
\end{math}
depende de poder hallar todos aquellos elementos $\varpi$ cuya existencia
est\'{a} garantizada por aproximaci\'{o}n fuerte. Estos elementos se pueden
caracterizar globalmente. La sumatoria se realiza sobre un sistema de
representantes de
\begin{align*}
	\Theta(\frak{p})_{\frak{a},\frak{b}} & \,=\,
		\Gamma_{\frak{b}}\backslash\Big\{\varpi\in
			I_{\frak{b}}I_{\frak{a}}^{-1}\cap B_{+}^{\times}\,:\,
			\nrd(\varpi)\frak{a}\frak{p}^{-1}=\frak{b}
		\Big\}
	\text{ ,}
\end{align*}
%
donde $I_{\frak{a}}=\hhat{\alpha}\Adfin{\cal{O}}\cap B$ e
$I_{\frak{b}}=\hhat{\beta}\Adfin{\cal{O}}\cap B$. El argumento es similar al
dado en la observaci\'{o}n \ref{obs:idelesdenormapporglobales}. El resultado
\ref{propo:descomposicioninducedescomposicion} tambi\'{e}n es v\'{a}lido en
este caso, es decir, siendo $B$ un \'{a}lgebra de divisi\'{o}n indefinida (ver
\cite[Propo.~2.3]{ShimuraDirichletSeriesAndAbelianVarieties}).

\begin{obsHeckeParaIndefinidosPorBloques}%
	\label{obs:heckeparaindefinidosporbloques}
	Los operadores de Hecke act\'{u}an por bloques en
	$\spitzH[B]{k}{\frak{N}}$, permutando en cierto sentido los sumandos
	$\spitzH[B]{k}{\frak{N},\frak{a}}$: dados
	$\frak{p}\nmid\frak{D}\frak{N}$ primo y $\frak{a}$ y $\frak{b}$ tales
	que $[\frak{b}]=[\frak{a}\frak{p}^{-1}]$, definimos
	\begin{align*}
		\big(T_{\frak{p}}\big)_{\frak{a},\frak{b}} & \,:\,
			\spitzH[B]{k}{\frak{N},\frak{b}}\,\rightarrow\,
			\spitzH[B]{k}{\frak{N},\frak{a}}
	\end{align*}
	%
	por
	\begin{align*}
		\big(T_{\frak{p}}\big)_{\frak{a},\frak{b}}f_{\frak{b}} & \,=\,
		\sum_{\varpi\in\Theta(\frak{p})_{\frak{a},\frak{b}}}\,
			f_{\frak{b}}\operadormatrices{\peso{k}}{\varpi}
		\text{ .}
	\end{align*}
	%
	Entonces el operador $T_{\frak{p}}$ act\'{u}a como la matriz de
	operadores
	\begin{math}
		\big[\big(T_{\frak{p}}\big)_{\frak{a},\frak{b}}
			\big]_{\frak{a},\frak{b}}
	\end{math}~,
	donde $\frak{a}$ y $\frak{b}$ recorren los representantes de las
	clases estrictas de $F$ y $(T_{\frak{p}})_{\frak{a},\frak{b}}$ es
	el operador reci\'{e}n definido, si $[\frak{a}]=[\frak{b}\frak{p}]$, y
	es igual a $0$, en caso contrario.
\end{obsHeckeParaIndefinidosPorBloques}


