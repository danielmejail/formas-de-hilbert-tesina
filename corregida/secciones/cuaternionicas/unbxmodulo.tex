
Las formas modulares cuaterni\'{o}nicas que vamos a definir est\'{a}n
dadas como funciones en $\ideles{B}$. Para poder tratarlas de manera
homog\'{e}nea, lo primero ser\'{a} definir el espacio de llegada de dichas
funciones.

Dado un entero $w\in\bb{Z}$, consideramos el $\bb{C}$-espacio vectorial de
polinomios homog\'{e}neos de grado $w$ en variables $X,Y$, con una acci\'{o}n a
derecha de $\GL_{2}(\bb{C})$ dada por
% a izquierda en los argumentos, a derecha en las funciones
\begin{equation}
	\label{eq:accionenpolinomioshomogeneos}
\begin{aligned}
	\begin{bmatrix} X \\ Y \end{bmatrix}\,\cdot\,\gamma \,:=\, &
		(\gamma^{\iota})^{t}\begin{bmatrix} X \\ Y \end{bmatrix}
		\,=\, \begin{bmatrix} d & -c \\ -b & a \end{bmatrix}
			\begin{bmatrix} X \\ Y \end{bmatrix} \\
	\,=\, & \begin{bmatrix} d X - c Y \\ -b X + a Y \end{bmatrix}
	\text{ .}
\end{aligned}
\end{equation}
%
En la expresi\'{o}n anterior, $\gamma$ es la matriz
\begin{math}
	\begin{bmatrix} a & b \\ c & d \end{bmatrix}
\end{math}
con coeficientes complejos, $\null^{t}$ denota transposici\'{o}n e
$\null^{\iota}$ denota la adjunta.
% en este caso, para no confundir el conjugado de un elemento del \'{a}lgebra de
% matrices (ejemplo \ref{ejemplo:matrices}) con conjugaci\'{o}n compleja.
Fijando un segundo entero $m\in\bb{Z}$, consideramos una acci\'{o}n modificada
a partir de la anterior: si $p$ es un polinomio homog\'{e}neo de grado $w$ y
$\gamma\in\GL_{2}(\bb{C})$, definimos
\begin{equation}
	\label{eq:accionenpolinomioscontwist}
	p(X,Y)\,\cdot\,\gamma \,=\,\det(\gamma)^{m}\,p(dX-cY,-bX+aY)
	\text{ .}
\end{equation}
%
Denotaremos el espacio de polinomios homog\'{e}neos de grado $w$ con esta
acci\'{o}n de $\GL_{2}(\bb{C})$ por $\wmpoli{w}{m}(\bb{C})$.
% Vale la pena notar que este mismo m\'{o}dulo (pero con una acci\'{o}n
% \emph{a izquierda}) es el descripto en \cite{EichlerBasisProblem}.

Sea $\peso{k}=(\lista{k}{n})$ un peso (ver
\S~\ref{sec:dehilbertformasparacongruencia}) y sean
\begin{align*}
	k_{0} \,=\,\mathrm{max}_{i}\,k_{i} & \text{ ,}\quad
		m_{i} \,=\,\frac{k_{0} - k_{i}}{2}\quad\text{y}\quad
		w_{i} \,=\,k_{i}-2
	\text{ .}
\end{align*}
%
Si
\begin{math}
	B\otimes_{\bb{Q}}\bb{R}=\MM_{2\times 2}(\bb{R})^{r}\times\bb{H}^{n-r}
\end{math}~,
se considera el $\GL_{2}(\bb{C})^{n-r}$-m\'{o}dulo
\begin{equation}
	\label{eq:unbxmodulo}
	W_{\peso{k}}(\bb{C}) \,:=\,
		\wmpoli{w_{r+1}}{m_{r+1}}(\bb{C})\,\otimes\,\cdots\,\otimes\,
		\wmpoli{w_{n}}{m_{n}}(\bb{C})
	\text{ .}
\end{equation}
%
Si $r=n$, se define $W_{\peso{k}}(\bb{C}):=\bb{C}$. Tambi\'{e}n se cumple
$W_{\peso{k}}(\bb{C})\simeq\bb{C}$, si $\peso{k}=(2,\,\dots,\,2)$.

Sean $\lista{v}{n}$ los lugares arquimedianos de $F$. Para cada $v_{j}$ podemos
escoger una inclusi\'{o}n
\begin{math}
	B_{v_{j}}\hookrightarrow\MM_{2\times 2}(\bb{C})
\end{math}
de manera que la imagen de un elemento
\begin{math}
	t\in\centre(B_{v_{j}}^{\times})=\bb{R}^{\times}
\end{math}
sea la matriz escalar
\begin{math}
	\begin{bmatrix} t & \\ & t \end{bmatrix}
\end{math}
y la denotamos $\gamma\mapsto\gamma_{j}$. El morfismo
\begin{math}
	B^{\times}\hookrightarrow\GL_{2}(\bb{C})^{n-r}
\end{math}
dado por
\begin{align*}
	\gamma\,\mapsto\, & (\gamma_{r+1},\,\dots,\,\gamma_{n})
\end{align*}
%
determina una estructura de $B^{\times}$-m\'{o}dulo en $W_{\peso{k}}(\bb{C})$.
Si $r=n$, por ejemplo si $B\simeq\MM_{2\times 2}(F)$,
$W_{\peso{k}}(\bb{C})=\bb{C}$ con la acci\'{o}n trivial de $B^{\times}$.
% Vale la pena observar que este m\'{o}dulo involucra
% \'{u}nicamente los lugares arquimedianos ramificados del \'{a}lgebra.
