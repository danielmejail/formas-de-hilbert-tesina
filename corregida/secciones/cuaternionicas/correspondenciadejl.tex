La interpretaci\'{o}n automorfa de las formas modulares cuaterni\'{o}nicas
permite definir de manera uniforme los operadores de Hecke para un \'{a}lgebra
definida o indefinida, tal como se hace para el \'{a}lgebra de matrices. Dados,
pues, un \'{a}lgebra de cuaterniones $B/F$, un orden de Eichler $\cal{O}$ en
$B$ y un id\`{e}le $\hhat{\pi}\in\Idfin{B}$, el subgrupo abierto
$\Idfin{\cal{O}}$ act\'{u}a a izquierda sobre el compacto
$\Idfin{\cal{O}}\hhat{\pi}\Idfin{\cal{O}}\subset\Idfin{B}$, de lo que se deduce
que existe una descomposici\'{o}n
\begin{equation}
	\label{eq:cuaternionesdescomposiciondecoclasedobleideles}
	\Idfin{\cal{O}}\hhat{\pi}\Idfin{\cal{O}} \,=\,
		\bigsqcup_{i}\,\Idfin{\cal{O}}\hhat{\pi}_{i}
\end{equation}
%
y el sistema de representantes $\{\hhat{\pi}_{i}\}_{i}$ es un conjunto finito.
Sea $\frak{N}$ el nivel del orden $\cal{O}$. Si $B$ es un \'{a}lgebra
indefinida y $f\in\modularH[B]{k}{\frak{N}}$, la expresi\'{o}n
\begin{equation}
	\label{eq:cuaternionicasoperadorcoclase}
	\big(T_{\hhat{\pi}}f\big)(z,\hhat{\alpha}\Idfin{\cal{O}}) \,=\,
		\sum_{i}\,f(z,\hhat{\alpha}\hhat{\pi}_{i}^{-1}\Idfin{\cal{O}})
\end{equation}
%
define un nuevo elemento de $\modularH[B]{k}{\frak{N}}$; si $B$ es un
\'{a}lgebra definida, la expresi\'{o}n an\'{a}loga, sin el argumento
arquimediano `$z$', tambi\'{e}n define una forma modular del mismo peso y del
mismo nivel que $f$. En todo caso, queda determinado, de esta manera, un
operador
\begin{math}
	T_{\hhat{\pi}}:\,\modularH[B]{k}{\frak{N}}\rightarrow
		\modularH[B]{k}{\frak{N}}
\end{math}
asociado al id\`{e}le $\hhat{\pi}$.

Sea $\frak{D}$ el discriminante de $B$ y sea $\frak{p}$ un ideal primo tal que
$\frak{p}\nmid\frak{D}\frak{N}$. Sea $p$ un uniformizador de
$\oka{F,\frak{p}}$, un generador de su ideal maximal, y sea
$\pi\in\cal{O}_{\frak{p}}$ el elemento dado por la matriz
\begin{align*}
	\pi & \,=\,\begin{bmatrix} p & \\ & 1 \end{bmatrix}
	\text{ .}
\end{align*}
%
Sea $\hhat{\pi}\in\Idfin{B}$ el id\`{e}le dado por $\hhat{\pi}_{v}=1$, si
$v\not=\frak{p}$ y tal que $\hhat{\pi}_{\frak{p}}=\pi$. Llamamos
\emph{operador de Hecke en $\frak{p}$} al operador
$T_{\frak{p}}:=T_{\hhat{\pi}}$, determinado por una descomposici\'{o}n de
$\Idfin{\cal{O}}\hhat{\pi}\Idfin{\cal{O}}$. Definimos, tambi\'{e}n,
\begin{align*}
	\frak{I}(\frak{p}) \,=\,\Big\{\hhat{x}\in\Adfin{\cal{O}}\,:\,
		\nrd(\hhat{x})\in\hhat{p}\Idfin{\oka{F}}\Big\}
		& \quad\text{y} \quad
	\Theta(\frak{p})\,=\,\Idfin{\cal{O}}\backslash\frak{I}(\frak{p})
	\text{ .}
\end{align*}
%
Entonces $\frak{I}(\frak{p})=\Idfin{\cal{O}}\hhat{x}\Idfin{\cal{O}}$ para todo
$\hhat{x}$ perteneciente a este conjunto, pues $(\frak{p},\frak{D}\frak{N})=1$,
y, en particular,
$\frak{I}(\frak{p})=\Idfin{\cal{O}}\hhat{\pi}\Idfin{\cal{O}}$. Adem\'{a}s, la
sumatoria en la definici\'{o}n de $T_{\frak{p}}$ se realiza sobre un sistema de
representantes de $\Theta(\frak{p})$, tanto en el caso indefinido, como en el
definido. Dado que para ideales primos distintos $\frak{p},\frak{q}$ los
operadores correspondientes $T_{\frak{p}}$ y $T_{\frak{q}}$ act\'{u}an en
distintas coordenadas, se deduce que conmutan. El \emph{\'{a}lgebra de Hecke}
actuando en $\modularH[B]{k}{\frak{N}}$ es el \'{a}lgebra de endomorfismos
generada por el conjunto
\begin{math}
	\big\{T_{\frak{p}}\,:\,\frak{p}\nmid\frak{D}\frak{N}\big\}
\end{math}~.
Denotamos por $\spitzH[B]{k}{\frak{N}}$ el espacio de formas cuaterni\'{o}nicas
\emph{cuspidales} de peso $\peso{k}$ y nivel $\frak{N}$ para $B$. En la
siguiente secci\'{o}n se explicar\'{a} lo que se quiere decir por ``cuspidal'';
a los fines de enunciar el siguiente teorema, es suficiente saber que
$\spitzH[B]{k}{\frak{N}}$ es un subespacio de $\modularH[B]{k}{\frak{N}}$, con
la estructura de m\'{o}dulo de Hecke dada por restringir los operadores
$T_{\frak{p}}$.

Sea $\frak{N}\subset\oka{F}$ un ideal \'{\i}ntegro y supongamos que
$\frak{N}=\frak{D}\frak{N}'$ con $\frak{D}$ y $\frak{N}'$ \'{\i}ntegros y
$\frak{D}$ libre de cuadrados. Por el Teorema de clasificaci\'{o}n
\ref{thm:clasificacionglobal}, existe al menos un \'{a}lgebra de cuaterniones
$B/F$ de discriminante $\frak{D}$. Supongamos, adem\'{a}s, que
$(\frak{D},\frak{N}')=1$.

\begin{teoJacquetLanglands}[Jacquet-Langlands]\label{thm:correspondenciajl}
	Existe un morfismo inyectivo
	\begin{math}
		\spitzH[B]{k}{\frak{N}'}
			\hookrightarrow\spitzH{k}{\frak{D}\frak{N}'}
	\end{math}
	que preserva la acci\'{o}n de Hecke y cuya imagen es el subespacio
	$\spitzH{k}{\frak{D}\frak{N}'}^{\frak{D}-\neue}$ de formas
	nuevas en todo ideal primo divisor de $\frak{D}$.
\end{teoJacquetLanglands}

El Teorema \ref{thm:correspondenciajl} nos da un procedimiento para recuperar
$\spitzH{k}{\frak{N}}$ a trav\'{e}s de formas modulares cuaterni\'{o}nicas.
Supongamos que $\frak{l}$ es un ideal primo que divide a $\frak{N}$ pero que al
cuadrado no lo divide. Por el Teorema de clasificaci\'{o}n global para
\'{a}lgebras de cuaterniones \ref{thm:clasificacionglobal}, sabemos que existe
un \'{a}lgebra de cuaterniones $B/F$ tal que
$\Ram(B)\cap\lugares[f]{F}=\{\frak{l}\}$, es decir, $\frak{l}$ es el \'{u}nico
lugar finito de $F$ en donde $B$ ramifica. Si elegimos $B$ de esta manera,
\begin{align*}
	\spitzH{k}{\frak{N}} & \,=\, \spitzH{k}{\frak{N}}^{\frak{l}-\neue}
		\,\oplus\,\spitzH{k}{\frak{N}}^{\frak{l}-\oude} \\
	& \,=\, \spitzH[B]{k}{\frak{N}'}\,\oplus\,
		\iota_{1}\left(\spitzH{k}{\frak{N}'}\right)\,\oplus\,
		\iota_{\frak{l}}\left(\spitzH{k}{\frak{N}'}\right)
	\text{ .}
\end{align*}
%
El problema de la descripci\'{o}n de $\spitzH{k}{\frak{N}}$ se reduce
entonces a poder determinar un subespacio en correspondencia con un espacio
de formas cuaterni\'{o}nicas y un subespacio de formas de Hilbert de nivel
``m\'{a}s bajo''.

Aun asumiendo que podemos \emph{calcular} sin problemas los espacios de formas
cuspidales cuaterni\'{o}nicas $\spitzH[B]{k}{\frak{N}'}$ para un nivel
arbitrario $\frak{N}'$, este procedimiento presenta un inconveniente.
Supongamos primero, para simplificar, que $\frak{N}=\frak{p}\frak{q}$, con
$\frak{p}$ y $\frak{q}$ ideales primos distintos. Entonces la
descomposici\'{o}n de $\spitzH{k}{\frak{p}\frak{q}}$ se puede hacer, en
principio, de varias maneras. Podemos elegir un \'{a}lgebra $B$ con
$\Ram(B)\cap\lugares[f]{F}=\{\frak{p}\}$ para obtener
\begin{align*}
	\spitzH{k}{\frak{p}\frak{q}} & \,=\, \spitzH[B]{k}{\frak{q}}\,\oplus\,
		\spitzH{k}{\frak{p}\frak{q}}^{\frak{p}-\oude}
	\text{ ;}
\end{align*}
%
si $n=[F:\bb{Q}]=1\text{ o }2$, hay una \'{u}nica \'{a}lgebra de tales
caracter\'{\i}sticas, pero si $n>2$ hay m\'{a}s de una elecci\'{o}n posible.
Podemos, tambi\'{e}n, intercambiar los roles de $\frak{p}$ y de $\frak{q}$.
O bien podemos elegir $B$ de manera que
$\Ram(B)\cap\lugares[f]{F}=\{\frak{p},\frak{q}\}$ --de nuevo, hay m\'{a}s
de una manera de hacer esta elecci\'{o}n, si $n\geq 2$-- y obtener as\'{\i}
\begin{align*}
	\spitzH{k}{\frak{p}\frak{q}} & \,=\, \spitzH[B]{k}{1}\,\oplus\,
		\spitzH{k}{\frak{p}\frak{q}}^{\frak{p}\frak{q}-\oude} \\
	& \,=\,\spitzH[B]{k}{1}\,\oplus\, \spitzH{k}{\frak{p}\frak{q}}^{\oude}
	\text{ .}
\end{align*}
%

Pero, si el nivel $\frak{N}$ no es libre de cuadrados, no tenemos tantas
elecciones. Por ejemplo, si $\frak{N}=\frak{p}^{2}$, el discriminante de $B$
debe ser $\frak{D}=1$, es decir, s\'{o}lo podemos permitir ramificaci\'{o}n en
infinito. Si $n=1$ esto es imposible y si $n=2$ hay una \'{u}nica elecci\'{o}n
posible. En general, los m\'{e}todos que describiremos en el
Cap\'{\i}tulo~\ref{cap:metodos} se basan en poder elegir $B$ de manera que $r$,
la cantidad de lugares arquimedianos en donde $B$ no ramifica, sea, o bien $0$,
o bien $1$. En ese caso, si $\frak{N}=\frak{p}^{2}$, hay una \'{u}nica forma de
realizar esa elecci\'{o}n.

En el cap\'{\i}tulo siguiente, describimos en detalle los m\'{o}dulos de Hecke
$\spitzH[B]{k}{\frak{N}}$, manteniendo, como hasta ahora, la distinci\'{o}n
entre las \'{a}lgebras definidas e indefinidas. La descripci\'{o}n en el caso
indefinido es muy similar a la del m\'{o}dulo de formas de Hilbert cuspidales.
V\'{\i}a los isomorfismos de Eichler-Shimura, los espacios de formas
cuatern\'{o}nicas para un \'{a}lgebra indefinida $B$ se realizan en la
cohomolog\'{\i}a de la variedad de Shimura $\shimura[B]{\frak{N}}$. Esta
reinterpretaci\'{o}n da lugar, en el caso particular en que $B$ ramifica en
todos excepto un \'{u}nico lugar arquimediano, a un m\'{e}todo para calcular
formas de Hilbert cuspidales. Por otra parte, la descripci\'{o}n cuando $B$ es
totalmente definida es lo suficientemente expl\'{\i}cita como para ser
implementada y obtener, as\'{\i}, un segundo m\'{e}todo.
