Sea $F$ un cuerpo de n\'{u}meros totalmente real de grado $[F:\bb{Q}]=n$ y
sea $B/F$ un \'{a}lgebra de cuaterniones. Cada una de las completaciones
$F_{v}$ con $v\in\lugares[\infty]{F}$ se identifica con $\bb{R}$ y $B_{v}$
con un \'{a}lgebra de cuaterniones real. Entonces, o bien
\begin{math}
	B_{v}\simeq\MM_{2\times 2}(\bb{R})
\end{math}
es un \'{a}lgebra de matrices, o bien $B_{v}\simeq\bb{H}$; en el primer caso,
decimos que el lugar \emph{$v$ es no ramificado}, o que $B$ no ramifica en $v$,
y, en el segundo, decimos que \emph{$v$ es ramificado}, o que $B$ ramifica en
$v$. Sean $\lista{v}{n}$ los lugares arquimedianos de $F$ y supongamos que
est\'{a}n ordenados de manera que, entre ellos, los lugares en donde $B$
ramifica son $\lista[r+1]{v}{n}$. Sea $\cal{O}$ un orden de $B$ y sea
$\ideles{B}$ el grupo de id\`{e}les de $B$. Entonces
\begin{math}
	\ideles{B}=\Idinf{B}\times\Idfin{B}
\end{math}~,
donde
\begin{equation}
	\label{eq:cuaternionesinfinitounidades}
	\Idinf{B} \,=\,\GL_{2}(\bb{R})^{r}\times (\bb{H}^{\times})^{n-r}
\end{equation}
%
y $\Idfin{B}$ es el producto restringido de $B_{v}^{\times}$ respecto de
los subgrupos compactos y abiertos $\cal{O}_{v}^{\times}$, donde $v$
recorre el conjunto de lugares finitos de $F$. Notemos que, si
en vez de $\cal{O}$, eligi\'{e}semos otro orden $\cal{O}'$ de $B$ en principio
distinto, por la Proposici\'{o}n \ref{propo:localgloballattices},
$\cal{O}_{v}=\cal{O}'_{v}$ para casi todo $v\in\lugares[f]{F}$, con lo cual
obtendr\'{\i}amos la misma \'{a}lgebra $\Idfin{B}$.

Supongamos que $B/F$ es un \'{a}lgebra indefinida distinta del \'{a}lgebra
de matrices, es decir, $r\geq 1$ y
\begin{math}
	B\not\simeq\MM_{2\times 2}(F)
\end{math}~.
La acci\'{o}n de $\GL_{2}(\bb{R})$ sobre $\hP^{\pm}$ se extiende a una
acci\'{o}n de $\Idinf{B}$ sobre el producto cartesiano $(\hP^{\pm})^{r}$: si
\begin{math}
	g_{\infty}=(\lista[v_{1}]{g}{v_{n}})\in\Idinf{B}
\end{math}
y $z=(\lista{z}{r})\in(\hP^{\pm})^{r}$ definimos
\begin{equation}
	\label{eq:acciondeidinf}
	g_{\infty}\cdot z \,=\, (g_{v_{1}}z_{1},\,\dots,\,g_{v_{r}}z_{r})
	\text{ .}
\end{equation}
%
Sea
\begin{math}
	C^{B}_{\mathbf{i}}\subset\Idinf{B}
\end{math}
el estabilizador del punto
\begin{math}
	\mathbf{i}=(\sqrt{-1},\,\dots,\,\sqrt{-1})\in(\hP^{\pm})^{r}
\end{math}~.
Como la acci\'{o}n \eqref{eq:acciondeidinf} es transitiva, podemos identificar
$(\hP^{\pm})^{r} =\Idinf{B}/C^{B}_{\mathbf{i}}$ v\'{\i}a
\begin{align*}
	& g_{\infty}\in\Idinf{B}\,\mapsto\, g_{\infty}\cdot\mathbf{i}
	\text{ .}
\end{align*}
%
El subgrupo estabilizador del punto $\mathbf{i}$ viene dado por
\begin{align*}
	C^{B}_{\mathbf{i}} & \,=\,\centre(\bb{R})\,
		\big( \SO{2}^{r}\times
		(\bb{H}^{1})^{n-r} \big)
	\text{ ,}
\end{align*}
%
donde
\begin{math}
	\centre(\bb{R})=\bb{R}^{\times}\times\,\cdots\,\times\bb{R}^{\times}
\end{math} --%
un factor por cada lugar arriba de $\infty$-- es el centro de $\Idinf{B}$,
\begin{math}
	\bb{H}^{1}=\{x\in\bb{H}:\conj{x}x=x\conj{x}=1\}
\end{math}~es el grupo de unidades de norma $1$ en el \'{a}lgebra de Hamilton
y $\SO{2}$ es el grupo ortogonal especial.

% Sea $K\subset\Idfin{B}$ un subgrupo compacto abierto de los id\`{e}les finitos.
% Como con ${\GL_{2}}_{/\bb{Q}}$, queremos estudiar un cociente de la forma
% \begin{align*}
	% \centre(\adeles{F})B^{\times}\backslash\ideles{B}/C^{B}_{\mathbf{i}}K
	% \text{ ,}
% \end{align*}
% %
% donde ahora $\ideles{B}$ juega el rol de $\GL_{2}(\adeles{\bb{Q}})$ (los
% puntos ad\'{e}licos del centro, $\centre(\adeles{F})$, se identifican con
% el grupo $\ideles{F}$ de ideles de $F$, siendo $B$ un \'{a}lgebra central).
% 
Sea $\cal{O}\subset B$ un orden de Eichler. Entonces
% Por aproximaci\'{o}n fuerte (Teorema \ref{thm:aproxfuerte}), [[NO]]
\begin{equation}
	\label{eq:descomposicionidelescuaterniones}
	\ideles{B} \,=\,\Idinf{B}\,\times\,\Idfin{B}
		\,=\, \bigsqcup_{g}\,
		B^{\times}g\Idinf{B}\Idfin{\cal{O}}
	\text{ ,}
\end{equation}
%
donde $g=(g_{v})_{v}\in\Idfin{B}$ recorre un sistema de representantes en
correspondencia con el conjunto de clases $\lClass{\cal{O}}$
--que es finito, seg\'{u}n el Corolario \ref{thm:numerodeclasesfinito}.
% Ese corolario s\'{\i} depende de aproximaci\'{o}n fuerte
%
% Pasando al cociente
% \begin{align*}
	% B^{\times}\backslash\Idinf{B}\times\Idfin{B}/
	% C^{B}_{\mathbf{i}}\Idfin{\cal{O}}
	% \,\simeq\, & \bigsqcup_{g}\,\Gamma_{g}\backslash (\hP^{\pm})^{r}
	% \text{ ,}
% \end{align*}
% %
% donde $\Gamma_{g}=g(\Idinf{B}\Idfin{\cal{O}})g^{-1}\cap B^{\times}$ es un
% subgrupo discreto de $\Idinf{B}$ actuando en $(\hP^{\pm})^{r}$. Dado que
% $B^{\times}/B_{+}^{\times}\simeq\{\pm\}^{r}$,
Sea $B_{\infty,+}^{\times}\subset\Idinf{B}$ el subgrupo de elementos cuyas
coordenadas tienen norma reducida positiva, es decir,
\begin{equation}
	\label{eq:cuaternionesinfinitounidadestotalmentepositivas}
	B_{\infty,+}^{\times} \,=\,
		\GLtp_{2}(\bb{R})^{r}\times (\bb{H}^{\times})^{n-r}
		\subset\Idinf{B}
	\text{ .}
\end{equation}
%
Por \eqref{eq:cuaternionesinfinitounidades},
\eqref{eq:cuaternionesinfinitounidadestotalmentepositivas} y la igualdad
\begin{math}
	\nrd(B^{\times})=F_{(+)}^{\times}
\end{math}
(ver el Teorema \ref{thm:eichlernorma}), pasando al cociente en
\eqref{eq:descomposicionidelescuaterniones}, se deduce, reemplazando
$\Idinf{B}$ por $B_{\infty,+}^{\times}$ (y $\hP^{\pm}$ por $\hP$), que
\begin{align*}
	B^{\times}\backslash
		\Idinf{B}\times\Idfin{B}/C^{B}_{\mathbf{i}}\Idfin{\cal{O}}
		& \,\simeq\,B_{+}^{\times}\backslash
		(\hP^{r}\times (\Idfin{B}/\Idfin{\cal{O}})) \\
	& \,=\,\bigsqcup_{g}\,\Gamma_{g}\backslash g\hP^{r}
	\text{ ,}
\end{align*}
%
donde ahora $g$ recorre un sistema de representantes en correspondencia
con $\pClass{F}$ el grupo de clases estrictas de $F$ y
\begin{math}
	\Gamma_{g}=g(\Idinf{B}\Idfin{\cal{O}})g^{-1}\cap B_{+}^{\times}
\end{math}
es un subgrupo discreto de $B_{\infty,+}^{\times}$ actuando en $\hP^{r}$.
Este cociente es la \emph{variedad de Shimura cuaterni\'{o}nica %
(de nivel $\frak{N}$)}\index{variedad de Shimura}
asociada a $B$ y al orden de Eichler $\cal{O}$. Es una variedad compleja
de dimensi\'{o}n $r$ y, salvo que $\#\pClass{F}=1$, no es conexa.
La denotaremos $\shimura[B]{\frak{N}}$. Si el n\'{u}mero de lugares del
infinito no ramificados es $r=1$, entonces se obtiene una curva. En general, si
$B$ es de divisi\'{o}n, $\shimura[B]{\frak{N}}$ es una variedad compacta.

\begin{obsGruposYOrdenesAsociadosAUnaClaseEstricta}%
	\label{obs:gruposyordenesasociadosaunaclaseestricta}
	Dado un ideal de $F$, $\frak{a}$, se puede elegir un elemento
	$\hhat{a}\in\Adfin{\oka{F}}$ tal que
	\begin{equation}
		\label{eq:ideleintegroasociado}
		\frak{a} \,=\,\hhat{a}\Adfin{\oka{F}}\,\cap\,F
		\text{ .}
	\end{equation}
	%
	Este id\`{e}le est\'{a} determinado salvo una unidad en
	$\Idfin{\oka{F}}$. Luego, existe $\hhat{\alpha}\in\Idfin{B}$ tal que
	$\nrd(\hhat{\alpha})=\hhat{a}$, pues la norma reducida es sobreyectiva
	en las componentes no arquimedeanas. A $\hhat{\alpha}$ le asociamos,
	una familia de ret\'{\i}culos locales
	$\Adfin{\cal{O}}_{\hhat{\alpha}}$, un ret\'{\i}culo global
	$\cal{O}_{\hhat{\alpha}}$ y un grupo $\Gamma_{\hhat{\alpha}}$ de la
	siguiente manera:
	\begin{equation}
		\label{eq:ordenasociadocuaterniones}
		\Adfin{\cal{O}}_{\hhat{\alpha}} \,=\,
			\hhat{\alpha}\Adfin{\cal{O}}\hhat{\alpha}^{-1}
			\text{ ,}\quad
		\cal{O}_{\hhat{\alpha}}\,=\,\Adfin{\cal{O}}_{\hhat{\alpha}}
			\,\cap\, B
			\quad\text{y}\quad
		\Gamma_{\hhat{\alpha}}\,=\,
			\cal{O}_{\hhat{\alpha},+}^{\times}\,=\,
			\cal{O}_{\hhat{\alpha}}^{\times}\cap B_{+}^{\times}
		\text{ .}
	\end{equation}
	%
	El ret\'{\i}culo $\cal{O}_{\hhat{\alpha}}$ es un orden de Eichler
	de nivel igual al nivel de $\cal{O}$; ambos \'{o}rdenes son
	\emph{localmente} conjugados, pero no necesariamente conjugados por un
	elemento de $B^{\times}$. El grupo $\cal{O}_{\hhat{\alpha},+}^{\times}$
	es el grupo de unidades ``totalmente positivas'' del orden
	$\cal{O}_{\hhat{\alpha}}$. Cambiando $\hhat{\alpha}$ por otro elemento
	perteneciente a la misma clase en
	\begin{math}
		B_{+}^{\times}\backslash\Idfin{B}/\Idfin{\cal{O}}
	\end{math}~,
	se obtiene un orden de $B$ conjugado a $\cal{O}_{\hhat{\alpha}}$ por un
	elemento de $B_{+}^{\times}$. Por esta raz\'{o}n, utilizaremos,
	principalmente, un sub\'{\i}ndice $\frak{a}$ en lugar de
	$\hhat{\alpha}$.
\end{obsGruposYOrdenesAsociadosAUnaClaseEstricta}

La elecci\'{o}n de los representantes $g$ en la descomposici\'{o}n de
$\shimura[B]{\frak{N}}$ se puede hacer expl\'{\i}cita en el siguiente sentido.
En primer lugar, se determina un sistema de representantes de $\pClass{F}$,
$\{\frak{a}\}_{[\frak{a}]\in\pClass{F}}$. Quedan determinados conjuntos
\begin{math}
	\{\hhat{a}\}_{\frak{a}}\subset\Adfin{\oka{F}}
\end{math}
y
\begin{math}
	\{\hhat{\alpha}\}_{\frak{a}}\subset\Idfin{B}
\end{math}~,
los \'{o}rdenes $\cal{O}_{\frak{a}}$ y los grupos $\Gamma_{\frak{a}}$
seg\'{u}n la Observaci\'{o}n
\ref{obs:gruposyordenesasociadosaunaclaseestricta}. Realizadas estas
elecciones, la variedad de Shimura $\shimura[B]{\frak{N}}$ se identifica con
una uni\'{o}n disjunta de variedades conexas:
\begin{equation}
	\label{eq:descomposicionvariedaddeshimuraindefinida}
\begin{aligned}
	\shimura[B]{\frak{N}} \,=\,
		\bigsqcup_{[\frak{a}]\in\pClass{F}}\,
			B_{+}^{\times}\backslash\big(\hP^{r}\times
				\hhat{\alpha}\Idfin{\cal{O}}\big)
		& \,\simeq\,
		\bigsqcup_{[\frak{a}]\in\pClass{F}}\,
		\Gamma_{\frak{a}}\backslash\hP^{r} \\
	B_{+}^{\times}(z,\hhat{\alpha}\Idfin{\cal{O}}) & \,\mapsto\,
		\Gamma_{\frak{a}}z
	\text{ .}
\end{aligned}
\end{equation}
%
Cada una de estas componentes es compacta, lo que implica que
$\shimura[B]{\frak{N}}$ es compacta.

\begin{obsVariedadDeShimuraEsCompacta}\label{obs:variedaddeshimuraescompacta}
	De hecho, si
	\begin{math}
		\cal{O}_{\hhat{\alpha}}^{1}=
			\{x\in\cal{O}_{\hhat{\alpha}}\,:\,\nrd(x)=1\}
	\end{math}~,
	el cociente
	$\cal{O}_{\hhat{\alpha}}^{1}\backslash\hP^{r}$ es compacto.
	Definimos
	\begin{align*}
		\adeles{B}^{(1)} \,=\,\big\{x\in\ideles{B}\,:\,
			|\nrd(x)|_{\adeles{F}}=1\big\}\text{ ,}\quad
		\adeles{B}^{1} \,=\,\big\{x\in\ideles{B}\,:\,\nrd(x)=1\big\}
	\end{align*}
	%
	y, dado un subgrupo $H\subset\ideles{B}$, subgrupos
	$H^{(1)}=H\cap\adeles{B}^{(1)}$ y $H^{1}=H\cap\adeles{B}^{1}$.
	Recordemos, tambi\'{e}n, el siguiente hecho: dados un grupo
	topol\'{o}gico $G$, un subgrupo abierto $H$ y un subconjunto arbitrario
	$A\subset G$, el producto $A\cdot H$ y su complemento son abiertos,
	puedi\'{e}ndose escribir como uniones de aquellas coclases $x\cdot H$
	que los intersecan:
	\begin{math}
		A\cdot H=\bigcup_{x\in A\cdot H}\,x\cdot H
	\end{math}~y
	\begin{math}
		G\setmin\big(A\cdot H\big)=
			\bigcup_{x\not\in A\cdot H}\,x\cdot H
	\end{math}~. Ahora, dado que $B$ es un \'{a}lgebra indefinida,
	$\lugares[\infty]{F}$ contiene, al menos, un lugar en donde el
	\'{a}lgebra no ramifica. En particular, por aproximaci\'{o}n fuerte
	(Teorema~\ref{thm:aproxfuerte}), el producto $B^{1}\cdot\Adinf{B}^{1}$
	es denso en $\adeles{B}^{1}$ y, puesto que
	\begin{math}
		\Adinf{B}^{1}\,\Adfin{\cal{O}}^{1}_{\hhat{\alpha}}=
			\adeles{B}^{1}\cap\big(B^{\times}_{\infty,+}\,
				\Idfin{\cal{O}}_{\hhat{\alpha}}\big)
	\end{math}
	es un subgrupo abierto de $\adeles{B}^{1}$, el complemento
	\begin{math}
		\adeles{B}^{1}\setmin B^{1}\cdot\big(\Adinf{B}^{1}\,
			\Adfin{\cal{O}}^{1}_{\hhat{\alpha}}\big)
	\end{math}
	es abierto y
	\begin{align*}
		\adeles{B}^{1} & \,=\, B^{1}\Adinf{B}^{1}
			\Adfin{\cal{O}}^{1}_{\hhat{\alpha}}
		\text{ .}
	\end{align*}
	%
	En consecuencia, la aplicaci\'{o}n $\adeles{B}^{1}\rightarrow\hP^{r}$
	dada por $g=(g_{\infty},\hhat{g})\mapsto g_{\infty}\cdot\mathbf{i}$
	induce una correspondencia
	\begin{align*}
		B^{1}\backslash\adeles{B}^{1}/K_{\infty}
			\Adfin{\cal{O}}^{1}_{\hhat{\alpha}} & \,\simeq\,
			\cal{O}^{1}_{\hhat{\alpha}}\backslash\hP^{r}
		\text{ ,}
	\end{align*}
	%
	donde $K_{\infty}=\SO{2}^{r}\times(\bb{H}^{1})^{n-r}$ es el subgrupo
	compacto maximal en $\Adinf{B}^{1}$. Ahora, como $B$ es de
	divisi\'{o}n, el cociente $B^{\times}\backslash\adeles{B}^{(1)}$ es
	compacto.%
	\footnote{
		\cite[Ch.~III, \S~1]{Vigneras}
		%(teorema de Fujisaki)
	}
	El subgrupo
	\begin{math}
		B^{(1)}_{\infty,+}\,\Idfin{\cal{O}}_{\hhat{\alpha}}=
			\adeles{B}^{(1)}\cap\big(B^{\times}_{\infty,+}\,
				\Idfin{\cal{O}}_{\hhat{\alpha}}\big)
	\end{math}
	es abierto, lo que implica que
	\begin{math}
		W=\adeles{B}^{(1)}\setmin B^{\times}\cdot\big(
			B^{(1)}_{\infty,+}\,
				\Idfin{\cal{O}}_{\hhat{\alpha}}\big)
	\end{math}
	sea abierto. En particular, $B^{\times}\backslash W$ es abierto en el
	cociente y su complemento,
	\begin{math}
		B^{\times}\backslash B^{\times}\cdot\big(B^{(1)}_{\infty,+}\,
			\Idfin{\cal{O}}_{\hhat{\alpha}}\big)
	\end{math}
	es compacto. Pero
	\begin{math}
		B_{\infty,+}^{(1)}=
			\big(\centre(\bb{R})\cap\adeles{B}^{(1)}\big)\cdot
				B_{\infty}^{1}
	\end{math}
	y, a fin de cuentas,
	\begin{align*}
		\big(\centre(\bb{R})\cap\adeles{B}^{(1)}\big)\,B^{\times}
			\backslash B^{\times}\cdot\big(B_{\infty,+}^{(1)}\,
				\Idfin{\cal{O}}_{\hhat{\alpha}}\big)/
				K_{\infty}\Idfin{\cal{O}}_{\hhat{\alpha}}
			& \,=\,
			B^{1}\backslash B^{1}\cdot\big(\Adinf{B}^{1}\,
				\Adfin{\cal{O}}^{1}_{\hhat{\alpha}}\big)/
				K_{\infty}\Adfin{\cal{O}}^{1}_{\hhat{\alpha}}
	\end{align*}
	%
	es compacto.
\end{obsVariedadDeShimuraEsCompacta}

Cuando $B/F$ es un \'{a}lgebra definida, $r=0$, no tenemos una acci\'{o}n sobre
$\hP$ y el cociente ad\'{e}lico
\begin{math}
	B^{\times}\backslash\Idinf{B}\times\Idfin{B}/\Idinf{B}\Idfin{\cal{O}}=
	B^{\times}\backslash\Idfin{B}/\Idfin{\cal{O}}
\end{math}
es simplemente un conjunto finito de puntos en correspondencia con el conjunto
$\lClass{\cal{O}}$ de clases de ideales cuyo orden a derecha es $\cal{O}$
(Teorema \ref{thm:numerodeclasesfinito}).
% En este caso, la variedad de Shimura asociada es una variedad compacta de
% dimensi\'{o}n $0$, que seguimos denotando $\shimura[B]{\frak{N}}$.
