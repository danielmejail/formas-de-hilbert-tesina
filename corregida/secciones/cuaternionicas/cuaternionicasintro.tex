% 
% En esta secci\'{o}n vemos c\'{o}mo asociarle a un \'{a}lgebra de cuaterniones
% sobre un cuerpo de n\'{u}meros totalmente real una variedad de Shimura.
% %Cuando el \'{a}lgebra es un \'{a}lgebra de matrices, recuperamos las
% %variedades $Y(\Gamma)$ definidas como el cociente de un producto de
% %copias del semiplano complejo por un subgrupo discreto de $\GLtp_{2}(F)$
% %(matrices invertibles cuyo determinante es totalmente positivo).
% % Las propiedades de estas variedades son marcadamente distintas, si el
% % \'{a}lgebra es totalmente definida, indefinida o un \'{a}lgebra de matrices.
% Luego definimos, usando el lenguaje ad\'{e}lico, los espacios de formas
% modulares cuaterni\'{o}nicas, objetos an\'{a}logos a las formas modulares
% cl\'{a}sicas, y los operadores de Hecke actuando en los correspondientes
% subespacios de formas cuspidales. El producto interno de Petersson nos
% permite separar los subespacios de formas nuevas y, finalmente, enunciamos
% el resultado que nos garantiza que las formas modulares para un \'{a}lgebra
% de cuaterniones se pueden ver como formas de Hilbert.

Comenzamos este cap\'{\i}tulo asociando un objeto geom\'{e}trico a un orden
de Eichler de nivel $\frak{N}$ en un \'{a}lgebra de cuaterniones $B/F$
sobre un cuerpo de n\'{u}meros totalmente real. Estos objetos se denominan
variedades de Shimura cuaterni\'{o}nicas. Estas variedades son compactas,
si y s\'{o}lo si $B$ es un \'{a}lgebra de divisi\'{o}n y esta propiedad se
ve reflejada en la definici\'{o}n de las formas automorfas (modulares)
correspondientes. Luego de introducir una representaci\'{o}n del grupo de
unidades $B^{\times}$, definimos las formas modulares asociadas a esta
representaci\'{o}n. Como en el caso de las formas de Hilbert, existe una
correspondencia entre formas modulares cuaterni\'{o}nicas y cierto espacio de
formas automorfas.

La correspondencia de Jacquet-Langlands, Teorema~\ref{thm:correspondenciajl},
permite reconstruir el m\'{o}dulo de Hecke $\spitzH{k}{\frak{N}}$ a partir de
espacios de formas cuaterni\'{o}nicas. \'{E}ste es el resultado fundamental en
el que se basan los m\'{e}todos para el c\'{a}lculo de las formas de Hilbert
descriptos en el Cap\'{\i}tulo~\ref{cap:metodos}, ya que reduce el problema
de determinar la acci\'{o}n de los operadores $T_{\frak{p}}$ al c\'{a}lculo de
los mismos en el espacio de formas modulares cuaterni\'{o}nicas. La naturaleza
de la correspondencia impone ciertas restricciones a la aplicabilidad de estos
m\'{e}todos.
% En lo que resta de la secci\'{o}n, se describir\'{a} la acci\'{o}n
% de Hecke en los espacios de formas cuaterni\'{o}nicas.

