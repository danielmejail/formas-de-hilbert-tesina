En lo que resta de esta introducci\'{o}n, intentaremos mostrar,
restringi\'{e}ndonos a la situaci\'{o}n en la que el cuerpo de base es
$\bb{Q}$, algunos puntos de la teor\'{\i}a que desarrollaremos m\'{a}s adelante
en un contexto m\'{a}s general. Antes, explicamos brevemente el contenido del
resto del presente trabajo.

En el Cap\'{\i}tulo~\ref{cap:preliminares}, recordamos algunos resultados
generales de la teor\'{\i}a de las \'{a}lgebras de cuaterniones. Nos
interesar\'{a}, especialmente, entender el caso en el que el cuerpo de base es
un cuerpo de n\'{u}meros. Para poder tratar este caso, repasamos las
definiciones b\'{a}sicas del lenguaje de los ad\`{e}les y, utilizando esta
herramienta, podremos enunciar los teoremas de clasificaci\'{o}n (Teorema~%
\ref{teo:clasificacion}), de aproximaci\'{o}n fuerte (Teorema~%
\ref{teo:aproxfuerte}) y de la norma (Teorema~\ref{teo:delanorma}). Dada un
\'{a}lgebra de cuaterniones y un orden en este \'{a}lgebra podemos definir un
objeto an\'{a}logo al grupo de clases de un cuerpo de n\'{u}meros, el
\emph{conjunto} de clases del orden. Los elementos de este conjunto son clases
de ideales \emph{invertibles} y, en el caso de un orden de Eichler, la cantidad
de dichas clases es finita. Hacia el final del cap\'{\i}tulo enunciamos un
resultado que relaciona el conjunto de clases de un orden de Eichler con un
grupo de clases del cuerpo de base.

En el Cap\'{\i}tulo~\ref{cap:dehilbert}, introducimos las formas de Hilbert
sobre un cuerpo totalmente real $F$; daremos una definici\'{o}n
``cl\'{a}sica'', como funciones en el producto $\hP^{n}$ de copias del
semiplano complejo superior, y una definici\'{o}n ``ad\'{e}lica'', como
funciones en el grupo $\GL_{2}(\adeles{F})$. Este cap\'{\i}tulo tambi\'{e}n
tiene por ojetivo introducir los operadores de Hecke y mostrar en qu\'{e}
sentido es posible diagonalizarlos simult\'{a}neamente y caracterizarlos por
los sistemas de autovalores asociados. Calcular los espacios de formas
de Hilbert, se reduce entonces a determinar los posibles sistemas de
autovalores para los operadores de Hecke.

Los \'{u}ltimos dos cap\'{\i}tulos est\'{a}n dedicados a las formas modulares
cuaterni\'{o}nicas. En el Cap\'{\i}tulo~\ref{cap:cuaternionicas} definimos
formas modulares para un \'{a}lgebra de cuaterniones arbitraria y enunciamos la
correspondencia de Jacquet-Langlands. Finalmente, en el Cap\'{\i}tulo~%
\ref{cap:metodos}, nos restringimos a las \'{a}lgebras definidas y a las
\'{a}lgebras indefinidas ramificadas en todos salvo un \'{u}nico lugar
arquimediano, describiendo en mayor detalle los objetos utiizados en la
pr\'{a}ctica para calcular efectivamente formas de Hilbert.
