A continuaci\'{o}n explicamos brevemente la correspondencia entre las formas
modulares el\'{\i}pticas ``cl\'{a}sicas'' y sus contrapartes ad\'{e}licas. En
esta secci\'{o}n, la expresi\'{o}n $f\operadormatrices{k}{\gamma}$ estar\'{a}
dada por \eqref{eq:operadormatricesusual} con $n=1$.

A una funci\'{o}n $f:\,\hP\rightarrow\bb{C}$, le podemos asociar, bajo ciertas
condiciones, una nueva funci\'{o}n
\begin{math}
	\phi_{f}:\,\GL_{2}(\adeles{\bb{Q}})\rightarrow\bb{C}
\end{math}~. Una funci\'{o}n definida en $\GL_{2}(\adeles{\bb{Q}})$ se denomina
\emph{funci\'{o}n ad\'{e}lica}. Si $g\in\GL_{2}(\adeles{\bb{Q}})$, por
\eqref{eq:aproximacionfuerteintro}, existen $\gamma\in\GL_{2}(\bb{Q})$,
$g_{\infty}\in\GLtp_{2}(\bb{R})$ y $\hhat{\alpha}\in K_{0}(N)$ tales que
$g=\gamma g_{\infty}\hhat{\alpha}$. Si $f\operadormatrices{k}{\gamma}=f$ para
toda $\gamma\in\Gamma_{0}(N)$, entonces
\begin{equation}
	\label{eq:funcionadelicacorrespondientesobreq}
	\phi_{f}(g) \,=\, \det(g_{\infty})^{k/2}j(g_{\infty},i)^{-k}
		f(g_{\infty}\cdot i)
\end{equation}
%
es independiente de la descomposici\'{o}n
\begin{math}
	g=\gamma g_{\infty}\hhat{\alpha}
\end{math}~. Se puede comprobar que $\phi_f$ verifica \textit{(i)} a
\textit{(iv)} de la Proposici\'{o}n~\ref{propo:introequivformaglq}. Queremos
ver qu\'{e} otras propiedades tiene esta funci\'{o}n.

Adem\'{a}s de estas condiciones de invarianza, $\phi_{f}$ satisface ciertas
condiciones de regularidad. Si $f$ es una funci\'{o}n meromorfa y de peso $k$
invariante para $\Gamma_{0}(N)$, entonces $f\in\spitz{k}{\Gamma_{0}(N)}$, si y
s\'{o}lo si la funci\'{o}n $\Gamma_{0}(N)$-invariante $f(z)\,\Im(z)^{k/2}$
est\'{a} acotada en $\hP$.%
\footnote{
	\cite[Ch.~2, \S~1]{MiyakeModular}
}
%En general, si $f$ es holomorfa en $\hP$ y de peso $k$
%invariante y satisface la condici\'{o}n de crecimiento moderado entonces $f$
%es modular.
En ese caso, la igualdad
\begin{align*}
	\big|\phi_{f}(\gamma g_{\infty}\hhat{\alpha})\big| & \,=\,
		\big|f(g_{\infty}\cdot i)\,j(g_{\infty},i)^{-k}\,
			\det(g_{\infty})^{k/2}\big|
		\,=\, \big|f(z)\,\Im(z)^{k/2}\big|
\end{align*}
%
($z=g_{\infty}\cdot i$) muestra que $\phi_{f}$ est\'{a} acotada.
Dado que el cociente
\begin{math}
	Z(\adeles{\bb{Q}})\,\GL_{2}(\bb{Q})\backslash
		\GL_{2}(\adeles{\bb{Q}})
\end{math}
tiene medida finita,%
\footnote{
	El cociente \eqref{eq:curvamodularglqadeles} se identifica con la curva
	modular $Y_{0}(N)$, que tiene volumen finito, y el grupo
	$K_{0}(N)\,\SO{2}$ es compacto.
}
se deduce que la integral de $|\phi_{f}|^{2}$ sobre este espacio es finita.

Ahora bien, la definici\'{o}n de forma cuspidal est\'{a} dada en t\'{e}rminos
de las expansiones de Fourier: si $f\in\modular{k}{\Gamma_{0}(N)}$ y
\begin{math}
	\sum_{n\geq 0}\,a_{n}(f\operadormatrices{k}{\gamma})\,q^{n}
\end{math}
es la expansi\'{o}n de Fourier de $f\operadormatrices{k}{\gamma}$, entonces
$f$ es cuspidal, si $a_{0}(f\operadormatrices{k}{\gamma})=0$ para toda
$\gamma\in\GLtp_{2}(\bb{Q})$. Podemos expresar los coeficientes de la siguiente
manera:
\begin{align*}
	a_{n}\big(f\operadormatrices{k}{\gamma}\big) & \,=\,\int_{-1/2}^{1/2}\,
		\big(f\operadormatrices{k}{\gamma}\big)(z+t)\,
			e^{-2\pi i n (z+t)}\,dt
\end{align*}
%
para cualquier $z\in\hP$. Para relacionar esto con el comportamiento de la
funci\'{o}n $\phi_{f}$, tomamos $x\in\adeles{\bb{Q}}$ y
$g\in\GL_{2}(\adeles{\bb{Q}})$. Sea $\Phi$ la funci\'{o}n
\begin{align*}
	\Phi(x) & \,=\,\phi_{f}\left(
		\left[\begin{matrix} 1 & x \\ & 1 \end{matrix}\right]\,g\right)
\end{align*}
%
y notemos que, si $a\in\bb{Q}$, entonces $\Phi(x+a)=\Phi(x)$. Podemos
considerarla como funci\'{o}n en el grupo compacto
$\bb{Q}\backslash\adeles{\bb{Q}}$ y admite una expansi\'{o}n de Fourier:
\begin{align*}
	\Phi(x) & \,=\,\sum_{a\in\bb{Q}}\,c_{a}\,\psi(ax)
	\text{ ,}
\end{align*}
%
donde $\psi$ es un car\'{a}cter de $\adeles{\bb{Q}}$ trivial en $\bb{Q}$.
El coeficiente $c_{a}$ est\'{a} dado por
\begin{align*}
	c_{a} & \,=\,\int_{\bb{Q}\backslash\adeles{\bb{Q}}}\,
		\phi_{f}\left(
			\begin{bmatrix} 1 & x \\ & 1 \end{bmatrix}\,g\right)\,
		\psi(-ax)\,dx
	\text{ .}
\end{align*}
%
Nos interesa, en particular, $c_{0}$. Sean
\begin{math}
	\gamma,\gamma'\in\GL_{2}(\bb{Q})
\end{math}~,
\begin{math}
	g_{\infty},g_{\infty}'\in\GLtp_{2}(\bb{R})
\end{math}
y
\begin{math}
	\hhat{\alpha},\hhat{\alpha}'\in K_{0}(N)
\end{math}
tales que
\begin{align*}
	g \,=\,\gamma g_{\infty}\hhat{\alpha} & \quad\text{y}\quad
	\begin{bmatrix} 1 & x \\ & 1 \end{bmatrix}\,
		\gamma g_{\infty}\hhat{\alpha} \,=\,
		\gamma' g_{\infty}'\hhat{\alpha}'
	\text{ .}
\end{align*}
%
En particular,
\begin{align*}
	g_{\infty}' & \,=\,(\gamma_{\infty}')^{-1}
		\begin{bmatrix} 1 & x_{\infty} \\ & 1 \end{bmatrix}
		\gamma_{\infty}g_{\infty}
	\text{ ,}
\end{align*}
%
donde $x_{\infty}$ denota la componente arquimediana de $x$ y
$\gamma_{\infty}=\inc[\infty](\gamma)\in\GL_{2}(\bb{R})$. Entonces
\begin{align*}
	\phi_{f}\left(\begin{bmatrix} 1 & x \\ & 1 \end{bmatrix}
		\gamma g_{\infty}\hhat{\alpha}\right) & \,=\,
		\phi_{f}(\gamma' g_{\infty}'\hhat{\alpha}') \,=\,
		f(g_{\infty}'\cdot i)\,
			j(g_{\infty}',i)^{-k}\det(g_{\infty}')^{k/2}
	\text{ .}
\end{align*}
%
Pero esto es igual a
\begin{align*}
	&
	\big(f\operadormatrices{k}{\gamma_{\infty}'^{-1}}\big)
		\big((\gamma_{\infty}g_{\infty}\cdot i)+x_{\infty}\big)\,
	j\left(\begin{bmatrix} 1 & x_{\infty} \\ & 1 \end{bmatrix}
		\gamma_{\infty} g_{\infty},i\right)^{-k}\,
		\det\left(\begin{bmatrix} 1 & x_{\infty} \\ & 1 \end{bmatrix}
		\gamma_{\infty} g_{\infty}\right)^{k/2} \\
	& \qquad\,=\,
	\big(f\operadormatrices{k}{\gamma_{\infty}'^{-1}}\big)
		\big((\gamma_{\infty}g_{\infty}\cdot i)+x_{\infty}\big)\,
		j(\gamma_{\infty} g_{\infty},i)^{-k}\,
		\det(\gamma_{\infty} g_{\infty})^{k/2}
	\text{ ,}
\end{align*}
%
por la propiedad de cociclo de $j$ y porque la matriz asociada a $x$ tiene
determinante $1$. Para recuperar los coeficientes de la expansi\'{o}n de
$\Phi$, integramos sobre el compacto
\begin{math}
	\big[-\frac{1}{2},\frac{1}{2}\big]\times\prod_{p}\,\bb{Z}_{p}
\end{math}~:
\begin{align*}
	c_{0} & \,=\,\int_{\bb{Q}\backslash\adeles{\bb{Q}}}\,
		\phi_{f}\left(
			\begin{bmatrix} 1 & x \\ & 1 \end{bmatrix} g\right)\,dx
		\,=\,C\,\int_{\left[-\frac{1}{2},\frac{1}{2}\right]\times%
					\prod_{p}\,\bb{Z}_{p}}\,
		\left(f\operadormatrices{k}{\gamma_{\infty}'^{-1}}\right)
		\big((\gamma_{\infty}g_{\infty}\cdot i)+x_{\infty}\big)\,dx \\
	& \,=\,C\,\int_{-1/2}^{1/2}\,
		f\operadormatrices{k}{\gamma_{\infty}'^{-1}} (z+t)\,dt
		\,=\,C\,a_{0}(f\operadormatrices{k}{\gamma_{\infty}'^{-1}})
		\,=\, 0
\end{align*}
%
($C$ es simplemente alguna constante y $z=\gamma_{\infty}g_{\infty}\cdot i$).
En definitiva, hemos demostrado que la funci\'{o}n $\phi_{f}$ asociada a una
$f$ cuspidal es \emph{de cuadrado integrable (m\'{o}dulo el centro)}:
\index{funcion de cuadrado integrable@funci\'{o}n de cuadrado integrable}
\begin{align*}
	\int_{Z(\adeles{\bb{Q}})\GL_{2}(\bb{Q})\backslash%
		\GL_{2}(\adeles{\bb{Q}})}\,|\phi_{f}(g)|^{2}\,dg & \,<\,\infty
\end{align*}
%
y es \emph{cuspidal} en tanto que
\index{funcion cuspidal@funci\'{o}n cuspidal}
\begin{align*}
	\int_{\bb{Q}\backslash\adeles{\bb{Q}}}\,
		\phi_{f}\left(\begin{bmatrix} 1 & x \\ & 1 \end{bmatrix} g
			\right)\,dx
		& \,=\,0
\end{align*}
%
para $g\in\GL_{2}(\adeles{\bb{Q}})$.

Esto no es suficiente para caracterizar la imagen de $\spitz{k}{\Gamma_{0}(N)}$
por la aplicaci\'{o}n $f\mapsto\phi_{f}$. Para ello, introducimos el operador
de Casimir. Usando la \emph{descomposici\'{o}n de Iwasawa}, si
$g_{\infty}\in\GLtp_{2}(\bb{R})$ entonces existen $x,\theta\in\bb{R}$,
$y,u\in\bb{R}_{>0}$ tales que
\index{descomposicion de Iwasawa@descomposici\'{o}n de Iwasawa}
\begin{equation}
	\label{eq:descomposiciondegldos}
	g_{\infty} \,=\,\begin{bmatrix} u & \\ & u \end{bmatrix}\,
		\begin{bmatrix} y^{1/2} & xy^{-1/2} \\
		& y^{-1/2} \end{bmatrix}\,
			\begin{bmatrix} \cos\,\theta & \sin\,\theta \\
			-\sin\,\theta & \cos\,\theta\end{bmatrix}
	\text{ .}
\end{equation}
%
Teniendo en cuenta esta descomposici\'{o}n, definimos el
\emph{operador de Casimir} como\index{operador de Casimir}
\begin{align*}
	\Delta & \,=\, -y^{2}\left(
		\frac{\partial^{2}}{\partial x^{2}}+
		\frac{\partial^{2}}{\partial y^{2}} \right)\,-\,
		y\frac{\partial^{2}}{\partial x\partial\theta}
	\text{ .}
\end{align*}
%
La funci\'{o}n $\phi_{f}$ satisface la ecuaci\'{o}n diferencial:
\begin{align*}
	\Delta\phi_{f} & \,=\, -\frac{k}{2}\,
		\left(\frac{k}{2}-1\right)\,\phi_{f}
	\text{ ,}
\end{align*}
%
es decir, vista como funci\'{o}n en $\GLtp_{2}(\bb{R})$, es $C^{\infty}$ y
es soluci\'{o}n de esta ecuaci\'{o}n. Dada $g_{\infty}\in\GLtp_{2}(\bb{R})$,
por \eqref{eq:descomposiciondegldos}, se cumple que
\begin{equation}
	\label{eq:funcioncorrespondientefactorizacionelipticas}
	\phi_{f}(g_{\infty}) \,=\,f(x+iy)\,y^{k/2}\,e^{ik\theta}
	\text{ .}
\end{equation}
%
Derivando respecto de las variables $x,y,\theta$,
\begin{align*}
	\frac{\partial^{2}}{\partial x^{2}}\,\phi_{f}(g_{\infty})
		& \,=\,\frac{\partial^{2}f}{\partial x^{2}}(x+iy)\,
			y^{k/2}\,e^{ik\theta}\text{ ,} \\
	\frac{\partial^{2}}{\partial x\partial\theta}\,\phi_{f}(g_{\infty})
		& \,=\,\frac{\partial f}{\partial x}(x+iy)\,
			y^{k/2}(ik)\,e^{ik\theta}\text{ ,} \\
	\frac{\partial^{2}}{\partial y^{2}}\,\phi_{f}(g_{\infty})
		& \,=\,\left(\frac{\partial^{2}f}{\partial y^{2}}(x+iy)\,
			y^{k/2}\,+\,\frac{\partial f}{\partial y}(x+iy)\,
			k\,y^{k/2-1}\right. \\
		& \qquad\,+\,\left. f(x+iy)\,
			\tfrac{k}{2}\,\big(\tfrac{k}{2}-1\big)\,
			y^{k/2-2}\right)\,e^{ik\theta}
	\text{ .}
\end{align*}
%
Al ser $f$ una funci\'{o}n holomorfa, se deduce que $\phi_{f}$ es
autofunci\'{o}n de $\Delta$ con autovalor
\begin{math}
	-\frac{k}{2}\left(\frac{k}{2}-1\right)
\end{math}~.

Diremos que una funci\'{o}n en
\begin{math}
	\GL_{2}(\adeles{\bb{Q}})=
		\GL_{2}(\bb{R})\times\GL_{2}(\Adfin{\bb{Q}})
\end{math}
a valores complejos es \emph{suave},
\index{funcion suave@funci\'{o}n suave en los ad\`{e}les}
si es $C^{\infty}$ en la primera variable y localmente constante en la segunda.

\begin{propoEquivFormaGLQIntro}[{\cite[\S~3]{GelbartAutomorphicOnAdeles}}]%
	\label{propo:introequivformaglq}
	La aplicaci\'{o}n que a una funci\'{o}n $f:\,\hP\rightarrow\bb{C}$
	le asigna la funci\'{o}n
	$\phi_{f}:\,\GL_{2}(\adeles{\bb{Q}})\rightarrow\bb{C}$ dada por
	\begin{equation}
		\label{eq:funcionadelicacorrespondienteelipticas}
		\phi_{f}(\gamma g_{\infty}\hhat{\alpha})
			\,=\,\det(g_{\infty})^{k/2}j(g_{\infty},i)^{-k}\,
				f(g_{\infty}\cdot i)
	\end{equation}
	%
	determina un isomorfismo entre el espacio de formas cuspidales
	$\spitz{k}{\Gamma_{0}(N)}$ y el espacio de funciones suaves
	\begin{math}
		\phi:\,\GL_{2}(\adeles{\bb{Q}})\rightarrow\bb{C}
	\end{math}
	que satisfacen
	\begin{itemize}
		\item[(i)] $\phi(\gamma g)=\phi(g)$ para toda
			$\gamma\in\GL_{2}(\bb{Q})$,
		\item[(ii)] $\phi(g\hhat{\beta})=\phi(g)$ para toda
			$\hhat{\beta}\in K_{0}(N)$,
			%\hhat{\cal{O}}^{\times}$,
		\item[(iii)] $\phi(gr_{\theta})=e^{ik\theta}\,\phi(g)$ para
			toda
			\begin{math}
				r_{\theta}=
				\left[\begin{smallmatrix}
					\cos\,\theta & \sin\,\theta\\[5pt]
					-\sin\,\theta & \cos\,\theta
				\end{smallmatrix}\right]
				\in\SO{2}
			\end{math}
			y
		\item[(iv)] $\phi(\eta g)=\phi(g)$ para toda
			$\eta\in Z(\adeles{\bb{Q}})$,
	\end{itemize}
	%
	y adem\'{a}s
	\begin{itemize}
		\item[(v)] $\phi$ es de \emph{crecimiento moderado}:
\index{funcion de crecimiento moderado@funci\'{o}n de crecimiento moderado}
			para todo
			compacto $\Omega\subset\GL_{2}(\adeles{\bb{Q}})$ y
			toda $c>0$, existen constantes $C$ y $N$ positivas
			tales que, para	$g\in\Omega$ y $a\in\ideles{\bb{Q}}$
			tal que $\left|a\right|>c$
			\begin{align*}
				\left|\phi\left(
					\begin{bmatrix} a & \\ & 1
						\end{bmatrix} g\right)\right|
					& \,\leq\,C|a|^{N}
				\text{ ,}
			\end{align*}
			%
		\item[(vi)] $\phi$, como funci\'{o}n de $\GLtp_{2}(\bb{R})$,
			es soluci\'{o}n de la ecuaci\'{o}n diferencial
			\begin{align*}
				\Delta\phi & \,=\, -\frac{k}{2}
					\left(\frac{k}{2}-1\right)\phi
				\text{ ,}
			\end{align*}
			%
		\item[(vii)] $\phi$ es \emph{cuspidal}, es decir,
			\begin{align*}
				\int_{\bb{Q}\backslash\adeles{\bb{Q}}}\,
					\phi\left(\begin{bmatrix} 1 & x \\ & 1
						\end{bmatrix} g\right)\,dx
					& \,=\,0
		  	\end{align*}
			%
			para casi todo $g$.
	\end{itemize}
	%
\end{propoEquivFormaGLQIntro}

% \begin{defFormaGLQIntro}\label{thm:defformaglqintro}
	% Diremos que una funci\'{o}n
	% \begin{math}
		% \phi:\,\GL_{2}(\adeles{\bb{Q}})\rightarrow\bb{C}
	% \end{math}
	% es una forma modular de peso $k$ y nivel $N$ para
	% ${\GL_{2}}_{/\bb{Q}}$, si cumple con las condiciones \textit{i} a
	% \textit{vi}. Diremos que es cuspidal, si adem\'{a}s cumple con
	% \textit{vii}.
% \end{defFormaGLQIntro}
