Sea $B=\varquatalg[\bb{Q}]{-1,-11}$ el \'{a}lgebra de cuaterniones de
discriminante $11$ sobre $\bb{Q}$, generada por elementos $\{1,i,j,k\}$ que
verifican $i^2=-1$, $j^2=-11$ y $k=ij=-ji$.
A modo de ejemplo, calculamos el espacio $\spitz[B]{2}{1}:=\Theta_0(11,1)'$.

En primer lugar, buscamos un orden maximal en $B$. El ret\'{\i}culo
$\cal{R}=\generado{1,i,j,k}$ es un orden en $B$, pero no es maximal
% , pues el discriminante de la $\bb{Z}$-base de $\cal{R}$ es igual a
($\discriminante{1,i,j,k}=4^2\,11^2$).
% Si $\cal{R}'$ es un orden generado como ret\'{\i}culo por una base que se
% obtiene a partir de $\cal{R}$ por una matriz $A$, entonces
% $\discriminante{\cal{R}'}=\det(A)^{2}\,\discriminante{\cal{R}}$.
Si definimos $\rho=\frac{1+i+j+k}{2}$ y $z=\frac{1+j}{2}$, entonces los
ret\'{\i}culos $\cal{R}'=\generado{1,i,j,\rho}$ y
$\cal{O}=\generado{1,i,z,iz}$ son \'{o}rdenes y se cumple
$\cal{O}\supset\cal{R}'\supset\cal{R}$, con inclusiones estrictas.
El orden $\cal{O} = \generado{1,i,z,iz}$ es maximal y, llamando $q_{\cal{O}}$ a
la forma cuadr\'{a}tica asociada,
\begin{align*}
	q_{\cal{O}}(u,v,w,t) & \,=\,\nrd(u\cdot 1+v\cdot i+w\cdot z+t\cdot iz)
		\\
	& \,=\,\big(u+\tfrac{w}{2}\big)^{2}
		+ 11\,\big(\tfrac{w}{2}\big)^{2}
		+ \big(v+\tfrac{t}{2}\big)^{2}
		+ 11\,\big(\tfrac{t}{2}\big)^{2} \\
	& \,=\,u^{2}+uw+3\,w^{2} + v^{2} +vt + 3\,t^{2}
	\text{ ,}
\end{align*}
%
las unidades del orden $\cal{O}$ son los elementos
$x=u\,1+v\,i+w\,z+t\,iz\in\cal{O}$ tales que $q_{\cal{O}}(u,v,w,t)=1$
(al ser $B$ definida, la forma es positiva). Se comprueba que
$\big|\cal{O}^{\times}\big|=4$.

En general, dado un ideal
\begin{math}
	I=\generado{\alpha_{1},\,\alpha_{2},\,\alpha_{3},\,\alpha_{4}}
\end{math}
de $B$, definimos la forma cuadr\'{a}tica $q_{I}(u,v,w,t)=\nrd(x)/\nrd(I)$,
donde
\begin{math}
	x=u\alpha_{1}+v\alpha_{2}+w\alpha_{3}+t\alpha_{4}
\end{math}~.
Definimos, tambi\'{e}n, el n\'{u}mero de representaci\'{o}n $n_{I}(p)$ como la
cantidad de soluciones a la ecuaci\'{o}n $q_{I}=p$. Si $\cal{O}$ es el orden a
derecha de $I$ y llamamos $\cal{O}_{I}$ a su orden a izquierda, entonces el
cociente
\begin{math}
	n_{I}(p)/\big|\cal{O}_{I}^{\times}\big|
\end{math}
es igual al n\'{u}mero de ideales de norma $p$, contenidos en el orden
$\cal{O}$ y pertenecientes a la clase del ideal $I$.%
\footnote{
	Una inclusi\'{o}n $J=b\,I\subset\cal{O}$ equivale a
	$b\,\cal{O}_{I}\subset I^{-1}$. Pero $I^{-1}=\nrd(I)^{-1}\,\conj{I}$,
	donde $\conj{I}$ es el ideal que se obtiene conjugando los elementos de
	$I$ (ver \cite[Propo.~1.17]{PizerAlgo} o el
	Corolario~\ref{coro:localmenteinvertibleesinvertible} del presente
	trabajo). En consecuencia, vale $b\,I\subset\cal{O}$, si y s\'{o}lo si
	$\cal{O}_{I}\,\big(\conj{b}\,\nrd(I)\big)\subset I$, ya que los
	\'{o}rdenes son estables por tomar conjugado. Pero
	$q_{I}(\conj{b}\,\nrd(I))=\nrd(b\,I)$ y dos representaciones
	$\cal{O}_{I}\,h=\cal{O}_{I}\,h'$ difieren en una unidad en
	$\cal{O}_{I}^{\times}$.
}
Si $p$ no divide al discriminante del \'{a}lgebra, entonces, eligiendo
representantes $\{\lista{I}{H}\}$ de $\lClass{\cal{O}}$,
\begin{align*}
	p+1 & \,=\,\sum_{i=1}^{H}\,\frac{n_{i}(p)}{%
		\big|\cal{O}_{i}^{\times}\big|}
\end{align*}
%
($n_{i}=n_{I_{i}}$, $q_{i}=q_{I_{i}}$ y $\cal{O}_{i}=\cal{O}_{I_{i}}$).

Volviendo al ejemplo, sea $p=2$. Entonces, se verifica que $n_{\cal{O}}(2)=4$
y que, de los tres ideales \'{\i}ntegros de norma $2$, s\'{o}lo uno es
principal. Por lo tanto, debe haber, al menos, otra clase en
$\lClass{\cal{O}}$. Antes de empezar a buscar otros ideales, observamos que el
elemento $1+i\in\cal{O}$ tiene norma reducida $\nrd(1+i)=2$ y que el ideal
\'{\i}ntegro de norma $2$ correspondiente a la clase principal es
$(1+i)\,\cal{O}$. Observamos tambi\'{e}n que, si dos ideales $I,J$ cumplen
$J\subset I$, entonces $\nrd(I)\mid\nrd(J)$. Buscamos, entonces, un ideal
$I\subset\cal{O}$ de norma $2$ que no sea principal. Para encontrarlo, notamos
que $\nrd(2)=\nrd(z-i)=4$. Entonces, los ideales principales $2\,\cal{O}$ y
$(i-z)\,\cal{O}$ tienen norma $4$ y son distintos. Consideramos
$I=2\,\cal{O}+(z-i)\,\cal{O}$. Calculando los productos de $(z-i)$ contra los
elementos de la base de $\cal{O}$ y ordenando los coeficientes, se puede
corroborar que $I=\generado{1-zi,i+z,i-z,1+zi}$. En particular,
$2\,\cal{O},(z-i)\,\cal{O}\subset I\subset\cal{O}$, con inclusiones estrictas e
$I\not=(1+i)\,\cal{O}$. Hemos encontrado dos ideales \'{\i}ntegros de norma $2$
pertenecientes a clases distintas. La forma cuadr\'{a}tica asociada a $I$ es
\begin{align*}
	q_{I}(u,v,w,t) % & \,=\,\tfrac{1}{2}\Big(
		% \big(u+t+\tfrac{v-w}{2}\big)^{2}
		% +\big(\tfrac{u-t}{2}+v+w\big)^{2}
		% +11\,\big(v-w\big)^{2} +11\,\big(-u+t\big)^{2}\Big) \\
	& \,=\,2\big(u^{2}+v^{2}+w^{2}+t^{2}\big)+uv-2\,ut-2\,vw-wt
	\text{ .}
\end{align*}
%
Debemos hallar $\big|\cal{O}_{I}^{\times}\big|$ y $n_{I}(2)$. En primer lugar,
$n_{I}(2)=12$. En cuanto a las unidades a izquierda,
% las relaciones
% \begin{align*}
	% (1\pm zi)\,i & \,=\,(i\mp z)
% \end{align*}
% %
% implican que, para hallar el orden a izquierda $\cal{O}_{I}$, alcanza con
% determinar los elementos $\alpha\in B$ tales que $\alpha (i\mp z)\in I$. Se
% cumple que $(1+zi)\,(i\mp z)\in 2\,I$, de lo que se deduce que
% $\tau=\frac{1+zi}{2}\in\cal{O}_{I}$. M\'{a}s aun, $\nrd(\tau)=1$, es decir,
% $\tau\in\cal{O}_{I}^{\times}$. Se verifica tambi\'{e}n que
% $(i\pm z)\in\cal{O}_{I}$ y que el ret\'{\i}culo $\generado{1,\tau,i-z,i+z}$
% es un orden. El discriminante es $\discriminante{1,\tau,i-z,i+z}=11^{2}$, con
% lo que
$\cal{O}_{I}=\generado{1,\tau,i-z,-i+z}$,
con $\tau = \frac{1+zi}{2}$.
% pues preserva $I$ y es maximal.
La forma cuadr\'{a}tica correspondiente est\'{a} dada por
\begin{align*}
	q_{\cal{O}_{I}} % & \,=\,\big(u+\tfrac{-v+w+t}{2}\big)^{2}
		% +\big(v+w+\tfrac{t}{4}\big)^{2}
		% +11\,\big(\tfrac{-v+w}{2}\big)^{2}
		% +11\,\big(\tfrac{t}{4}\big)^{2} \\
	& \,=\,u^{2}+4\,\big(v^{2}+w^{2}\big)+t^{2}-uv+uw+ut-4\,vw+wt
	\text{ .}
\end{align*}
%
El grupo de unidades
% de $\cal{O}_{I}$
tiene orden $\big|\cal{O}_{I}^{\times}\big|=6$. Verificamos la igualdad
\begin{align*}
	\frac{n_{\cal{O}}(2)}{\big|\cal{O}^{\times}\big|}
		\,+\,\frac{n_{I}(2)}{\big|\cal{O}_{I}^{\times}\big|}
		& \,=\,3
\end{align*}
%
Esto no quiere decir que $[\cal{O}]$ y $[I]$ sean las \'{u}nicas clases en
$\lClass{\cal{O}}$. Para determinar el n\'{u}mero de clases, recurrimos a la
f\'{o}rmula de masa:%
\footnote{
	ver \cite[Propo.~25]{PizerArithmeticII}, o \cite[\S~5]{KirschmerVoight}
	para el caso de un cuerpo de n\'{u}meros
}
% dado que
si $B$ ramifica en un \'{u}nico lugar finito, $l$, y $\cal{O}$ es un
orden maximal, en este caso se debe cumplir que
\begin{align*}
	\frac{l-1}{12} & \,=\,\sum_{i=1}^{H}\,\frac{1}{%
		\big|\cal{O}_{i}^{\times}:\bb{Z}^{\times}\big|}
	\text{ .}
\end{align*}
%
De esta igualdad (con $l=11$) y habiendo determinado los \'{o}rdenes
$|\cal{O}^{\times}|=4$ y $|\cal{O}_{I}^{\times}|=6$, se deduce que $H=2$ y un
conjunto de representantes est\'{a} dado por $\{[\cal{O}],[I]\}$.

En cuanto a los ideales \'{\i}ntegros de norma $2$, s\'{o}lo uno de ellos
pertenece a $[\cal{O}]$ y la clase $[I]$ da cuenta de los otros dos ideales.
Inmediatamente, deducimos que
\begin{align*}
	B_{2}[\cal{O}] & \,=\,[\cal{O}]+2\,[I]
	\text{ ,}
\end{align*}
%
donde $B_p=B_{0}(p)$ (no el \'{a}lgebra sobre $\bb{Q}_{p}$). Para calcular el
valor del operador $B_2$ en la clase $[I]$, basta con contar la cantidad de
ideales principales de norma $2\,\nrd(I)$ contenidos en $I$, ya que s\'{o}lo
hay dos clases de equivalencia y el total de ideales de norma $2$ contenidos en $I$ es $2+1=3$. Pero
\begin{align*}
	\#\Big\{J\subset I\,:\,\nrd(J)=2\,\nrd(I),\, [J]=[\cal{O}]\Big\}
		& \,=\,	\frac{n_{I}(2)}{\big|\cal{O}^{\times}\big|}
\end{align*}
%
(los ideales contabilizados son $b\,\cal{O}\subset I$ con $\nrd(b)=2\,\nrd(I)$
y $b\,\cal{O}=b'\,\cal{O}$, si y s\'{o}lo si $b^{-1}b'\in\cal{O}^{\times}$).
Seg\'{u}n las cuentas realizadas anteriormente,
\begin{align*}
	B_{2}[I] & \,=\,3\,[\cal{O}]
	\text{ .}
\end{align*}
%

Del c\'{a}lculo de $B_{2}$ deducimos, adem\'{a}s, que, para calcular $B_{p}$,
con $p\not=11$, basta con determinar el n\'{u}mero de representaci\'{o}n
$n_{\cal{O}}(p)$, pues, en general, si $J=\cal{O}$ o $J=I$,
\begin{align*}
	B_{p}[J] \,=\,\alpha\,[\cal{O}]+\beta\,[I]
	\text{ ,}
\end{align*}
%
donde $\alpha +\beta=p+1$, $\alpha=n_{J}(p)/|\cal{O}^{\times}|$ y,
si $J=I$,
\begin{math}
	p+1=\frac{n_{\cal{O}}(p)}{|\cal{O}^{\times}|}
		+\frac{n_{I}(p)}{|\cal{O}_{I}^{\times}|}
\end{math}~.
As\'{\i}, por ejemplo, usando la forma $q_{\cal{O}}$, calculamos
$n_{\cal{O}}(p)$ para $p\leq 17$:
\begin{center}
	\begin{tabular}{c|cccccc}
		$p$ & $2$ & $3$ & $5$ & $7$ & $13$ & $17$ \\
		\hline
		$n_{\cal{O}}(p)$ & $4$ & $8$ & $16$ & $16$ & $40$ & $40$
	\end{tabular}
\end{center}
Esto nos permite expresar los operadores $B_{p}$ para $p\leq 17$, $p\not=11$,
en la base $\{[\cal{O}],[I]\}$:
\begin{align*}
	B_{2}\,=\,\begin{bmatrix} 1 & 2 \\ 3 & 0 \end{bmatrix} & \text{ ,}\quad
		B_{3}\,=\,\begin{bmatrix} 2 & 2 \\ 3 & 1 \end{bmatrix}
			\text{ ,}\quad
		B_{5}\,=\,\begin{bmatrix} 4 & 2 \\ 3 & 3 \end{bmatrix}
			\text{ ,}\quad
		B_{7}\,=\,\begin{bmatrix} 4 & 4 \\ 6 & 2 \end{bmatrix}
			\text{ ,} \\
	& \qquad
	B_{13}\,=\,\begin{bmatrix} 10 & 4 \\ 6 & 8 \end{bmatrix}
			\quad\text{y}\quad
		B_{17}\,=\,\begin{bmatrix} 10 & 8 \\ 12 & 6 \end{bmatrix}
	\text{ .}
\end{align*}
%

La matriz $B_{2}$ tiene autovalores: $3$ y $-2$, con autovectores
$e_1=\frac{1}{2}\,[\cal{O}]+\frac{1}{3}\,[I]$ y $f_1=[I]-[\cal{O}]$,
respectivamente. El espacio $\spitz[B]{2}{1}$ admite un producto interno
(\emph{producto interno de Petersson}), con respecto al cual las formas $e_1$ y
$f_1$ son ortogonales (ver \eqref{eq:peterssonpesoparalelodosdefinida}).
% 
% La forma $e_1$ es la funci\'{o}n $\ideales{\cal{O}}\rightarrow\bb{C}$ dada por
% \eqref{eq:eisensteincuaternionicas} ($\pClass{\bb{Q}}=1$ y $[\frak{a}]=[1]$).
% Por otro lado, con respecto al producto interno de Petersson
% \begin{align*}
	% \langle e_1,e_1\rangle \,=\,\frac{1}{2}+\frac{1}{3} & \text{ ,}\quad
		% \langle f_1,f_1\rangle \,=\,3+2 \quad\text{y}\quad
		% \langle f_1,e_1\rangle \,=\,0
	% \text{ .}
% \end{align*}
% %
% Observamos que $\langle e_1,e_1\rangle$ es, simplemente, la masa del orden
% maximal $\cal{O}$ y, tambi\'{e}n, que $f_1\in\spitz[B]{2}{1}$.
% 
El espacio de formas cuspidales tiene dimensi\'{o}n $1$ y $f_1$ es un
generador. La forma $f_1$ es autofunci\'{o}n para los operadores
$B_{2},B_{3},B_{5},\dots$ con autovalores $a_p=-2,-1,1,\,\dots$
% $a_p=-2,-1,1,-2,4,-2,\,\dots$
Por la correspondencia de Jacquet-Langlands, el espacio
$\spitz{2}{11}^{11-\neue}$ es de dimensi\'{o}n $1$, generado por una
autofunci\'{o}n para los operadores de Hecke con autovalores
$\{a_p\}_{p\not=11}$. Usando las relaciones para los operadores de Hecke,
obtenemos los primeros t\'{e}rminos para la correspondiente autoforma
normalizada:
\begin{align*}
	F_1 & \,=\,q-2\,q^2-q^3+2\,q^4+q^5+2\,q^6-2\,q^7
		-2\,q^9-2\,q^{10} + O(q^{11})
	\text{ .}
\end{align*}
%
% 
% Como podemos ver ya en este ejemplo sencillo, para poder aprovechar el
% m\'{e}todo definido para el c\'{a}lculo de formas de Hilbert, es necesario
% contar con una serie de algoritmos adicionales que permitan explorar la
% estructura aritm\'{e}tica de los \'{o}rdenes e ideales en las \'{a}lgebras de
% cuaterniones totalmente definidas. Muchos de los resultados necesarios se
% pueden encontrar en \cite{PacettiSirolli}, \cite{KirschmerVoight} y en
% \cite{VoightIdentifying}.
