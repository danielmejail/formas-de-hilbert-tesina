A lo largo del presente trabajo utilizaremos la siguiente notaci\'{o}n sin
nuevas aclaraciones. Los s\'{\i}mbolos $\bb{Z}$, $\bb{Q}$, $\bb{R}$, $\bb{C}$ y
$\bb{H}$ denotan, respectivamente, el anillo de n\'{u}meros enteros racionales,
los cuerpos de n\'{u}meros racionales, reales y complejos y el \'{a}lgebra de
cuaterniones de Hamilton. El semiplano complejo superior lo denotamos $\hP$. Si
$A$ es un anillo, $A^{\times}$ denotar\'{a} su grupo de unidades. Dado un
cuerpo de n\'{u}meros denotado por $K$, utilizamos $\oka{K}$ para referirnos a
su anillo de enteros. En general, letras como $\frak{n}$, $\frak{m}$,
$\frak{N}$ o $\frak{M}$ denotan ideales (\'{\i}ntegros o fraccionarios) en un
cuerpo de n\'{u}meros y $\frak{p}$ o $\frak{q}$ ideales primos. Por \'{u}ltimo,
$\GL_{2}$ denota el grupo de matrices invertibles de tama\~{n}o $2\times 2$.
