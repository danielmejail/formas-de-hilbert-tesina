Usando las $n$ inmersiones de $F$ en $\bb{R}$, queda determinada una
inclusi\'{o}n de $\GLtp_{2}(F)$ en el producto $\GL_{2}(\bb{R})^{n}$ y, de esta
manera, una acci\'{o}n de $\GLtp_{2}(F)$ en $\hP^{n}$ dada por:
\begin{equation}
	\label{eq:accionenelsemiplanodehilbert}
	\begin{bmatrix} a & b \\ c & d \end{bmatrix}\,z
		\,=\, \left(
		\frac{a_{1}z_{1}+b_{1}}{c_{1}z_{1}+d_{1}},\,\dots,\,
		\frac{a_{n}z_{n}+b_{n}}{c_{n}z_{n}+d_{n}}
		\right)
	\text{ .}
\end{equation}
%
El espacio $\hP^{n}$ posee una medida invariante por la acci\'{o}n de
$\GLtp_{2}(\bb{R})^{n}$:
\begin{equation}
	\label{eq:medidaenelsemiplanodehilbert}
	d\mu \,=\, \prod_{i=1}^{n}\,y_{i}^{-2}dx_{i}dy_{i}
	\text{ .}
\end{equation}
%
En particular, por \eqref{eq:accionenelsemiplanodehilbert} esta medida es
invariante por $\GLtp_{2}(F)$.

Sea $\cal{O}_{0}(1)$ el orden maximal
\begin{equation}
	\label{eq:ordenmaximalmatrices}
	\cal{O}_{0}(1) \,:=\, \MM_{2\times 2}(\oka{F}) \,=\,
	\begin{bmatrix} \oka{F} & \oka{F} \\ \oka{F} & \oka{F} \end{bmatrix}
\end{equation}
%
contenido en $B=\MM_{2\times 2}(F)$. Manteniendo este orden fijo,
consideramos los \'{o}rdenes de la forma
\begin{equation}
	\label{eq:ordendeeichlermatrices}
	\cal{O}_{0}(\frak{N}) \,:=\,
	\begin{bmatrix} \oka{F} & \oka{F} \\ \frak{N} & \oka{F} \end{bmatrix}
		\,\subset\,\cal{O}_{0}(1)
		\text{ ,}
\end{equation}
%
que resultan ser \'{o}rdenes de Eichler para los distintos ideales
\'{\i}ntegros $\frak{N}\subset\oka{F}$. Si no hay ambig\"{u}edad, escribiremos
$\cal{O}$ en lugar de $\cal{O}_{0}(\frak{N})$. Dado un ideal fraccionario
$\frak{a}$ de $F$, fijamos un id\`{e}le finito $\hhat{a}\in\Idfin{F}$ que
cumpla
\begin{equation}
	\label{eq:ideleasociadomatrices}
	\frak{a} \,=\,\hhat{a}\Adfin{\oka{F}}\,\cap\,F
\end{equation}
%
y $\hhat{\alpha}\in\GL_{2}(\Adfin{F})$ dado por
\begin{equation}
	\label{eq:matrizasociadaaidelematrices}
	\hhat{\alpha} \,=\,\begin{bmatrix} \hhat{a} & \\ & 1 \end{bmatrix}
	\text{ .}
\end{equation}
%
En particular, $\nrd(\hhat{\alpha})=\hhat{a}$. De esta manera, a cada ideal
$\frak{a}$, se le asocian el orden de Eichler
\begin{equation}
	\label{eq:ordenasociadomatrices}
	\cal{O}_{\frak{a}} \,:=\,\big(\hhat{\alpha}\Adfin{\cal{O}}
			\hhat{\alpha}^{-1}\big)\,\cap\,\MM_{2\times 2}(F)
		\,=\,\begin{bmatrix} \oka{F} & \frak{a} \\
			\frak{N}\frak{a}^{-1} & \oka{F}\end{bmatrix}
\end{equation}
%
y el grupo de unidades totalmente positivas
\begin{equation}
	\label{eq:unidadestotalmentepositivasasociadasmatrices}
	\Gamma_{0}(\frak{N},\frak{a}) \,:=\,
		\cal{O}_{\frak{a}}^{\times}\,\cap\,\GLtp_{2}(F)
	\text{ .}
\end{equation}
%
Escribiremos ocasionalmente, $\Gamma_{\frak{a}}$ o
$\cal{O}_{\frak{a},+}^{\times}$ para denotar este mismo grupo.

% \begin{obsDefinicionOrdenAsociadoMatrices}%
	% \label{obs:definicionordenasociadomatrices}
	% Como el id\`{e}le $\hhat{a}$ en \eqref{eq:ideleintegroasociadomatrices}
	% est\'{a} determinado salvo una unidad en $\Idfin{\oka{F}}$, el orden
	% $\cal{O}_{\frak{a}}$ y el grupo de unidades $\Gamma_{\frak{a}}$, no
	% dependen de la elecci\'{o}n de $\hhat{a}$, pero s\'{\i} dependen de la
	% elecci\'{o}n del elemento $\hhat{\alpha}\in\GL_{2}(\Adfin{F})$ tal que
	% $\nrd(\hhat{\alpha})=\hhat{a}$. Si
	% $\hhat{\alpha}'\in\GL_{2}(\Adfin{F})$ es tal que el ideal determinado
	% por $\nrd(\hhat{\alpha}')$ pertenece a la clase reducida de $\frak{a}$,
	% entonces, por el Corolario~\ref{thm:normaclasesestrictas}, existen
	% $\gamma\in\GLtp_{2}(F)$ y $\hhat{\beta}\in\Idfin{\cal{O}}$ tales que
	% $\hhat{\alpha}'=\gamma\hhat{\alpha}\hhat{\beta}$ y los \'{o}rdenes
	% asociados por \eqref{eq:ordenasociadomatrices} son conjugados por un
	% elemento de $\GLtp_{2}(F)$.
% \end{obsDefinicionOrdenAsociadoMatrices}
% 
Al \'{a}lgebra $\MM_{2\times 2}(F)$ y al orden $\cal{O}$ se les asocia la
variedad
\begin{align*}
	Y_{0}(\frak{N}) & \,=\,
		\bigsqcup_{\frak{a}}\,Y_{0}(\frak{N},\frak{a})
	\text{ ,}
\end{align*}
%
donde $\frak{a}$ recorre un sistema de representantes del grupo de clases
estrictas $\pClass{F}$ e $Y_{0}(\frak{N},\frak{a})$ se define como el cociente
$\Gamma_{0}(\frak{N},\frak{a})\backslash\hP^{n}$ (comparar con
\S~\ref{sec:cuaternionicasvariedadesdeshimura}). Esta variedad no es
compacta pero, v\'{\i}a la determinaci\'{o}n de un dominio fundamental para
cada componente $Y_{0}(\frak{N},\frak{a})$, se demuestra que es de volumen
finito.%
\footnote{\cite[Ch.~IV,\S~1]{vanDerGeerSurfaces}}
% Por ejemplo, con respecto a la medida \eqref{eq:medidaenelsemiplanodehilbert},
% la variedad $Y_{0}(1,1)$ correspondiente al grupo $\Gamma_{0}(1,1)$ (matrices
% con coeficientes enteros y determinante una unidad totalmente positiva), tiene
% volumen
% \begin{align*}
	% \mu\big(Y_{0}(1,1)\big) & \,=\,2\zeta_{F}(-1)
	% \text{ ,}
% \end{align*}
% %
% donde
% \begin{math}
	% \zeta_{F}(s)=\sum_{\frak{b}\subset\oka{F}}\,\norma(\frak{b})^{-s}
% \end{math}
% es la funci\'{o}n zeta de Dedekind del cuerpo $F$
% \cite[Ch.~IV,\S~1]{vanDerGeerSurfaces}.
%
% En general, los grupos $\Gamma_{0}(\frak{N},\frak{a})$ son conmensurables (?)
% con $\Gamma_{0}(1,1)$ y el volumen de la variedad asociada est\'{a} dado por
% \begin{align*}
	% \mu\big(Y_{0}(\frak{N},\frak{a})\big) & \,=\,
		% \frac{\big|\Gamma_{0}(1,1):\Gamma_{0}(1,1)\cap%
			% \Gamma_{0}(\frak{N},\frak{a})\big|}{%
		% \big|\Gamma_{0}(\frak{N},\frak{a}):
			% \Gamma_{0}(1,1)\cap\Gamma_{0}(\frak{N},\frak{a})\big|}
		% \,\mu\big(Y_{0}(1,1)\big)
	% \text{ .}
% \end{align*}
% %
Para obtener una variedad compacta, se considera el grupo $\GLtp_{2}(F)$
actuando en $\bb{P}^{1}(F)$ por:
\begin{equation}
	\label{eq:accionenlascuspides}
	\begin{bmatrix} a & b \\ c & d \end{bmatrix}\,[\alpha:\beta]
		\,=\,[a\alpha+b\beta:c\alpha+d\beta]
\end{equation}
%
Como tambi\'{e}n $\bb{P}^{1}(F)\hookrightarrow\bb{P}(\bb{R})^{n}$ v\'{\i}a las
inmersiones de $F$ en $\bb{R}$, podemos obtener una compactificaci\'{o}n de
$Y_{0}(\frak{N},\frak{a})$ agregando las c\'{u}spides de
$\Gamma_{0}(\frak{N},\frak{a})$, las \'{o}rbitas de
$\Gamma_{0}(\frak{N},\frak{a})$ actuando en $\bb{P}^{1}(F)$ --esta
compactificaci\'{o}n se conoce como \emph{compactificaci\'{o}n de %
Baily-Borel}%
\footnote{\cite[Ch.~II, \S~7]{vanDerGeerSurfaces}}
-- que denotamos
\index{compactificacion de Baily Borel@compactificaci\'{o}n de Baily-Borel}
\begin{align*}
	\shimura{\frak{N},\frak{a}} & \,:=\,
		\Gamma_{0}(\frak{N},\frak{a})\backslash
		\left(\hP^{n}\cup\bb{P}^{1}(F)\right)
	\text{ .}
\end{align*}
%
A diferencia de las \emph{curvas} modulares, estas variedades tienen
puntos singulares: los puntos el\'{\i}pticos y las (finitas) c\'{u}spides
que agregamos. Cuando $F=\bb{Q}$, para cada nivel $N>1$ hay una \'{u}nica
curva $X_{0}(N)$. En general, a cada ideal \'{\i}ntegro $\frak{N}$, le
asociamos en principio una variedad $\shimura{\frak{N},\frak{a}}$ por cada
$\frak{a}$ en un sistema de representantes de las clases en $\pClass{F}$.

Adem\'{a}s de los grupos $\Gamma_{0}(\frak{N},\frak{a})$, aparecer\'{a}n
naturalmente otros, por ejemplo, de la forma
\begin{math}
	\Gamma_{0}(\frak{N},\frak{a})\cap
		\big(\gamma\Gamma_{0}(\frak{N},\frak{b})\gamma^{-1}\big)
\end{math}~.
Las propiedades fundamentales de las formas de Hilbert: existencia
de expansi\'{o}n de Fourier y principio de Koecher, dependen de la existencia
de suficientes ``traslaciones'' y de suficientes ``homotecias''
(respectivamente). Estas transformaciones surgen de considerar el estabilizador
de la c\'{u}spide $\infty$ en $\Gamma_{0}(\frak{N},\frak{a})$. Por otra
parte, para poder relacionar el producto interno de Petersson con los
operadores de Hecke en los espacios de formas cuspidales, necesitaremos
garantizar que estas transformaciones est\'{e}n presentes, tambi\'{e}n, en
otros grupos. Con este fin, introducimos los subgrupos de congruencia y la
relaci\'{o}n de conmensurabilidad.

Dados $\frak{N}$ y $\frak{a}$, respectivamente un ideal \'{\i}ntegro y un
ideal fraccionario en $F$, definimos un grupo $\Gamma_{0}(\frak{N},\frak{a})$
por la expresi\'{o}n \eqref{eq:unidadestotalmentepositivasasociadasmatrices}
(la diferencia est\'{a} en que $\frak{a}$ es, en principio, arbitrario).
Fijado $\frak{a}$, los grupos $\Gamma_{0}(\frak{N},\frak{a})$ constituyen
una familia ordenada por
\begin{math}
	\frak{M}\mid\frak{N}\Rightarrow\Gamma_{0}(\frak{M},\frak{a})
		\supset\Gamma_{0}(\frak{N},\frak{a})
\end{math}~,
con $\Gamma_{0}(1,\frak{a})$ conteniendo a todos. El anillo
$\cal{O}_{0}(1)_{\frak{a}}$ (\eqref{eq:ordenasociadomatrices} con $\frak{N}=1$)
act\'{u}a por endomorfismos en el $\oka{F}$-m\'{o}dulo
$\frak{a}\oplus\oka{F}$:
\begin{align*}
	\begin{bmatrix} a & b \\ c & d \end{bmatrix}(x,y) & \,=\,
			(ax+by,cx+dy)
	\text{ .}
\end{align*}
%
Evaluando en un punto $(x,0)$ y en un punto $(0,y)$ ($x,y\not=0$), se ve que
el \'{u}nico elemento que act\'{u}a trivialmente es la matriz identidad
$I\in\cal{O}_{0}(1)_{\frak{a}}$. En particular, el grupo
$\Gamma_{0}(1,\frak{a})$ se realiza como un subgrupo de automorfismos de
este $\oka{F}$-m\'{o}dulo. Para cada ideal \'{\i}ntegro
$\frak{N}\subset\oka{F}$, el subm\'{o}dulo $\frak{N}\frak{a}\oplus\frak{N}$ es
$\cal{O}_{0}(1)_{\frak{a}}$-invariante:
\begin{align*}
	\begin{bmatrix} \oka{F} & \frak{a} \\
		\frak{a}^{-1} & \oka{F} \end{bmatrix}
			(\frak{N}\frak{a}\oplus\frak{N}) & \,\subset\,
		\frak{N}\frak{a}\oplus\frak{N}
	\text{ .}
\end{align*}
%
En particular, $\Gamma_{0}(1,\frak{a})$ preserva este subm\'{o}dulo y la
acci\'{o}n del grupo desciende a una acci\'{o}n en el cociente
\begin{math}
	\big(\frak{a}\oplus\oka{F}\big)/
		\big(\frak{N}\frak{a}\oplus\frak{N}\big)\simeq
		\frak{a}/\frak{N}\frak{a}\,\oplus\,\oka{F}/\frak{N}
\end{math}~.
La imagen del correspondiente morfismo
\begin{math}
	\Gamma_{0}(1,\frak{a})\rightarrow
		\Auto(\frak{a}/\frak{N}\frak{a}\,\oplus\,\oka{F}/\frak{N})
\end{math}
es finita y su n\'{u}cleo est\'{a} dado por
\begin{align*}
	\Gamma(\frak{N},\frak{a}) & \,:=\,
		\bigg\{\begin{bmatrix} a & b \\ c & d \end{bmatrix}\in
			\Gamma_{0}(1,\frak{a})\,:\,
		\begin{bmatrix} a & b \\ c & d \end{bmatrix}
			\equiv
			\begin{bmatrix} 1 & \\ & 1 \end{bmatrix}\,
				\bigg(\frak{N}\cdot
				\begin{bmatrix} \oka{F} & \frak{a} \\
				\frak{a}^{-1} & \oka{F} \end{bmatrix}
				\bigg)
			\bigg\}% \\
	%& \,=\,\bigg\{\begin{bmatrix} a & b \\ c & d \end{bmatrix}\in
			%\begin{bmatrix} \oka{F} & \frak{a} \\
			%\frak{a}^{-1} & \oka{F} \end{bmatrix}\,:\,
			%ad-bc\in\oka{F,+}^{\times},\,a\equiv d\equiv 1
				%(\frak{N}),\,
				%c\in\frak{N}\frak{a}^{-1},\,
				%b\in\frak{N}\frak{a}
			%\bigg\}
	\text{ .}
\end{align*}
%
El grupo $\Gamma(\frak{N},\frak{a})$ es, pues, normal en
$\Gamma_{0}(1,\frak{a})$ y de \'{\i}ndice finito. Adem\'{a}s, la
condici\'{o}n $c\in\frak{N}\frak{a}^{-1}$ implica que
$\Gamma(\frak{N},\frak{a})\subset\Gamma_{0}(\frak{N},\frak{a})$. Un subgrupo
$\Gamma\subset\GLtp_{2}(F)$ se dice \emph{de congruencia}, si
\index{subgrupo de congruencia}
existen ideales $\frak{a},\frak{a}'$ y $\frak{N},\frak{N}'\subset\oka{F}$ y
$A\in\GLtp_{2}(F)$ tales que
\begin{align*}
	A^{-1}\,\Gamma_{0}(\frak{N},\frak{a})\,A & \,\supset\,
		\Gamma \,\supset\,
		\Gamma(\frak{N}',\frak{a}')
	\text{ .}
\end{align*}
%
Los grupos $\Gamma_{0}(\frak{N},\frak{a})$ son subgrupos de congruencia. Se
puede demostrar que la clase de subgrupos de congruencia es cerrada por
conjugaci\'{o}n en $\GLtp_{2}(F)$ y que todos los subgrupos de congruencia son
conmensurables entre s\'{\i}.%
\footnote{
	Ver \cite[Ch.~I, \S~3]{FreitagForms} para una idea de la
	demostraci\'{o}n.
}

En general, si $\Gamma\subset\GLtp_{2}(F)$ es conmensurable con
$\Gamma_{0}(1,1)$ (y, por lo tanto, con todo subgrupo de congruencia), entonces
el cociente $\Gamma\backslash\hP^{n}$ es una variedad no compacta y de volumen
finito. Si $\Gamma'\subset\Gamma$ es un subgrupo de \'{\i}ndice finito,
entonces%
\footnote{
	\cite[Ch.~IV,\S~1]{vanDerGeerSurfaces}
}
\begin{align*}
	\mu\big(\Gamma'\backslash\hP^{n}\big) & \,=\,
		\big|\Gamma:\Gamma'\big|\cdot
			\mu\big(\Gamma\backslash\hP^{n}\big)
	\text{ .}
\end{align*}
%
