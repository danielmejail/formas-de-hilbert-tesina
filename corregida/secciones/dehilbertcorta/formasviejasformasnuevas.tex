% Sean $M$ y $N$ enteros positivos tales que $M\mid N$. Si $d$ es un divisor
% de $N/M$ y $\alpha_{d}=\begin{bmatrix} d & \\ & 1 \end{bmatrix}$, la
% transformaci\'{o}n $\iota_{d}:\,f\mapsto f\barra{k}{\alpha_{d}}$ determina
% una inclusi\'{o}n del espacio $\spitz{k}{M}$ de formas cuspidales de peso
% $k$ para el subgrupo de congruencia $\Gamma_{0}(M)\subset\SLZ$ en
% $\spitz{k}{N}$, el espacio correspondiente a $\Gamma_{0}(N)$. Notemos que
% \begin{align*}
	% \Gamma_{0}(N) \,=\, & \begin{bmatrix} d & \\ & 1 \end{bmatrix}^{-1}
	% \Gamma_{0}(M) \begin{bmatrix} d & \\ & 1 \end{bmatrix}
	% \,\cap\,\Gamma_{0}(M)
	% \text{ .}
% \end{align*}
% %
% Dado un primo racional $l\geq 1$ divisor de $N$, el subespacio de
% formas viejas en $l$ en $\spitz{k}{N}$ est\'{a} dado por
% \begin{align*}
	% \spitz{k}{N}^{l-\oude} \,=\, &
	% \iota_{1}\left(\spitz{k}{N/l}\right) \,+\,
	% \iota_{l}\left(\spitz{k}{N/l}\right)
% \end{align*}
% %
% y el subespacio $\spitz{k}{N}^{l-\neue}$, de formas nuevas en $l$, se define
% como el complemento ortogonal de $\spitz{k}{N}^{l-\oude}$ respecto del
% producto interno de Petersson.
% 
% Sean ahora $\frak{M},\frak{q}\subset\oka{F}$ ideales \'{\i}ntegros y sea
% $\frak{N}=\frak{q}\frak{M}$. Si $\hhat{q}\in\Idfin{F}$ es tal
% $\frak{q}=\hhat{q}\Idfin{\oka{F}}\cap\oka{F}$,
% 
% 
% Sea $f\in\modularH{k}{\frak{M}}$ una forma modular y sea $\phi=\phi_{f}$
% la forma correspondiente definida en los id\`{e}les de
% $B\simeq\MM_{2\times 2}(F)$.
%%
Las formas de Hilbert admiten una teor\'{\i}a de formas nuevas similar a la
teor\'{\i}a para formas modulares el\'{\i}pticas. Aqu\'{\i} incluimos algunos
resultados de esta teor\'{\i}a que sirven como fundamento para los m\'{e}todos
expl\'{\i}citos descriptos m\'{a}s adelante. Tomamos como referencia principal
\cite{ShemanskeWalling}. Otra referencia importante es
\cite{MiyakeOnAutomorphicFormsOnGL}.
%% EN SHEMANSKE-WALLING, SE CONSIDERA UN CAR\'ACTER M\'ODULO N. AC\'A SE ASUME
%% QUE ESTE CAR\'ACTER ES TRIVIAL

Dados dos subgrupos compactos y abiertos $K_{1}$ y $K_{2}$ de
$\GL_{2}(\Adfin{F})$ y dado $\hhat{\pi}\in\GL_{2}(\Adfin{F})$, al igual que en
el caso $K_{1}=K_{2}=\Idfin{\cal{O}}$, la coclase doble $K_{1}\hhat{\pi}K_{2}$
es un conjunto compacto y, al ser $K_{1}$ abierto,
$K_{1}\backslash K_{1}\hhat{\pi}K_{2}$ es finito. Esto quiere decir, como
antes, que existe un conjunto finito $\{\hhat{\pi}_{i}\}_{i}$ tal que
\begin{align*}
	K_{1}\hhat{\pi}K_{2} & \,=\,\bigsqcup_{i}\,K_{1}\hhat{\pi}_{i}
	\text{ .}
\end{align*}
%
Si $\phi:\,\GL_{2}(\adeles{F})\rightarrow\bb{C}$ es una funci\'{o}n
que verifica $\phi\operadormatrices{\peso{k}}{(1,\hhat{\beta})}=\phi$ para
toda $\hhat{\beta}\in K_{1}$, entonces definimos una nueva funci\'{o}n
\begin{math}
	\phi\operadorcoclases{\peso{k}}{K_{1}\hhat{\pi}K_{2}}
\end{math}
por una expresi\'{o}n an\'{a}loga a \eqref{eq:operadordecoclasedobledehilbert}.

\begin{obsDefinicionOperadorDeCoclaseDoble}%
	\label{obs:definicionoperadordecoclasedoble}
	Esta definici\'{o}n no depende de la elecci\'{o}n de los representantes
	$\hhat{\pi}_{i}$. Adem\'{a}s, si $\hhat{\beta}\in K_{2}$, entonces
	\begin{equation}
		\label{eq:invarianzaaderechaluegodecolcasedoblematrices}
		\Big(\phi\operadorcoclases{\peso{k}}{K_{1}\hhat{\pi}K_{2}}
			\Big)\operadormatrices{\peso{k}}{(1,\hhat{\beta})}
		\,=\,\phi\operadorcoclases{\peso{k}}{K_{1}\hhat{\pi}K_{2}}
		\text{ .}
	\end{equation}
	%
	La demostraci\'{o}n de esta afirmaci\'{o}n es an\'{a}loga a la de la
	afirmaci\'{o}n correspondiente en la Proposici\'{o}n~%
	\ref{propo:coclasedoblepreservaelespaciodeformascuspidales} (propiedad
	(ii)). M\'{a}s aun, si $\phi:\,\GL_{2}(\adeles{F})\rightarrow\bb{C}$
	verifica las propiedades (i) a (vi) de la Proposici\'{o}n~%
	\ref{propo:equivalenciaautomorfasformascuspidales} (reemplazando (ii)
	por la condici\'{o}n an\'{a}loga de que $\phi$ sea invariante a derecha
	por $K_{1}$) entonces
	$\phi\operadorcoclases{\peso{k}}{K_{1}\hhat{\pi}K_{2}}$ verifica (i) a
	(vi). Si $\phi$ es, adem\'{a}s, cuspidal, entonces
	$\phi\operadorcoclases{\peso{k}}{K_{1}\hhat{\pi}K_{2}}$ es cuspidal,
	tambi\'{e}n.
\end{obsDefinicionOperadorDeCoclaseDoble}

% \begin{obsRepresentantesDeLasCoclases}\label{obs:representantesdelascoclases}
	% Como en la demostraci\'{o}n de la Proposici\'{o}n
	% \ref{thm:descomposiciondelacoclasedoblematrices}, para obtener los
	% representantes $\hhat{\pi}_{i}$ tales que
	% $K_{1}\hhat{\pi}K_{2}=\bigsqcup_{i}\,K_{1}\hhat{\pi}_{i}$,
	% podemos, primero, elegir un sistema de representantes de las
	% clases en $(\hhat{\pi}^{-1}K_{1}\hhat{\pi}\cap K_{2})\backslash K_{2}$
	% y luego multiplicar a izquierda por $\hhat{\pi}$ para obtener un
	% sistema de representantes de $K_{1}\backslash K_{1}\hhat{\pi}K_{2}$.
% \end{obsRepresentantesDeLasCoclases}
% 
Sean, ahora, $\frak{N}$, $\frak{M}$ y $\frak{l}$ ideales \'{\i}ntegros de
$\oka{F}$ con $\frak{l}$ primo y tales que $\frak{N}=\frak{l}\,\frak{M}$. Sean
$\cal{O}_{0}(\frak{N})$ y $\cal{O}_{0}(\frak{M})$ los \'{o}rdenes de Eichler en
$\MM_{2\times 2}(F)$ y sea $\phi:\,\GL_{2}(\adeles{F})\rightarrow\bb{C}$ una
funci\'{o}n invariante a derecha por $\Idfin{\cal{O}_{0}(\frak{M})}$. Fijemos
$\hhat{l}\in\Adfin{\oka{F}}$ tal que $\frak{l}=\hhat{l}\Adfin{\oka{F}}\cap F$ y
\begin{math}
	\hhat{\lambda}=\begin{bmatrix} \hhat{l} & \\ & 1 \end{bmatrix}
\end{math}~.
Para obtener una funci\'{o}n invariante por la acci\'{o}n del compacto
$\Idfin{\cal{O}_{0}(\frak{N})}$ a partir de $\phi$, observamos que
\begin{align*}
	\Idfin{\cal{O}_{0}(\frak{N})} \,\subset\,
		\Idfin{\cal{O}_{0}(\frak{M})} & \quad\text{y que}\quad
	\Idfin{\cal{O}_{0}(\frak{N})} \,=\,
		\hhat{\lambda}^{-1}\Idfin{\cal{O}_{0}(\frak{M})}\hhat{\lambda}
		\,\cap\,\Idfin{\cal{O}_{0}(\frak{M})}
	\text{ .}
\end{align*}
%
Llamando $\hhat{1}$ al id\`{e}le que es la identidad en todos los lugares,
\begin{align*}
	\Idfin{\cal{O}_{0}(\frak{M})}\hhat{1}\Idfin{\cal{O}_{0}(\frak{N})}
		\,=\, \Idfin{\cal{O}_{0}(\frak{M})}\hhat{1}
	& \quad\text{y}\quad
	\Idfin{\cal{O}_{0}(\frak{M})}\hhat{\lambda}
			\Idfin{\cal{O}_{0}(\frak{N})}
		\,=\,\Idfin{\cal{O}_{0}(\frak{M})}\hhat{\lambda}
	\text{ .}
\end{align*}
%
Definimos inclusiones $\iota_{1}$ y $\iota_{\frak{l}}$ respectivamente por
\begin{align*}
	\big(\iota_{1}\phi\big)(g,\hhat{\alpha}) & \,=\,\phi\operadorcoclases{\peso{k}}{%
		\Idfin{\cal{O}_{0}(\frak{M})}\hhat{1}%
			\Idfin{\cal{O}_{0}(\frak{N})}}
				(g_{\infty},\hhat{\alpha})
		\,=\, \phi(g_{\infty},\hhat{\alpha}) \quad\text{y} \\
	\big(\iota_{\frak{l}}\phi\big)(g,\hhat{\alpha}) & \,=\,
		\phi\operadorcoclases{\peso{k}}{%
		\Idfin{\cal{O}_{0}(\frak{M})}\hhat{\lambda}%
			\Idfin{\cal{O}_{0}(\frak{N})}}
				(g_{\infty},\hhat{\alpha})
		\,=\, \phi(g_{\infty},\hhat{\alpha}\hhat{\lambda}^{-1})
	\text{ .}
\end{align*}
%

Trasladando estas definiciones hacia el lado de las formas de Hilbert,
identificamos funciones en $\spitzH{k}{\frak{N}}$ que vienen de niveles
\emph{m\'{a}s bajos}. Si $f\in\spitzH{k}{\frak{M}}$ es una forma de Hilbert
(cuspidal) de nivel $\frak{M}$ y $\frak{N}=\frak{l}\,\frak{M}$, con
$\frak{l}$ un ideal primo de $\oka{F}$, entonces las expresiones
\begin{align*}
	\big(\iota_{1}f\big) (z,\hhat{\alpha}\Idfin{\cal{O}_{0}(\frak{N})})
		& \,=\, \bigg(\prod_{i=1}^{n}\,J_{i}(g_{i},\sqrt{-1})\bigg)\,
			\phi_{f}(g_{\infty},\hhat{\alpha}) \quad\text{e} \\
	\big(\iota_{\frak{l}}f\big) (z,
		\hhat{\alpha}\Idfin{\cal{O}_{0}(\frak{N})})
		& \,=\, \bigg(\prod_{i=1}^{n}\,J_{i}(g_{i},\sqrt{-1})\bigg)\,
			\phi_{f}(g_{\infty},\hhat{\alpha}\hhat{\lambda}^{-1})
	\text{ ,}
\end{align*}
%
donde $g_{\infty}\in\GL_{2}(\bb{R})^{n}$ es tal que
$g_{\infty}\cdot\mathbf{i}=z$, determinan elementos del espacio
$\spitzH{k}{\frak{N}}$.

\begin{defFormasViejasFormasNuevas}\label{def:formasviejasformasnuevas}
	Dado un ideal primo $\frak{l}\subset\oka{F}$ divisor de $\frak{N}$,
	una forma de Hilbert cuspidal en $\spitzH{k}{\frak{N}}$ es una
	\emph{forma vieja en $\frak{l}$}, si es combinaci\'{o}n lineal de
	\index{forma vieja en l@forma vieja en $\frak{l}$}
	formas del tipo $\iota_{1}f$ y $\iota_{\frak{l}}f$ con
	$f\in\spitzH{k}{\frak{M}}$, donde $\frak{l}\,\frak{M}=\frak{N}$.
	El \emph{subespacio de formas viejas en $\frak{l}$} se define como
	\begin{align*}
		\spitzH{k}{\frak{N}}^{\frak{l}-\oude} & \,:=\,
			\iota_{1}\left(\spitzH{k}{\frak{M}}\right)\,+\,
			\iota_{\frak{l}}\left(\spitzH{k}{\frak{M}}\right)
		\text{ y}
	\end{align*}
	%
	el \emph{subespacio de formas viejas} en $\spitzH{k}{\frak{N}}$ es
	el subespacio
	\begin{align*}
		\spitzH{k}{\frak{N}}^{\oude} & \,:=\,
			\sum_{\frak{l}\mid\frak{N}}\,
			\spitzH{k}{\frak{N}}^{\frak{l}-\oude}
		\text{ .}
	\end{align*}
	%
	El \emph{subespacio de formas nuevas en $\frak{l}$} se define
	como el complemento ortogonal
	\begin{align*}
		\spitzH{k}{\frak{N}}^{\frak{l}-\neue} &  \,:=\,
			\Big(\spitzH{k}{\frak{N}}^{\frak{l}-\oude}\Big)^{\perp}
	\end{align*}
	%
	respecto del producto interno de Petersson y el
	\emph{subespacio de formas nuevas} es
	\begin{align*}
		\spitzH{k}{\frak{N}}^{\neue} & \,:=\,
			\bigcap_{\frak{l}\mid\frak{N}}\,
			\spitzH{k}{\frak{N}}^{\frak{l}-\neue}
			\,=\,\Big(\spitzH{k}{\frak{N}}^{\oude}\Big)^{\perp}
		\text{ .}
	\end{align*}
	%
\end{defFormasViejasFormasNuevas}

% Si $\frak{p}$ es un primo de $\oka{F}$ coprimo con el nivel $\frak{N}$,
% y $\frak{N}=\frak{l}\frak{M}$ con $\frak{l}$ primo como en la definici\'{o}n
% ???, entonces contamos con un operador $T_{\frak{p}}$ definido en
% $\spitzH{k}{\frak{N}}$ y otro definido en $\spitzH{k}{\frak{M}}$. Ambos
% se definen de manera similar como operadores de coclases dobles. Por otro
% lado, como vimos, tenemos inclusiones
% \begin{align*}
	% \iota_{1},\,\iota_{\frak{l}} \,:\, &
	% \spitzH{k}{\frak{M}}\,\hookrightarrow\,
	% \spitzH{k}{\frak{N}}
	% \text{ .}
% \end{align*}
% %
Sean $\frak{l}$ y $\frak{M}$ como en la definici\'{o}n
\ref{def:formasviejasformasnuevas} y sea $\frak{N}=\frak{l}\,\frak{M}$. Sea
$\frak{p}\nmid\frak{N}$ y sean $T_{\frak{p}}^{\frak{N}}$ y
$T_{\frak{p}}^{\frak{M}}$ los operadores de Hecke asociados al primo $\frak{p}$
en $\spitzH{k}{\frak{N}}$ y en $\spitzH{k}{\frak{M}}$, respectivamente. Estos
operadores est\'{a}n definidos de la misma manera. Es decir, dado que
$\frak{p}$ es coprimo con $\frak{N}$, es coprimo con $\frak{M}$ y si
$\{\hhat{\pi}_{i}\}_{i}$ es un sistema de representantes de
$\frak{I}(\frak{p})$ en nivel $\frak{M}$, entonces el mismo conjunto es un
sistema de representantes en nivel $\frak{N}$. As\'{\i}, si
$f\in\spitzH{k}{\frak{M}}$, entonces
\begin{align*}
	T_{\frak{p}}^{\frak{N}}(\iota_{1}f) \,=\,
		\iota_{1}(T_{\frak{p}}^{\frak{M}}f) & \quad\text{y}\quad
	T_{\frak{p}}^{\frak{N}}(\iota_{\frak{l}}f) \,=\,
		\iota_{\frak{l}}(T_{\frak{p}}^{\frak{M}}f)
	\text{ ,}
\end{align*}
%
ya que los id\`{e}les que aparecen en las definiciones de $\iota_{1}$,
$\iota_{\frak{l}}$ y los operadores $T_{\frak{p}}$ conmutan entre s\'{\i}. Un
argumento an\'{a}logo muestra que $S_{\frak{p}}$ tambi\'{e}n conmuta con las
inclusiones $\iota_{1}$ e $\iota_{\frak{l}}$. De esto se deduce que los
operadores de Hecke y sus adjuntos preservan los espacios de formas viejas. En
consecuencia, los espacios de formas nuevas tambi\'{e}n son invariantes por los
operadores de Hecke para todos los primos $\frak{p}\nmid\frak{N}$.

% \begin{obsBaseDeAutoformasParaHecke}\label{obs:basedeautoformasparahecke}
	% ?`Son normales los operadores de Hecke? La demostraci\'{o}n en 
	% \cite{ShimuraSpecialValuesOfZetaAssociatedWithHilbert} no parece
	% poder replicarse. En cada bloque, es decir, para cada par
	% $\frak{a},\frak{b}$ tal que $[\frak{a}]=[\frak{b}\frak{p}]$, aparece un
	% factor $\norma(\lambda)^{k_{0}-2}$ que viene de
	% $\eta^\iota=\det(\eta)\eta^{-1}$ ($\det(\eta)=\lambda\xi$ con
	% $\xi\in\oka{F,+}^{\times}$) y de que
	% \begin{math}
		% \langle f\operadormatrices{\peso{k}}{\eta},g\rangle=
		% \langle f,g\operadormatrices{\peso{k}}{\eta^\iota}\rangle=
		% \norma(\det(\eta))^{k_{0}-2}
		% \langle f,g\operadormatrices{\peso{k}}{\eta^{-1}}\rangle
	% \end{math}~.
	% Sin este factor extra\~{n}o,
	% \begin{align*}
		% T_{\frak{p}}^{*} & \,=\, T_{\frak{p}^{-1}}
			% \,=\, S_{\frak{p}}T_{\frak{p}}
		% \text{ ,}
	% \end{align*}
	% %
	% donde, si $T_{\frak{p}}=T_{\hhat{\pi}}$, entonces
	% $T_{\frak{p}^{-1}}$ es $T_{\hhat{\pi}^{-1}}$, es decir, sumar sobre
	% los representantes invertidos (esto viene de tomar un sistema de
	% representantes $\{\hhat{\pi}_{i}\}_{i}$ tanto de
	% $\Idfin{\cal{O}}\backslash\frak{I}(\frak{p})$ como de
	% $\frak{I}(\frak{p})/\Idfin{\cal{O}}$, o, lo que es lo mismo, un sistema
	% $\{\eta_{i}\}_{i}$ tal que
	% \begin{math}
		% \frak{I}(\frak{p})_{\frak{a},\frak{b}}=
			% \bigsqcup_{i}\,\Gamma_{\frak{b}}\eta_{i}=
			% \bigsqcup_{i}\,\eta_{i}\Gamma_{\frak{a}}
	% \end{math}~.
	% La igualdad de los cardinales de las \'{o}rbitas a izquierda y de las
	% \'{o}rbitas a derecha se deduce de
	% \begin{math}
		% \mu\big(\Gamma_{\frak{b}}\backslash\hP\big)=
			% \mu\big(\Gamma_{\frak{a}}\backslash\hP\big)
	% \end{math}~).
	% % Los operadores de Hecke $T_{\frak{p}}$ son autoadjuntos respecto del
	% % producto interno de Petersson
	% % (\cite[Prop.~2.4]{ShimuraSpecialValuesOfZetaAssociatedWithHilbert}) y
	% % preservan los espacios de formas viejas. En consecuencia, preservan
	% % tambi\'{e}n los espacios de formas nuevas. Como la familia de
	% % operadores $\big\{T_{\frak{p}}\,:\,\frak{p}\nmid\frak{N}\big\}$ es
	% % conmutativa y la dimensi\'{o}n de $\spitzH{k}{\frak{N}}$ es finita,
	% % existe una base ortognal de autofunciones para todos los operadores
	% % $T_{\frak{p}}$ coprimos con el nivel.
% \end{obsBaseDeAutoformasParaHecke}

\begin{teoCoeficientesNulosFormaVieja}[{\cite[Thm.~3.1]{ShemanskeWalling}}]%
	\label{thm:coeficientesnulosformavieja}
	Sea $f\in\spitz{k}{\frak{N},\omega}$ y sean $c(\frak{m},f)$ sus
	coeficientes de Fourier. Si existe un ideal \'{\i}ntegro
	$\frak{l}\subset\oka{F}$ tal que $c(\frak{m},f)=0$ siempre que
	$(\frak{m},\frak{l})=1$, entonces
	$f\in\spitzH{k}{\frak{N},\omega}^{\oude}$.
\end{teoCoeficientesNulosFormaVieja}

\begin{defFormaNueva}\label{def:formanueva}
	% Decimos que una forma cuspidal $f\in\spitzH{k}{\frak{N},\omega}$ es
	% una \emph{autoforma de Hecke}, si es una autoforma para todos los
	% operadores $T_{\frak{p}}$ tales que $\frak{p}\nmid\frak{N}$.
	%%EN GENERAL SE SUELE INCLUIR LA CONDICI\'ON DE SER AUTOFUNCI\'ON DE
	%% *TODOS* LOS T_p
	Decimos que una forma cuspidal $f$ es una \emph{forma nueva}, si
	$f\in\spitzH{k}{\frak{N},\omega}^{\neue}$ y $f$ es una autofunci\'{o}n
	de $T_{\frak{p}}$ para todo $\frak{p}\nmid\frak{N}$.
\end{defFormaNueva}

Si $f\in\spitzH{k}{\frak{N},\omega}$ es una forma nueva y
$T_{\frak{p}}f=\lambda_{\frak{p}}f$, entonces los coeficientes de Fourier de
$f$ verifican la relaci\'{o}n
\begin{equation}
	\label{eq:coeficientesdefourierformanueva}
	c(\frak{p},f) \,=\,\lambda_{\frak{p}}\,c(\oka{F},f)
	\text{ ,}
\end{equation}
%
para todo $\frak{p}\nmid\frak{N}$. En particular, $c(\oka{F},f)\not=0$. Decimos
que una forma nueva $f$ \emph{est\'{a} normalizada} (o que $f$ es una forma
(nueva) \emph{normalizada}), si $c(\oka{F},f)=1$.

\begin{coroCoeficientesNulosFormaVieja}%
	\label{coro:coeficientesnulosformavieja}
	Si $f,g\in\spitzH{k}{\frak{N},\omega}^{\neue}$ son dos formas nuevas
	normalizadas y con los mismos autovalores, entonces $f=g$.
\end{coroCoeficientesNulosFormaVieja}

\begin{proof}
	Es consecuencia de \ref{thm:coeficientesnulosformavieja} con
	$\frak{l}=\frak{N}$.
\end{proof}

El resultado anterior se puede expresar diciendo que \emph{dentro de cada %
espacio $\spitzH{k}{\frak{N},\omega}$} vale la propiedad de
\emph{multiplicidad uno} para formas nuevas.
% El espacio $\spitzH{k}{\frak{N},\omega}^{\neue}$ se descompone como suma
% directa de autoespacios ortogonales para el \'{a}lgebra de Hecke; cada uno de
% estos autoespacios tiene asociado un \emph{sistema de autovalores}:
% $\{a_{\frak{p}}\}_{\frak{p}\nmid\frak{N}}$; dos autoespacios distintos
% determinan sistemas de autovalores distintos. En t\'{e}rminos de una base de
% autoformas, 
El espacio $\spitzH{k}{\frak{N},\omega}^\neue$ admite una base ortogonal de
formas nuevas que podemos asumir normalizadas; cada una de estas autofunciones
tiene asociado un \emph{sistema de autovalores}:
$\{a_{\frak{p}}(f)\}_{\frak{p}}$ (en principio, $\frak{p}\nmid\frak{N}$), donde
$a_{\frak{p}}=c(\frak{p},f)$; dos formas nuevas determinan el mismo sistema de
autovalores, si una es un m\'{u}ltiplo de la otra.
Pero el Corolario~\ref{coro:coeficientesnulosformavieja} no dice nada en el
caso en que $f$ y $g$ pertenezcan a espacios asociados a caracteres diferentes.
El siguiente resultado resulve este problema.

\begin{teoFormasNuevasMismosAutovalores}%
	[ver {\cite[Thm.~3,6]{ShemanskeWalling}}]%
	\label{thm:formasnuevasmismosautovalores}
	Sean $\omega,\omega':\,\ideles{F}\rightarrow\bb{C}^{\times}$
	cuasicaracteres triviales en $F^{\times}$. Sean
	$f\in\spitzH{k}{\frak{N},\omega}$ y $g\in\spitzH{k}{\frak{N},\omega'}$
	formas nuevas (normalizadas) con los mismos autovalores para
	$T_{\frak{p}}$ para todo primo $\frak{p}\nmid\frak{N}$
	(equivalentemente, $c(\frak{m},f)=c(\frak{m},g)$ para todo ideal
	\'{\i}ntegro $\frak{m}$ tal que $(\frak{m},\frak{N})=1$). Entonces
	$\omega=\omega'$ y $f=g$.
\end{teoFormasNuevasMismosAutovalores}
