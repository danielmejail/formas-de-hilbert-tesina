Ahora, si $\tau_{\infty}\in\Idinf{F}$, se debe cumplir
\begin{math}
	\omega(\tau_{\infty})=\prod_{i=1}^{n}\,\tau_{i}^{k_{0}-2}
\end{math} y el correspondiente car\'{a}cter $\chi$ debe verificar
\begin{math}
	\chi(\tau_{\infty})=\prod_{i=1}^{n}\,
		\signo(\tau_{i})^{k_{0}-2}
\end{math}~.
En particular, $\chi$ no s\'{o}lo es trivial en $F^{\times}$, sino tambi\'{e}n
en
\begin{math}
	F_{\infty,+}^{\times}=\bb{R}_{>0}\times\cdots\times\bb{R}_{>0}
\end{math} y en $\Idfin{\oka{F}}$ (la representaci\'{o}n $\rho$ era trivial en
este subgrupo). En otras palabras, $\chi$ define un car\'{a}cter en el grupo de
clases estrictas
\begin{math}
	\pClass{F}=F^{\times}\backslash\ideles{F}/
		F^{\times}_{\infty,+}\Idfin{\oka{F}}
\end{math}~.
% Esto se puede usar para definir un car\'{a}cter del grupo de clases
% % (?)
% Si denotamos, para cada lugar $v$ de $F$, por
% $\omega_{v}:\,F_{v}^{\times}\rightarrow\bb{C}^{\times}$ el
% cuasicar\'{a}cter inducido, vale que $\omega=\prod_{v}\,\omega_{v}$.
% En paritcular,
% $\omega(\tau)=\psi_{\infty}(\tau_{\infty})\psi_{0}(\hhat{\tau})$,
% separando la parte arquimediana de la parte finita.
Por esta raz\'{o}n, si $\hhat{\tau}\in\Idfin{F}$, el valor de
$\chi(\hhat{\tau})$ depende \'{u}nicamente de la clase de este id\`{e}le en
$\Idfin{F}/\Idfin{\oka{F}}$. Si $\frak{t}=\hhat{\tau}\Adfin{\oka{F}}\cap F$ es
el ideal fraccionario determinado por $\hhat{\tau}$, entonces podemos definir
\begin{align*}
	\chi(\frak{t}) \,:=\,\chi(\hhat{\tau})
		& \quad\text{y}\quad\omega(\frak{t})\,:=\,\omega(\hhat{\tau})
	\text{ .}
\end{align*}
%
En particular, en ideales principales,
\begin{align*}
	\chi((\lambda)) & \,=\,\chi(\lambda_{0})\,=\,
		\chi(\lambda_{\infty})^{-1}\,=\,
		\prod_{i=1}^{n}\,\signo(\lambda_{i})^{-(k_{0}-2)}
	\text{ ,}
\end{align*}
%
que s\'{o}lo depende del peso $\peso{k}$. En general, si $\idnorm(\frak{t})$
denota la norma del ideal $\frak{t}$,
\begin{align*}
	|\omega(\frak{t})| & \,=\,|\hhat{\tau}|_{\adeles{F}}^{k_{0}-2}
		\,=\,\frac{1}{\idnorm(\frak{t})^{k_{0}-2}}
	\text{ .}
\end{align*}
%
