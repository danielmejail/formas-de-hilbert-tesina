% 
% En la secci\'{o}n \ref{subsec:cuaternionicasformasmodulares}, definimos formas
% modulares cuaterni\'{o}nicas de peso $\peso{k}$ y nivel $\frak{N}$ para un
% \'{a}lgebra de cuaterniones de divisi\'{o}n $B$ y vimos c\'{o}mo el espacio
% $\modularH[B]{k}{\frak{N}}$ se descompon\'{\i}a, reflejando la
% descomposici\'{o}n de la variedad de Shimura $\shimura[B]{\frak{N}}$. En esta
% secci\'{o}n, procedimos en el sentido inverso, asociando, en primer lugar, a
% cada par $(\frak{N},\frak{a})$ el grupo $\Gamma_{0}(\frak{N},\frak{a})$ y, en
% segunda instancia, a este grupo el espacio de formas modulares
% $\modularH{k}{\frak{N},\frak{a}}$. Mencionamos tambi\'{e}n que las variedades
% $\shimura[B]{\frak{N}}$ son compactas, pero que la variedad
% $Y_{0}(\frak{N})$ asociada por un procedimiento an\'{a}logo al
% \'{a}lgebra de matrices $\MM_{2\times 2}(F)$ no lo es. Aun as\'{\i},
% vimos que es posible compactificarla agregando una cantidad finita de
% puntos, que denominamos c\'{u}spides. A diferencia lo que sucede con las
% formas modulares el\'{\i}pticas, la holomorf\'{\i}a en las c\'{u}spides
% estaba garantizada por el pricipio de Koecher; pero esto no les restaba
% importancia. La existencia de estas c\'{u}spides la relacionamos con la
% existencia de expansiones de Fourier para las formas
% $f\in\modularH{k}{\frak{N},\frak{a}}$. Finalmente, estas expansiones nos
% permitieron identificar los subespacios $\spitzH{k}{\frak{N},\frak{a}}$
% de formas cuspidales. Llegamos entonces a la definici\'{o}n central de
% esta secci\'{o}n.

Hemos mencionado al comienzo de esta secci\'{o}n que un orden de Eichler
$\cal{O}\subset\MM_{2\times 2}(F)$ de nivel $\frak{N}$ tiene asociada cierta
variedad $Y_{0}(\frak{N})$. Esta variedad no es conexa, pero es una uni\'{o}n
disjunta de componentes conexas $Y_{0}(\frak{N},\frak{a})$ indexadas por el
grupo de clases estrictas del cuerpo $F$. En general, dado un subgrupo de
congruencia $\Gamma\subset\GLtp_{2}(F)$, la variedad
$Y(\Gamma)=\Gamma\backslash\hP^{n}$ no es compacta, pero es posible
compactificarla agregando una cantidad finita de puntos, que hemos denominado
c\'{u}spides. A diferencia de lo que sucede con las formas modulares
el\'{\i}pticas, la holomorf\'{\i}a en las c\'{u}spides est\'{a} garantizada por
el principio de Koecher; pero esto no les resta importancia. La existencia de
estas c\'{u}spides la hemos relacionado con la existencia de expansiones de
Fourier para las formas $f\in\modularH{k}{\Gamma}$. Finalmente, estas
expansiones nos han permitido identificar los subespacios $\spitzH{k}{\Gamma}$.
% Llegamos, entonces, a la definici\'{o}n central de esta secci\'{o}n.

\begin{defFormaModularDeHilbertDeNivelN}%
	\label{def:formamodulardehilbertdeniveln}
	Una \emph{forma de Hilbert} de peso $\peso{k}$ y nivel $\frak{N}$
	es una funci\'{o}n
	\index{forma modular!de Hilbert}
	\begin{align*}
		f & \,:\,(\hP^{\pm})^{n}
			\,\times\,(\GL_{2}(\Adfin{F})/\Idfin{\cal{O}})
			\,\rightarrow\,\bb{C}
	\end{align*}
	%
	holomorfa en la primera variable y localmente constante en la
	segunda tal que $f\operadormatrices{\peso{k}}{\gamma}=f$ para toda
	matriz $\gamma\in\GL_{2}(F)$, donde
	\begin{equation}
		\label{eq:matricesoperadordepesokcompleto}
		\big(f\operadormatrices{\peso{k}}{\gamma}\big)
			(z,\hhat{\alpha}\Idfin{\cal{O}})
		\,=\,\bigg(\prod_{i=1}^{n}\,
			\frac{\det(\gamma_{i})^{k_{i}+m_{i}-1}}{%
				j(\gamma_{i},z_{i})^{k_{i}}}
			\bigg)\,f(\gamma z,\gamma\hhat{\alpha}\Idfin{\cal{O}})
	\end{equation}
	%
	y los valores $m_{i}$ est\'{a}n dados por \eqref{eq:pesomatrices}.
	Denotamos este espacio por $\modularH{k}{\frak{N}}$. Como en las
	secciones anteriores, usaremos la notaci\'{o}n
	\begin{math}
		J_{i}(\gamma_{i},z_{i})=
			\frac{\det(\gamma_{i})^{k_{i}+m_{i}-1}}{%
				j(\gamma_{i},z_{i})^{k_{i}}}
	\end{math}~.
\end{defFormaModularDeHilbertDeNivelN}

\begin{obsEleccionDelFactorParaAdelicas}%
	\label{obs:elecciondelfactorparaadelicas}
	En relaci\'{o}n con la Observaci\'{o}n~\ref{obs:elecciondelfactor},
	podemos definir, m\'{a}s en general,
	$f\operadormatrices{\peso{k},\peso{m}}{\gamma}$ por la misma
	expresi\'{o}n \eqref{eq:matricesoperadordepesokcompleto}, pero con
	$\peso{k}\in\bb{Z}^{n}$ y $\peso{m}\in\bb{R}^{n}$ arbitrarios, y
	$\modular{\peso{k},\peso{m}}{\frak{N}}$ como el espacio de aquellas
	funciones que cumplen $f\operadormatrices{\peso{k},\peso{m}}{\gamma}=f$
	siempre que $\gamma\in\GL_{2}(F)$. Este espacio es trivial, a menos que
	$(\peso{k},\peso{m})$ sea un peso razonable. Si $h\in\bb{R}$, entonces
	\begin{align*}
		f\operadormatrices{\peso{k},\peso{m}+h\cdot\peso{1}}{\gamma} &
			\,=\,\norma(\det\,\gamma)^{h}\,
			f\operadormatrices{\peso{k},\peso{m}}{\gamma}
		\text{ ,}
	\end{align*}
	%
	con lo cual, si $f\in\modular{\peso{k},\peso{m}}{\frak{N}}$, la
	funci\'{o}n
	\begin{math}
		\tilde{f}(z,\hhat{\alpha}\Idfin{\cal{O}})=
			\big(\signo(\Im\,z)\,
			|\det\,\hhat{\alpha}|_{\adeles{F}}\big)^{h}\,
			f(z,\hhat{\alpha}\Idfin{\cal{O}})
	\end{math} verifica
	\begin{align*}
		\tilde{f}(\gamma z,\gamma\hhat{\alpha}\Idfin{\cal{O}}) & \,=\,
			\big(\signo(\norma(\det\,\gamma))\,
				|\det\,\gamma_{0}|_{\adeles{F}}\big)^{h}\,
			\bigg(\prod_{i=1}^{n}\,
				\frac{j(\gamma_{i},z_{i})^{k_{i}}}{%
					\det(\gamma_{i})^{k_{i}+m_{i}-1}}
				\bigg)\,
				\tilde{f}(z,\hhat{\alpha}\Idfin{\cal{O}}) \\
		& \,=\,\norma(\det\,\gamma)^{-h}\,\bigg(\prod_{i=1}^{n}\,
				\frac{j(\gamma_{i},z_{i})^{k_{i}}}{%
					\det(\gamma_{i})^{k_{i}+m_{i}-1}}
				\bigg)\,
				\tilde{f}(z,\hhat{\alpha}\Idfin{\cal{O}})
		\text{ ,}
	\end{align*}
	%
	donde $\gamma_{0}$ denota la proyecci\'{o}n de $\gamma$ a
	$\GL_{2}(\Adfin{F})$. Es decir, $f\mapsto\tilde{f}$ es una
	biyecci\'{o}n entre $\modular{\peso{k},\peso{m}}{\frak{N}}$ y
	$\modular{\peso{k},\peso{m}+h\cdot\peso{1}}{\frak{N}}$.
\end{obsEleccionDelFactorParaAdelicas}

Consideraremos exclusivamente formas de Hilbert en el sentido de la
Definici\'{o}n~\ref{def:formamodulardehilbertdeniveln}.

\begin{propoDescomposicionDelEspacioDeFormasModularesDeHilbert}%
	\label{thm:descomposiciondelespaciodeformasmodularesdehilbert}
	Existe un isomorfismo
	\begin{equation}
		\label{eq:descomposicionmodularesmatrices}
		\modularH{k}{\frak{N}} \,\simeq\, \bigoplus_{\frak{a}}\,
			\modularH{k}{\frak{N},\frak{a}}
		\text{ ,}
	\end{equation}
	%
	donde $\frak{a}$ recorre un sistema de representantes de las clases
	estrictas en $\pClass{F}$.
\end{propoDescomposicionDelEspacioDeFormasModularesDeHilbert}

\begin{proof}
	En primer lugar, por el Corolario~\ref{coro:normaclasesestrictas}, la
	norma reducida induce una biyecci\'{o}n
	\begin{align*}
		\GLtp_{2}(F)\backslash\GL_{2}(\Adfin{F})/\Idfin{\cal{O}}
			& \,\simeq\,
			F_{+}^{\times}\backslash\Idfin{F}/\Idfin{\oka{F}}
			\,\simeq\,\pClass{F}
		\text{ .}
	\end{align*}
	%
	Eligiendo un sistema de representantes $\{\frak{a}\}$ de las clases
	estrictas de $F$, v\'{\i}a \eqref{eq:ideleasociadomatrices} y
	\eqref{eq:matrizasociadaaidelematrices} queda determinado un conjunto
	$\{\hhat{\alpha}\}_{\frak{a}}$ que constituye un sistema de
	representantes de las clases en
	\begin{math}
		\GLtp_{2}(F)\backslash\GL_{2}(\Adfin{F})/\Idfin{\cal{O}}
	\end{math}~,
	en correspondencia con $\{\frak{a}\}$.
	% y, en particular, contiene un sistema de representantes, si se
	% permiten matrices con determinante no necesariamente totalmente
	% positivo.
	En consecuencia, dada $\hhat{\alpha}'\in\GL_{2}(\Adfin{F})$, existen un
	representante $\hhat{\alpha}$ y una matriz $\gamma\in\GLtp_{2}(F)$
	tales que
	\begin{math}
		\gamma\hhat{\alpha}'\Idfin{\cal{O}}=
			\hhat{\alpha}\Idfin{\cal{O}}
	\end{math}~.
	En segundo lugar, notamos que, por la
	Observaci\'{o}n~\ref{obs:eichlernorma}, existen matrices
	$\lista{\rho}{n}\in \GL_{2}(F)$ tales que
	\begin{math}
		\det(\inc[i](\rho_{j}))>0
	\end{math} para $i\not=j$ y
	\begin{math}
		\det(\inc[j](\rho_{j}))<0
	\end{math}~. Usando estas matrices, podemos ver que toda
	$f\in\modularH{k}{\frak{N}}$ est\'{a} determinada por los valores que
	toma en el producto de los semiplanos superiores $\hP^{n}$.
	% Entonces
	% \begin{align*}
		% f(z,\hhat{\alpha}'\Idfin{\cal{O}}) & \,=\,
			% \big(f\operadormatrices{\peso{k}}{\gamma}\big)
				% (z,\hhat{\alpha}'\Idfin{\cal{O}})
		% \,=\,\bigg(\prod_{i=1}^{n}\,J_{i}(\gamma_{i},z_{i})^{-1}
			% \bigg)\,f(\gamma z, \hhat{\alpha}\Idfin{\cal{O}})
		% \text{ .}
	% \end{align*}
	% %
	De estas dos observaciones, se deduce que $f\in\modularH{k}{\frak{N}}$
	est\'{a} determinada por los valores que toma en pares
	$(z,\hhat{\alpha}\Idfin{\cal{O}})$, con $z\in\hP^{n}$ y $\hhat{\alpha}$
	tomado del conjunto de representantes $\{\hhat{\alpha}\}_{\frak{a}}$.

	Ahora, dada $f\in\modularH{k}{\frak{N}}$ y uno de los representantes
	$\frak{a}$, sea $f_{\frak{a}}:\,\hP^{n}\rightarrow\bb{C}$ la
	funci\'{o}n
	\begin{align*}
		f_{\frak{a}}(z) & \,=\,f(z,\hhat{\alpha}\Idfin{\cal{O}})
		\text{ .}
	\end{align*}
	%
	Esta funci\'{o}n $f_{\frak{a}}$ es holomorfa, porque $f$ lo es y, si
	$\gamma\in\Gamma_{0}(\frak{N},\frak{a})$, entonces
	\begin{align*}
		\big(f_{\frak{a}}\operadormatrices{\peso{k}}{\gamma}\big)(z)
		& \,=\,\bigg(\prod_{i=1}^{n}\,J_{i}(\gamma_{i},z_{i})^{-1}
			\bigg)\,f(\gamma z,\hhat{\alpha}\Idfin{\cal{O}})
			\,=\,\big(f\operadormatrices{\peso{k}}{\gamma}\big)
				(z,\gamma^{-1}\hhat{\alpha}\Idfin{\cal{O}}) \\
		& \,=\,\big(f\operadormatrices{\peso{k}}{\gamma}\big)
			(z,\hhat{\alpha}\Idfin{\cal{O}})
			\,=\,f(z,\hhat{\alpha}\Idfin{\cal{O}})
			\,=\,f_{\frak{a}}(z)
		\text{ .}
	\end{align*}
	%
	La aplicaci\'{o}n $f\mapsto f_{\frak{a}}$ es lineal e induce una
	transformaci\'{o}n
	\begin{math}
		\modularH{k}{\frak{N}}\rightarrow\bigoplus_{\frak{a}}\,
			\modularH{k}{\frak{N},\frak{a}}
	\end{math}~.
	El argumento del p\'{a}rrafo anterior muestra que
	$f\mapsto (f_{\frak{a}})_{\frak{a}}$ es inyectiva. Rec\'{\i}procamente,
	dadas $f_{\frak{a}}\in\modularH{k}{\frak{N},\frak{a}}$, las condiciones
	\begin{align*}
		f(z,\hhat{\alpha}\Idfin{\cal{O}}) & \,=\,f_{\frak{a}}(z)
			\qquad\text{si } z\in\hP^{n} \quad\text{y} \\
		f\operadormatrices{\peso{k}}{\gamma} & \,=\,f
			\qquad\text{si }\gamma\in\GL_{2}(F)
	\end{align*}
	%
	determinan un\'{\i}vocamente una funci\'{o}n
	\begin{math}
		f:\,(\hP^{\pm})^{n}\times(\GL_{2}(\Adfin{F})/\Idfin{\cal{O}})
			\rightarrow\bb{C}
	\end{math}~.
	Veamos esto: si
	\begin{math}
		(z,\hhat{\alpha}'\Idfin{\cal{O}})\in
			(\hP^{\pm})^{n}\times\GL_{2}(\Adfin{F})/\Idfin{\cal{O}}
	\end{math}~,
	podemos tomar alg\'{u}n producto de las matrices $\rho_{j}$,
	llam\'{e}moslo $\rho$, de manera que $\rho\,z\in\hP^{n}$. Ahora, porque
	$\{\hhat{\alpha}\}_{\frak{a}}$ es un sistema de representantes de
	\begin{math}
		\GLtp_{2}(F)\backslash\GL_{2}(\Adfin{F})/\Idfin{\cal{O}}
	\end{math}~,
	existen $\gamma\in\GLtp_{2}(F)$ y $\hhat{\alpha}$ en dicho conjunto,
	tales que
	\begin{math}
		\gamma\rho\hhat{\alpha}'\Idfin{\cal{O}}=
			\hhat{\alpha}\Idfin{\cal{O}}
	\end{math}~.
	Entonces $f(z,\hhat{\alpha}'\Idfin{\cal{O}})$ es igual a
	$f(\gamma\rho\,z,\hhat{\alpha}\Idfin{\cal{O}})$, m\'{o}dulo alguna
	constante. Esta funci\'{o}n es una forma modular de Hilbert: es
	holomorfa en la primera variable porque las $f_{\frak{a}}$ lo son,
	es localmente constante porque, dado $\hhat{\alpha}'$, es constante en
	el entorno $\hhat{\alpha}'\Idfin{\cal{O}}$ y verifica, por
	definici\'{o}n, la condici\'{o}n de invarianza.
\end{proof}

\begin{defFormaDeHilbertCuspidalDeNivelN}%
	\label{def:formadehilbertcuspidaldeniveln}
	Una forma de Hilbert $f=(f_{\frak{a}})_{\frak{a}}\in%
	\modularH{k}{\frak{N}}$ es una \emph{forma cuspidal}, si
	\index{forma modular!de Hilbert!cuspidal}
	$f_{\frak{a}}\in\spitzH{k}{\frak{N},\frak{a}}$ para cada
	representante $\frak{a}$ de las clases estrictas de $F$.
	El subespacio de formas cuspidales lo denotamos
	$\spitzH{k}{\frak{N}}$ y vale
	\begin{equation}
		\label{eq:descomposicioncuspidalesmatrices}
		\spitzH{k}{\frak{N}} \,\simeq\, \bigoplus_{\frak{a}}\,
			\spitzH{k}{\frak{N},\frak{a}}
		\text{ .}
	\end{equation}
	%
\end{defFormaDeHilbertCuspidalDeNivelN}

\begin{obsMockProductoInterno}\label{obs:mockproductointerno}
	Teniendo en cuenta el isomorfismo
	\eqref{eq:descomposicioncuspidalesmatrices}, se podr\'{\i}a definir un
	producto interno en $\spitzH{k}{\frak{N}}$ por
	\begin{math}
		\langle f,g\rangle =\sum_{\frak{a}}\,
		\langle f_{\frak{a}},g_{\frak{a}}\rangle_{\frak{a}}
	\end{math}~,
	donde
	\begin{math}
		\langle\cdot,\cdot\rangle_{\frak{a}}=
		\langle\cdot,\cdot\rangle_{\Gamma_{0}(\frak{N},\frak{a})}
	\end{math}~.
	Pero esta expresi\'{o}n \emph{depende de los representantes}
	$\{\frak{a}\}$. Si $\{\frak{a_{1}}\}$ es otra elecci\'{o}n y
	$[\frak{a}_{1}]=[\frak{a}]$, entonces existen
	$\gamma_{\frak{a},\frak{a}_{1}}\in\GLtp_{2}(F)$ que verifican
	\begin{math}
		\hhat{\alpha}_{1}\Idfin{\cal{O}}=
		\gamma_{\frak{a},\frak{a}_{1}}^{-1}\hhat{\alpha}\Idfin{\cal{O}}
	\end{math}~.%
	\footnote{
		Si $\lambda_{\frak{a},\frak{a}_{1}}\in F_{+}^{\times}$ verifica
		$\lambda_{\frak{a},\frak{a}_{1}}\,\frak{a}_{1}=\frak{a}$,
		podemos elegir la matriz diagonal
		\begin{math}
			\left[\begin{smallmatrix}
				\lambda_{\frak{a},\frak{a}_{1}} & \\
				& 1
			\end{smallmatrix}\right]
		\end{math}~.
	}
	En particular,
	\begin{math}
		\det(\gamma_{\frak{a},\frak{a}_{1}})\,\frak{a}_{1}=\frak{a}
	\end{math}~y, por la Proposici\'{o}n~\ref{propo:peterssonpropiedades}
	(i),
	\begin{align*}
		\langle f_{\frak{a}_{1}},g_{\frak{a}_{1}}
			\rangle_{\frak{a}_{1}} & \,=\,
		\langle f_{\frak{a}}\operadormatrices{\peso{k}}{%
				\gamma_{\frak{a},\frak{a}_{1}}},
		g_{\frak{a}}\operadormatrices{\peso{k}}{%
			\gamma_{\frak{a},\frak{a}_{1}}}\rangle_{\frak{a}_{1}}
		\,=\,\norma(
			\det(\gamma_{\frak{a},\frak{a}_{1}}))^{k_{0}-2}\,
		\langle f_{\frak{a}},g_{\frak{a}}\rangle_{\frak{a}}
		\text{ .}
	\end{align*}
	%
	Para obtener un producto invariante, se debe definir
	\begin{equation}
		\label{eq:productointernodepeterssondehilbert}
		\langle f,g\rangle \,:=\,\sum_{\frak{a}}\,
			\idnorm(\frak{a})^{k_{0}-2}\,\langle f_{\frak{a}},
				g_{\frak{a}}\rangle_{\frak{a}}
		\text{ .}
	\end{equation}
	%
	Esta definici\'{o}n quedar\'{a} mejor justificada una vez realizada la
	interpretaci\'{o}n automorfa de las formas de Hilbert
	(\S~\ref{subsec:enlosadelescuadradointegrable}).
\end{obsMockProductoInterno}

Sea $f\in\modularH{k}{\frak{N}}$. Cada componente $f_{\frak{a}}$ admite un
desarrollo de Fourier, que, por el Teorema~\ref{thm:principiodekoecher}, es de
la forma
\begin{align*}
	f_{\frak{a}}(z) & \,=\,a_{0}(f_{\frak{a}})+
		\sum_{\nu\frak{a}^{\perp},\,\nu\gg 0}\,a_{\nu}(f_{\frak{a}})\,
			e^{2\pi i\traza(\nu z)}
	\text{ .}
\end{align*}
%
Los coeficientes de $f_{\frak{a}}$ verifican
\begin{math}
	a_{\nu\epsilon}(f_{\frak{a}})=
		\big(\prod_{i=1}^{n}\,\epsilon_{i}^{m_{i}+k_{i}-1}\big)\,
		a_{\nu}(f_{\frak{a}})
\end{math}~,
para toda unidad totalmente positiva $\epsilon\in\oka{F,+}^{\times}$. En
particular, el valor de la expresi\'{o}n
\begin{equation}
	\label{eq:coeficientesdefouriertilde}
	\tilde{c}(\nu,f_{\frak{a}}) \,=\,
		\bigg(\prod_{i=1}^{n}\,\nu_{i}^{m_{i}+k_{i}-1}\bigg)^{-1}\,
			a_{\nu}(f_{\frak{a}})
\end{equation}
%
es invariante por multiplicar $\nu$ por una unidad totalmente positiva.

Por otro lado, como el conjunto $\{\frak{a}\}$ es un sistema de representantes
del grupo de clases estrictas de $F$, el conjunto $\{\frak{a}\diferente\}$
(donde $\diferente$ denota el diferente de $F$) tambi\'{e}n lo es. Dado un
ideal fraccionario $\frak{m}$ de $\oka{F}$, existe un \'{u}nico representante
$\frak{a}$ tal que $[\frak{m}]=[\frak{a}\diferente]$ en $\pClass{F}$, es decir,
existe $\nu\gg 0$ tal que $\frak{m}=\nu\frak{a}\diferente$. Si $\frak{m}$ es
\'{\i}ntegro, $\nu\frak{a}\diferente\subset\oka{F}$, de lo que se deduce que
$\nu\in(\frak{a}\diferente)^{-1}=\frak{a}^{\perp}$. Definimos, entonces, una
funci\'{o}n en ideales
\begin{equation}
	\label{eq:coeficientesdefourierformamodular}
	c(\frak{m},f) \,:=\,
		\begin{cases}
			\tilde{c}(\nu,f_{\frak{a}}) & \quad\text{si }
					\frak{m}=\nu\cdot\frak{a}\diferente
					\text{ (\phantom)}
					\nu\in\frak{a}^{\perp},\,\nu\gg 0
					\text{\phantom()} \\
			0 & \quad\text{si }\frak{m}\text{ no es \'{\i}ntegro}
		\end{cases}
	\text{ .}
\end{equation}
%
Llamamos \emph{coeficientes de Fourier de $f$} a los valores de esta
funci\'{o}n \cite{ShimuraSpecialValuesOfZeta}.
Dado que $\tilde{c}(\nu\epsilon,f_{\frak{a}})=\tilde{c}(\nu,f_{\frak{a}})$,
si $\epsilon\in\oka{F,+}^{\times}$, los coeficientes de Fourier est\'{a}n bien
definidos.
