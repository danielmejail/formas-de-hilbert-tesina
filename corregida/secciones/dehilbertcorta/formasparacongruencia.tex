Entenderemos por \emph{peso} un vector\index{peso}
\begin{math}
	\peso{k}=(\lista{k}{n})\in(\bb{Z}_{\geq 2})^{n}
\end{math}
tal que $k_{i}\equiv k_{j}\,(\modulo\,2)$. Dado un peso $\peso{k}$, definimos
\begin{equation}
	\label{eq:pesomatrices}
	k_{0} \,=\,\mathrm{max}_{i}\,k_{i} \text{ ,}\quad
		m_{i} \,=\,\frac{k_{0} - k_{i}}{2}\quad\text{y}\quad
		w_{i} \,=\,k_{i}-2
	\text{ .}
\end{equation}
%

\begin{defPesoKInvariante}\label{def:pesokinvariante}
	Sea $\peso{k}$ un peso y sea
	\begin{math}
		J_{i}:\,\GL_{2}(\bb{R})\times\hP^{\pm}\rightarrow\bb{C}
	\end{math}
	el factor de automorf\'{\i}a
	\index{factor de automorfia@factor de automorf\'{\i}a}
	\begin{equation}
		\label{eq:factordeautomorfiamatrices}
		J_{i}(\gamma,z) \,=\, \frac{j(\gamma,z)^{k_{i}}}{%
			\det(\gamma)^{m_{i}+k_{i}-1}}
		\text{ .}
	\end{equation}
	%
	Dado un subgrupo $\Gamma\subset\GLtp_{2}(F)$, una funci\'{o}n
	$f:\,\hP^{n}\rightarrow\bb{C}$ ($n>1$) se dice \emph{de peso %
	$\peso{k}$ invariante para $\Gamma$}, si, para toda $\gamma\in\Gamma$,
	satisface
	\index{peso k invariante@peso $\peso{k}$ invariante}
	\begin{align*}
		f(\gamma z) & \,=\,\bigg(
			\prod_{i=1}^{n}\,J_{i}(\gamma_{i},z_{i})\bigg)\,f(z)
		\text{ ,}
	\end{align*}
	%
	donde $z=(\lista{z}{n})\in\hP^{n}$ y $\gamma_{i}\in\GL_{2}(\bb{R})$
	es la matriz que se obtiene aplicando la $i$-\'{e}sima inmersi\'{o}n
	a las coordenadas de $\gamma$.
\end{defPesoKInvariante}

Las funciones $J_{i}$ verifican
\begin{equation}
	\label{eq:factordeautomorfiacociclomatrices}
	J_i(\gamma\gamma',z) \,=\,J_i(\gamma,\gamma'z)\,J_i(\gamma',z)
\end{equation}
%
para todo $z\in\hP^{n}$ y todas $\gamma,\gamma'\in\GL_{2}(\bb{R})$.

\begin{defOperadorDePesoKMatrices}\label{def:operadordepesokmatrices}
	Dada una matriz $\gamma\in\GLtp_{2}(F)$, definimos el \emph{operador %
	de peso $\peso{k}$} actuando en una funci\'{o}n
	$f:\,\hP^{n}\rightarrow\bb{C}$ como
	\begin{equation}
		\label{eq:matricesoperadordepesok}
		\big(f\operadormatrices{\peso{k}}{\gamma}\big) (z)\,=\,
			\bigg(\prod_{i=1}^{n}\,J_{i}(\gamma_{i},z_{i})^{-1}
			\bigg)\,f(\gamma z)
		\text{ .}
	\end{equation}
	%
\end{defOperadorDePesoKMatrices}

Entonces, una funci\'{o}n $f$ es de peso $\peso{k}$ invariante para un subgrupo
de congruencia $\Gamma$, si $f\operadormatrices{\peso{k}}{\gamma}=f$ para toda
matriz $\gamma\in\Gamma$ y la relaci\'{o}n
\eqref{eq:factordeautomorfiacociclomatrices} implica que
\begin{math}
	\big(f\operadormatrices{\peso{k}}{\gamma}\big)
		\operadormatrices{\peso{k}}{\gamma'}=
		f\operadormatrices{\peso{k}}{(\gamma\gamma')}
\end{math}~. En particular, si $f$ es invariante de peso $\peso{k}$ para un
subgrupo $\Gamma$, entonces $f\operadormatrices{\peso{k}}{A}$ es de peso
$\peso{k}$ invariante para el conjugado $A^{-1}\Gamma A$.

\begin{obsEleccionDelFactor}\label{obs:elecciondelfactor}
	Un comentario acerca de la elecci\'{o}n de los factores
	\eqref{eq:factordeautomorfiamatrices}. Si $\Gamma\subset\GLtp_{2}(F)$
	es un subgrupo de congruencia, entonces $\Gamma$ contiene un subgrupo
	de \'{\i}ndice finito en el grupo de unidades del cuerpo:
	espec\'{\i}ficamente,
	\begin{math}
		|\oka{F}^{\times}:\Gamma\cap\oka{F}^{\times}|<\infty
	\end{math}~, identificando $x\in F^{\times}$ con
	\begin{math}
		\left[\begin{smallmatrix}
			x & \\ & x
		\end{smallmatrix}\right]\in
			\GL_{2}(F)
	\end{math}~.
	% \footnote{
		% Porque, por ejemplo,
		% $\oka{F}^{\times}\subset\Gamma_{0}(\frak{N},\frak{a})$ y
		% $\Gamma$ y $\Gamma_{0}(\frak{N},\frak{a}) son conmesurables.
	% }
	Consideremos la siguiente variaci\'{o}n de la Definici\'{o}n~%
	\ref{def:pesokinvariante}. Llamamos \emph{peso} a un par
	$(\peso{k},\peso{m})$, donde $\peso{k}\in\bb{Z}^{n}$ y
	$\peso{m}\in\bb{R}^{n}$ y definimos una acci\'{o}n a derecha de
	$\GLtp_{2}(F)$ en funciones $f:\,\hP^{n}\rightarrow\bb{C}$ torcida por
	un peso $(\peso{k},\peso{m})$: dada $\gamma\in\GLtp_{2}(F)$ y
	$z\in\hP^{n}$,
	\begin{align*}
		\big(f\operadormatrices{\peso{k},\peso{m}}{\gamma}\big)(z)
			& \,=\,\bigg(\prod_{i=1}^{n}\,
				\frac{\det(\gamma_{i})^{k_{i}+m_{i}-1}}{%
					j(\gamma_{i},z_{i})^{k_{i}}}\bigg)\,
			f(\gamma\cdot z)
		\text{ .}
	\end{align*}
	%
	Los elementos centrales
	\begin{math}
		x=\left[\begin{smallmatrix}
			x & \\ & x
		\end{smallmatrix}\right]\in F^{\times}
	\end{math} act\'{u}an por multiplicaci\'{o}n por escalares:
	\begin{math}
		f\operadormatrices{\peso{k},\peso{m}}{x}=
			x^{(\peso{k}+2\cdot\peso{m})-2\cdot\peso{1}}\,f
	\end{math}~, donde
	\begin{math}
		x^{(\lista{v}{n})}=\prod_{i}\,x_{i}^{v_{i}}
	\end{math} y $\peso{1}=(\lista[\null]{1}{\null})$ es el vector de peso
	$1$ en todas las componentes. Decimos que $f$ es
	\emph{de peso $(\peso{k},\peso{m})$ invariante para $\Gamma$}, si
	$f\operadormatrices{\peso{k},\peso{m}}{\gamma}=f$ para toda
	$\gamma\in\Gamma$. Recuperamos la Definici\'{o}n~%
	\ref{def:pesokinvariante} eligiendo $m_i=\frac{k_{0}-k_{i}}{2}$, con
	$k_{0}=\max_{i}\,k_{i}$ y recuperamos la definici\'{o}n usual
	mencionada en la introducci\'{o}n, determinada por la regla de
	transformaci\'{o}n \eqref{eq:operadormatricesusual}, tomando
	$m_i=1-\frac{k_i}{2}$. Notamos que, en ambos casos, $x\in F^{\times}$
	act\'{u}a por multiplicaci\'{o}n por una potencia de la norma
	$\norma[F/\bb{Q}](x)$, pues el valor de $k_i+2m_i$ es independiente de
	de la inmersi\'{o}n $i$. Un peso $(\peso{k},\peso{m})$ tal que
	$k_i+2m_i$ no depende de $i$ se dice \emph{razonable}.%
	\footnote{Ver \cite{DembeleLoefflerPacettiNonparitious}.}

	En general, dados un subgrupo de congruencia $\Gamma$ y un peso
	$(\peso{k},\peso{m})$, las \'{u}nicas funciones de peso
	$(\peso{k},\peso{m})$ invariantes para $\Gamma$ son las funciones
	nulas, a menos que el peso sea razonable; esto es consecuencia de que,
	si $f\operadormatrices{\peso{k},\peso{m}}{\gamma}=f$ para toda
	$\gamma\in\Gamma$ con $f\not=0$, entonces
	\begin{math}
		\varepsilon^{(\peso{k}+2\cdot\peso{m})-2\cdot\peso{1}}=1
	\end{math} para toda unidad
	\begin{math}
		\varepsilon\in U=\Gamma\cap\oka{F}^{\times}
	\end{math}~, pero $|\oka{F}^{\times}:U|<\infty$ implica que $U$ es un
	subgrupo de rango m\'{a}ximo en $\oka{F}^{\times}$ y esto, a su vez,
	fuerza que el valor $k_i+2m_i$ sea independiente de $i$.

	Dos pesos razonables $(\peso{k},\peso{m})$ y $(\peso{k},\peso{m}')$
	asociados al mismo vector de enteros $\peso{k}$ difieren en un vector
	constante $\peso{m}-\peso{m}'=h\cdot\peso{1}$, con $h\in\bb{R}$. Si
	$\gamma\in\GLtp_{2}(F)$, la relaci\'{o}n entre las correspondientes
	acciones a derecha est\'{a} dada por:
	\begin{align*}
		f\operadormatrices{\peso{k},\peso{m}+h\cdot\peso{1}}{\gamma}
			& \,=\,\norma(\det\,\gamma)^h\,
				f\operadormatrices{\peso{k},\peso{m}}{\gamma}
		\text{ .}
	\end{align*}
	%
	En particular, si $\gamma\in\Gamma$ entonces
	$\det(\gamma)\in\oka{F,+}^{\times}$ y su norma es $1$, con lo que las
	definiciones de invarianza para un subgrupo de congruencia $\Gamma$ no
	se ve modificada por aplicar una traslaci\'{o}n constante a la
	componente $\peso{m}$ del peso.
	Al pasar a la interpretaci\'{o}n
	ad\'{e}lica (\S~\ref{sec:dehilbertformasmodulares}), podremos observar
	que los espacios de formas modulares asociados a pesos
	$(\peso{k},\peso{m})$ y $(\peso{k},\peso{m}')$ son isomorfos; desde un
	punto de vista anal\'{\i}tico, no habr\'{a} raz\'{o}n para preferir uno
	sobre el otro.

	La diferencia surge del lado algebraico, al observar que los subgrupos
	de congruencia contienen matrices de la forma
	\begin{math}
		\left[\begin{smallmatrix}
			\varepsilon & \\ & 1
		\end{smallmatrix}\right]
	\end{math}~, donde $\varepsilon$ es cualquier unidad totalmente
	positiva perteneciente a cierto subgrupo de \'{\i}ndice finito (rango
	m\'{a}ximo) en $\oka{F,+}^{\times}$ dependiendo del subgrupo de
	congruencia. Por ejemplo, supongamos que $f$ verifica
	\begin{math}
		f(\gamma\cdot z)=\bigg(\prod_{i=1}^{n}\,
			\frac{j(\gamma_{i},z_{i})^{k_{i}}}{%
				\det(\gamma_{i})^{k_{i}/2}}\bigg)\,f(z)
	\end{math}~, para toda $\gamma\in\Gamma$, entonces, para toda unidad
	totalmente positiva $\varepsilon$ perteneciente a un subgrupo de
	\'{\i}ndice finito en $\oka{F,+}^{\times}$, la ecuaci\'{o}n funcional
	es
	\begin{math}
		f(\varepsilon\,z)=\varepsilon^{\peso{k}/2}\,f(z)
	\end{math}~; si ls enteros $k_{i}$ son impares, aparecen ra\'{\i}ces
	cuadradas de estas unidades. Si queremos ``corregir'' esto, necesitamos
	elegir $h$ tal que $k_{i}/2 +h\in\bb{Z}$ para todo $i$; y esto s\'{o}lo
	se puede hacer, si los $k_{i}$ \emph{tienen todos la misma paridad}.
	Podemos tomar $h=k_{0}/2 -1$ (y obtenemos el factor en la
	Definici\'{o}n~\ref{def:pesokinvariante}) o una traslaci\'{o}n de
	\'{e}ste por cualquier entero $t$: $(k_{0}/2 - 1)+t$. Esto corresponde
	a tomar pesos de la forma $(\peso{k},\peso{m})$, con
	$m_{i}=\frac{k_{0}-k_{i}}{2}+t$, $t\in\bb{Z}$ independiente de $i$. La
	ecuaci\'{o}n funcional que cumple $f$ se lee
	\begin{align*}
		f(\varepsilon\,z) & \,=\,\bigg(\prod_{i=1}^{n}\,
			\varepsilon_{i}^{1-\big(\frac{k_{0}-k_{i}}{2}+t\big)}
			\bigg)\,f(z)
		\text{ .}
	\end{align*}
	%
	Con cualquier elecci\'{o}n de entero $t$ (y $h=k_{0}/2 -1 +t$), los
	vectores $\peso{m}$ son enteros;%
	\footnote{
		Ver \cite{DembeleLoefflerPacettiNonparitious} para una
		discusi\'{o}n m\'{a}s detallada de los pesos.
	}
	tomando $t=0$, se consigue que, adem\'{a}s, $m_{i}\geq 0$ para todo
	$i$ y $m_{j}=0$ para, al menos, alg\'{u}n $j$.
	% En la introducci\'{o}n, se mencionaron dos posibles convenciones al
	% momento de definir qu\'{e} se entiende por funci\'{o}n de peso
	% $\peso{k}$ invariante. Para cada $i\in[\![1,n]\!]$, llamemos,
	% temporariamente,
	% $I_{i}:\GL_{2}(\bb{R})\times\hP^{\pm}\rightarrow\bb{C}$ a la
	% funci\'{o}n dada por la expresi\'{o}n
	% \begin{equation}
		% \label{eq:factordeautomorfiamatricesfalso}
		% I_{i}(\gamma,z) \,=\,\frac{j(\gamma,z)^{k_{i}}}{%
			% \det(\gamma)^{k_{i}/2}}
	% \end{equation}
	% %
	% y sea $f\doblebarra{\peso{k}}{\gamma}$ el ``operdor de peso
	% $\peso{k}$'' definido usando \eqref{eq:factordeautomorfiamatricesfalso}
	% en lugar de \eqref{eq:factordeautomorfiamatrices}.
\end{obsEleccionDelFactor}

De ahora en adelante, a menos que indiquemos otra cosa, consideraremos
funciones invariantes en el sentido de la Definici\'{o}n~%
\ref{def:pesokinvariante}.

Al igual que las formas modulares el\'{\i}pticas, las funciones
holomorfas que satisfacen esta condici\'{o}n de invarianza tambi\'{e}n
admiten desarrollos en series de Fourier. Para poder entender esto, necesitamos
analizar el grupo de isotrop\'{\i}a de la c\'{u}spide en infinito. Nos
restringimos, en primer lugar, al caso $\Gamma=\Gamma_{0}(\frak{N},\frak{a})$.

Sea $\infty=[1:0]\in\bb{P}^{1}(F)$. Entonces una matriz
\begin{math}
	\gamma=\left[\begin{smallmatrix} a & b \\
		c & d \end{smallmatrix}\right]
\end{math}
deja fija la c\'{u}spide $\infty$ exactamente cuando $c=0$. En particular,
\begin{align*}
	\Gamma_{0}(\frak{N},\frak{a})_{\infty} & \,=\,\left\lbrace
		\begin{bmatrix} a & b \\ & d \end{bmatrix}\,:\,
		a,\,d\in\oka{F},\,b\in\frak{a},\,ad\in\oka{F,+}^{\times}
		\right\rbrace
	\text{ ,}
\end{align*}
%
cuyos elementos act\'{u}an por $z\mapsto\epsilon z+\mu$, donde $\mu\in\frak{a}$
y $\epsilon$ es una unidad totalmente positiva del anillo de enteros de $F$. En
general, si $\Gamma\subset\GLtp_{2}(F)$ es un subgrupo de congruencia, entonces
existe un ret\'{\i}culo $M\subset F$ y existe un subgrupo de unidades
totalmente positivas $V\subset\oka{F,+}^{\times}$ de rango m\'{a}ximo, $n-1$,
que preservan $M$ y tales que%
\footnote{
	Ver \cite[Ch.~I.3]{FreitagForms} o \cite[Ch.~II.1]{vanDerGeerSurfaces}.
}
\begin{align*}
	G(M,V) & \,=\,\bigg\{\begin{bmatrix}\epsilon & \mu \\ & 1\end{bmatrix}
			\,:\, \mu\in M,\,\epsilon\in V\bigg\}
		\,\subset\,\Gamma
\end{align*}
%
El subgrupo $G(M,V)$ es isomorfo al producto semidirecto $M\rtimes V$ y es un
subgrupo del grupo de isotrop\'{\i}a $\Gamma_{\infty}$. El grupo $V$
podr\'{\i}a ser un subgrupo propio del grupo completo de elementos de $F$
(necesariamente unidades) totalmente positivos que preservan el ret\'{\i}culo
$M$. Esto es consecuencia de la conmensurabilidad de los subgrupos de
congruencia. Toda funci\'{o}n $f:\,\hP^{n}\rightarrow\bb{C}$ de peso $\peso{k}$
invariante para $\Gamma$ es, pues, invariante por las traslaciones
$z\mapsto z+\mu$ en un ret\'{\i}culo $M$ y admitir\'{a}, en consecuencia, una
expansi\'{o}n de Fourier:\index{desarrollo de Fourier}
\begin{equation}
	\label{eq:expansiondefourier}
	f(z) \,=\,\sum_{\nu\in M^{\perp}}\,a_{\nu}(f)\,e^{2\pi i\traza(\nu z)}
	\text{ ,}
\end{equation}
%
donde $\traza(\nu z):=\nu_{1}z_{1}+\cdots+\nu_{n}z_{n}$ y la suma se realiza
sobre el ret\'{\i}culo dual de $M$ respecto de la forma traza.
\index{reticulo@ret\'{\i}culo!dual}
Nos referiremos a los coeficientes $a_{\nu}= a_{\nu}(f)$ como \emph{los %
coeficientes de Fourier de $f$}.\index{coeficientes de Fourier}
Diremos que $f$ es \emph{holomorfa en $\infty$}, si $a_{\nu}(f)\not =0$ implica
$\nu\gg 0$ o $\nu=0$ y que \emph{se anula en $\infty$}, si, adem\'{a}s,
$a_{0}(f)=0$.
\index{holomorfa en infinito@holomorfa en $\infty$}
\index{se anula en infinito@se anula en $\infty$}
\index{en infinito@en $\infty$!holomorfa}
\index{en infinito@en $\infty$!se anula}

\begin{obsReticuloDual}[El ret\'{\i}culo dual $M^{\perp}$]
	Recordemos que los elementos de $F$ se pueden ver como vectores en
	$\bb{R}^{n}$ v\'{\i}a $\mu\mapsto(\lista{\mu}{n})$. A trav\'{e}s de
	esta identificaci\'{o}n, los ideales de $F$ pasan a ser ret\'{\i}culos
	en $\bb{R}^{n}$ y, dados $\mu,\nu\in F$, la traza
	$\traza:\,F\rightarrow\bb{Q}$ del producto $\nu\mu$ est\'{a} dada por
	\begin{math}
		\traza(\nu\mu)=\nu_{1}\mu_{1}+\,\cdots\,+\nu_{n}\mu_{n}
	\end{math}~,
	es decir, el producto interno usual de los vectores correspondientes a
	$\nu$ y a $\mu$. El ret\'{\i}culo dual a $\frak{a}$ es entonces:
	\begin{align*}
		M^{\perp} & \,=\,
			\left\lbrace \nu\in F\,:\,\traza(\nu\mu)\in\bb{Z}\,
			\forall\mu\in M\right\rbrace
		\text{ .}
	\end{align*}
	%
	Si $M=\frak{a}$, entonces
	$\frak{a}^{\perp}=\frak{a}^{-1}\diferente^{-1}$, donde
	$\diferente=(\oka{F}^{*})^{-1}$ es el diferente (absoluto) de $F$.
\end{obsReticuloDual}

Hemos visto que el grupo de isotrop\'{\i}a de la c\'{u}spide en infinito consta
de dos partes: traslaciones $z\mapsto z+\mu$ por elementos $\mu$ de un
ret\'{\i}culo en $F$ y homotecias $z\mapsto \epsilon z$ por unidades totalmente
positivas. Hasta ahora, solamente hemos analizado las consecuencias de la
invarianza por traslaciones. La presencia de las unidades juega un papel
importante en la teor\'{\i}a de formas de Hilbert:

\begin{teoPrincipioDeKoecher}\label{thm:principiodekoecher}
	Sea $M$ un ret\'{\i}culo (completo) en $F$ y sea
	$V\subset\oka{F,+}^{\times}$ un subgrupo de unidades totalmente
	positivas, de rango m\'{a}ximo, tales que $\epsilon M=M$. Si
	$f:\,\hP^{n}\rightarrow\bb{C}$ es una funci\'{o}n holomorfa tal que
	$f\operadormatrices{\peso{k}}{\gamma}=f$ para toda matriz $\gamma$ de
	la forma
	\begin{align*}
		\gamma & \,=\,\begin{bmatrix} \epsilon & \mu \\
					& 1 \end{bmatrix}
	\end{align*}
	%
	con $\mu\in M$ y $\epsilon\in V$, entonces
	\begin{itemize}
		\item[(i)]
			\begin{math}
				a_{\nu\epsilon}(f) =\big(\prod_{i=1}^{n}\,
					\epsilon_{i}^{m_{i}+k_{i}-1}\big)\,
					a_{\nu}(f)
			\end{math}
			para todo $\nu\in M^{\perp}$ y $\epsilon\in V$;
		\item[(ii)]
			\begin{math}
				a_{\nu}(f)\not =0\Rightarrow\nu = 0\text{ o }
					\nu\gg 0
			\end{math}~.
	\end{itemize}
	%
\end{teoPrincipioDeKoecher}

Si tomamos
\begin{math}
	\gamma=\begin{bmatrix} \epsilon & \\ & 1\end{bmatrix}
\end{math}~,
entonces
\begin{math}
	f(z)=\sum_{\nu\in M^{\perp}}\,a_{\nu}e^{2\pi i\traza(\nu z)}
\end{math}
y
\begin{align*}
	\big(f\operadormatrices{\peso{k}}{\gamma}\big) (z) & \,=\,
		\bigg(\prod_{i=1}^{n}\,\epsilon_{i}^{m_{i}+k_{i}-1}\bigg)\,
		\sum_{\nu\in M^{\perp}}\,a_{\nu}e^{2\pi i\traza(\nu\epsilon z)}
	\text{ ,}
\end{align*}
%
de donde se deduce \textit{(i)}, comparando los coeficientes de Fourier
($\nu\epsilon\in M^{\perp}$). Para probar \textit{(ii)}, supongamos que
$\nu\in M^{\perp}$ es tal que $\nu_{1}<0$. Dado que $V$ es de rango $n-1$,
existe $\epsilon\in V$ con $0<\epsilon_{j}<1$ para $j\not =1$ y,
necesariamente, $\epsilon_{1}>1$. Evaluando en $\mathbf{i}$, como la serie de
Fourier de $f$ es absolutamente convergente, tambi\'{e}n lo es
\begin{math}
	\sum_{m\geq 1}\,a_{\nu\epsilon^{m}}\,
		e^{2\pi i\traza(\nu\epsilon^{m}\mathbf{i})}
\end{math}~.
Pero, por \textit{(i)},
\begin{align*}
	\sum_{m\geq 1}\,\big|a_{\nu\epsilon^{m}}\,
		e^{2\pi i\traza(\nu\epsilon^{m}\mathbf{i})}\big|
		& \,=\, \sum_{m\geq 1}\,|a_{\nu}|\,
		\Big(\prod_{i=1}^{n}\,\epsilon_{i}^{m(m_{i}+k_{i}-1)}\Big)\,
			e^{-2\pi\traza(\nu\epsilon^{m})}
	\text{ .}
\end{align*}
%
Fijada alguna constante $\delta>0$ tal que $\delta<|\nu_{1}|$, para $m$
suficientemente grande, se cumple
\begin{math}
	\traza(\nu\epsilon^{m})<\epsilon_{1}^{m}\,(\nu_{1}+\delta)
\end{math}~.
En particular, el t\'{e}rmino general de la serie diverge, a menos que
$a_{\nu}$ sea nulo.

\begin{obsPrincipioDeKoecher}
	\label{obs:principiodekoecher}
	Podemos expresar \textit{(ii)} de la siguiente manera: toda funci\'{o}n
	holomorfa $f:\,\hP^{n}\rightarrow\bb{C}$ ($n>1$) de peso $\peso{k}$
	invariante para $\Gamma$ es autom\'{a}ticamente holomorfa en $\infty$.
	Este es el denominado \emph{principio de Koecher}.
	\index{principio de Koecher}
\end{obsPrincipioDeKoecher}

\begin{defFormaModularDeHilbertParaCongruencia}%
	\label{def:formamodulardehilbertparacongruencia}
	Sea $\Gamma\subset\GLtp_{2}(F)$ un subgrupo de congruencia. Emulando la
	definici\'{o}n de forma modular el\'{\i}ptica, decimos que una
	funci\'{o}n $f:\,\hP^{n}\rightarrow\bb{C}$ es una
	\emph{forma modular de Hilbert de peso $\peso{k}$ para $\Gamma$}, si
	es holomorfa y satisface:
	\index{forma modular!de Hilbert}
	\begin{itemize}
		\item[(i)] $f$ es de peso $\peso{k}$ invariante para $\Gamma$ y
		\item[(ii)] $f\operadormatrices{\peso{k}}{A}$ es holomorfa en
			$\infty$ para toda $A\in\GLtp_{2}(F)$.
	\end{itemize}
	%
	Denotamos por $\modularH{k}{\Gamma}$ el espacio de formas modulares de
	peso $\peso{k}$ para $\Gamma$.
\end{defFormaModularDeHilbertParaCongruencia}

\begin{obsDefinicionFormasParaCongruencia}%
	\label{obs:definicionformasparacongruencia}
	La condici\'{o}n \textit{(ii)} de la definici\'{o}n anterior tiene
	sentido: si $\Gamma$ es de congruencia,
	$A^{-1}\,\Gamma\,A\conmensurable\Gamma$ para toda $A\in\GLtp_{2}(F)$ y
	la transformada $f\operadormatrices{\peso{k}}{A}$ admite un desarrollo
	de Fourier en $\infty$. Pero, por la misma raz\'{o}n, el principio de
	Koecher es v\'{a}lido para el conjugado y la condici\'{o}n
	\textit{(ii)} es redundante.
\end{obsDefinicionFormasParaCongruencia}

\begin{defFormaDeHilbertCuspidalParaCongruencia}%
	\label{def:formadehilbertcuspidalparacongruencia}
	Una forma modular $f$ de peso $\peso{k}$ invariante para $\Gamma$
	se dice \emph{forma cuspidal},\index{forma modular!de Hilbert!cuspidal}
	si $f\operadormatrices{\peso{k}}{A}$ se anula en $\infty$ para toda
	matriz $A\in\GLtp_{2}(F)$, es decir, el coeficiente
	$a_{0}\big(f\operadormatrices{\peso{k}}{A}\big)$ de la
	correspondiente expansi\'{o}n de Fourier es nulo. Las formas cuspidales
	constituyen un subespacio de $\modularH{k}{\Gamma}$, el cual ser\'{a}
	denotado por $\spitzH{k}{\Gamma}$.
\end{defFormaDeHilbertCuspidalParaCongruencia}

Si $\Gamma=\Gamma_{0}(\frak{N},\frak{a})$, escribimos
$\modularH{k}{\frak{N},\frak{a}}$ y $\spitzH{k}{\frak{N},\frak{a}}$ en lugar de
$\modularH{k}{\Gamma_{0}(\frak{N},\frak{a})}$ y de
$\spitzH{k}{\Gamma_{0}(\frak{N},\frak{a})}$, respectivamente.

\begin{obsCuandoNoEsParalelo}\label{obs:cuandonoesparalelo}
	Decimos que el peso $\peso{k}$ de una forma modular es
	\emph{paralelo},\index{peso paralelo} si $\peso{k}=(k,\,\dots,\,k)$
	para alg\'{u}n entero $k$. Del \'{\i}tem \textit{(i)} con $\nu=0$
	se deduce que, si el peso no es paralelo, entonces toda forma
	es cuspidal.
	% Si el peso es paralelo, debe ser
	% $\norma(\epsilon)^{k_{0}}=1$ para toda unidad de $\oka{F}$, a
	% menos que $a_{0}=0$, con lo cual, toda forma es cuspidal
	% tambi\'{e}n cuando el peso es paralelo e impar y el anillo de
	% enteros de $F$ contiene una unidad de norma negativa.
\end{obsCuandoNoEsParalelo}

\begin{teoLaDimEsFinita}\label{thm:ladimensionesfinita}
	El espacio de formas modulares $\modularH{k}{\Gamma}$ es de
	dimensi\'{o}n finita.
\end{teoLaDimEsFinita}

\begin{proof}[Demostraci\'{o}n]
	Ver \cite{BruinierFormsAndApplications},
	\cite[\S~I.6 y Ch.~II]{FreitagForms} o
	\cite{vanDerGeerSurfaces}.
\end{proof}

% \begin{propoDeCongruenciaReticuloYUnidades}%
	% \label{propo:decongruenciareticuloyunidades}
	% Sea $\Gamma\subset\GLtp_{2}(F)$ un subgrupo de congruencia. Existe un
	% ret\'{\i}culo $M\subset F$ y un subgrupo $V\subset\oka{F,+}^{\times}$
	% de unidades totalmente positivas que preservan $M$ de rango m\'{a}ximo,
	% tales que $G(M,V)\subset\Gamma$.
% \end{propoDeCongruenciaReticuloYUnidades}
% 
% \begin{proof}
	% Sea $\Gamma'=\Gamma_{0}(\frak{N},\frak{a})\cap\Gamma$. Consideramos el
	% subgrupo de matrices triangulares superiores
	% \begin{align*}
		% \Delta & \,=\,\bigg\{\begin{bmatrix} 1 & \mu \\
				% & 1 \end{bmatrix}\,:\,\mu\in F\bigg\}
			% \,\subset\,\GLtp_{2}(F)
		% \text{ .}
	% \end{align*}
	% %
	% El subgrupo de traslaciones de $\Gamma_{0}(\frak{N},\frak{a})$ est\'{a}
	% representado por el subgrupo de matrices triangulares superiores con
	% $\mu\in\frak{a}$. Abusando de la notaci\'{o}n, denotaremos,
	% temporariamente, $\frak{a}$ a este subgrupo. Notamos que
	% $\frak{a}=\Gamma_{0}(\frak{N},\frak{a})\cap\Delta$. Sean
	% $M=\Gamma\cap\Delta$ y $M'=\Gamma'\cap\Delta$. Al igual que $\frak{a}$,
	% los grupos $M$ y $M'$ representan traslaciones en $\Gamma$ y en
	% $\Gamma'$, respectivamente, y son subgrupos aditivos de $F$ (de
	% $\bb{R}^{n}$). Por otro lado, dado que
	% $\big|\Gamma_{0}(\frak{N},\frak{a}):\Gamma'\big|$ es finito,
	% $\big|\frak{a}:\Gamma'\cap\frak{a}\big|$ es finito, tambi\'{e}n. Pero
	% $\frak{a}$ es un grupo abeliano libre y $\Gamma'\cap\frak{a}=M'$, con
	% lo que $M'$ tambi\'{e}n es un grupo abeliano libre y del mismo rango
	% que $\frak{a}$. An\'{a}logamente, $\big|\Gamma:\Gamma'\big|<\infty$
	% implica $\big|M:\Gamma'\cap M\big|<\infty$, $\Gamma'\cap M=M'$ y $M$ es
	% un grupo abeliano libre del mismo rango que $M'$ (y, en particular, que
	% $\frak{a}$). En definitiva, $M$ es un ret\'{\i}culo completo en $F$.
% 
	% El grupo $\Gamma_{0}(\frak{N},\frak{a})$ contiene un subgrupo isomorfo
	% a $\oka{F,+}^{\times}$:
	% \begin{align*}
		% U & \,=\,\bigg\{\begin{bmatrix} \epsilon & \\
			% & 1 \end{bmatrix}\,:\,\epsilon\in\oka{F,+}^{\times}
			% \bigg\}\,\subset\,\Gamma_{0}(\frak{N},\frak{a})
		% \text{ .}
	% \end{align*}
	% %
	% Manteniendo la notaci\'{o}n del p\'{a}rrafo anterior, definimos
	% $V=\Gamma\cap U$ y $V'=\Gamma'\cap U$. La relaci\'{o}n entre los grupos
	% $\frak{a}$, $M$ y $M'$ es similar a la relaci\'{o}n entre $U$, $V$ y
	% $V'$: el hecho de que
	% $\Gamma\conmensurable\Gamma_{0}(\frak{N},\frak{a})$ implica que los
	% \'{\i}ndices $|U:V'|$ y $|V:V'|$ sean finitos. En particular, como $U$
	% es un grupo abeliano de rango $n-1$, tambi\'{e}n lo son $V'$ y, por lo
	% tanto, $V$. Es decir, $V$ se identifica con un subgrupo de unidades
	% totalmente positivas de rango m\'{a}ximo. Por otro lado, la
	% relaci\'{o}n
	% \begin{align*}
		% \begin{bmatrix} \lambda & \\ & 1 \end{bmatrix}
			% \begin{bmatrix} 1 & \mu \\ & 1 \end{bmatrix}
			% \begin{bmatrix} \lambda^{-1} & \\ & 1 \end{bmatrix}
		% & \,=\,\begin{bmatrix} 1 & \lambda\mu \\ & 1 \end{bmatrix}
	% \end{align*}
	% %
	% muestra que $U$ se identifica con el grupo de elementos totalmente
	% positivos $\lambda\in F_{+}$ tales que $\lambda\frak{a}=\frak{a}$.
	% Notemos que cualquier elemento con esta propiedad debe ser una unidad
	% y, si la matriz correspondiente ha de pertenecer a $\GLtp_{2}(F)$,
	% entonces debe ser una unidad totalmente positiva. El grupo $V$ se puede
	% describir, de manera an\'{a}loga, como un subgrupo de unidades
	% totalmente positivas de rango $n-1$ que preserva el ret\'{\i}culo $M$
	% (pero $V$ podr\'{\i}a ser un subgrupo propio del grupo de todas las
	% unidades totalmente positivas que preservan $M$). Dado que
	% $M=\Gamma\cap\Delta$, si un elemento de $\mu\in M$ se conjuga por un
	% elemento $\epsilon\in V$, se obtiene un matriz en $\Gamma$ que,
	% adem\'{a}s, pertenece a $\Delta$: espec\'{\i}ficamente, se obtiene la
	% matriz correspondiente a la traslaci\'{o}n por $\epsilon\mu$ (conjugar
	% las matrices se corresponde con aplicar una homotecia, con lo que la
	% matriz que se obtiene es triangular, con $1$ en la diagonal y
	% $\epsilon\mu$ en la esquina superior derecha). Por definici\'{o}n,
	% $\epsilon\mu\in M$.
% \end{proof}
