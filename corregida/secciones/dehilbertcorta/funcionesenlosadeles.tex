A continuaci\'{o}n enunciamos la correspondencia entre formas modulares de
Hilbert y formas automorfas en el grupo $\GL_{2}(\adeles{F})$. Su
demostraci\'{o}n es an\'{a}loga a la de la Proposici\'{o}n~%
\ref{propo:introequivformaglq}.
% (ver \cite[\S~3]{GelbartAutomorphicOnAdeles}).
Para cada lugar arquimediano $v\in\lugares[\infty]{F}$, definimos
$K_{v}=\SO{2}$. El producto $K_{\infty}=\prod_{v\in\lugares[\infty]{F}}\,K_{v}$
es el subgrupo compacto y conexo maximal del estabilizador
del punto $\mathbf{i}=(\sqrt{-1},\,\dots,\,\sqrt{-1})\in(\hP^{\pm})^{n}$ por la
acci\'{o}n de $\GL_{2}(\bb{R})^{n}$.

\begin{propoEquivalenciaAutomorfasFormasCuspidales}
	\label{propo:equivalenciaautomorfasformascuspidales}
	La aplicaci\'{o}n $f\mapsto\phi_{f}$ que a una forma de Hilbert
	cuspidal le asigna la funci\'{o}n
	\begin{equation}
		\label{eq:funcionadelicacorrespondientedehilbert}
		\phi_{f}(g_{\infty},\hhat{\alpha}) \,=\,
		\bigg(\prod_{i=1}^{n}\,J_{i}(g_{i},\sqrt{-1})^{-1}\bigg)\,
		f(g_{\infty}\cdot\mathbf{i},\hhat{\alpha}\Idfin{\cal{O}})
	\end{equation}
	%
	determina una correspondencia entre el espacio $\spitzH{k}{\frak{N}}$
	y el espacio de funciones $\phi:\,\GL_{2}(\adeles{F})%
	\rightarrow\bb{C}$ que cumplen con:
	\begin{itemize}
		\item[(i)] $\phi(\gamma g)=\phi(g)$ para toda
			$\gamma\in\GL_{2}(F)$;
		\item[(ii)] $\phi(g\hhat{\beta})=\phi(g)$ si
			\begin{math}
				\hhat{\beta}\in\Idfin{\cal{O}}=
					\prod_{\frak{p}}\,
					\cal{O}_{\frak{p}}^{\times}
			\end{math}~;
		\item[(iii)]
			\begin{math}
				\phi(gh)=\prod_{i=1}^{n}\,
					e^{\sqrt{-1}k_{i}\theta_{i}}\,\phi(g)
			\end{math}
			si
			\begin{math}
				h=(\lista[\theta_{1}]{r}{\theta_{n}})
					\in K_{\infty}
			\end{math}~;
		\item[(iv)]
			\begin{math}
				\phi(\tau_{\infty}g)=\prod_{i=1}^{n}\,
				\tau_{i}^{k_{0}-2}\phi(g)
			\end{math}
			si
			\begin{math}
				\tau_{\infty}=\bigg(
				\left[\begin{matrix} \tau_{1} & \\
					& \tau_{1} \end{matrix}\right],
					\,\dots,\,
				\left[\begin{matrix} \tau_{n} & \\
					& \tau_{n}\end{matrix}\right]
				\bigg)\in\centre(\bb{R})
			\end{math}~;
			% no hace falta el valor absoluto, porque los k_{i}
			% son todos de igual paridad y no hay problema con las
			% potencias.
		% \item[v.b] $\phi(\hhat{\eta}g)=\prod_{i=1}^{n}\,%
			% t_{i}^{-(k_{0}-2)}%
			% \phi(g\hhat{a})$ si $\hhat{\eta}=t\hhat{a}\hhat{r}\in%
			% F_{+}^{\times}\hhat{a}\Idfin{\oka{F}}\subset%
			% \centre(\Idfin{F})$;
		\item[(v)] $\phi$ es de \emph{crecimiento moderado}: si
			\index{crecimiento moderado}
			$\Omega\subset\GL_{2}(\adeles{F})$ es un subconjunto
			compacto y $c>0$, existen constantes $C$ y $N$ tales
			que
			\begin{align*}
				\left|\phi\bigg(
				\begin{bmatrix} a & \\ & 1 \end{bmatrix}
					g\bigg)\right| & \,\leq\,
					C\left|a\right|^{N}
			\end{align*}
			%
			para toda $g\in\Omega$ y $a\in\ideles{F}$ con $|a|>c$;
			y
		\item[(vi)] como funci\'{o}n de $\GLtp_{2}(\bb{R})^{n}$,
			$\phi$ es $C^{\infty}$ y verifica la ecuaci\'{o}n
			diferencial
			\begin{align*}
				\Delta_{i}\phi\,=\, & -\frac{k_{i}}{2}
				\left(\frac{k_{i}}{2}\,-\,1\right)\phi
				\text{ ;}
			\end{align*}
		\item[(vii)] $\phi$ es \emph{cuspidal}, es decir,
			\index{forma automorfa!cuspidal}
			\begin{align*}
				\int_{F\backslash\adeles{F}}\,
				\phi\bigg(
				\begin{bmatrix} 1 & x \\ & 1 \end{bmatrix}
					g\bigg)\,dx & \,=\, 0
			\end{align*}
			%
			para casi todo $g$.
	\end{itemize}
\end{propoEquivalenciaAutomorfasFormasCuspidales}

% \begin{proof}
	% Los puntos \textit{(i)}, \textit{(ii)}, \textit{(iii)} y \textit{(v)}
	% son consecuencia de la invarianza de una forma $f$ respecto de la
	% acci\'{o}n de $\GL_{2}(F)$. Los puntos restantes se demuestran de
	% manera similar al caso $F=\bb{Q}$. Para ver \textit{(iv)}, sea
	% $f\in\spitzH{k}{\frak{N}}$ y sea $\phi_{f}$ la funci\'{o}n ad\'{e}lica
	% correspondiente. Para ver que satisface \emph{(iv)}, sea
	% $g_{\infty}=(\lista{g}{n})\in\GLtp_{2}(\bb{R})^{n}$ y sea
	% $\hhat{\alpha}\in\GL_{2}(\Adfin{F})$ arbitrario. Cada matriz $g_{i}$ se
	% puede escribir como
	% \begin{equation}
		% \label{eq:iwasawa}
		% g_{i} \,=\,
		% \begin{bmatrix} u_{i} & \\ & u_{i} \end{bmatrix}
		% \begin{bmatrix} y^{1/2} & xy^{-1/2} \\ & y^{-1/2} \end{bmatrix}
		% \begin{bmatrix} \cos(\theta_{i}) & \sin(\theta_{i}) \\
			% -\sin(\theta_{i}) & \cos(\theta_{i}) \end{bmatrix}
		% \text{ .}
	% \end{equation}
	% %
	% Entonces, llamando $s=k_{0}-2$,
	% \begin{align*}
		% \phi_{f}(g_{\infty},\hhat{\alpha}) & \,=\,
			% \bigg(\prod_{i=1}^{n}\,
			% \frac{\det(g_{i})^{m_{i}+k_{i}-1}}{%
				% j(g_{i},\sqrt{-1})^{k_{i}}}\bigg)\,
			% f(g_{\infty}\cdot\mathbf{i},
				% \hhat{\alpha}\Idfin{\cal{O}}) \\
		% & \,=\,\bigg(\prod_{i=1}^{n}\,\det(g_{i})^{s/2}\bigg)\,
			% \bigg(\prod_{i=1}^{n}\,
			% \frac{\det(g_{i})^{k_{i}/2}}{%
				% j(g_{i},\sqrt{-1})^{k_{i}}}\bigg)\,
			% f((x_{i}+\sqrt{-1}y_{i})_{i},
				% \hhat{\alpha}\Idfin{\cal{O}}) \\
		% & \,=\, \bigg(\prod_{i=1}^{n}\,u_{i}^{s}\bigg)\,
			% \bigg(\prod_{i=1}^{n}\,y_{i}^{k_{i}/2}\,
				% e^{\sqrt{-1}k_{i}\theta_{i}}\bigg)\,
			% f((x_{i}+\sqrt{-1}y_{i})_{i},
				% \hhat{\alpha}\Idfin{\cal{O}})
		% \text{ .}
	% \end{align*}
	% %
	% El operador de Casimir $\Delta_{j}$ est\'{a} dado por
	% \begin{align*}
		% \Delta_{j} & \,=\,
			% -y_{j}^{2}\left(\frac{\partial^{2}}{\partial x_{j}^{2}}
			% \,+\,\frac{\partial^{2}}{\partial y_{j}^{2}}\right)
			% \,-\, y_{j}
			% \frac{\partial^{2}}{\partial x_{j}\partial\theta_{j}}
		% \text{ .}
	% \end{align*}
	% %
	% En la introducci\'{o}n, mencionamos que el factor de automorf\'{\i}a
	% suele estar definido de una manera levemente distinta (la potencia en
	% el determinante de la matriz involucrada suele ser $k/2$, en lugar de
	% algo de la forma $m+k-1$, como en nuestra definici\'{o}n). Esta
	% diferencia tiene como consecuencia la aparici\'{o}n de un factor
	% correspondiente a la parte central en la descomposici\'{o}n
	% \eqref{eq:descomposiciondegldos} de $\GLtp_{2}(\bb{R})$,
	% $\prod_{i=1}^{n}\,u_{i}^{s}$, que no estar\'{\i}a presente, si
	% hubi\'{e}semos definido $\phi_{f}$ como en
	% \eqref{eq:funcionadelicacorrespondienteelipticas}. Aun as\'{\i}, como
	% se hizo en la introducci\'{o}n, se deduce que $\phi_{f}$ es una
	% autofunci\'{o}n para $\Delta_{j}$ con autovalor
	% $-\frac{k_{j}}{2}\Big(\frac{k_{j}}{2}-1\Big)$.
% \end{proof}
%
% \paragraph{Formas automorfas}
% \input{./secciones/comentarioFormasAutomorfas.tex}

Teniendo en cuenta la correspondencia entre formas cuspidales y formas
automorfas, si $f\in\spitzH{k}{\frak{N}}$, llamaremos tambi\'{e}n forma de
Hilbert cuspidal a la correspondiente funci\'{o}n $\phi_{f}$ dada por
\eqref{eq:funcionadelicacorrespondientedehilbert}.

\subsection{Funciones de cuadrado integrable}%
	\label{subsec:enlosadelescuadradointegrable}
Usando el lenguaje ad\'{e}lico, definimos un producto interno
$\langle f,g\rangle$ para formas cuspidales. Fijamos un sistema de
representantes $\{\frak{a}\}$ del grupo de clases estrictas de $F$ y denotamos
por $\hhat{\alpha}$ las matrices dadas por
\eqref{eq:matrizasociadaaidelematrices}.

Si bien, puede parecer natural definir un producto interno usando el
isomorfismo \eqref{eq:descomposicioncuspidalesmatrices} y la expresi\'{o}n
\eqref{eq:productodepeterssongamacero} para los grupos
$\Gamma_{0}(\frak{N},\frak{a})$, una definici\'{o}n as\'{\i} depender\'{\i}a
del sistema de representantes de $\pClass{F}$ que parametriza las componentes
de la variedad $Y_{0}(\frak{N})$. Para llegar a una definici\'{o}n adecuada,
observamos, en primer lugar, que dicha parametrizaci\'{o}n est\'{a} relacionada
con una descomposici\'{o}n similar del grupo $\GL_{2}(\adeles{F})$. Por
aproximaci\'{o}n fuerte (ver tambi\'{e}n
\S~\ref{sec:cuaternionicasvariedadesdeshimura}),
\begin{align*}
	\GL_{2}(\adeles{F}) & \,=\,\bigsqcup_{\frak{a}}\,\centre(\adeles{F})\,
		\GL_{2}(F)\,\GLtp_{2}(\bb{R})^{n}\,\hhat{\alpha}\Idfin{\cal{O}}
	\text{ .}
\end{align*}
%
Entonces hay una identificaci\'{o}n
\begin{equation}
	\label{eq:variedadmodularycocienteadelico}
	\centre(\adeles{F})\,\GL_{2}(F)\backslash\GL_{2}(\adeles{F})/
		K_{\infty}\Idfin{\cal{O}} \,\simeq\,\bigsqcup_{\frak{a}}\,
			\Gamma_{0}(\frak{N},\frak{a})\backslash\hP^{n}
		\,=\,Y_{0}(\frak{N})
	\text{ .}
\end{equation}
%
En segundo lugar, para cada representante $\frak{a}$, si
$f,g\in\spitzH{k}{\frak{N}}$ el producto interno
$\langle f_{\frak{a}},g_{\frak{a}}\rangle_{\frak{a}}$ se define en t\'{e}rminos
de una integral sobre la componente correspondiente de la variedad
$Y_{0}(\frak{N})$. Del lado ad\'{e}lico, teniendo en cuenta la
identificaci\'{o}n \eqref{eq:variedadmodularycocienteadelico}, debemos poder
integrar sobre el cociente
\begin{math}
	\centre(\adeles{F})\,\GL_{2}(F)\backslash\GL_{2}(\adeles{F})
\end{math}~.
El problema es que, si bien las funciones $\phi_{f}$ y $\phi_{g}$ son
invariantes por $\GL_{2}(F)$, no es obvio c\'{o}mo obtener a partir de ellas un
integrando definido m\'{o}dulo el centro $\centre(\adeles{F})$.

Sea $f\in\spitzH{k}{\frak{N}}$ y sea $\phi=\phi_{f}$ la funci\'{o}n definida
por \eqref{eq:funcionadelicacorrespondientedehilbert}. Seg\'{u}n el \'{\i}tem
(iv) de la Proposici\'{o}n~\ref{propo:equivalenciaautomorfasformascuspidales},
sabemos que
\begin{math}
	\phi(\tau_{\infty}g)=\Big(\prod_{i=1}^{n}\,\tau_{i}^{k_{0}-2}\Big)\,
		\phi(g)
\end{math}~, si $\tau_{\infty}\in\centre(\bb{R})=(\bb{R}^{\times})^{n}$. Pero
no hemos dicho nada acerca de c\'{o}mo afecta multiplicar por un elemento
central de la parte no arquimediana. Para poder describir esta acci\'{o}n,
definimos, para cada $\tau\in\centre(\adeles{F})$ y cada $\phi$, una
funci\'{o}n $\rho(\tau)\phi$ por
\begin{align*}
	\big(\rho(\tau)\phi\big)(g) & \,=\,|\tau|_{\adeles{F}}^{-(k_{0}-2)}\,
		\phi(\tau g)
	\text{ ,}
\end{align*}
%
donde $|\tau|_{\adeles{F}}$ denota el valor absoluto del ad\`{e}le $\tau$.%
\footnote{
	Recordemos que este valor absoluto est\'{a} dado por
	$|\tau|_{\adeles{F}}=\prod_{v}\,|\tau_{v}|_{v}$, donde $|\cdot|_{v}$
	denota el valor absoluto en la completaci\'{o}n $F_{v}$.
}

Sea $\{\hhat{a}\}\subset\Idfin{F}$ un sistema de representantes de las clases
en $\pClass{F}$ y sea $\hhat{\tau}\in\Idfin{F}$. Existen
$\lambda\in F_{+}^{\times}$, $\hhat{r}\in\Idfin{\oka{F}}$ y un representante
$\hhat{a}$ tales que $\hhat{\tau}=\lambda_{0}\hhat{a}\hhat{r}$, donde
$\lambda_{0}$ denota la parte finita de $\lambda$. La parte arquimediana la
denotamos $\lambda_{\infty}$. Entonces, dado que
\begin{math}
	\Idfin{\oka{F}}\subset\Idfin{\cal{O}}\cap\centre(\adeles{F})
\end{math}~, $\phi(\lambda g)=\phi(g)=\phi(g\hhat{r})$,
$|\lambda|_{\adeles{F}}=1=|\hhat{r}|_{\adeles{F}}$ y $\lambda\gg 0$, podemos
deducir que
\begin{align*}
	\big(\rho(\hhat{\tau})\phi\big)(g) & \,=\,
		|\lambda_{0}\hhat{a}\hhat{r}|_{\adeles{F}}^{-(k_{0}-2)}\,
			\phi(\lambda_{0}\hhat{a}\hhat{r}g) \,=\,
		|\lambda_{\infty}|_{\adeles{F}}^{k_{0}-2}
			|\hhat{a}|_{\adeles{F}}^{-(k_{0}-2)}\,
			\phi(\lambda\lambda_{\infty}^{-1}\hhat{a}g\hhat{r}) \\
	& \,=\,|\norma(\lambda)|^{k_{0}-2}
		|\hhat{a}|_{\adeles{F}}^{-(k_{0}-2)}\,
			\phi(\lambda_{\infty}^{-1}\hhat{a}g) \,=\,
		\signo(\norma(\lambda))^{-(k_{0}-2)}
		|\hhat{a}|_{\adeles{F}}^{-(k_{0}-2)}\,\phi(\hhat{a}g) \\
	& \,=\,\big(\rho(\hhat{a})\phi\big)(g)
	\text{ .}
\end{align*}
%
Si llamamos
\begin{math}
	\chi_{\infty}(\tau_{\infty})=\prod_{i=1}^{n}\,
		\signo(\tau_{i})^{k_{0}-2}
\end{math}~, entonces, dado $\tau=(\tau_{\infty},\hhat{\tau})$, vale que
\begin{math}
	\rho(\tau)=\chi_{\infty}(\tau_{\infty})\rho(\hhat{\tau})
\end{math}~. De lo anterior, se deduce que la restricci\'{o}n de $\rho$ a la
parte finita $\Idfin{F}$ es, en otras palabras, una representaci\'{o}n del
grupo abeliano finito $\pClass{F}$. Podemos concluir que $\spitzH{k}{\frak{N}}$
se descompone como suma directa de subrepresentaciones irreducibles, cada una
de grado $1$:%
\footnote{Ver, por ejemplo, \cite{Etingof} o \cite{SerreScott}}
\begin{equation}
	\label{eq:descomposicioncuspidalescuasicaracteres}
	\spitzH{k}{\frak{N}} \,=\,\bigoplus_{\omega}\,
		\spitzH{k}{\frak{N},\omega}
	\text{ ,}
\end{equation}
%
donde
\begin{math}
	\omega:\,\ideles{F}\rightarrow\bb{C}^{\times}
\end{math} es el cuasicar\'{a}cter
\begin{math}
	\omega=|\cdot|_{\adeles{F}}^{k_{0}-2}\,\chi_{\infty}\chi_{0}
\end{math}~, $\chi_{0}$ recorre los caracteres del grupo $\pClass{F}$ y
\begin{align*}
	\spitzH{k}{\frak{N},\omega} & \,=\,\Big\{\phi\in\spitzH{k}{\frak{N}}
		\,:\,\phi(\tau g)=\omega(\tau)\cdot\phi(g)
			\text{ , si }\tau\in\centre(\adeles{F})
		\Big\}
	\text{ .}
\end{align*}
%
Notamos que
\begin{math}
	|\omega|=|\cdot|_{\adeles{F}}^{k_{0}-2}
\end{math} y que
\begin{math}
	|\omega|^{-1}\omega=\chi_{\infty}\chi_{0}:\,
		\ideles{F}\rightarrow S^1
\end{math} es un car\'{a}cter de $\ideles{F}$ trivial en $F^{\times}$ y en
$\Idfin{\oka{F}}$. En general, escribiremos $\chi$ para denotar dicho
car\'{a}cter. Si $\hhat{\tau}\in\Idfin{F}$ y
\begin{math}
	\frak{t}=\hhat{\tau}\Adfin{\oka{F}}\,\cap\,F
\end{math} es el ideal fraccionario determinado por $\hhat{\tau}$, podemos
definir
\begin{align*}
	\chi(\frak{t}) \,:=\,\chi(\hhat{\tau})
		& \quad\text{y}\quad\omega(\frak{t})\,:=\,\omega(\hhat{\tau})
	\text{ .}
\end{align*}
%
El valor de $\chi(\frak{t})$ depende \'{u}nicamente de la clase de $\frak{t}$
en $\pClass{F}$ y
\begin{math}
	|\omega(\frak{t})|=\idnorm(\frak{t})^{-(k_{0}-2)}
\end{math}~, donde $\idnorm(\frak{t})$ denota la norma del ideal.
%%
% \{\{\{
% Observaci\'{o}n al margen. Si $f=(f_{\frak{a}})_{\frak{a}}$ es una forma
% modular,
% \begin{align*}
	% \rho(\hhat{\tau})f(z,\hhat{\alpha}\Idfin{\cal{O}}) & \,:=\,
		% f(z,\hhat{\tau}\hhat{\alpha}\Idfin{\cal{O}})
	% \text{ .}
% \end{align*}
% %
% Entonces $\rho(\hhat{\tau})f$ tambi\'{e}n es una forma modular, del mismo peso
% que $f$. Sea $\frak{t}\subset F$ el ideal correspondiente a $\hhat{\tau}$.
% Tomando norma reducida, el ideal fraccionario determinado por
% $\nrd(\hhat{\tau}\hhat{\alpha}\Idfin{\cal{O}})$ es $\frak{t}^{2}\frak{a}$. Sea
% $\frak{b}$ el \'{u}nico representante de las clases estrictas tal que
% $[\frak{t}^{2}\frak{a}]=[\frak{b}]$. Por aproximaci\'{o}n fuerte, existe
% $\gamma\in\GLtp_{2}(F)$ tal que
% $\gamma\hhat{\tau}\hhat{\alpha}\Idfin{\cal{O}}=\hhat{\beta}\Idfin{\cal{O}}$ y,
% por lo tanto,
% \begin{align*}
	% \big(\rho(\hhat{\tau})f\big)_{\frak{a}}(z) & \,=\,
		% f(z,\gamma^{-1}\hhat{\beta}\Idfin{\cal{O}})\,=\,
		% f_{\frak{b}}\operadormatrices{\peso{k}}{\gamma}
	% \text{ .}
% \end{align*}
% %
% La relaci\'{o}n entre los representantes del grupo de clases estrictas est\'{a}
% dada por
% \begin{align*}
	% \lambda\frak{t}^{2}\frak{a} & \,=\,\frak{b}
	% \text{ ,}
% \end{align*}
% %
% para alg\'{u}n elemento totalmente positivo $\lambda\gg 0$. La
% representaci\'{o}n $\rho$ permuta las componentes de $f$ correspondientes a
% ideales pertenecientes al mismo ``g\'{e}nero''.
% \}\}\}
%%

Dado un cuasicar\'{a}cter $\omega:\,\ideles{F}\rightarrow\bb{C}^{\times}$
(trivial en $F^{\times}$), escribimos $f\in\spitzH{k}{\frak{N},\omega}$, si
$f\in\spitzH{k}{\frak{N}}$ es tal que $\phi_{f}\in\spitzH{k}{\frak{N},\omega}$.
Denotamos por $\cuadradointegrables{\omega}$ el espacio de funciones medibles
$\phi:\,\GL_{2}(\adeles{F})\rightarrow\bb{C}^{\times}$ tales que:%
\footnote{Comparar con \cite[Ch.~3]{Bump}}
\begin{itemize}
	\item[(i)] $\phi(\tau g)=\omega(\tau)\cdot\phi(g)$ para todo
		$\tau\in\centre(\adeles{F})$,
	\item[(ii)] $\phi(\gamma g)=\phi(g)$ para toda $\gamma\in\GL_{2}(F)$ y
	\item[(iii)] $\phi$ es de \emph{cuadrado integrable m\'{o}dulo el %
		centro}, es decir,
		\begin{align*}
			\int_{\centre(\adeles{F})\,\GL_{2}(F)\backslash%
				\GL_{2}(\adeles{F})}\,|\omega(\det\,g)|^{-1}
				|\phi(g)|^{2}\,dg & \,<\,\infty
			\text{ .}
		\end{align*}
		%
\end{itemize}
%
Si $\omega=\psi\cdot\chi$ y si $f\in\spitzH{k}{\frak{N},\omega}$, entonces
$\phi=\phi_{f}$ pertenece a este espacio; lo \'{u}nico que hay que verificar es
la \'{u}ltima condici\'{o}n. Para ver esto, sean $f_{\frak{a}}$ las componentes
dadas por el isomorfismo \eqref{eq:descomposicioncuspidalesmatrices}. El
integrando $|\omega(\det(g))|^{-1}|\phi(g)|^{2}$ es invariante por la
acci\'{o}n del centro, es decir, por $\rho$, con lo cual, la integral anterior
est\'{a} bien definida. Llamemos temporariamente
$\tilde{Y}_{\hhat{\alpha}}(\Idfin{\cal{O}})$ a la componente
\begin{align*}
	\tilde{Y}_{\hhat{\alpha}}(\Idfin{\cal{O}}) & \,=\,
		\centre(\adeles{F})\,\GL_{2}(F)\,\GLtp_{2}(\bb{R})^{n}\,
			\hhat{\alpha}\Idfin{\cal{O}}
\end{align*}
%
de $\GL_{2}(\adeles{F})$. Con $g_{\infty}\in\GLtp_{2}(\bb{R})^{n}$, $\hhat{u}$
variando en $\Idfin{\cal{O}}$, $d\theta$ y $d\hhat{u}$ las medidas que cumplen
$d\theta(\SO{2}^{n})=1$ y $d\hhat{u}(\Idfin{\cal{O}})=1$ y usando la
descomposici\'{o}n de Iwasawa \eqref{eq:descomposiciondegldos},
\begin{align*}
	& \int_{\centre(\adeles{F})\,\GL_{2}(F)\backslash%
		\tilde{Y}_{\hhat{\alpha}}(\Idfin{\cal{O}})}\,
		|\omega(\det(g_{\infty}\hhat{\alpha}\hhat{u}))|^{-1}
		|\phi(g_{\infty},\hhat{\alpha}\hhat{u})|^{2}\,
		dg_{\infty}\,d\hhat{u} \\
	& \qquad\,=\, \frac{1}{|\omega(\frak{a})|}\,
		\int_{\centre(\adeles{F})\,\GL_{2}(F)\backslash%
		\tilde{Y}_{\hhat{\alpha}}(\Idfin{\cal{O}})}\,
			\Big|\phi\Big(
			\left[\begin{smallmatrix}
				y^{1/2} & xy^{-1/2} \\
				& y^{-1/2}
			\end{smallmatrix}\right]\,h(\theta),
			\hhat{\alpha}\hhat{u}
			\Big)\Big|^{2}
		d\mu\,d\theta\,d\hhat{u} \\[5pt]
	& \qquad\,=\,\idnorm(\frak{a})^{k_{0}-2}\,
		\int_{\Gamma_{0}(\frak{N},\frak{a})\backslash%
		\GLtp_{2}(\bb{R})^{n}/\centre(\bb{R})\,\SO{2}^{n}}\,
		\big|f_{\frak{a}}(x+y\mathbf{i})\big|^{2}\,y^{\peso{k}}\,d\mu
	\text{ .}
\end{align*}
%
Pero este \'{u}ltimo t\'{e}rmino es finito, porque la forma cuspidal
$f_{\frak{a}}$ es de cuadrado integrable. Esto demuestra que $\phi_{f}$
pertenece al espacio $\cuadradointegrables{\omega}$. En realidad,
$\phi\in\cuadradocuspidales{\omega}$, el subespacio de funciones
\emph{cuspidales}.%
\footnote{
	Ver (vii) de la Proposici\'{o}n~%
	\ref{propo:equivalenciaautomorfasformascuspidales}
}
El producto interno en este espacio est\'{a} dado por
\begin{equation}
	\label{eq:peterssonenadelesmatrices}
	\langle \phi,\phi'\rangle \,:=\,
	\int_{\centre(\adeles{F})\,\GL_{2}(F)\backslash\GL_{2}(\adeles{F})}\,
		|\omega(\det\,g)|^{-1}\,\phi(g)\lconj{\phi'(g)}\,dg
	\text{ ,}
\end{equation}
%

Finalmente, observamos que
\begin{math}
	\phi(g_{\infty},\hhat{g}) =\sum_{\frak{a}}\,
		\phi(g_{\infty},\hhat{g})\,
		[\hhat{\alpha}\Idfin{\cal{O}}](\hhat{g})
\end{math}~,
donde $[\hhat{\alpha}\Idfin{\cal{O}}]$ denota la funci\'{o}n
caracter\'{\i}stica del conjunto $\hhat{\alpha}\Idfin{\cal{O}}$. De esto
deducimos que, dadas $f,g\in\spitzH{k}{\frak{N},\omega}$, se cumple
\begin{align*}
	\langle f,g\rangle & \,\equiv\,\langle\phi_{f},\phi_{g}\rangle\,=\,
		\sum_{\frak{a}}\,\idnorm(\frak{a})^{k_{0}-2}\,
		\langle f_{\frak{a}},g_{\frak{a}}\rangle_{\frak{a}}
	\text{ .}
\end{align*}
%

\subsection{Una acci\'{o}n a derecha}%
	\label{subsec:operadoresdepesokadelesmatrices}
Ser\'{a} \'{u}til, tambi\'{e}n, traducir y extender la definici\'{o}n de los
operadores de peso $\peso{k}$ de $\modularH{k}{\frak{N}}$ a funciones
ad\'{e}licas. Dada $\phi:\,\GL_{2}(\adeles{F})\rightarrow\bb{C}$ y
$(h,\hhat{\beta})\in\GL_{2}(\adeles{F})$, con $h\in\GL_{2}(\bb{R})^{n}$ y
$\hhat{\beta}\in\GL_{2}(\Adfin{F})$, definimos
\begin{equation}
	\label{eq:operadordepesokenadelesmatrices}
	\big(\phi\operadormatrices{\peso{k}}{(h,\hhat{\beta})}\big)
		(g_{\infty},\hhat{\alpha}) \,=\,
			% \big|\omega(\det(h\hhat{\beta}))\big|^{1/2}\,
		\bigg(\prod_{i=1}^{n}\,J_{i}(h_{i},\sqrt{-1})^{-1}\bigg)\,
		\phi(g_{\infty}h^{-1},\hhat{\alpha}\hhat{\beta}^{-1})
	\text{ .}
\end{equation}
%
Esta expresi\'{o}n define una acci\'{o}n del grupo $\GL_{2}(\Adfin{F})$. La
propiedad (ii) en la Proposici\'{o}n~%
\ref{propo:equivalenciaautomorfasformascuspidales} se puede expresar entonces
\begin{equation}
	\label{eq:matricesinvariantedepesokadelicas}
	\phi\operadormatrices{\peso{k}}(1,\hhat{\beta}) \,=\, \phi
\end{equation}
%
para toda $\hhat{\beta}\in\Idfin{\cal{O}}$.
% (notemos que el peso es irrelevante en esta expresi\'{o}n).
Si una funci\'{o}n $\phi$ cumple \eqref{eq:matricesinvariantedepesokadelicas}
con $\hhat{\beta}$ variando en alg\'{u}n subgrupo $K\subset\GL_{2}(\Adfin{F})$,
decimos que es invariante a derecha por el grupo $K$.

\begin{obsOperadorDePesoKAdjunto}\label{obs:operadordepesokadjunto}
	En relaci\'{o}n con el producto interno, dado un id\`{e}le finito
	$\hhat{\pi}\in\GL_{2}(\Adfin{F})$ y dadas
	$\phi,\phi'\in\cuadradointegrables{\omega}$, se deduce de
	\eqref{eq:peterssonenadelesmatrices} y de
	\eqref{eq:operadordepesokenadelesmatrices} que
	\begin{align*}
		\langle \phi\operadormatrices{\peso{k}}{(1,\hhat{\pi})},
			\phi'\rangle
			& \,=\,|\omega(\det\,\hhat{\pi})|^{-1}\,
				\langle\phi,\phi'\operadormatrices{\peso{k}}{%
				(1,\hhat{\pi}^{-1})}\rangle
			\,=\,\lconj{\chi(\det\,\hhat{\pi})}\,
			\langle\phi,\phi'\operadormatrices{\peso{k}}{%
				(1,\hhat{\pi}^\iota)}\rangle
		\text{ ,}
	\end{align*}
	%
	donde $\chi=|\omega|^{-1}\,\omega$. La \'{u}ltima igualdad es
	consecuencia de que el conjugado del id\`{e}le $\hhat{\pi}$ est\'{a}
	dado por $\hhat{\pi}^\iota=\det(\hhat{\pi})\,\hhat{\pi}^{-1}$.
\end{obsOperadorDePesoKAdjunto}
