Las formas de Hilbert se pueden ver, en analog\'{\i}a con las formas
modulares el\'{\i}pticas, como funciones holomorfas en un producto de
copias del semiplano complejo superior y que verifican una condici\'{o}n
de invarianza respecto de un subgrupo discreto de $\GLtp_{2}(F)$.
% Pero tambi\'{e}n se puede definir una forma de Hilbert como una forma
% cuaterni\'{o}nica para el \'{a}lgebra $\MM_{2\times 2}(F)$.
Aqu\'{\i}, $F$ denotar\'{a} un cuerpo de n\'{u}meros totalmente real de grado
$[F:\bb{Q}]=n$ y $\oka{F}$ su anillo de enteros, $\frak{N}$ denotar\'{a} un
ideal \'{\i}ntegro de $F$ y $\frak{a}$ un ideal fraccionario.
% , a menos que se indique algo distinto.
%
% En la secci\'{o}n ??? definimos las formas de Hilbert para un cuerpo
% de n\'{u}meros totalmente real $F$ como funciones holomorfas
% $f:\,\hP^{n}\rightarrow\bb{C}$ que son de peso $\peso{k}$ invariante para
% un subgrupo $\Gamma\subset\GLtp_{2}(F)$ que act\'{u}a de manera propiamente
% discontinua en $\hP^{n}$. Espec\'{\i}ficamente, hicimos esto para los grupos
% $\Gamma=\Gamma_{0}(\frak{N},\frak{a})$. El \'{a}lgebra de matrices sobre el
% cuerpo $F$, $\MM_{2\times 2}(F)$, tambi\'{e}n es un \'{a}lgebra de
% cuaterniones (totalmente indefinida), con lo cual tenemos, en principio,
% dos nociones distintas de formas modulares (automorfas) para este
% \'{a}lgebra.
%%%
% Al igual que en el caso de las \'{a}lgebras de divisi\'{o}n indefinidas,
% ciertos subgrupos de unidades totalmente positivas,
% $\Gamma_{0}(\frak{N},\frak{a})\subset\GLtp_{2}(F)$, asociados a representantes
% de las clases estrictas jugar\'{a}n un rol fundamental, en tanto que determinan
% descomposiciones de la variedad $\shimura{\frak{N}}$ (an\'{a}loga a las
% variedades de Shimura para un \'{a}lgebra de divisi\'{o}n) y, a su vez, de los
% espacios de formas modulares y cuspidales.
