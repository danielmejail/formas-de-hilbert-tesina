
\theoremstyle{plain}
\newtheorem{teoPrincipioDeKoecher}{Teorema}[chapter]
\newtheorem{propoGamaCeroIndiceFinito}[teoPrincipioDeKoecher]{Proposici\'{o}n}
\newtheorem{propoDeCongruenciaConjugadoConmensurable}[teoPrincipioDeKoecher]%
	{Proposici\'{o}n}
\newtheorem{propoGamaCeroAGamaCeroUno}[teoPrincipioDeKoecher]{Proposici\'{o}n}
\newtheorem{coroGamaCeroAGamaCeroBConjugado}[teoPrincipioDeKoecher]%
	{Corolario}
\newtheorem{teoLaDimEsFinita}[teoPrincipioDeKoecher]{Teorema}
\newtheorem{propoDeCongruenciaReticuloYUnidades}[teoPrincipioDeKoecher]%
	{Proposici\'{o}n}
\newtheorem{propoDescomposicionDelEspacioDeFormasModularesDeHilbert}%
	[teoPrincipioDeKoecher]{Proposici\'{o}n}
\newtheorem{propoPeterssonPropiedades}%
	[teoPrincipioDeKoecher]{Proposici\'{o}n}
\newtheorem{propoEquivalenciaAutomorfasFormasCuspidales}%
	[teoPrincipioDeKoecher]{Proposici\'{o}n}
\newtheorem{propoCoclaseDoblePreservaElEspacioDeFormasCuspidales}%
	[teoPrincipioDeKoecher]{Proposici\'{o}n}
\newtheorem{propoOperadoresDeHeckeSonNormales}[teoPrincipioDeKoecher]%
	{Proposici\'{o}n}
\newtheorem{propoDescomposicionInduceDescomposicion}%
	[teoPrincipioDeKoecher]{Proposici\'{o}n}
\newtheorem{teoCoeficientesNulosFormaVieja}[teoPrincipioDeKoecher]{Teorema}
\newtheorem{coroCoeficientesNulosFormaVieja}[teoPrincipioDeKoecher]{Corolario}
\newtheorem{teoFormasNuevasMismosAutovalores}[teoPrincipioDeKoecher]{Teorema}

\theoremstyle{definition}
\newtheorem{obsSubgruposConjugadosAGamaCero}[teoPrincipioDeKoecher]%
	{Observaci\'{o}n}
\newtheorem{obsDefinicionOrdenAsociadoMatrices}[teoPrincipioDeKoecher]%
	{Observaci\'{o}n}
\newtheorem{defPesoKInvariante}[teoPrincipioDeKoecher]{Definici\'{o}n}
\newtheorem{defOperadorDePesoKMatrices}[teoPrincipioDeKoecher]{Definici\'{o}n}
\newtheorem{obsEleccionDelFactor}[teoPrincipioDeKoecher]{Observaci\'{o}n}
\newtheorem{obsReticuloDual}[teoPrincipioDeKoecher]{Observaci\'{o}n}
\newtheorem{obsDesarrolloReticulo}[teoPrincipioDeKoecher]{Observaci\'{o}n}
\newtheorem{obsPrincipioDeKoecher}[teoPrincipioDeKoecher]{Observaci\'{o}n}
\newtheorem{defFormaModularDeHilbertParaCongruencia}[teoPrincipioDeKoecher]%
	{Definici\'{o}n}
\newtheorem{obsDefinicionFormasParaCongruencia}[teoPrincipioDeKoecher]%
	{Observaci\'{o}n}
\newtheorem{defFormaDeHilbertCuspidalParaCongruencia}[teoPrincipioDeKoecher]%
	{Definici\'{o}n}
\newtheorem{obsCuandoNoEsParalelo}[teoPrincipioDeKoecher]{Observaci\'{o}n}
\newtheorem{defFormaModularDeHilbertDeNivelN}[teoPrincipioDeKoecher]%
	{Definici\'{o}n}
\newtheorem{obsEleccionDelFactorParaAdelicas}[teoPrincipioDeKoecher]%
	{Observaci\'{o}n}
\newtheorem{defFormaDeHilbertCuspidalDeNivelN}[teoPrincipioDeKoecher]%
	{Definici\'{o}n}
\newtheorem{obsMockProductoInterno}[teoPrincipioDeKoecher]{Observaci\'{o}n}
\newtheorem{obsCuasicaracterCaracter}[teoPrincipioDeKoecher]{Observaci\'{o}n}
\newtheorem{obsCuasicaracter}[teoPrincipioDeKoecher]{Observaci\'{o}n}
\newtheorem{obsOperadorDePesoKAdjunto}[teoPrincipioDeKoecher]{Observaci\'{o}n}
\newtheorem{obsOperadorCoclaseAdelesAdjunto}[teoPrincipioDeKoecher]%
	{Observaci\'{o}n}
\newtheorem{obsHeckeEsConvolucion}[teoPrincipioDeKoecher]{Observaci\'{o}n}
\newtheorem{defOperadoresDeHeckeMatrices}[teoPrincipioDeKoecher]%
	{Definici\'{o}n}
\newtheorem{obsShimuraDiferenciaEnElAutovalor}[teoPrincipioDeKoecher]%
	{Observaci\'{o}n}
\newtheorem{obsIdelesDeNormaP}[teoPrincipioDeKoecher]{Observaci\'{o}n}
\newtheorem{obsBaseDeAutofunciones}[teoPrincipioDeKoecher]{Observaci\'{o}n}
\newtheorem{obsHeckeActuaPorBloquesMatrices}[teoPrincipioDeKoecher]%
	{Observaci\'{o}n}
\newtheorem{obsIdelesDeNormaPPorGlobales}[teoPrincipioDeKoecher]%
	{Observaci\'{o}n}
\newtheorem{obsIdealesDeNormaP}[teoPrincipioDeKoecher]{Observaci\'{o}n}
\newtheorem{obsDefinicionOperadorDeCoclaseDoble}[teoPrincipioDeKoecher]%
	{Observaci\'{o}n}
\newtheorem{obsRepresentantesDeLasCoclases}[teoPrincipioDeKoecher]%
	{Observaci\'{o}n}
\newtheorem{obsShimuraDiferenciaEnLosCoeficientes}[teoPrincipioDeKoecher]%
	{Observaci\'{o}n}
\newtheorem{defFormasViejasFormasNuevas}[teoPrincipioDeKoecher]%
	{Definici\'{o}n}
\newtheorem{obsBaseDeAutoformasParaHecke}[teoPrincipioDeKoecher]%
	{Observaci\'{o}n}
\newtheorem{defFormaNueva}[teoPrincipioDeKoecher]{Definici\'{o}n}

%-------------
