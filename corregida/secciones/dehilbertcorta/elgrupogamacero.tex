
Usando las $n$ inmersiones de $F$ en $\bb{R}$, queda determinada una
inclusi\'{o}n de $B_{+}^{\times}=\GLtp_{2}(F)$ en el producto
$\GL_{2}(\bb{R})^{n}$ y, de esta manera una acci\'{o}n de $\GLtp_{2}(F)$
en $\hP^{n}$ dada por:
\begin{equation}
	\label{eq:accionenelsemiplanodehilbert}
	\begin{bmatrix} a & b \\ c & d \end{bmatrix}\,z
		\,=\, \left(
		\frac{a_{1}z_{1}+b_{1}}{c_{1}z_{1}+d_{1}},\,\dots,\,
		\frac{a_{n}z_{n}+b_{n}}{c_{n}z_{n}+d_{n}}
		\right)
	\text{ .}
\end{equation}
%
El espacio $\hP^{n}$ posee una medida invariante por la acci\'{o}n de
$\GLtp_{2}(\bb{R})^{n}$:
\begin{equation}
	\label{eq:medidaenelsemiplanodehilbert}
	d\mu \,=\, \prod_{i=1}^{n}\,y_{i}^{2}dx_{i}dy_{i}
	\text{ .}
\end{equation}
%
En particular, por \eqref{eq:accionenelsemiplanodehilbert} esta medida es
invariante por $\GLtp_{2}(F)$.

Sea $\cal{O}_{0}(1)$ el orden maximal
\begin{equation}
	\label{eq:ordenmaximalmatrices}
	\cal{O}_{0}(1) \,:=\, \MM_{2\times 2}(\oka{F}) \,=\,
	\begin{bmatrix} \oka{F} & \oka{F} \\ \oka{F} & \oka{F} \end{bmatrix}
\end{equation}
%
contenido en $B=\MM_{2\times 2}(F)$. Manteniendo este orden fijo,
consideramos los \'{o}rdenes de la forma
\begin{equation}
	\label{eq:ordendeeichlermatrices}
	\cal{O}_{0}(\frak{N}) \,:=\,
	\begin{bmatrix} \oka{F} & \oka{F} \\ \frak{N} & \oka{F} \end{bmatrix}
		\,\subset\,\cal{O}_{0}(1)
		\text{ ,}
\end{equation}
%
que resultan ser \'{o}rdenes de Eichler para los distintos ideales
\'{\i}ntegros $\frak{N}\subset\oka{F}$. Tomamos entonces
$\cal{O}=\cal{O}_{0}(\frak{N})$ y un conjunto de representantes
$\frak{a}\subset\oka{F}$ de las clases estrictas de $F$. Como en la
observaci\'{o}n \ref{obs:gruposyordenesasociadosaunaclaseestricta}, para cada
representante $\frak{a}$, fijamos un id\`{e}le finito (\'{\i}ntegro)
$\hhat{a}\in\Adfin{\oka{F}}$ que cumpla
\begin{equation}
	\label{eq:ideleintegroasociadomatrices}
	\frak{a} \,=\,\hhat{a}\Adfin{\oka{F}}\,\cap\,F
\end{equation}
y $\hhat{\alpha}\in\GL_{2}(\Adfin{F})$ tal que $\nrd(\hhat{\alpha})=\hhat{a}$.
Concretamente, sea
\begin{equation}
	\label{eq:matrizasociadaaideleintegromatrices}
	\hhat{\alpha} \,=\,\begin{bmatrix} \hhat{a} & \\ & 1 \end{bmatrix}
	\text{ .}
\end{equation}
%
Con estas elecciones, a cada representante $\frak{a}$, se le asocian el orden
de Eichler
\begin{equation}
	\label{eq:ordenasociadomatrices}
	\cal{O}_{\frak{a}} \,:=\,\big(\hhat{\alpha}\Adfin{\cal{O}}
			\hhat{\alpha}^{-1}\big)\,\cap\,\MM_{2\times 2}(F)
		\,=\,\begin{bmatrix} \oka{F} & \frak{a} \\
			\frak{N}\frak{a}^{-1} & \oka{F}\end{bmatrix}
\end{equation}
%
y el grupo de unidades totalmente positivas
\begin{equation}
	\label{eq:unidadestotalmentepositivasasociadasmatrices}
	\Gamma_{0}(\frak{N},\frak{a}) \,=\,
		\Gamma_{\frak{a}} \,:=\,\cal{O}_{\frak{a}}^{\times}\,\cap\,
				\GLtp_{2}(F)
	\text{ .}
\end{equation}
%
% Estas definiciones son an\'{a}logas a \eqref{eq:ordenasociado} y
% \eqref{eq:unidadestotalmentepositivasasociadas}.

% \begin{obsDefinicionOrdenAsociadoMatrices}%
	% \label{obs:definicionordenasociadomatrices}
	% Como el id\`{e}le $\hhat{a}$ en \eqref{eq:ideleintegroasociadomatrices}
	% est\'{a} determinado salvo una unidad en $\Idfin{\oka{F}}$, el orden
	% $\cal{O}_{\frak{a}}$ y el grupo de unidades $\Gamma_{\frak{a}}$, no
	% dependen de la elecci\'{o}n de $\hhat{a}$, pero s\'{\i} dependen de la
	% elecci\'{o}n del elemento $\hhat{\alpha}\in\GL_{2}(\Adfin{F})$ tal que
	% $\nrd(\hhat{\alpha})=\hhat{a}$. Si
	% $\hhat{\alpha}'\in\GL_{2}(\Adfin{F})$ es tal que el ideal determinado
	% por $\nrd(\hhat{\alpha}')$ pertenece a la clase reducida de $\frak{a}$,
	% entonces, por el corolario \ref{thm:normaclasesestrictas}, existen
	% $\gamma\in\GLtp_{2}(F)$ y $\hhat{\beta}\in\Idfin{\cal{O}}$ tales que
	% $\hhat{\alpha}'=\gamma\hhat{\alpha}\hhat{\beta}$ y los \'{o}rdenes
	% asociados por \eqref{eq:ordenasociadomatrices} son conjugados por un
	% elemento de $\GLtp_{2}(F)$.
% \end{obsDefinicionOrdenAsociadoMatrices}
% 
Al \'{a}lgebra $\MM_{2\times 2}(F)$ y al orden $\cal{O}$ se les asocia la
variedad
\begin{align*}
	Y_{0}(\frak{N}) & \,=\,
		\bigsqcup_{[\frak{a}]\in\pClass{F}}\,
		Y_{0}(\frak{N},\frak{a})
	\text{ ,}
\end{align*}
%
donde $Y_{0}(\frak{N},\frak{a})$ se define como el cociente
$\Gamma_{0}(\frak{N},\frak{a})\backslash\hP^{n}$. Esta variedad no es compacta
pero, v\'{\i}a la determinaci\'{o}n de un dominio fundamental para cada
componente $Y_{0}(\frak{N},\frak{a})$, se demuestra que es de volumen finito.
\cite[Ch.~IV,\S~1]{vanDerGeerSurfaces}.
% Por ejemplo, con respecto a la medida \eqref{eq:medidaenelsemiplanodehilbert},
% la variedad $Y_{0}(1,1)$ correspondiente al grupo $\Gamma_{0}(1,1)$ (matrices
% con coeficientes enteros y determinante una unidad totalmente positiva), tiene
% volumen
% \begin{align*}
	% \mu\big(Y_{0}(1,1)\big) & \,=\,2\zeta_{F}(-1)
	% \text{ ,}
% \end{align*}
% %
% donde
% \begin{math}
	% \zeta_{F}(s)=\sum_{\frak{b}\subset\oka{F}}\,\norma(\frak{b})^{-s}
% \end{math}
% es la funci\'{o}n zeta de Dedekind del cuerpo $F$
% \cite[Ch.~IV,\S~1]{vanDerGeerSurfaces}.
%
% En general, los grupos $\Gamma_{0}(\frak{N},\frak{a})$ son conmensurables (?)
% con $\Gamma_{0}(1,1)$ y el volumen de la variedad asociada est\'{a} dado por
% \begin{align*}
	% \mu\big(Y_{0}(\frak{N},\frak{a})\big) & \,=\,
		% \frac{\big|\Gamma_{0}(1,1):\Gamma_{0}(1,1)\cap%
			% \Gamma_{0}(\frak{N},\frak{a})\big|}{%
		% \big|\Gamma_{0}(\frak{N},\frak{a}):
			% \Gamma_{0}(1,1)\cap\Gamma_{0}(\frak{N},\frak{a})\big|}
		% \,\mu\big(Y_{0}(1,1)\big)
	% \text{ .}
% \end{align*}
% %
Para obtener una variedad compacta, se considera el grupo $\GLtp_{2}(F)$
actuando en $\bb{P}^{1}(F)$ por:
\begin{equation}
	\label{eq:accionenlascuspides}
	\begin{bmatrix} a & b \\ c & d \end{bmatrix}\,[\alpha:\beta]
		\,=\,[a\alpha+b\beta:c\alpha+d\beta]
\end{equation}
%
Como tambi\'{e}n $\bb{P}^{1}(F)\hookrightarrow\bb{P}(\bb{R})^{n}$ v\'{\i}a las
inmersiones de $F$ en $\bb{R}$, podemos obtener una compactificaci\'{o}n de
$Y_{0}(\frak{N},\frak{a})$ agregando las c\'{u}spides de
$\Gamma_{0}(\frak{N},\frak{a})$, las \'{o}rbitas de
$\Gamma_{0}(\frak{N},\frak{a})$ actuando en $\bb{P}^{1}(F)$. Esta
compactificaci\'{o}n se conoce como \emph{compactificaci\'{o}n de %
Baily-Borel} \cite[Ch.~II, \S~7]{vanDerGeerSurfaces} y la denotamos
\index{compactificacion de Baily Borel@compactificaci\'{o}n de Baily-Borel}
\begin{align*}
	\shimura{\frak{N},\frak{a}} & \,:=\,
		\Gamma_{0}(\frak{N},\frak{a})\backslash
		\left(\hP^{n}\cup\bb{P}^{1}(F)\right)
	\text{ .}
\end{align*}
%
A diferencia de las \emph{curvas} modulares, estas variedades tienen
puntos singulares: los puntos el\'{\i}pticos y las (finitas) c\'{u}spides
que agregamos. Cuando $F=\bb{Q}$, para cada nivel $N>1$ hay una \'{u}nica
curva $X_{0}(N)$. En general, a cada ideal \'{\i}ntegro $\frak{N}$, le
asociamos en principio una variedad $\shimura{\frak{N},\frak{a}}$ por cada
$\frak{a}$ en un sistema de representantes de las clases en $\pClass{F}$.
