Adem\'{a}s de los grupos $\Gamma_{0}(\frak{N},\frak{a})$, aparecer\'{a}n
naturalmente otros, por ejemplo, de la forma
\begin{math}
	\Gamma_{0}(\frak{N},\frak{a})\cap
		\big(\gamma\Gamma_{0}(\frak{N},\frak{b})\gamma^{-1}\big)
\end{math}~.
Las propiedades fundamentales de las formas de Hilbert: existencia
de expansi\'{o}n de Fourier y principio de Koecher, dependen de la existencia
de suficientes ``traslaciones'' y de suficientes ``homotecias''
(respectivamente). Estas transformaciones surgen de considerar el estabilizador
de la c\'{u}spide $\infty$ en $\Gamma_{0}(\frak{N},\frak{a})$. Por otra
parte, para poder relacionar el producto interno de Petersson con los
operadores de Hecke en los espacios de formas cuspidales, necesitaremos
garantizar que estas transformaciones est\'{e}n presentes, tambi\'{e}n, en
otros grupos. Con este fin, introducimos los subgrupos de congruencia y la
relaci\'{o}n de conmensurabilidad.

Dados $\frak{N}$ y $\frak{a}$, respectivamente un ideal \'{\i}ntegro y un
ideal fraccionario en $F$, definimos un grupo $\Gamma_{0}(\frak{N},\frak{a})$
por la expresi\'{o}n \eqref{eq:unidadestotalmentepositivasasociadasmatrices}
(la diferencia est\'{a} en que $\frak{a}$ es, en principio, arbitrario).
Fijado $\frak{a}$, los grupos $\Gamma_{0}(\frak{N},\frak{a})$ constituyen
una familia ordenada por
\begin{math}
	\frak{M}\mid\frak{N}\Rightarrow\Gamma_{0}(\frak{M},\frak{a})
		\supset\Gamma_{0}(\frak{N},\frak{a})
\end{math}~,
con $\Gamma_{0}(1,\frak{a})$ conteniendo a todos. El anillo
$\cal{O}_{0}(1)_{\frak{a}}$ (\eqref{eq:ordenasociadomatrices} con $\frak{N}=1$)
act\'{u}a por endomorfismos en el $\oka{F}$-m\'{o}dulo
$\frak{a}\oplus\oka{F}$:
\begin{align*}
	\begin{bmatrix} a & b \\ c & d \end{bmatrix}(x,y) & \,=\,
			(ax+by,cx+dy)
	\text{ .}
\end{align*}
%
Evaluando en un punto $(x,0)$ y en un punto $(0,y)$ ($x,y\not=0$), se ve que
el \'{u}nico elemento que act\'{u}a trivialmente es la matriz identidad
$I\in\cal{O}_{0}(1)_{\frak{a}}$. En particular, el grupo
$\Gamma_{0}(1,\frak{a})$ se realiza como un subgrupo de automorfismos de
este $\oka{F}$-m\'{o}dulo. Para cada ideal \'{\i}ntegro
$\frak{N}\subset\oka{F}$, el subm\'{o}dulo $\frak{N}\frak{a}\oplus\frak{N}$ es
$\cal{O}_{0}(1)_{\frak{a}}$-invariante:
\begin{align*}
	\begin{bmatrix} \oka{F} & \frak{a} \\
		\frak{a}^{-1} & \oka{F} \end{bmatrix}
			(\frak{N}\frak{a}\oplus\frak{N}) & \,\subset\,
		\frak{N}\frak{a}\oplus\frak{N}
	\text{ .}
\end{align*}
%
En particular, $\Gamma_{0}(1,\frak{a})$ preserva este subm\'{o}dulo y la
acci\'{o}n del grupo desciende a una acci\'{o}n en el cociente
\begin{math}
	\frak{a}\oplus\oka{F}/\frak{N}\frak{a}\oplus\frak{N}\simeq
		\frak{a}/\frak{N}\frak{a}\oplus\oka{F}/\frak{N}
\end{math}~.
La imagen del correspondiente morfismo
\begin{math}
	\Gamma_{0}(1,\frak{a})\rightarrow
		\Auto(\frak{a}/\frak{N}\frak{a}\oplus\oka{F}/\frak{N})
\end{math}
es finita y su n\'{u}cleo est\'{a} dado por
\begin{align*}
	\Gamma(\frak{N},\frak{a}) & \,:=\,
		\bigg\{\begin{bmatrix} a & b \\ c & d \end{bmatrix}\in
			\Gamma_{0}(1,\frak{a})\,:\,
		\begin{bmatrix} a & b \\ c & d \end{bmatrix}
			\equiv
			\begin{bmatrix} 1 & \\ & 1 \end{bmatrix}\,
				\bigg(\frak{N}\cdot
				\begin{bmatrix} \oka{F} & \frak{a} \\
				\frak{a}^{-1} & \oka{F} \end{bmatrix}
				\bigg)
			\bigg\}% \\
	%& \,=\,\bigg\{\begin{bmatrix} a & b \\ c & d \end{bmatrix}\in
			%\begin{bmatrix} \oka{F} & \frak{a} \\
			%\frak{a}^{-1} & \oka{F} \end{bmatrix}\,:\,
			%ad-bc\in\oka{F,+}^{\times},\,a\equiv d\equiv 1
				%(\frak{N}),\,
				%c\in\frak{N}\frak{a}^{-1},\,
				%b\in\frak{N}\frak{a}
			%\bigg\}
	\text{ .}
\end{align*}
%
El grupo $\Gamma(\frak{N},\frak{a})$ es, pues, normal en
$\Gamma_{0}(\frak{N},\frak{a})$ y de \'{\i}ndice finito. Adem\'{a}s, la
condici\'{o}n $c\in\frak{N}\frak{a}^{-1}$ implica que
$\Gamma(\frak{N},\frak{a})\subset\Gamma_{0}(\frak{N},\frak{a})$. Un subgrupo
$\Gamma\subset\GLtp_{2}(F)$ se dice \emph{de congruencia}, si
\index{subgrupo de congruencia}
existen ideales $\frak{a},\frak{a}'$ y $\frak{N},\frak{N}'\subset\oka{F}$ y
$A\in\GLtp_{2}(F)$ tales que
\begin{align*}
	A^{-1}\,\Gamma_{0}(\frak{N},\frak{a})\,A & \,\supset\,
		\Gamma \,\supset\,
		\Gamma(\frak{N}',\frak{a}')
	\text{ .}
\end{align*}
%
Los grupos $\Gamma_{0}(\frak{N},\frak{a})$ son subgrupos de congruencia. Se
puede demostrar que la clase de subgrupos de congruencia es cerrada por
conjugaci\'{o}n en $\GLtp_{2}(F)$ y que todos los subgrupos de congruencia son
conmensurables entre s\'{\i}.%
\footnote{
	Ver \cite{Freitag} para una idea de la demostraci\'{o}n.
}

En general, si $\Gamma\subset\GLtp_{2}(F)$ es conmensurable con
$\Gamma_{0}(1,1)$ (y, por lo tanto, con todo subgrupo de congruencia), entonces
el cociente $\Gamma\backslash\hP^{n}$ es una variedad no compacta y de volumen
finito. Si $\Gamma'\subset\Gamma$ es un subgrupo de \'{\i}ndice finito,
entonces%
\footnote{
	\cite[Ch.~IV,\S~1]{vanDerGeerSurfaces}.
}
\begin{align*}
	\mu\big(\Gamma'\backslash\hP^{n}\big) & \,=\,
		\big|\Gamma:\Gamma'\big|\cdot
			\mu\big(\Gamma\backslash\hP^{n}\big)
\end{align*}
%
% Si, por otro lado, $A\in\GLtp_{2}(F)$
% y $\Gamma'=A^{-1}\,\Gamma\,A$, entonces la funci\'{o}n
% $(\tau\mapsto A^{-1}\tau):\,\hP^{n}\rightarrow\hP^{n}$ tiene las siguentes
% propiedades: preserva la medida $\mu$ en $\hP^{n}$ y, si dos puntos
% $\tau,\tau'\in\hP^{n}$ pertenecen a la misma \'{o}rbita con respecto a la
% acci\'{o}n de $\Gamma$, entonces los puntos $A^{-1}\tau$ y $A^{-1}\tau'$
% pertenecen a la misma \'{o}rbita con respecto a la acci\'{o}n de $\Gamma'$. Se
% deduce entonces que $\tau\mapsto A^{-1}\tau$ induce una funci\'{o}n biholomorfa
% \begin{math}
	% \Gamma\backslash\hP^{n}
		% \xrightarrow{\sim}\Gamma'\backslash\hP^{n}
% \end{math}
% que preserva la medida natural en los cocientes. En particular,
% \begin{math}
	% \mu\big((A^{-1}\,\Gamma\,A)\backslash\hP^{n}\big)=
		% \mu\big(\Gamma\backslash\hP^{n}\big)
% \end{math}~.
