Sean $\frak{p}\subset\oka{F}$ un ideal primo, $p\in\oka{F,\frak{p}}$ un
generador del ideal maximal y sea
\begin{math}
	\pi=\begin{bmatrix} p & \\ & 1 \end{bmatrix}
\end{math}~.
Sea $\hhat{p}\in\Idfin{F}$ el id\`{e}le (\'{\i}ntegro) dado por
$\hhat{p}_{v}=1$, si $v\not=\frak{p}$, e igual a $p$ en $\frak{p}$;
% En particular, $\hhat{p}\Adfin{\oka{F}}\cap F=\frak{p}$.
identificamos $\hhat{p}$ con el correspondiente elemento del centro de
$\Idfin{B}$. Sea $\hhat{\pi}\in\Idfin{B}$ el id\`{e}le dado por
$\hhat{\pi}_{v}=1$, si $v\not=\frak{p}$, e igual a $\pi$ en $\frak{p}$.

\begin{defOperadoresDeHeckeMatrices}\label{def:operadoresdeheckematrices}
	Llamaremos \emph{operadores de Hecke (en \frak{p})} a los operadores
	\begin{align*}
		T_{\frak{p}}\,=\,T_{\hhat{\pi}} & \quad\text{y}\quad
			S_{\frak{p}}\,=\,T_{\hhat{p}}
		\qquad\qquad (\,(\frak{p},\frak{N})=1\,)
	\end{align*}
	%
	en $\modularH{k}{\frak{N}}$.\index{operador de Hecke}
	El \emph{\'{a}lgebra de Hecke} de $\modularH{k}{\frak{N}}$ es el
	\'{a}lgebra sobre $\bb{C}$ generada por el conjunto
	\begin{math}
		\big\{T_{\frak{p}},\,S_{\frak{p}}
			\,:\,\frak{p}\nmid\frak{N}\big\}
	\end{math}
	en $\Endo[\bb{C}](\modularH{k}{\frak{N}})$.%
	\footnote{
		La definici\'{o}n de operadores de Hecke se extiende a los
		primos divisores del nivel con algunas modificaciones (ver
		\cite{ShemanskeWalling}, \cite{MiyakeOnAutomorphicFormsOnGL}).
	}
	Los operadores de Hecke preservan los espacios de formas cuspidales
	$\spitzH{k}{\frak{N}}$ y $\spitzH{k}{\frak{N},\omega}$ para cada
	cuasicar\'{a}cter $\omega$.
\end{defOperadoresDeHeckeMatrices}

Si $\frak{p}$ y $\frak{q}$ son dos primos distintos, dado que act\'{u}an
en coordenadas distintas, los operadores $T_{\frak{p}}$ y $T_{\frak{q}}$
conmutan. Los operadores $S_{\frak{p}}$ conmutan con $T_{\frak{q}}$,
$S_{\frak{q}}$ y tambi\'{e}n con $T_{\frak{p}}$, porque $\hhat{p}$ es central.

Para poder expresar el operador de Hecke $T_{\frak{p}}$ como un operador de
coclase doble, es necesario hallar un conjunto de representantes de las clases
\begin{math}
	\Idfin{\cal{O}}\backslash\Idfin{\cal{O}}\hhat{\pi}\Idfin{\cal{O}}
\end{math}~,
con $\hhat{\pi}$ como en la Definici\'{o}n \ref{def:operadoresdeheckematrices}.
Como el \'{u}nico lugar en donde $\hhat{\pi}$ no es trivial es el lugar
correspodiente al ideal primo $\frak{p}$, ser\'{a} suficiente hallar una
descomposici\'{o}n de
\begin{math}
	\cal{O}_{\frak{p}}^{\times}\pi\cal{O}_{\frak{p}}^{\times}
\end{math}
como uni\'{o}n disjunta de conjuntos de la forma
$\cal{O}_{\frak{p}}^{\times}\beta$. Si $\frak{p}\nmid\frak{N}$, entonces
$\cal{O}_{\frak{p}}^{\times}=\GL(\oka{F,\frak{p}})$ y, tomando un sistema de
representantes $\{b_{i}\}_{i}$ de las clases $i\in\oka{F,\frak{p}}/(p)$, se
verifica
\begin{align*}
	\cal{O}_{\frak{p}}^{\times}
		\begin{bmatrix} p & \\ & 1 \end{bmatrix}
			\cal{O}_{\frak{p}}^{\times} & \,=\,
		\bigsqcup_{i\in\oka{F,\frak{p}}/(p)}\,
			\cal{O}_{\frak{p}}^{\times}
		\begin{bmatrix} 1 & b_{i} \\ & p \end{bmatrix}\,\sqcup\,
			\cal{O}_{\frak{p}}^{\times}
		\begin{bmatrix} p & \\ & 1\end{bmatrix}
	\text{ .}
\end{align*}
%
En cuanto a $S_{\frak{p}}$, notamos simplemente que
\begin{align*}
	\cal{O}_{\frak{p}}^{\times}
		\begin{bmatrix} p & \\ & p \end{bmatrix}
			\cal{O}_{\frak{p}}^{\times} & \,=\,
		\cal{O}_{\frak{p}}^{\times}
			\begin{bmatrix} p & \\ & p \end{bmatrix}
	\text{ .}
\end{align*}
%
As\'{\i}, si $f\in\modularH{k}{\frak{N}}$, entonces
\begin{math}
	\big(S_{\frak{p}}f\big)(z,\hhat{\alpha}\Idfin{\cal{O}})=
		f(z,\hhat{\alpha}\hhat{p}^{-1}\Idfin{\cal{O}})
\end{math}~.
En particular, si $f\in\spitzH{k}{\frak{N},\omega}$, entonces
$S_{\frak{p}}f=\omega(\frak{p})^{-1}\cdot f$.%
\footnote{\label{obs:shimuradiferenciaenelautovalor}
	En \cite{ShimuraSpecialValuesOfZeta},
	Proposici\'{o}n~2.1, se afirma algo en apariencia distinto: el
	autovalor de $S_{\frak{p}}$ en una $f$ perteneciente a uno de los
	espacios an\'{a}logos a los que aqu\'{\i} denotamos
	$\spitzH{k}{\frak{N},\omega}$ es $\omega(\frak{p})$ en lugar de
	$\omega(\frak{p})^{-1}$. Esta discrepancia se debe a que nosotros hemos
	definido la acci\'{o}n de $\GL_{2}(\Adfin{F})$ a derecha en funciones
	por
	\begin{math}
		\phi\operadormatrices{\peso{k}}{(1,\hhat{\pi})}
			(g,\hhat{\alpha})=\phi(g,\hhat{\alpha}\hhat{\pi}^{-1})
	\end{math}~,
	mientras que Shimura define la acci\'{o}n con el conjugado
	$\hhat{\pi}^\iota$ en lugar del inverso. Esta diferencia tambi\'{e}n
	aparecer\'{a} m\'{a}s adelante cuando describamos el efecto de los
	operadores $T_{\frak{p}}$ en los coeficientes de Fourier
	(\S~\ref{subsec:relacionconfourier}).
}

\begin{obsIdelesDeNormaP}\label{obs:idelesdenormap}
	Dado $\hhat{p}\in\Adfin{\oka{F}}$, el id\`{e}le $\hhat{\pi}$ pertenece
	al anillo $\Adfin{\cal{O}}$. En particular, los elementos de la coclase
	doble $\Idfin{\cal{O}}\hhat{\pi}\Idfin{\cal{O}}$ pertenecen a
	$\Adfin{\cal{O}}$ y sus normas reducidas pertenecen a
	$\hhat{p}\,\Idfin{\oka{F}}$.
	% (Localmente,
	% \begin{math}
		% \big(\Idfin{\cal{O}}\hhat{\pi}\Idfin{\cal{O}}\big)_{\frak{q}}
	% \end{math}
	% es igual a $\cal{O}_{\frak{q}}^{\times}$, si $\frak{q}\not= \frak{p}$,
	% y a
	% \begin{math}
		% \GL_{2}(\oka{F,\frak{p}})
			% \begin{bmatrix} p & \\ & 1 \end{bmatrix}
			% \GL_{2}(\oka{F,\frak{p}})
	% \end{math}~,
	% si $\frak{q}=\frak{p}$, pues $\frak{p}$ es coprimo con el nivel del
	% orden $\cal{O}$).
	Rec\'{\i}procamente, si $\hhat{x}\in\Adfin{\cal{O}}$
	y $\nrd(\hhat{x})\in\hhat{p}\,\Idfin{\oka{F}}$, entonces
	$x_{\frak{q}}\in\cal{O}_{\frak{q}}^{\times}$, si
	$\frak{q}\not=\frak{p}$, y, por eliminaci\'{o}n, existen
	$v_{\frak{p}},w_{\frak{p}}\in\GL_{2}(\oka{F,\frak{p}})$ tales que
	\begin{math}
		v_{\frak{p}}x_{\frak{p}}w_{\frak{p}}=\pi
	\end{math}~.
	En definitiva,
	\begin{align*}
		\Idfin{\cal{O}}\hhat{\pi}\Idfin{\cal{O}} & \,=\,
			\Big\{\hhat{x}\in\Adfin{\cal{O}}\,:\,
				\nrd(\hhat{x})\in\hhat{p}\,\Idfin{\oka{F}}
			\Big\}
		\text{ .}
	\end{align*}
	%
	Escribimos $\frak{I}(\frak{p})$ para referirnos a este
	conjunto y notamos que
	\begin{math}
		\frak{I}(\frak{p})=\Idfin{\cal{O}}\hhat{x}\Idfin{\cal{O}}
	\end{math}
	para cualquier elemento $\hhat{x}\in\frak{I}(\frak{p})$. Definimos,
	tambi\'{e}n,
	\begin{align*}
		\Theta(\frak{p}) & \,=\,
			\Idfin{\cal{O}}\backslash\frak{I}(\frak{p})
		\text{ .}
	\end{align*}
	%
\end{obsIdelesDeNormaP}

\begin{propoOperadoresDeHeckeSonNormales}%
	\label{propo:operadoresdeheckesonnormales}
	Los operadores de Hecke $T_{\frak{p}}$ y $S_{\frak{p}}$ son normales,
	si $\frak{p}\nmid\frak{N}$.
\end{propoOperadoresDeHeckeSonNormales}

\begin{proof}
	Notamos que, con $\hhat{\pi}$ y $\hhat{p}$ como en la Definici\'{o}n
	\ref{def:operadoresdeheckematrices}, $\hhat{p}^\iota=\hhat{p}$ y, por
	la Observaci\'{o}n~\ref{obs:idelesdenormap},
	\begin{math}
		\Idfin{\cal{O}}\hhat{\pi}^\iota\Idfin{\cal{O}}=
			\Idfin{\cal{O}}\hhat{\pi}\Idfin{\cal{O}}
	\end{math}~, lo que implica que $T_{\hhat{p}^\iota}=T_{\hhat{p}}$ y que
	$T_{\hhat{\pi}^\iota}=T_{\hhat{\pi}}$. De la expresi\'{o}n
	\eqref{eq:operadorcoclaseformasadjunto}, se deduce que
	\begin{align*}
		S_{\frak{p}}^{*} \,=\,\chi(\frak{p})^{2}\,S_{\frak{p}}
			&\quad\text{y}\quad
		T_{\frak{p}}^{*} \,=\,\chi(\frak{p})\,T_{\frak{p}}
		\text{ ,}
	\end{align*}
	%
	en cada componente $\spitzH{k}{\frak{N},\omega}$
	($\chi=|\omega|^{-1}\,\omega$).
\end{proof}

Cada espacio $\spitzH{k}{\frak{N},\omega}$ es de dimensi\'{o}n finita y la
familia de operadores $\big\{T_{\frak{p}}\,:\,\frak{p}\nmid\frak{N}\big\}$ es
una familia de operadores normales que conmutan entre s\'{\i}. Concluimos,
entonces, que existen bases ortogonales para los espacios de formas cuspidales
conformadas por autofunciones para todos los operadores $T_{\frak{p}}$ con
$\frak{p}$ que no divide al nivel.
