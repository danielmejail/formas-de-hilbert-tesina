Sea $\frak{p}\subset\oka{F}$ un primo $\frak{p}\nmid\frak{N}$ y sean
$f,g\in\spitzH{k}{\frak{N}}$. Por definici\'{o}n del producto interno de
Petersson,
\begin{math}
	\langle T_{\frak{p}}f,g\rangle=\sum_{\frak{a}}\,
		\langle\big(T_{\frak{p}}f\big)_{\frak{a}},g_{\frak{a}}
			\rangle_{\frak{a}}
\end{math}~.

Para cada representante $\frak{a}$, sea $\frak{b}$ el representante que cumple
$[\frak{a}]=[\frak{b}\frak{p}]$ en $\pClass{F}$. Sea
$\gamma\in\frak{I}(\frak{p})_{\frak{a},\frak{b}}$ un elemento arbitrario, de
manera que
\begin{math}
	\frak{I}(\frak{p})_{\frak{a},\frak{b}}=
		\Gamma_{\frak{b}}\gamma\Gamma_{\frak{a}}
\end{math}~.
El grupo $\Gamma_{\frak{b}}$ act\'{u}a a izquierda en este conjunto y las
\'{o}rbitas por dicha acci\'{o}n (las coclases a derecha de $\Gamma_{\frak{b}}$
contenidas en el mismo) est\'{a}n en biyecci\'{o}n con las clases en
\begin{math}
	\big(\Gamma_{\frak{a}}\cap\gamma^{-1}\Gamma_{\frak{b}}\gamma\big)
		\backslash\Gamma_{\frak{a}}
\end{math}~.
En t\'{e}rminos de representantes, un conjunto $\{\gamma_{i}\}_{i}$ verifica
\begin{math}
	\Gamma_{\frak{b}}\gamma\Gamma_{\frak{a}}=
		\bigsqcup_{i}\,\Gamma_{\frak{b}}\gamma_{i}
\end{math}~,
si y s\'{o}lo si $\{\delta_{i}\}_{i}$ verifica
\begin{math}
	\Gamma_{\frak{a}}=\bigsqcup_{i}\,\big(\Gamma_{\frak{a}}\cap
		\gamma^{-1}\Gamma_{\frak{b}}\gamma\big)\,\delta_{i}
\end{math}~,
donde $\delta_{i}=\gamma^{-1}\gamma_{i}$. De manera an\'{a}loga, el grupo
$\Gamma_{\frak{a}}$ act\'{u}a a derecha en la coclase doble
$\Gamma_{\frak{b}}\gamma\Gamma_{\frak{a}}$ y las \'{o}rbitas por dicha
acci\'{o}n est\'{a}n en biyecci\'{o}n con las clases en
\begin{math}
	\Gamma_{\frak{b}}/\big(\gamma\Gamma_{\frak{a}}\gamma^{-1}\cap
		\Gamma_{\frak{b}}\big)
\end{math}~.
Un conjunto $\{\tilde{\gamma}_{j}\}_{j}$ cumple con
\begin{math}
	\Gamma_{\frak{b}}\gamma\Gamma_{\frak{a}}=
		\bigsqcup_{j}\,\gamma_{j}\Gamma_{\frak{a}}
\end{math}~,
si y s\'{o}lo si $\{\tilde{\delta}_{j}\}_{j}$ cumple con
\begin{math}
	\Gamma_{\frak{b}}=\bigsqcup_{j}\,\tilde{\delta}_{j}\,\big(
		\gamma\Gamma_{\frak{a}}\gamma^{-1}\cap\Gamma_{\frak{b}}\big)
\end{math}~.
Ahora bien, dado que
\begin{math}
	\mu\big(\Gamma_{\frak{a}}\backslash\hP^{n}\big)=
		\mu\big(\Gamma_{\frak{b}}\backslash\hP^{n}\big)
\end{math}~,
vale que
\begin{math}
	\big|\Gamma_{\frak{b}}:\gamma\Gamma_{\frak{a}}\gamma^{-1}\cap
		\Gamma_{\frak{b}}\big|=
	\big|\Gamma_{\frak{a}}:\Gamma_{\frak{a}}\cap\gamma^{-1}
		\Gamma_{\frak{b}}\gamma\big|
\end{math}~.
Por lo tanto, es posible hallar un sistema de representantes com\'{u}n a
ambas descomposiciones de la coclase doble, es decir, un conjunto
$\{\eta_{i}\}_{i}\subset\frak{I}(\frak{p})_{\frak{a},\frak{b}}$ tal que
\begin{align*}
	\Gamma_{\frak{b}}\gamma\Gamma_{\frak{a}} & \,=\,
		\bigsqcup_{i}\,\Gamma_{\frak{b}}\eta_{i} \,=\,
		\bigsqcup_{i}\,\eta_{i}\Gamma_{\frak{a}}
	\text{ .}
\end{align*}
%
Sean $\{\delta_{i}\}_{i}\subset\Gamma_{\frak{a}}$ y
$\{\tilde{\delta}_{i}\}_{i}\subset\Gamma_{\frak{b}}$ tales que
$\eta_{i}=\tilde{\delta}_{i}\gamma\delta_{i}$.

Volviendo a $T_{\frak{p}}$, sabemos que
\begin{math}
	\big(T_{\frak{p}}f\big)_{\frak{a}}=
		\sum_{i}\,f_{\frak{b}}\operadormatrices{\peso{k}}{\eta_{i}}
\end{math}~.
Entonces, por las observaciones del p\'{a}rrafo anterior, las propiedades del
producto de Petersson (Proposici\'{o}n~\ref{propo:peterssonpropiedades}),
\begin{align*}
	& \Big\langle\sum_{i}\,f_{\frak{b}}\operadormatrices{\peso{k}}{%
			\eta_{i}},g_{\frak{a}}\Big\rangle_{\frak{a}} \,=\,
	\Big\langle\sum_{i}\,\big(f_{\frak{b}}\operadormatrices{\peso{k}}{%
		\gamma}\big)\operadormatrices{\peso{k}}{\delta_{i}},
		g_{\frak{a}}\Big\rangle_{\frak{a}} \,=\,
	\big|\Gamma_{\frak{a}}:\Gamma_{\frak{a}}\cap\gamma^{-1}
		\Gamma_{\frak{b}}\gamma\big|\,\big\langle f_{\frak{b}}
		\operadormatrices{\peso{k}}{\gamma},g_{\frak{a}}
		\big\rangle_{\Gamma_{\frak{a}}\cap\gamma^{-1}\Gamma_{\frak{b}}%
			\gamma} \\
	&\qquad\qquad\,=\,
	\big|\Gamma_{\frak{b}}:\gamma\Gamma_{\frak{a}}\gamma^{-1}
		\cap\Gamma_{\frak{b}}\big|\,\big\langle f_{\frak{b}},
		g_{\frak{a}}\operadormatrices{\peso{k}}{\gamma^\iota}
		\big\rangle_{\gamma\Gamma_{\frak{a}}\gamma^{-1}\cap%
			\Gamma_{\frak{b}}} \,=\,
	\Big\langle f_{\frak{b}},\sum_{i}\,\big(g_{\frak{a}}
		\operadormatrices{\peso{k}}{\gamma^\iota}\big)
		\operadormatrices{\peso{k}}{\tilde{\delta}_{i}^\iota}
		\Big\rangle_{\frak{b}} \\
	& \qquad\qquad\,=\,
	\Big\langle f_{\frak{b}},\sum_{i}\,g_{\frak{a}}
		\operadormatrices{\peso{k}}{\eta_{i}^\iota}
		\Big\rangle_{\frak{b}}
	\text{ .}
\end{align*}
%
Finalmente, reescribimos la sumatoria del \'{u}ltimo t\'{e}rmino:
\begin{align*}
	\sum_{i}\,\big(g_{\frak{a}}\operadormatrices{\peso{k}}{\eta_{i}^\iota}
		\big)(z) & \,=\,
		\sum_{i}\,J(\eta_{i}^\iota,z)^{-1}\,
			g_{\frak{a}}(\eta_{i}^\iota z) \,=\,
		\sum_{i}\,J(\eta_{i}^\iota,z)^{-1}\,g(\eta_{i}^\iota z,
			\hhat{\alpha}\Idfin{\cal{O}}) \\
	& \,=\,\sum_{i}\,g(z,\eta_{i}^{-\iota}\hhat{\alpha}\Idfin{\cal{O}})
		\,=\,\sum_{i}\,g(z,\hhat{\beta}\hhat{b}^{-1}\hhat{a}
			(\hhat{\beta}^{-1}\eta_{i}\hhat{\alpha})^{\iota (-1)}
				\Idfin{\cal{O}}) \\
	& \,=\,\big(T_{\frak{p}}g\big)(z,\hhat{\beta}\hhat{b}^{-1}\hhat{a}
		\Idfin{\cal{O}}) \,=\,
		\big(S_{\frak{p}}^{-1}T_{\frak{p}}g\big)(z,\lambda^{-1}
			\hhat{\beta}\Idfin{\cal{O}})
	\text{ ,}
\end{align*}
%
donde $\det(\eta_{i})=\lambda\xi_{i}$, con $\xi_{i}\in\oka{F,+}^{\times}$ y
$J(\eta,z)$ denota el producto de los factores de automorf\'{\i}a. La
\'{u}ltima expresi\'{o}n es igual a
\begin{math}
	\norma(\lambda)^{k_{0}-2}\,\big(S_{\frak{p}}^{-1}T_{\frak{p}}g
		\big)_{\frak{b}}(z)
\end{math}~.
En definitiva,
\begin{align*}
	\langle T_{\frak{p}}f,g\rangle & \,=\,\sum_{\frak{a}}\,
		\big\langle \big(T_{\frak{p}}f\big)_{\frak{a}},g_{\frak{a}}
			\big\rangle_{\frak{a}} \,=\,
		\sum_{\frak{b}}\,\norma(\lambda)^{k_{0}-2}\,
		\big\langle f_{\frak{b}},\big(S_{\frak{p}}^{-1}T_{\frak{p}}g
			\big)_{\frak{b}}
			\big\rangle_{\frak{b}}
	\text{ .}
\end{align*}
%
