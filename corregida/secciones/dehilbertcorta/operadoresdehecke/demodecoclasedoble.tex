La condici\'{o}n
\begin{math}
	\hhat{\alpha}\hhat{\pi}_{i}^{-1}\Idfin{\cal{O}}=
		\gamma_{i}^{-1}\hhat{\beta}\Idfin{\cal{O}}
\end{math}
equivale a
\begin{math}
	\gamma_{i}\in
		\hhat{\beta}\Idfin{\cal{O}}\hhat{\pi}_{i}\hhat{\alpha}^{-1}
\end{math}~.
En particular,
\begin{align*}
	\Gamma_{\frak{b}}\gamma_{i} & \,=\,\big(
		\Idfin{\cal{O}}_{\frak{b}}\,\cap\,\GLtp_{2}(F)
		\big)\gamma_{i} \,=\,
		\big(\hhat{\beta}\Idfin{\cal{O}}
			\hhat{\pi}_{i}\hhat{\alpha}^{-1}\big)
			\,\cap\,\GLtp_{2}(F)
	\text{ .}
\end{align*}
%
Si $\delta\in\Gamma_{\frak{b}}\gamma_{i}$ y
$\tilde{\gamma}\in\Gamma_{\frak{a}}$, entonces
$\delta\tilde{\gamma}\in\GLtp_{2}(F)$ y
$\tilde{\gamma}=\hhat{\alpha}\hhat{u}\hhat{\alpha}^{-1}$ para cierta unidad
$\hhat{u}\in\Idfin{\cal{O}}$. En particular,
\begin{align*}
	\delta\tilde{\gamma} & \,\in\,\hhat{\beta}\Idfin{\cal{O}}
		\hhat{\pi}_{i}\hhat{\alpha}^{-1}
		(\hhat{\alpha}\hhat{u}\hhat{\alpha}^{-1})\,=\,
		\hhat{\beta}\Idfin{\cal{O}}\hhat{v}
			\hhat{\pi}_{j}\hhat{\alpha}^{-1}\,=\,
		\hhat{\beta}\Idfin{\cal{O}}
			\hhat{\pi}_{j}\hhat{\alpha}^{-1}
	\text{ .}
\end{align*}
%
El elemento $\hhat{v}$ pertenece a $\Idfin{\cal{O}}$ y el \'{\i}ndice $j$
depende \'{u}nicamente de $\tilde{\gamma}$ y de $i$. Es decir, dado
$\tilde{\gamma}\in\Gamma_{\frak{a}}$, para cada $i$ existe un \'{u}nico $j$ tal
que
\begin{align*}
	\big(\Gamma_{\frak{b}}\gamma_{i}\big)\tilde{\gamma}
		& \,\subset\,\Gamma_{\frak{b}}\gamma_{j}
	\text{ .}
\end{align*}
%
En consecuencia,
\begin{math}
	\Gamma_{\frak{b}}\gamma_{i} =
		\big(\Gamma_{\frak{b}}\gamma_{i}\tilde{\gamma}\big)
			\tilde{\gamma}^{-1} \subset
		\big(\Gamma_{\frak{b}}\gamma_{j}\big)
			\tilde{\gamma}^{-1} \subset
		\Gamma_{\frak{b}}\gamma_{i'}
\end{math}~.
Pero, entonces $i'=i$ (tomando $\tilde{\gamma}=I$) y
\begin{math}
	\big(\Gamma_{\frak{b}}\gamma_{i}\big)\tilde{\gamma} =
		\Gamma_{\frak{b}}\gamma_{j}
\end{math}~.
De esto se deduce que
\begin{align*}
	\Gamma_{\frak{b}}\gamma_{i}\Gamma_{\frak{a}} & \,=\,
		\bigsqcup_{k}\,\Gamma_{\frak{b}}\gamma_{i}^{k}
	\text{ ,}
\end{align*}
%
con $\{\gamma_{i}^{k}\}_{k}\subset \{\gamma_{j}\}_{j}$. Por otro lado,
\begin{align*}
	\Idfin{\cal{O}}_{\frak{b}}\gamma_{i}\Gamma_{\frak{a}}
	& \,=\,\Idfin{\cal{O}}_{\frak{b}}\cdot
		\Gamma_{\frak{b}}\gamma_{i}\Gamma_{\frak{a}}
		\,=\,\bigsqcup_{k}\,\Idfin{\cal{O}}_{\frak{b}}
					\gamma_{i}^{k}
		\,=\,\hhat{\beta}\bigsqcup_{k}\,
			\Idfin{\cal{O}}\hhat{\pi}_{i}^{k}
				\hhat{\alpha}^{-1}
	\text{ .}
\end{align*}
%
La uni\'{o}n es disjunta:
\begin{math}
	\Idfin{\cal{O}}\hhat{\beta}\gamma_{i}^{k}=
		\Idfin{\cal{O}}\hhat{\beta}\gamma_{i}^{l}
\end{math}~,
si y s\'{o}lo si
\begin{math}
	\gamma_{i}^{l}(\gamma_{i}^{k})^{-1}\in
	\Idfin{\cal{O}}_{\frak{b}}\cap\GLtp_{2}(F)=\Gamma_{\frak{b}}
\end{math}~.
Para concluir la demostraci\'{o}n, afirmamos que
\begin{equation}
	\label{eq:coclasedoblelocalglobal}
	\Idfin{\cal{O}}_{\frak{b}}\gamma_{i}\Idfin{\cal{O}}_{\frak{a}}
	\,=\,\Idfin{\cal{O}}_{\frak{b}}\gamma_{i}\Gamma_{\frak{a}}
	\text{ .}
\end{equation}
%
Asumiendo que esto es cierto,
\begin{align*}
	\Idfin{\cal{O}}\hhat{\pi}_{i}\Idfin{\cal{O}}
	& \,=\,\hhat{\beta}^{-1}\Idfin{\cal{O}}_{\frak{b}}\gamma_{i}
			\Idfin{\cal{O}}_{\frak{a}}\hhat{\alpha}
		\,=\,\bigsqcup_{k}\,\Idfin{\cal{O}}\hhat{\pi}_{i}^{k}
	\text{ .}
\end{align*}
%
Pero
\begin{math}
	\Idfin{\cal{O}}\hhat{\pi}_{i}\Idfin{\cal{O}}=
	\Idfin{\cal{O}}\hhat{\pi}\Idfin{\cal{O}}
\end{math}~,
cualquiera sea $i$. As\'{\i}, $\{\hhat{\pi}_{i}^{k}\}=\{\hhat{\pi}_{j}\}_{j}$ y
$\{\gamma_{i}^{k}\}_{k}=\{\gamma_{j}\}_{j}$.

La demostraci\'{o}n de \eqref{eq:coclasedoblelocalglobal} est\'{a} adaptada de
\cite[Thm.~5.3.5]{MiyakeModular}. Para cada $\gamma_{i}$, existen
$\hhat{u}_{i}\in\Idfin{\cal{O}}$ tales que
$\gamma_{i}=\hhat{\beta}\hhat{u}_{i}\hhat{\pi}_{i}\hhat{\alpha}^{-1}$. Sea
$\hhat{h}\in\Idfin{\cal{O}}$. Demostraremos que
\begin{align*}
	\gamma_{i}(\hhat{\alpha}\hhat{h}\hhat{\alpha}^{-1})
	& \,\in\,\Idfin{\cal{O}}_{\frak{b}}\gamma_{i}\Gamma_{\frak{a}}
	\text{ .}
\end{align*}
%
Como $\hhat{\pi}_{i}\in\Idfin{\cal{O}}\hhat{\pi}\Idfin{\cal{O}}$, existen
unidades $\hhat{v},\hhat{w}\in\Idfin{\cal{O}}$ tales que
\begin{math}
	\hhat{v}\hhat{\pi}_{i}\hhat{w} =
		\begin{bmatrix} \hhat{p} & \\ & 1 \end{bmatrix}
\end{math}~;
como $\nrd(h)\in\Idfin{\oka{F}}$, el elemento
\begin{math}
	\hhat{k} =\hhat{w}\begin{bmatrix} \nrd(h)^{-1} & \\
		& 1 \end{bmatrix}\hhat{w}^{-1}
\end{math}
pertenece a $\Idfin{\cal{O}}$. Recordando que
$\cal{O}=\cal{O}_{0}(\frak{N})\subset\cal{O}_{0}(1)$, para cada divisor primo
$\frak{q}\mid\frak{N}\cdot\frak{p}$, existe una potencia suficientemente grande
$f\geq 1$, tal que
\begin{equation}
	\label{eq:potenciasufigrandecontenida}
\begin{aligned}
	\frak{q}^{f}\cdot\big(\gamma_{i}\alpha_{\frak{q}}
		\cal{O}_{0}(1)_{\frak{q}}\,(k_{\frak{q}}
			h_{\frak{q}})^{-1}\alpha_{\frak{q}}^{-1}
			\gamma_{i}^{-1}\big)
	& \,\subset\,\frak{q}\cdot\big(\beta_{\frak{q}}
		\cal{O}_{0}(\frak{N})_{\frak{q}}\beta_{\frak{q}}^{-1}
			\big) \quad\text{y} \\
	\frak{q}^{f}\cdot\big(\alpha_{\frak{q}}
		\cal{O}_{0}(1)_{\frak{q}}\alpha_{\frak{q}}^{-1}\big)
	& \,\subset\,\alpha_{\frak{q}}\cal{O}_{0}(\frak{N})_{\frak{q}}
		\alpha_{\frak{q}}^{-1}
	\text{ .}
\end{aligned}
\end{equation}
%
Como $\cal{O}_{0}(1)$ es un orden maximal en $\MM_{2\times 2}(F)$, podemos
aplicar el teorema de aproximaci\'{o}n \ref{thm:deaproximaciondelanorma} para
concluir que existe
\begin{math}
	\tilde{\gamma}\in\cal{O}_{0}(1)_{\frak{a}}
\end{math}
tal que $\nrd(\tilde{\gamma})=1$ y, para todo
$\frak{q}\mid\frak{N}\cdot\frak{p}$,
\begin{align*}
	\tilde{\gamma} & \,\equiv\,\alpha_{\frak{q}}
		(k_{\frak{q}}h_{\frak{q}})\alpha_{\frak{q}}^{-1}
		\quad\modulo \frak{q}^{f}\big(\alpha_{\frak{q}}
			\cal{O}_{0}(1)_{\frak{q}}\alpha_{\frak{q}}^{-1}
			\big)
	\text{ .}
\end{align*}
%
En particular, por \eqref{eq:potenciasufigrandecontenida}, para los primos
divisores de $\frak{N}\cdot\frak{p}$, la diferencia
\begin{math}
	\tilde{\gamma}-\alpha_{\frak{q}}
		(k_{\frak{q}}h_{\frak{q}})\alpha_{\frak{q}}^{-1}
\end{math}
pertenece a
\begin{math}
	\alpha_{\frak{q}}\cal{O}_{0}(\frak{N})_{\frak{q}}
		\alpha_{\frak{q}}^{-1}
\end{math}~,
pero, $k_{\frak{q}}h_{\frak{q}}\in\cal{O}_{0}(\frak{N})_{\frak{q}}$ implica
\begin{math}
	\tilde{\gamma}\in\alpha_{\frak{q}}
		\cal{O}_{0}(\frak{N})_{\frak{q}}\alpha_{\frak{q}}^{-1}
\end{math}~.
Como esto es cierto para todo primo divisor del nivel, $\frak{N}$,
\begin{math}
	\tilde{\gamma}\in\cal{O}_{0}(\frak{N})_{\frak{a}}
\end{math}~.
Como $\nrd(\tilde{\gamma})=1\in\oka{F,+}^{\times}$, se deduce que
\begin{align*}
	\tilde{\gamma} & \,\in\,\big(\hhat{\alpha}
		\Idfin{\cal{O}_{0}(\frak{N})}\hhat{\alpha}^{-1}\big)
		\,\cap\,\GLtp_{2}(F) \,=\,\Gamma_{\frak{a}}
	\text{ .}
\end{align*}
%
Sea, ahora, $z\in\GL_{2}(\Adfin{F})$ el elemento
\begin{align*}
	z & \,=\,\gamma_{i}\tilde{\gamma}\hhat{\alpha}
		(\hhat{k}\hhat{h})^{-1}\hhat{\alpha}^{-1}
			\gamma_{i}^{-1}
	\text{ .}
\end{align*}
%
Si $\frak{q}\not =\frak{p}$, entonces
\begin{math}
	z_{\frak{q}}\in\beta_{\frak{q}}\cal{O}^{\times}_{\frak{q}}
		\beta_{\frak{q}}^{-1}
\end{math}~.
Si, en cambio, $\frak{q}=\frak{p}$,
\begin{align*}
	z_{\frak{p}}-1 & \,=\,\gamma_{i}(\tilde{\gamma}
		\alpha_{\frak{p}}\,(k_{\frak{p}}h_{\frak{p}})^{-1}
		\alpha_{\frak{p}}^{-1} - 1)\,\gamma_{i}^{-1} \\
	& \,=\,\gamma_{i} (\tilde{\gamma}-
			\alpha_{\frak{p}}\,(k_{\frak{p}}h_{\frak{p}})
			\alpha_{\frak{p}}^{-1}) \,
		\alpha_{\frak{p}}(k_{\frak{p}}h_{\frak{p}})^{-1}
		\alpha_{\frak{p}}^{-1}\gamma_{i}^{-1}
	\text{ .}
\end{align*}
%
Pero,
\begin{math}
	\tilde{\gamma}-\alpha_{\frak{p}} (k_{\frak{p}}h_{\frak{p}})
		\alpha_{\frak{p}}^{-1}
\end{math}
pertenece a
\begin{math}
	\frak{p}^{f}\cdot\big(\alpha_{\frak{p}}
		\cal{O}_{0}(1)^{\times}_{\frak{p}}
		\alpha_{\frak{p}}^{-1}\big)
\end{math}
y
\begin{align*}
	z_{\frak{p}}-1 & \,\in\,\frak{p}\cdot \big(
		\beta_{\frak{p}}\cal{O}^{\times}_{\frak{p}}
			\beta_{\frak{p}}^{-1}\big)
	\text{ .}
\end{align*}
%
De esto se deduce que
\begin{math}
	z_{\frak{p}}\in\beta_{\frak{p}}\cal{O}^{\times}_{\frak{p}}
		\beta_{\frak{p}}^{-1}
\end{math}~,
tambi\'{e}n. En definitiva,
\begin{math}
	z\in\hhat{\beta}\Idfin{\cal{O}}_{\frak{b}}
\end{math}~.
Por otra parte,
\begin{align*}
	\gamma_{i} (\hhat{\alpha}\hhat{k}\hhat{\alpha}^{-1})
	& \,=\,\hhat{\beta}\hhat{u}_{i} \hhat{v}^{-1}
		\begin{bmatrix}\hhat{p} & \\ & 1\end{bmatrix}
		\begin{bmatrix} \nrd(h)^{-1} & \\ & 1\end{bmatrix}
		\hhat{w}^{-1}\hhat{\alpha}^{-1} \\
	& \,=\,\hhat{\beta}\hhat{u}_{i}\hhat{v}^{-1}
		\begin{bmatrix} \nrd(h)^{-1} & \\ & 1\end{bmatrix}
		\hhat{v}\hhat{u}_{i}^{-1}\hhat{\beta}^{-1}
		\gamma_{i}
	\text{ .}
\end{align*}
%
Pero el factor a la izquierda de $\gamma_{i}$ pertenece a
$\Idfin{\cal{O}}_{\frak{b}}$. En consecuencia,
\begin{align*}
	\gamma_{i} (\hhat{\alpha}\hhat{k}\hhat{\alpha}^{-1})
		(\hhat{\alpha}\hhat{h}\hhat{\alpha}^{-1})
	& \,=\, z^{-1}\gamma_{i}\tilde{\gamma}\,\in\,
		\Idfin{\cal{O}}_{\frak{b}}\gamma_{i}\Gamma_{\frak{a}}
		\quad\text{y} \\
	\gamma_{i} (\hhat{\alpha}\hhat{h}\hhat{\alpha}^{-1})
	& \,\in\,\Idfin{\cal{O}}_{\frak{b}}\gamma_{i}
		(\hhat{\alpha}\hhat{k}\hhat{\alpha}^{-1})
		(\hhat{\alpha}\hhat{h}\hhat{\alpha}^{-1})
		\,\subset\,\Idfin{\cal{O}}_{\frak{b}}\gamma_{i}
			\Gamma_{\frak{a}}
	\text{ .}
\end{align*}
%

