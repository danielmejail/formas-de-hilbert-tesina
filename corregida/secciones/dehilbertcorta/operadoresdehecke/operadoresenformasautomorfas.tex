% Un operador de coclase doble es, simplemente, una transformaci\'{o}n
% determinada por/asociada a un objeto de la forma $K_{1}\hhat{\pi}K_{2}$ y
% calculada/expresada en t\'{e}rminos de una descomposici\'{o}n de la forma
% $K_{1}\hhat{\pi}K_{2}=\bigsqcup_{i}\,K_{1}\hhat{\pi}_{i}$.
%%
%% NO S\'E SI LLAMAR A ESTOS OPERADORES `DE COCLASE DOBLE'. LOS DE COCLASE
%% DOBLE SON, MEJOR DICHO LOS \Gamma_b\gamma\Gamma_a
% 
Sea $\cal{O}=\cal{O}_{0}(\frak{N})$ el orden dado por la expresi\'{o}n
\eqref{eq:ordendeeichlermatrices} y sea $\hhat{\pi}\in\GL_{2}(\Adfin{F})$. El
grupo de unidades $\Idfin{\cal{O}}$ act\'{u}a a izquierda en el conjunto
$\Idfin{\cal{O}}\hhat{\pi}\Idfin{\cal{O}}$ por multiplicaci\'{o}n. En
particular, eligiendo un sistema de representantes, $\{\hhat{\pi}_{i}\}_{i}$,
es posible descomponer $\Idfin{\cal{O}}\hhat{\pi}\Idfin{\cal{O}}$ en una
uni\'{o}n disjunta:
\begin{equation}
	\label{eq:descomposiciondecoclasedoble}
	\Idfin{\cal{O}}\hhat{\pi}\Idfin{\cal{O}} \,=\,\bigsqcup_{i}\,
		\Idfin{\cal{O}}\hhat{\pi}_{i}
	\text{ .}
\end{equation}
%
Ahora bien, la multiplicaci\'{o}n
\begin{math}
	(\hhat{\alpha},\hhat{\beta})\mapsto\hhat{\alpha}\hhat{\beta}
\end{math} en $\GL_{2}(\Adfin{F})$ es una aplicaci\'{o}n continua, las
traslaciones $\hhat{\alpha}\mapsto\hhat{\alpha}\hhat{\beta}$ son homeomorfismos
y el grupo de unidades $\Idfin{\cal{O}}$ es un subespacio abierto y compacto.
Por lo tanto, el conjunto $\Idfin{\cal{O}}\hhat{\pi}\Idfin{\cal{O}}$, producto
de los compactos $\Idfin{\cal{O}}$ y $\hhat{\pi}\Idfin{\cal{O}}$, es compacto y
$\big\{\Idfin{\cal{O}}\hhat{\pi}_{i}\big\}_{i}$ es un cubrimiento por abiertos
disjuntos. En consecuencia, la uni\'{o}n
\eqref{eq:descomposiciondecoclasedoble} es finita.

Dada una forma de Hilbert $\phi:\,\GL_{2}(\adeles{F})\rightarrow\bb{C}$ (no
necesariamente cuspidal) de peso $\peso{k}$ y nivel $\frak{N}$, se define una
nueva funci\'{o}n en $\GL_{2}(\adeles{F})$ por la expresi\'{o}n
\begin{equation}
	\label{eq:operadordecoclasedobledehilbert}
	\phi\operadorcoclases{\peso{k}}{%
		\Idfin{\cal{O}}\hhat{\pi}\Idfin{\cal{O}}}(g,\hhat{\alpha})
	\,=\,\sum_{i}\,\big(
		\phi\operadormatrices{\peso{k}}{(1,\hhat{\pi}_{i})}
			\big)(g,\hhat{\alpha})
	\,=\, \sum_{i}\,\phi(g,\hhat{\alpha}\hhat{\pi}_{i}^{-1})
	\text{ .}
\end{equation}
%
Como $\phi$ es invariante por la acci\'{o}n de $\Idfin{\cal{O}}$ a derecha,
esta nueva funci\'{o}n est\'{a} bien definida en el sentido de que la suma es
finita y no depende de los representantes $\hhat{\pi}_{i}$ elegidos.

\begin{propoCoclaseDoblePreservaElEspacioDeFormasCuspidales}
	\label{propo:coclasedoblepreservaelespaciodeformascuspidales}
	Sea $\phi:\,\GL_{2}(\adeles{F})\rightarrow\bb{C}$ una funci\'{o}n
	con las propiedades (i) a (vi) de la
	Proposici\'{o}n~\ref{propo:equivalenciaautomorfasformascuspidales} y
	sea $\hhat{\pi}\in\GL_{2}(\Adfin{F})$. Entonces la funci\'{o}n
	ad\'{e}lica
	\begin{math}
		\phi\operadorcoclases{\peso{k}}{%
			\Idfin{\cal{O}}\hhat{\pi}\Idfin{\cal{O}}}
	\end{math}
	tambi\'{e}n verifica (i) a (vi). Si, adem\'{a}s, $\phi$ verifica
	(vii), es decir, es cuspidal, entonces la nueva funci\'{o}n tambi\'{e}n
	es cuspidal.
\end{propoCoclaseDoblePreservaElEspacioDeFormasCuspidales}

En otras palabras,
\begin{math}
	\phi\mapsto\phi\operadorcoclases{\peso{k}}{%
		\Idfin{\cal{O}}\hhat{\pi}\Idfin{\cal{O}}}
\end{math}
define un operador en el espacio $\modularH{k}{\frak{N}}$ de formas modulares
de Hilbert que se restringe a un operador en el subespacio
$\spitzH{k}{\frak{N}}$ de formas cuspidales. Estos operadores tambi\'{e}n
respetan la descomposici\'{o}n
\eqref{eq:descomposicioncuspidalescuasicaracteres} en t\'{e}rminos de los
cuasicaracteres $\omega:\,\ideles{F}/F^{\times}\rightarrow\bb{C}^{\times}$.

\begin{proof}
	Sea $\phi:\,\GL_{2}(\adeles{F})\rightarrow\bb{C}$ una
	funci\'{o}n que satisface las propiedades (i) a (vi) de
	\ref{propo:equivalenciaautomorfasformascuspidales}. Sea
	$\gamma\in\GL_{2}(F)$. Entonces
	\begin{align*}
		\phi\operadorcoclases{\peso{k}}{%
			\Idfin{\cal{O}}\hhat{\pi}\Idfin{\cal{O}}}
			(\gamma g,\gamma\hhat{\alpha}) & \,=\,
			\sum_{i}\,\phi(\gamma g,
				\gamma\hhat{\alpha}\hhat{\pi}_{i}^{-1})
			\,=\,\sum_{i}\,\phi(g,\hhat{\alpha}\hhat{\pi}_{i}^{-1})
		\text{ ,}
	\end{align*}
	%
	de lo que se deduce que se cumple (i). Por otro lado, de
	\eqref{eq:operadordecoclasedobledehilbert}, se deduce que la
	aplicaci\'{o}n
	\begin{math}
		\phi\mapsto\phi\operadorcoclases{\peso{k}}{%
			\Idfin{\cal{O}}\hhat{\pi}\Idfin{\cal{O}}}
	\end{math}
	solamente afecta las coordenadas no arquimedianas de $\phi$. Por lo
	tanto, las propiedades (iii), (iv) y (vi) se preservan. En cuanto a
	(ii), si $\hhat{\beta}\in\Idfin{\cal{O}}$, entonces
	\begin{align*}
		\phi\operadorcoclases{\peso{k}}{%
			\Idfin{\cal{O}}\hhat{\pi}\Idfin{\cal{O}}}
			(g,\hhat{\alpha}\hhat{\beta}^{-1}) & \,=\,
			\sum_{i}\,\phi(g,\hhat{\alpha}\hhat{\beta}^{-1}
				\hhat{\pi}_{i}^{-1})
			\,=\,\sum_{i}\,\phi(g,\hhat{\alpha}
				(\hhat{\pi}_{i}\hhat{\beta})^{-1})
		\text{ .}
	\end{align*}
	%
	Pero el conjunto $\{\hhat{\pi}_{i}\hhat{\beta}\}_{i}$ sigue siendo un
	conjunto de representantes de las clases en
	\begin{math}
		\Idfin{\cal{O}}\backslash
			\Idfin{\cal{O}}\hhat{\pi}\Idfin{\cal{O}}
	\end{math}~,
	ya que $\hhat{\pi}_{i}\hhat{\beta}$ pertenece a
	$\Idfin{\cal{O}}\hhat{\pi}_{j}\hhat{\beta}$, si y s\'{o}lo si
	$\hhat{\pi}_{i}\hhat{\pi}_{j}^{-1}$ pertenece a $\Idfin{\cal{O}}$ y la
	finitud de la descomposici\'{o}n
	\eqref{eq:descomposiciondecoclasedoble} implica que la aplicaci\'{o}n
	inyectiva
	\begin{math}
		\Idfin{\cal{O}}\hhat{\pi}_{i}\mapsto
			\Idfin{\cal{O}}\hhat{\pi}_{i}\hhat{\beta}
	\end{math}
	sea, tambi\'{e}n, sobreyectiva. Por otra parte, la finitud de la
	sumatoria correspodiente \eqref{eq:operadordecoclasedobledehilbert}
	implica (v) y, si, adem\'{a}s, $\phi$ cumple con (vii) entonces
	\begin{math}
		\phi\operadorcoclases{\peso{k}}{%
			\Idfin{\cal{O}}\hhat{\pi}\Idfin{\cal{O}}}
	\end{math}
	tambi\'{e}n posee esta propiedad.
\end{proof}

\begin{obsOperadorCoclaseAdelesAdjunto}%
	\label{obs:operadorcoclaseadelesadjunto}
	Sea $\hhat{\pi}\in\GL_{2}(\Adfin{F})$. Sea $\omega$ un
	cuasicar\'{a}cter de $\ideles{F}$ trivial en $F^{\times}$ y sean
	$\phi,\phi'\in\spitzH{k}{\frak{N},\omega}$. Entonces
	\begin{align*}
		\Big\langle \phi\operadorcoclases{\peso{k}}{%
			\Idfin{\cal{O}}\hhat{\pi}\Idfin{\cal{O}}},\phi'
			\Big\rangle & \,=\,|\omega(\det\,\hhat{\pi})|^{-1}\,
		\Big\langle\phi,\phi'\operadorcoclases{\peso{k}}{%
				\Idfin{\cal{O}}\hhat{\pi}^{-1}\Idfin{\cal{O}}}
			\Big\rangle \\
		& \,=\,\lconj{\chi(\det\,\hhat{\pi})}\,
			\Big\langle\phi,\phi'\operadorcoclases{\peso{k}}{%
				\Idfin{\cal{O}}\hhat{\pi}^\iota\Idfin{\cal{O}}}
			\Big\rangle
		\text{ .}
	\end{align*}
	%
	Esto se deduce de la expresi\'{o}n
	\eqref{eq:operadordecoclasedobledehilbert} para el operador, de la
	Observaci\'{o}n~\ref{obs:operadordepesokadjunto}, elegiendo un sistema
	de representantes $\{\hhat{\pi}_{i}\}_{i}$ tal que
	\begin{math}
		\Idfin{\cal{O}}\hhat{\pi}\Idfin{\cal{O}}=
		\bigsqcup_{i}\,\Idfin{\cal{O}}\hhat{\pi}_{i}=
		\bigsqcup_{i}\,\hhat{\pi}_{i}\Idfin{\cal{O}}
	\end{math}~, y de que $\omega(\det\,\hhat{x})=\omega(\det\,\hhat{\pi})$
	para todo
	\begin{math}
		\hhat{x}\in\Idfin{\cal{O}}\hhat{\pi}\Idfin{\cal{O}}
	\end{math}~.
	% Recordando que el conjugado de un elemento $\hhat{x}$ es
	% $\hhat{x}^\iota=\det(\hhat{x})\,\hhat{x}^{-1}$, deducimos que
	% (si $\chi=|\omega|^{-1}\,\omega$),
	% \begin{align*}
		% \Big\langle\phi\operadorcoclases{\peso{k}}{%
			% \Idfin{\cal{O}}\hhat{\pi}\Idfin{\cal{O}}},
				% \phi'\Big\rangle & \,=\,
		% \chi(\det\,\hhat{\pi})\,
			% \Big\langle\phi,\phi'\operadorcoclases{\peso{k}}{%
				% \Idfin{\cal{O}}\hhat{\pi}^\iota\Idfin{\cal{O}}}
				% \Big\rangle
		% \text{ .}
	% \end{align*}
	% %
\end{obsOperadorCoclaseAdelesAdjunto}

Usando el isomorfismo de la Proposici\'{o}n~%
\ref{propo:equivalenciaautomorfasformascuspidales}, se traducen las
definiciones anteriores al contexto de formas de Hilbert cl\'{a}sicas. Sea
$f\in\modularH{k}{\frak{N}}$ una forma de Hilbert y sea $\phi_{f}$ la
funci\'{o}n ad\'{e}lica correspondiente. Entonces
\begin{align*}
	\phi_{f}\operadorcoclases{\peso{k}}{%
		\Idfin{\cal{O}}\hhat{\pi}\Idfin{\cal{O}}} (g,\hhat{\alpha})
	& \,=\, \sum_{i}\,\phi_{f}(g,\hhat{\alpha}\hhat{\pi}_{i}^{-1})
	\,=\, \sum_{i}\,\bigg(\prod_{t=1}^{n}\,
			J_{t}(g_{t},\sqrt{-1})^{-1}\bigg)\,
		f(g\cdot\mathbf{i},
			\hhat{\alpha}\hhat{\pi}_{i}^{-1}\Idfin{\cal{O}})
	\text{ .}
\end{align*}
%
Se define una funci\'{o}n
\begin{math}
	T_{\hhat{\pi}}f:\,(\hP^{\pm})^{n}\times\big(
			\GL_{2}(\Adfin{F})/\Idfin{\cal{O}}\big)
		\rightarrow\bb{C}
\end{math}
por
\begin{align*}
	\big(T_{\hhat{\pi}}f\big)(z,\hhat{\alpha}\Idfin{\cal{O}}) & \,=\,
		\sum_{i}\,f(z,\hhat{\alpha}\hhat{\pi}_{i}^{-1}\Idfin{\cal{O}})
	\text{ .}
\end{align*}
%
Entonces $T_{\hhat{\pi}}f$ es una forma de Hilbert tambi\'{e}n y
$f\mapsto T_{\hhat{\pi}}f$ define un operador en $\modularH{k}{\frak{N}}$.
% Dicho de otra manera, $T_{\hhat{\pi}}f$ es la forma modular correspondiente a
% \begin{math}
	% \phi\operadorcoclases{\peso{k}}{%
			% \Idfin{\cal{O}}\hhat{\pi}\Idfin{\cal{O}}}
% \end{math}~.
Por la Observaci\'{o}n~\ref{obs:operadorcoclaseadelesadjunto}, el adjunto del
operador $T_{\hhat{\pi}}$ est\'{a} dado por
\begin{equation}
	\label{eq:operadorcoclaseformasadjunto}
	T_{\hhat{\pi}}^{*}\,=\,\chi(\det\,\hhat{\pi})\,T_{\hhat{\pi}^\iota}
	\text{ .}
\end{equation}
%
