% Sea $\hhat{\pi}\in\GL_{2}(\Adfin{F})$ y sea
% $\hhat{\pi}'\in\Idfin{\cal{O}}\hhat{\pi}\Idfin{\cal{O}}$. Como la norma
% reducida de un elemento en el grupo $\Idfin{\cal{O}}$ es una unidad en el
% anillo $\Adfin{\oka{F}}$, el ideal determinado por el id\`{e}le
% $\nrd(\hhat{\pi}')$ es igual al ideal determinado por $\nrd(\hhat{\pi})$.
% En particular, se puede decir que este ideal est\'{a} determinado por la
% coclase doble $\Idfin{\cal{O}}\hhat{\pi}\Idfin{\cal{O}}$.
%
Supongamos elegidos un sistema de representantes $\{\frak{a}\}$ de las clases
estrictas de $F$, $\hhat{a}\in\Idfin{F}$ tales que
$\frak{a}=\hhat{a}\Adfin{\oka{F}}\cap F$ y elementos $\hhat{\alpha}$ como en
\eqref{eq:matrizasociadaaidelematrices}, de manera que
$\nrd(\hhat{\alpha})=\hhat{a}$. Dado un representante $\frak{a}$, definimos:
\begin{align*}
	\Adfin{\cal{O}}_{\frak{a}}\,=\,
		\hhat{\alpha}\Adfin{\cal{O}}\hhat{\alpha}^{-1} & \text{ ,}\quad
	\cal{O}_{\frak{a}}\,=\,\Adfin{\cal{O}}_{\frak{a}}\,\cap\,
		\MM_{2\times 2}(F) \quad\text{y}\quad
	\Gamma_{\frak{a}} \,=\,\Gamma_{0}(\frak{N},\frak{a})
		\,=\,\Idfin{\cal{O}}_{\frak{a}}\,\cap\,\GLtp_{2}(F)
	\text{ .}
\end{align*}
%

Sean $\frak{a}$ un representante, $\frak{p}$ un primo que no divide al nivel
$\frak{N}$ de $\cal{O}$ y $\frak{b}$ el ideal en el mismo conjunto de
representantes de las clases estrictas tal que
\begin{math}
	[\frak{b}]=[\frak{a}][\frak{p}]^{-1}
\end{math}
en $\pClass{F}$. Sean $\hhat{b}$ y $\hhat{\beta}$ el id\`{e}le y la matriz
correspondientes al ideal $\frak{b}$. Para cada representante $\hhat{\pi}$ de
las \'{o}rbitas en $\frak{I}(\frak{p})$, se cumple que
\begin{math}
	[\nrd(\hhat{\alpha}\hhat{\pi}^{-1})]=[\frak{a}][\frak{p}]^{-1}=
		[\frak{b}]
\end{math}~.
Por el Corolario~\ref{coro:normaclasesestrictas}, existe
$\gamma\in\GLtp_{2}(F)$ tal que
\begin{math}
	\hhat{\alpha}\hhat{\pi}^{-1}\Idfin{\cal{O}}=
		\gamma^{-1}\hhat{\beta}\Idfin{\cal{O}}
\end{math}~.
Dada $f\in\modularH{k}{\frak{N}}$, la forma $T_{\frak{p}}f$ est\'{a}
determinada por sus proyecciones en cada uno de los espacios
$\modularH{k}{\frak{N},\frak{a}}$. De lo mencionado anteriormente y de la
invarianza de $f$, se deduce que
\begin{align*}
	\big(T_{\frak{p}}f\big)_{\frak{a}} (z) & \,=\,
		\big(T_{\frak{p}}f\big)(z,\hhat{\alpha}\Idfin{\cal{O}}) \,=\,
		\sum_{\hhat{\pi}}\,
			f(z,\hhat{\alpha}\hhat{\pi}^{-1}\Idfin{\cal{O}}) \,=\,
		\sum_{\gamma}\,f(z,\gamma^{-1}\hhat{\beta}\Idfin{\cal{O}}) \\
	& \,=\,\sum_{\gamma}\,
		\bigg(\prod_{i=1}^{n}\,J_{i}(\gamma_{i},z_{i})^{-1}\bigg)\,
		f(\gamma z,\hhat{\beta}\Idfin{\cal{O}}) \,=\,
		\sum_{\gamma}\,\big(f_{\frak{b}}\operadormatrices{\peso{k}}{%
				\gamma}\big)(z)
	\text{ ,}
\end{align*}
%
donde $\hhat{\pi}$ recorre un sistema de representantes de $\Theta(\frak{p})$.
Es decir, esencialmente, los operadores de Hecke permutan las componentes de
una forma de Hilbert. Esto justifica considerar conjuntamente todas las
variedades $Y_{0}(\frak{N},\frak{a})$.

\begin{obsIdelesDeNormaPPorGlobales}\label{obs:idelesdenormapporglobales}
	Dados $\frak{a}$ y $\frak{b}$ como arriba, sea
	\begin{align*}
		\frak{I}(\frak{p})_{\frak{a},\frak{b}} & \,=\,\Big\{
			\delta\in\GLtp_{2}(F)\,:\,
			\hhat{\beta}^{-1}\delta\hhat{\alpha}\in\Adfin{\cal{O}}
			\text{ y }
			\det(\delta)\frak{a}=\frak{b}\frak{p} \Big\}
		\text{ .}
	\end{align*}
	%
	Llamemos $\hhat{\pi}_{i}$ a los representantes de $\Theta(\frak{p})$ y
	$\gamma_{i}$ a las matrices correspondientes en $\GLtp_{2}(F)$. Las
	matrices $\delta\in\GLtp_{2}(F)$ que verifican
	\begin{math}
		\hhat{\alpha}\hhat{x}^{-1}\Idfin{\cal{O}}=
			\delta^{-1}\hhat{\beta}\Idfin{\cal{O}}
	\end{math}
	para alg\'{u}n $\hhat{x}\in\frak{I}(\frak{p})$ son aquellas tales que
	\begin{math}
		\hhat{\beta}^{-1}\delta\hhat{\alpha}\in\frak{I}(\frak{p})
	\end{math}~,
	es decir, por la Observaci\'{o}n~\ref{obs:idelesdenormap},
	$\hhat{\beta}^{-1}\delta\hhat{\alpha}\in\Adfin{\cal{O}}$ y
	$\det(\delta)\frak{a}=\frak{b}\frak{p}$. Es decir,
	\begin{align*}
		\frak{I}(\frak{p})_{\frak{a},\frak{b}} & \,=\,
			\hhat{\beta}\frak{I}(\frak{p})\hhat{\alpha}^{-1}\,\cap\,
			\GLtp_{2}(F)
		\text{ .}
	\end{align*}
	%
	Las matrices $\gamma_{i}$ pertencen a este conjunto. Si
	$\delta\in\frak{I}(\frak{p})_{\frak{a},\frak{b}}$, existe un
	representante $\hhat{\pi}_{i}$ tal que
	$\hhat{\beta}^{-1}\delta\hhat{\alpha}\in\Idfin{\cal{O}}\hhat{\pi}_{i}$.
	Pero $\hhat{\beta}^{-1}\gamma_{i}\hhat{\alpha}$ tambi\'{e}n pertenece a
	esta clase, lo que implica que
	\begin{align*}
		\delta & \,\in\,\big(\hhat{\beta}\Idfin{\cal{O}}
			\hhat{\beta}^{-1}\big)\,\gamma_{i}\,\cap\,\GLtp_{2}(F)
		\,=\,\Gamma_{\frak{b}}\gamma_{i}
		\text{ .}
	\end{align*}
	%
	Rec\'{\i}procamente, si $\gamma'\in\Gamma_{\frak{b}}$, entonces
	\begin{math}
		\hhat{\beta}^{-1}\gamma'\gamma_{i}\hhat{\alpha}\in
		\Idfin{\cal{O}}\hhat{\beta}^{-1}\gamma_{i}\hhat{\alpha}
	\end{math}~,
	que est\'{a} contenido en el conjunto $\frak{I}(\frak{p})$. En
	definitiva,
	\begin{align*}
		\frak{I}(\frak{p})_{\frak{a},\frak{b}} & \,=\,
			\bigcup_{i}\Gamma_{\frak{b}}\gamma_{i}
		\text{ .}
	\end{align*}
	%
	La Proposici\'{o}n~\ref{propo:descomposicioninducedescomposicion}
	garantiza que esta uni\'{o}n es, en realidad, disjunta e igual a una
	coclase doble en $\GLtp_{2}(F)$.
\end{obsIdelesDeNormaPPorGlobales}

\begin{propoDescomposicionInduceDescomposicion}%
	\label{propo:descomposicioninducedescomposicion}
	Sea $\frak{p}\nmid\frak{N}$ un ideal primo. Sean $\frak{a}$ y
	$\frak{b}$ dos representantes de las clases estrictas de $F$ tales que
	$[\frak{a}][\frak{p}]^{-1}=[\frak{b}]$. Sea $\{\hhat{\pi}_{i}\}_{i}$
	un sistema de representantes de las clases en
	\begin{math}
		\Idfin{\cal{O}}\backslash
			\Idfin{\cal{O}}\hhat{\pi}\Idfin{\cal{O}}
	\end{math}
	y, para cada $i$, sea $\gamma_{i}\in\GLtp_{2}(F)$ tal que
	\begin{math}
		\hhat{\alpha}\hhat{\pi}_{i}^{-1}\Idfin{\cal{O}}=
			\gamma_{i}^{-1}\hhat{\beta}\Idfin{\cal{O}}
	\end{math}~.
	Entonces
	\begin{align*}
		\frak{I}(\frak{p})_{\frak{a},\frak{b}} & \,=\,
			\bigsqcup_{i}\,\Gamma_{\frak{b}}\gamma_{i} \,=\,
			\Gamma_{\frak{b}}\gamma\Gamma_{\frak{a}}
		\text{ ,}
	\end{align*}
	%
	donde $\gamma$ es cualquier elemento en el conjunto
	$\frak{I}(\frak{p})_{\frak{a},\frak{b}}$.
\end{propoDescomposicionInduceDescomposicion}

\begin{proof}
	Ver \cite[Thm.~5.3.5]{MiyakeModular} para el caso $F=\bb{Q}$,
	% La demostraci\'{o}n se puede adaptar usando el Teorema de
	% aproximaci\'{o}n \ref{thm:deaproximaciondelanorma}. Comparar
	o bien \cite[Propo.~2.3]{ShimuraDirichletSeriesAbelianVarieties}.
	% \input{./secciones/dehilbertcorta/operadoresdehecke/%
	% demodecoclasedoble.tex}
	%% EN CASO DE INCLUIRLA, DEJARLA PARA EL FINAL.
\end{proof}

Rec\'{\i}procamente, dada cualquier
$\gamma\in\frak{I}(\frak{p})_{\frak{a},\frak{b}}$, si $\{\gamma_{i}\}_{i}$ es un
sistema de representantes de las clases en
\begin{math}
	\Gamma_{\frak{b}}\backslash\Gamma_{\frak{b}}\gamma\Gamma_{\frak{a}}
\end{math}~,
entonces $\{\hhat{\pi}_{i}=\hhat{\beta}^{-1}\gamma_{i}\hhat{\alpha}\}_{i}$ es
un sistema de representantes de las clases en
\begin{math}
	\Idfin{\cal{O}}\backslash\Idfin{\cal{O}}\hhat{\pi}\Idfin{\cal{O}}
\end{math}~.
Dicho de otra manera, el grupo de unidades $\Gamma_{\frak{b}}$ act\'{u}a a
izquierda en $\frak{I}(\frak{p})_{\frak{a},\frak{b}}$ y la
Proposici\'{o}n~\ref{propo:descomposicioninducedescomposicion} afirma que la
aplicaci\'{o}n
\begin{math}
	\frak{I}(\frak{p})_{\frak{a},\frak{b}}\rightarrow
		\frak{I}(\frak{p})
\end{math}
dada por $\delta\mapsto\hhat{\beta}^{-1}\delta\hhat{\alpha}$ induce una
biyecci\'{o}n
\begin{align*}
	& \Gamma_{\frak{b}}\backslash\frak{I}(\frak{p})_{\frak{a},\frak{b}}
		\,\xrightarrow{\sim}\,
		\Idfin{\cal{O}}\backslash\frak{I}(\frak{p})
	\text{ .}
\end{align*}
%
Denotamos por $\Theta(\frak{p})_{\frak{a},\frak{b}}$ el conjunto de \'{o}rbitas
$\Gamma_{\frak{b}}\backslash\frak{I}(\frak{p})_{\frak{a},\frak{b}}$.

\begin{obsHeckeActuaPorBloquesMatrices}\label{obs:heckeactuaporbloquesmatrices}
	Los operadores de Hecke
	\begin{math}
		T_{\frak{p}}:\,\modularH{k}{\frak{N}}\rightarrow
			\modularH{k}{\frak{N}}
	\end{math}
	act\'{u}an por bloques permutando los sumandos de la descomposici\'{o}n
	\eqref{eq:descomposicionmodularesmatrices}. Dado un ideal primo
	$\frak{p}\nmid\frak{N}$ e ideales $\frak{a}$ y $\frak{b}$ tales que
	$[\frak{b}]=[\frak{a}][\frak{p}]^{-1}$ en $\pClass{F}$, definimos
	\begin{align*}
		(T_{\frak{p}})_{\frak{a},\frak{b}} & \,:\,
			\modularH{k}{\frak{N},\frak{b}}\,\rightarrow\,
			\modularH{k}{\frak{N},\frak{a}}
	\end{align*}
	%
	por
	\begin{equation}
		\label{eq:heckeporbloquesmatrices}
		(T_{\frak{p}})_{\frak{a},\frak{b}}f \,=\,
			\sum_{\gamma\in\Theta(\frak{p})_{\frak{a},\frak{b}}}\,
				f\operadormatrices{\peso{k}}{\gamma}
		\text{ ,}
	\end{equation}
	%
	Entonces, dada $f=(f_{\frak{a}})_{\frak{a}}\in\modularH{k}{\frak{N}}$,
	se cumple
	\begin{math}
		\big(T_{\frak{p}}f\big)_{\frak{a}}=
			(T_{\frak{p}})_{\frak{a},\frak{b}}f_{\frak{b}}
	\end{math}~.
	El operador $T_{\frak{p}}$ act\'{u}a como la matriz de operadores
	$\big[(T_{\frak{p}})_{\frak{a},\frak{b}}\big]_{\frak{a},\frak{b}}$,
	donde $\frak{a}$ y $\frak{b}$ recorren los representantes de las clases
	estrictas de $F$ y $(T_{\frak{p}})_{\frak{a},\frak{b}}$ es el operdor
	definido por \eqref{eq:heckeporbloquesmatrices}, si
	$[\frak{a}]=[\frak{b}\frak{p}]$, y es igual a cero, en otro caso.
	En t\'{e}rminos de una descomposici\'{o}n
	\begin{math}
		\Gamma_{\frak{b}}\gamma\Gamma_{\frak{a}}=\bigsqcup_{i}\,
			\Gamma_{\frak{b}}\gamma_{i}
	\end{math}~,
	vale que
	\begin{math}
		(T_{\frak{p}})_{\frak{a},\frak{b}}f=
			f\operadorcoclases{\peso{k}}{%
				\Gamma_{\frak{b}}\gamma\Gamma_{\frak{a}}}
	\end{math}~,
	donde
	\begin{equation}
		\label{eq:coclasedobleporbloquesmatrices}
		f\operadorcoclases{\peso{k}}{%
			\Gamma_{\frak{b}}\gamma\Gamma_{\frak{a}}} \,=\,
			\sum_{i}\,f\operadormatrices{\peso{k}}{\gamma_{i}}
		\text{ .}
	\end{equation}
	%
\end{obsHeckeActuaPorBloquesMatrices}

%% Realci\'on con ideales a izquierda o a derecha en O_a o en O_b.
%% ?`Hay n(p)+1 de ellos en los cuatro casos, para p\nmid N?
%% De ser as\'\i, se puede relacionar con el p.i. de Petersson y deducir que
%% T_p es autoadjunto v\'\ia los (T_p)_a,b.
%% En este lugar hab\'\ia una observaci\'on al respecto de esto (ver versiones
%% anteriores).
