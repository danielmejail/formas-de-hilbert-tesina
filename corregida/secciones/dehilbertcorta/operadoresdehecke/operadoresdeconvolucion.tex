Temporariamente, para no sobrecargar la notaci\'{o}n, denotamos
$\MM_{2\times 2}(F)$ por $B$. Las combinaciones usuales mantienen el mismo
significado, por ejemplo, $B_{v}^{\times}=\GL_{2}(F_{v})$, en este caso.

% DE MANTENER TODA LA EXPLICACI\'ON, SER\'A MEJOR INTRODUCIR LA NOTACI\'ON
% PARA EL OPERADOR AL PRINCIPIO
A continuaci\'{o}n reinterpretamos el operador asociado a
$\Idfin{\cal{O}}\hhat{\pi}\Idfin{\cal{O}}$ como un operador de convoluci\'{o}n.
Para cada lugar finito $v\in\lugares[f]{F}$, sea $du_{v}^{\times}$ una
medida de Haar en $B_{v}^{\times}$. Por ejemplo, para fijar alguna referencia,
podemos suponer que $du_{v}^{\times}=du_{v}/|\nrd(x)|_{v}^{2}$, donde $du_{v}$
es la medida de Haar en $B_{v}$ que cumple con $|\cal{O}_{0}(1)_{v}|=1$, donde
$\cal{O}_{0}(1)\subset B$ es el orden maximal dado por
\eqref{eq:ordenmaximalmatrices}. Denotamos $d\hhat{u}^{\times}$ la medida de
Haar $\prod_{v}\,du_{v}^{\times}$ en $\Idfin{B}$.

Si $\hhat{\pi}=(\pi_{v})_{v}\in\Idfin{B}$, entonces
\begin{align*}
	[\Idfin{\cal{O}}\hhat{\pi}\Idfin{\cal{O}}] & \,=\,
		\prod_{v}\,[\cal{O}_{v}^{\times}\pi_{v}\cal{O}_{v}^{\times}]
	\text{ .}
\end{align*}
%
Para todo lugar $v$ salvo finitos, $\pi_{v}\in\cal{O}_{v}^{\times}$ y
$\cal{O}_{v}^{\times}=\cal{O}_{0}(1)_{v}^{\times}$ para casi todo $v$,
tambi\'{e}n. En particular,
\begin{align*}
	\cal{O}_{v}^{\times}\pi_{v}\cal{O}_{v}^{\times} & \,=\,
		\cal{O}_{v}^{\times}\,=\,
		\cal{O}_{0}(1)_{v}^{\times}
\end{align*}
%
para casi todo lugar finito $v$. Esto implica que, dada una forma modular
$\phi$, la convoluci\'{o}n
\begin{equation}
	\label{eq:operadordecoclasedobledehilbertconvolucion}
	\phi * [\Idfin{\cal{O}}\hhat{\pi}\Idfin{\cal{O}}] (g,\hhat{\alpha})
		\,=\, \int_{\Idfin{B}}\,\phi(g,\hhat{\alpha}\hhat{u}^{-1})\,
			[\Idfin{\cal{O}}\hhat{\pi}\Idfin{\cal{O}}](\hhat{u})\,
			d\hhat{u}^{\times}
\end{equation}
%
est\'{e} definida. De la descomposici\'{o}n
\eqref{eq:descomposiciondecoclasedoble}, se deduce que
\begin{align*}
	\phi * [\Idfin{\cal{O}}\hhat{\pi}\Idfin{\cal{O}}](g,\hhat{\alpha})
	& \,=\,\sum_{i}\,\int_{\Idfin{B}}\,
		\phi(g,\hhat{\alpha}\hhat{u}^{-1})
			[\Idfin{\cal{O}}\hhat{\pi}_{i}](\hhat{u})\,
			d\hhat{u}^{\times} \\
	& \,=\, \sum_{i}\,\phi(g,\hhat{\alpha}\hhat{\pi}_{i}^{-1})
		|\Idfin{\cal{O}}\hhat{\pi}_{i}|
	\text{ ,}
\end{align*}
%
donde la medida $|\Idfin{\cal{O}}\hhat{\pi}_{i}|$ del trasladado es
independiente de $\hhat{\pi}_{i}$ e igual a $|\Idfin{\cal{O}}|$ y, en
definitiva,
\begin{equation}
	\label{eq:igualdadentreoperadordecoclasedobleyconvolucion}
	\phi\operadorcoclases{\peso{k}}{%
			\Idfin{\cal{O}}\hhat{\pi}\Idfin{\cal{O}}}
		\,=\,\frac{1}{|\Idfin{\cal{O}}|}\,
			\phi * [\Idfin{\cal{O}}\hhat{\pi}\Idfin{\cal{O}}]
	\text{ .}
\end{equation}
%

Los operadores de convoluci\'{o}n $[\Idfin{\cal{O}}\hhat{\pi}\Idfin{\cal{O}}]$,
se factorizan como producto de operadores que act\'{u}an en lugares distintos.
Dado un lugar $v$ y un elemento $\pi_{v}\in B_{v}^{\times}$, sea
$\hhat{\pi_{v}}\in\Idfin{B}$, el id\`{e}le dado por $(\hhat{\pi_{v}})_{w}=1$,
si $w\not=v$, e igual a $\pi_{v}$ en $v$. Dado $\hhat{\pi}\in\Idfin{B}$,
escribimos
\begin{math}
	\tilde{T}_{\hhat{\pi}}\phi=\phi\operadorcoclases{\peso{k}}{%
		\Idfin{\cal{O}}\hhat{\pi}\Idfin{\cal{O}}}
\end{math}~.
Si $\hhat{\pi}=(\pi_{v})_{v}$, sean $\lista{w}{t}\in\lugares[f]{F}$ los lugares
finitos para los cuales $\pi_{w_{i}}\not\in\cal{O}_{w_{i}}^{\times}$ o
$\cal{O}_{w_{i}}^{\times}\not=\cal{O}_{0}(1)_{w_{i}}^{\times}$. Para todo
$v\in\lugares[f]{F}$, se cumple que
\begin{align*}
	\big(\tilde{T}_{\hhat{\pi_{v}}}\phi\big)(g,\hhat{\alpha}) & \,=\,
		\frac{1}{|\cal{O}_{v}^{\times}|}\,\int_{%
			\cal{O}_{v}^{\times}\pi_{v}\cal{O}_{v}^{\times}}\,
			\phi(g,\hhat{\alpha}h_{v}^{-1})\,dh_{v}^{\times}
	\text{ .}
\end{align*}
%
Si $v\not= w$, vale que
\begin{math}
	\tilde{T}_{\hhat{\pi_{v}}}\circ\tilde{T}_{\hhat{\pi_{w}}}=
		\tilde{T}_{\hhat{\pi_{w}}}\circ\tilde{T}_{\hhat{\pi_{v}}}
\end{math}~, porque $\tilde{T}_{\hhat{\pi_{v}}}$ y $\tilde{T}_{\hhat{\pi_{w}}}$
act\'{u}an en coordenadas distintas. Adem\'{a}s, si $v\not\in\{\lista{w}{t}\}$,
$T_{\hhat{\pi_{v}}}=\id$. As\'{\i},
\begin{align*}
	\tilde{T}_{\hhat{\pi}} & \,=\,\tilde{T}_{\hhat{\pi_{w_{1}}}}
		\circ\,\cdots\,\,\circ\tilde{T}_{\hhat{\pi_{w_{t}}}}
	\text{ .}
\end{align*}
%
No volveremos a utilizar la notaci\'{o}n $\tilde{T}_{\hhat{\pi}}$ para
referirnos a estos operadores.

