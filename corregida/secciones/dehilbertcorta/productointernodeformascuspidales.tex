Dado $z\in\hP^{n}$, definimos
\begin{math}
	\Im(z)^{\peso{k}}:=
		\prod_{i=1}^{n}\,\Im(z_{i})^{k_{i}}
\end{math}~.
Sea $\Gamma\subset\GLtp_{2}(F)$ un subgrupo de congruencia y sea
$f\in\modularH{k}{\Gamma}$. Definimos una funci\'{o}n
\begin{equation}
	\label{eq:funcionmodularasociadadehilbert}
	\varphi(z) \,:=\,f(z)\,\Im(z)^{\peso{k}/2}
\end{equation}
%
y vemos que, si $\gamma\in\Gamma$,
\begin{align*}
	\varphi(\gamma z) & \,=\,
		\bigg(\prod_{i=1}^{n}\frac{j(\gamma_{i},z_{i})^{k_{i}}}%
		{\det(\gamma_{i})^{m_{i}+k_{i}-1}}\bigg)\, f(z)
	\prod_{i=1}^{n}\,\frac{\Im(z)^{k_{i}/2}}{|j(\gamma_{i},z_{i})|^k_{i}}\,
		\det(\gamma_{i})^{k_{i}/2} \\
	& \,=\, \bigg(\prod_{i=1}^{n}\,\frac{1}{\det(\gamma_{i})^{k_{0}/2-1}}
	\Big(\frac{j(\gamma_{i},z_{i})}{|j(\gamma_{i},z_{i})|}\Big)^{k_{i}}
		\bigg)\,f(z)\Im(z)^{\peso{k}/2} \\
	& \,=\, \frac{1}{\norma(\det\,\gamma)^{k_{0}/2-1}}\,
	\bigg(\prod_{i=1}^{n}\,\Big(\frac{j(\gamma_{i},z_{i})}{%
			|j(\gamma_{i},z_{i})|}\Big)^{k_{i}}\bigg)\,\varphi(z)
	\text{ .}
\end{align*}
%
En particular, como $\det(\gamma)\in\oka{F,+}^{\times}$,
$\norma(\det\,\gamma)=1$ y $|\varphi(z)|$ es constante en las \'{o}rbitas
de $\Gamma$ en $\hP^{n}$. Entonces $|\varphi(z)|$ define una funci\'{o}n
continua en el cociente $\Gamma\backslash\hP^{n}$.
% Si, adem\'{a}s, $f$ es una
% forma cuspidal, entonces $|\varphi(z)|$ se extiende por cero a la
% compactificaci\'{o}n $X(\Gamma)$.

Sean $f,g\in\spitzH{k}{\Gamma}$ dos formas cuspidales y sean $\varphi_{f}$ y
$\varphi_{g}$ las funciones asociadas a $f$ y a $g$ por
\eqref{eq:funcionmodularasociadadehilbert}. El producto
\begin{math}
	\varphi_{f}\overline{\varphi_{g}}=
		f(z)\overline{g(z)}\,\Im(z)^{\peso{k}}
\end{math}
es $\Gamma$-invariante y acotado en $\hP^{n}$.
% De acuerdo con la \'{u}ltima afirmaci\'{o}n del p\'{a}rrafo anterior, este
% producto determina una funci\'{o}n en la variedad $X(\Gamma)$.
Definimos
\begin{equation}
	\label{eq:productodepeterssongamacero}
	\langle f,g\rangle_{\Gamma}\,:=\,
		%\frac{1}{\mu\big(\Gamma\backslash\hP^{n}\big)}
		\int_{\Gamma\backslash\hP^{n}}\,f(z)\overline{g(z)}\,
		\Im(z)^{\peso{k}}\,d\mu
	\text{ ,}
\end{equation}
%
donde $d\mu$ es la medida invariante \eqref{eq:medidaenelsemiplanodehilbert}.
Con respecto a esta medida, $\Gamma\backslash\hP^{n}$ tiene volumen
finito y la integral anterior define un producto interno en el espacio
$\spitzH{k}{\Gamma}$ de formas cuspidales que llamamos \emph{producto interno %
de Petersson}.\index{producto interno de Petersson}%
\footnote{
	\cite[Ch.~II]{FreitagForms}
}
En particular, las formas cuspidales son de cuadrado integrable:
\begin{align*}
	\int_{\Gamma\backslash\hP^{n}}\,|f(z)|^{2}|\Im(z)|^{\peso{k}}
		\,d\mu & \,<\, \infty
	\text{ .}
\end{align*}
%

\begin{propoPeterssonPropiedades}\label{propo:peterssonpropiedades}
	Sean $\Gamma'\subset\Gamma$ subgrupos de congruencia y sea
	$A\in\GLtp_{2}(F)$.
	\begin{itemize}
		\item[(i)] si $f\in\spitzH{k}{\Gamma}$ y
			$g\in\spitzH{k}{A^{-1}\,\Gamma\,A}$, entonces
			\begin{align*}
				\langle f\operadormatrices{\peso{k}}{A},
					g\rangle_{A^{-1}\Gamma A} & \,=\,
				\langle f,g\operadormatrices{\peso{k}}{A^\iota}
					\rangle_{\Gamma}
				\text{ ,}
			\end{align*}
			%
			donde $A^\iota = (\det\,A)\,A^{-1}$;
		\item[(ii)] si $f,g\in\spitzH{k}{\Gamma}$, entonces
			$f,g\in\spitzH{k}{\Gamma'}$ y
			\begin{align*}
				\langle f,g\rangle_{\Gamma'} & \,=\,
				|\Gamma:\Gamma'|\,\langle f,g\rangle_{\Gamma}
				\text{ ;}
			\end{align*}
			%
		\item[(iii)] si $f\in\spitzH{k}{\Gamma}$,
			$g\in\spitzH{k}{\Gamma'}$ y
			$\Gamma=\bigsqcup_{i}\,\Gamma'\delta_{i}$ es una
			descomposici\'{o}n en coclases, entonces
			\begin{math}
				h=\sum_{i}\,g\operadormatrices{\peso{k}}{%
						\delta_{i}}\in
					\spitzH{k}{\Gamma}
			\end{math}
			y
			\begin{align*}
				\langle f,h\rangle_{\Gamma} & \,=\,
					%|\Gamma:\Gamma'|\,
					\langle f,g\rangle_{\Gamma'}
				\text{ .}
			\end{align*}
			%
	\end{itemize}
	%
\end{propoPeterssonPropiedades}

\begin{proof}
	Es an\'{a}loga al caso de formas modulares sobre $\bb{Q}$. Se puede
	consultar, por ejemplo, \cite[\S\S~5.4 y 5.5]{DiamondShurman}.
\end{proof}
